\documentclass[output=paper]{langsci/langscibook}
\ChapterDOI{10.5281/zenodo.3972830}

\author{Susana Bejar\affiliation{University of Toronto}\and Diane Massam\affiliation{University of Toronto}\and Ana-Teresa Pérez-Leroux\affiliation{University of Toronto}\lastand Yves Roberge\affiliation{University of Toronto}}
\title{Rethinking complexity}

% \chapterDOI{} %will be filled in at production

\abstract{This paper addresses the nature of complexity of \isi{recursion}. We
    consider four asymmetries involving caps on \isi{recursion} observed in
    previous experimental acquisition\is{language acquisition} studies, which
    argue that complexity cannot be characterized exclusively in terms of the
    number of iterations of \isi{Merge}.  While \isi{recursion} is essentially
    syntactic and allowed for by the minimalist toolkit via \isi{Merge},
    selection, and labeling\is{labelling} or projection, the complexity of
recursive outputs arises at the interface.}

\maketitle

\begin{document}\glsresetall

\section{Introduction}

\textcite{WatHauRobHor2014} (WHRH\nocite{WatHauRobHor2014}) discuss three
criterial properties of \isi{recursion} and argue that “by these necessary and
sufficient criteria, the grammars of all natural languages are recursive” (p.\
1). Phrases and sentences are defined recursively “in a stepwise strongly
generative process creating increasing complexity” (p.\ 6). We focus here on
this notion of \isi{complexity}, since, from the perspective that recursive
structures are the result of repeated applications of \isi{Merge} operations,
structures arising from similar derivational steps should all be derivationally
equally complex.  This squib sheds light on the nature of the \isi{complexity}
of recursion in human grammar through a theoretically-based exploration of four
asymmetries observed in a series of experimental studies on the
acquisition\is{language acquisition} of self-embedding structures we have
conducted in the last few years. Note that, while \enquote{self-embedding}
often refers to complement structures, our use of the term generalizes over
\isi{adjunction} as well.

WHRH\nocite{WatHauRobHor2014} emphasize that \isi{recursion} is an
architectural property of the language faculty as opposed to a characterization
of output structures, pointing to two correlates of this view: (i)
\isi{recursion} is an architectural universal, not an emergent property; (ii)
the caps on \isi{recursion} that are observable in output structures re\-sult
from arbitrary external factors. Here, our work contrasts with
WHRH\nocite{WatHauRobHor2014} in two points. First, we investigate
\isi{recursion} as a property of outputs while what matters for
WHRH\nocite{WatHauRobHor2014} is the \isi{complexity} of the recursive\is{recursion}
procedure itself.\footnote{This is not to say of course that the issue of the
    \isi{complexity} of the recursive\is{recursion} program is of no interest but the
    goal of our research is to identify the source of the difficulties that
complex structures create for children (and adults).} Second, we examine caps
on \isi{recursion} in child language as a window into development of the
language faculty. Nonetheless, we seek to explore the links between our studies
and the positions articulated in WHRH\nocite{WatHauRobHor2014}. In particular,
we examine the connection between children’s capacity to produce self-embedding
structures and the notion of \isi{complexity}. We argue that while recursion is
essentially syntactic and allowed for by the minimalist toolkit, via
\isi{Merge}, selection, and labeling\is{labelling} or projection (cf.
\citealt{HauChoFit2002}), the \isi{complexity} of recursive\is{recursion} outputs
arises at the interface.\footnote{The view that \isi{recursion} is in narrow
    syntax we share with WHRH\nocite{WatHauRobHor2014} and many others (e.g.\
    \citealt{Moro2008,NevPesRod2009}); however, it has also been proposed to be
    in the discourse (\citealt{EvaLev2009,Koschmann2010}), or a consequence of
phasal architecture and the interface (\citealt{ArsHin2010}).}

The growth of grammatical competence gives rise to the ability to produce
longer and more complex sentences. Although there is little consensus about
what constitutes complexity
\parencite{Culicover2013,RoeSpea2014,TroBay2015,NewPre2014,McWhorter2011}, most
discussions agree that embedding increases complexity
\parencite{CulJac2006,Givon2009}. However, in the narrow syntax, embedding by
itself should not determine \isi{complexity}, as it is given by recursive\is{recursion} \isi{Merge}. We
argue that \isi{complexity}, rather than being strictly correlated with recursive
iterations of \isi{Merge}, arises at the interface.  Moreover, because recursive
iterations of \isi{Merge} can result in different varieties of recursively embedded
output structures, some structural elaborations turn out to be more complex
than others.

At the outset, the language of young children does not include structurally
elaborate expressions; various forms of structural elaboration emerge during
the preschool years. Absence of a structure leads to the attribution of the
property of \isi{complexity} to that structure, but often without a clear notion of
what \isi{complexity} is. Here, we discuss four aspects of \isi{complexity} in recursive
structures that present challenges for a simple definition. We consider these
issues in the context of recursive\is{recursion} NP embedding, including conjunction,
genitives, PP structures, and \isi{relative clauses}. In previous work
(\citealt{Perez-LerouxEtAl2012,Perez-LerouxEtAl2018a,Perez-LerouxEtAl2018b}) we
observed that recursive\is{recursion} conjunction seems simpler than recursive\is{recursion} PP
modification, that sequential double modification is less complex than
twice-embedded modification, and that the combination of relative clause\is{relative clauses} and PP
modification is somehow less complex than twice-embedded PP modification, at
least in some of the languages studied. From this, we argue that \isi{complexity} is
not uniform, and that the \isi{complexity} emerging from recursive\is{recursion} embedding is a
property of the interface, and not a property of narrow syntax.

We now turn to a discussion of four contrasts that shed light on the nature of
complexity.

\section{Coordination and modification}

Children learn the basic ingredients required for NP elaboration quite early,
including relevant functional elements \citep{Brown1973}, and semantic
relations \citep{BloomEtAl1975}. \citet{Perez-LerouxEtAl2012} investigated the
points when children learn to iterate forms of NP elaboration. Using a
referential task, we elicited twice-embedded genitives\is{genitive case} \REF{ex:bejar:1a} and
modificational PPs \REF{ex:bejar:1b}. Contexts were set up so twice-embedded
modification was needed to disambiguate target referents from other competing
referents. For instance, we need something like \REF{ex:bejar:1b} to uniquely
describe the target in a scenario with two girls, each with a dog, where the
only difference is a hat on one of the dogs.  We controlled for whether
children could produce utterances with three NPs, by testing
\isi{coordination}, as in \REF{ex:bejar:2}, which matched the utterance length of
the recursively embedded conditions.

\ea\label{ex:bejar:1}
    \ea\label{ex:bejar:1a} the boy’s cat’s tail
    \ex\label{ex:bejar:1b} the girl with a dog with a hat
    \z
\ex\label{ex:bejar:2}
    a boy, a bicycle, and a doll
\z

Of key importance is the following result: children had no difficulties
producing coordinate NPs, but had substantial difficulties with NP embedding.
Two-thirds of the younger children produced no NP embedding at all. This does
not follow from current assumptions about \isi{coordination} structures.
Recently, the goal has been to integrate \isi{coordination} into X-bar theory
(contra, e.g.\ \citealt{Jackendoff1977}), whether by \isi{adjunction} \citep{Munn1993}
or complementation \citep{Johannessen1998}. Under this approach, coordinates
are structurally equivalent to either of the twice-embedded structures in
\REF{ex:bejar:1}. This precludes a purely structural explanation of the relative
difficulty of the PP and genitive\is{genitive case} recursive\is{recursion}
structures.

The NP embedding/coordination contrast is thus placed squarely in the domains
of processing and/or semantics, i.e. interpretive \isi{complexity} at the interface.
Coordinating three NPs just augments a set. Embedding, via either \isi{adjunction} or
complementation, reformulates the description of a set. The descriptive content
of lower referents serves to restrict the domain of the higher nominal.

\section{Sequential and recursive PP modification}

A subsequent study explored the next logical question
\parencite{Perez-LerouxEtAl2018b}. Does each step in embedding increase the
complexity of the nominal structure? We set up a minimal comparison between two
types of doubly modified structures involving locatives, relying on a similar
referential task to the one previously employed, but contrasting two types of
contexts. One condition required two PPs modifying the same head noun as in
\REF{ex:bejar:3a}, whereas in the other \REF{ex:bejar:3b}, the head noun is
modified by a PP, itself modified by a lower PP.

\ea\label{ex:bejar:3}
    \ea\label{ex:bejar:3a} the plate [ under the table ] [ with oranges ]
    \ex\label{ex:bejar:3b} the bird [ on the alligator [ in the water ]]
    \z
\z

A detailed comparison of these two constructions reveals that, syntactically
and semantically, they are equally complex, at least in principle. Their
generation involves not only the same core operations (e.g.\ \isi{Merge}, predicate
modification), but also the same number of core operations. Given the formal
parallels of the two constructions, we would expect comparable patterns of
production. However, a strong asymmetry arises. Both children and adults
produced twice-embedded PP modification at half the rates of double sequential
modification. Since everything else is held constant, productivity can be
interpreted as a reflection of less \isi{complexity}.~Given the comparability
between the task and the structure, this suggests that depth of embedding
results in more complex configurations. What might account for this difference?
Again, we must look to the interface to explain this. Under the logic of
phase\is{phases} theory, a phasally complete functional domain like DP should
cease to function as a complex object (\glsdesc{PIC},
\glsunset{PIC}\gls{PIC}).\is{Phase Impenetrability Condition} While
\REF{ex:bejar:3a} and \REF{ex:bejar:3b} are equivalent with respect to the number
of phasal domains (assuming one views DP as a phase\is{phases}), in
\REF{ex:bejar:3b}, but not \REF{ex:bejar:3a}, the referent of the head noun is
restricted by an expression that is inaccessible under the \gls{PIC}\is{Phase
Impenetrability Condition}.  In fact, the descriptive content of the lower
phase\is{phases} \emph{in the water} in \REF{ex:bejar:3b} was essential for
success in the experimental task: other alligators lurked on land. We submit
that this is the source of the added \isi{complexity} of these structures, but
note that this is not \isi{complexity} in the narrow syntax -- the narrow
syntax freely generates such structures -- the challenge rests in interpretive
requirements at the interface.\is{PIC|see{Phase Impenetrability Condition}}

\section{PP/relative clause modification and recursive PP
modification}\label{sec:02.4}

A third observation in support of our view of \isi{complexity} also originates
from \citet{Perez-LerouxEtAl2018b}.  In lieu of the target PP modifiers
\REF{ex:bejar:4a}, speakers commonly substituted \isi{relative clauses}
\REF{ex:bejar:4b} and a mix of PP and relative clause\is{relative clauses}
constructions \REF{ex:bejar:4c}.

\ea\label{ex:bejar:4}
    \ea\label{ex:bejar:4a} The one on the plate with the apple.
    \ex\label{ex:bejar:4b} The bird that’s on the crocodile that’s in the water.
    \ex\label{ex:bejar:4c} The one on the one on the crocodile’s eyes that was in the   water.
    \z
\z

That adults were prone to use the more elaborate \glspl{RC} where simple PPs
would do the work was a surprise. That children did so too was more so, given
the extensive literature on children’s difficulties with \isi{relative clauses}
(see references in \citealt{FriedmannEtAl2009,Givon2009}). Interestingly, these
expansions were particularly frequent when the target was a twice-embedded PP
structure.  There, the relative and mixed PP/relative strategies represented over
40\% of adults’ and children’s target responses. This was true in \ili{English} as
well as in recent data from \ili{German} preschoolers, obtained with the same methods
\citep{Lowles2016}. These responses are perfectly natural, and certainly
successful in the context of our task. From a \isi{complexity} perspective,
they are perplexing~-- especially in the case of children~-- inasmuch as they
constitute longer and structurally more elaborate constructions that,
importantly, do not informationally add anything when compared to PP responses.
The additional syntactic and semantic \isi{complexity} introduced by \glspl{RC}
is not limited to the additional lexical material but is also due to the fact
that they involve displacement and dependencies in syntax as well as additional
semantic operations. Yet their use strongly suggests that the modification
relation is not problematic. This leaves us with a mystery: Why should children
and adults frequently use the structurally more elaborate relative clause\is{relative clauses}
strategy to express modification?

If \isi{complexity} is not computed in narrow syntax as the result of a number of
recursive applications of \isi{Merge}, then this result can be interpreted from
a different angle. Several possibilities arise which differ with respect to
how \enquote{detached} from the computational component the \isi{complexity} issue
really is. For instance, as early as 1963, \citeauthor{ChoMil1963} argued that
the \isi{complexity} of recursive\is{recursion} \isi{self-embedding} results
from performance processes, not formal grammar.  In contrast, \citet{ArsHin2012}
note that instances of X directly dominating another instance of X are rare:
the common strategy is for referential expressions to dominate others of the
same type indirectly, via sequences of functional categories. For them this is
a direct result of the phasal architecture of the computational component.
Everything seems to function as if to create a structural contour between
referential expressions in a phrase.

On a final note, our conclusion that \isi{complexity} of
recursive\is{recursion} embedding does not reside in narrow syntax is supported
by comparable data recently collected from French and \ili{Japanese}
\parencite{BambaEtAl2016,Robergeetal2018}. In these languages, children do not
readily rely on the relative clause\is{relative clauses} strategy; they
incorporate it gradually, as one would expect. One possible explanation route is
to link this cross-linguistic difference to uniformity in the directionality of
embedding: \ili{French} and \ili{Japanese} are uniformly right- and
left-embedding, whereas German and \ili{English} mix branching directionality in
their nominal syntax.  If this is confirmed by further studies on additional
languages, we would conclude that recursive\is{recursion} PP embedding is not
computationally more complex than any other applications of \isi{Merge} and
avoidance of twice-embedded PPs in our experiments must be accounted for by
recourse to other considerations.

\section{Genitives and PPs}

The cases discussed so far implicitly follow a quantity metric, comparing the
target structures in the two types of double-modification contexts with respect
to the number of noun phrases, embedding steps, layers of functional structure,
and steps required for semantic derivation. Let us now turn to qualitative
differences. Do different types of NP embedding yield differences in complexity
for reasons unrelated to structural metrics? Here we focus on possessive
embedding \REF{ex:bejar:1a}, which differs from comitative PPs \REF{ex:bejar:1b} in
terms of directionality and case marking. Again, on minimalist assumptions
about \isi{recursion}, the answer should be no.  However, accounts of acquisition
difficulties often rely on notions of uniformity, and the basic typology of the
target language. It is conceivable that in \ili{English}, a fundamentally
right-branching and analytic language, the genitive\is{genitive case} \emph{'s}
construction might be constrained in acquisition\is{language acquisition}. It is, after all, constrained
in related languages. \citegen{RoeSny2004} observation that the cognate
possessive form in \ili{German} does not iterate (i.e., \ili{German} allows \emph{NP’s NP}
but not \emph{NP’s NP’s NP}) was the starting point in the study of the
acquisition of recursive\is{recursion} \isi{self-embedding} structures.  Such
language differences prove that rule acquisition\is{language acquisition}
(i.e., possessive \emph{-s}, in this case) is a learning step distinct from the
acquisition\is{language acquisition} of rule iteration (allowing multiple
instances of the embedding process).  The data in \citet{Perez-LerouxEtAl2012}
suggested a delay. First-level embedding appeared simultaneously for
genitives\is{genitive case} and PP modifiers. Second-level of embedding was a
distinct stage, attained first for PP modifiers. Since few children attained
the second stage in the development of complex NPs, this was clearly worth
further investigation.  We recently elicited data on the production of
recursive\is{recursion} possessives and PPs in a group of seventy-one
English-speaking children in Toronto \parencite{Perez-LerouxEtAlinprep}. While
overall rates of production success were slightly higher for
recursive\is{recursion} comitative PPs, children did not acquire them earlier
than genitives\is{genitive case}.  In fact, the converse was true.
Individually, more children could produce recursive\is{recursion} sequences of
possessive \emph{-s} than of comitatives (\emph{NP with NP with NP}) at a ratio
of 5 to 1 compared to the converse pattern.  This is due to the PP/\gls{RC}
trade-off described in \Cref{sec:02.4}.  Possessives were rarely substituted by other
forms, so a child could more easily embed possessives twice. We can safely
conclude that the structurally distinct properties of the possessive
construction do not constrain children’s ability to iterate
genitive\is{genitive case} embedding.

\section{Conclusion}

The notion of \isi{complexity} -- often loosely defined and used intuitively -- is
illuminated by the consideration of caps on \isi{recursion} as observed in
acquisition studies. Four cases were discussed, all pointing to the conclusion
that \isi{complexity} cannot be characterized exclusively in terms of the number of
iterations of \isi{Merge}. In closing, we return to
WHRH\nocite{WatHauRobHor2014} and the view of \isi{recursion} articulated
therein.  WHRH\nocite{WatHauRobHor2014} take \isi{complexity} to correlate with
iterations of the recursively defined generative structure-building procedure,
with caps on recursion/complexity reducing to (arbitrary) extra-linguistic
considerations.  Couched in the traditional dichotomy, their focus is on
competence. We argued that this view of \isi{complexity} does not shed light on
the nature of caps on recursion observed in the language
acquisition\is{language acquisition} studies reported here. However, we believe
our results are consistent with the overall view of recursion articulated in
WHRH\nocite{WatHauRobHor2014}. The absence of a correlation between
\isi{complexity} and recursive\is{recursion} iterations of \isi{Merge} is
exactly what one might expect if the recursive\is{recursion} nature of grammar
is an architectural universal and hence unlearnable (as
WHRH\nocite{WatHauRobHor2014} say). Likewise, WHRH\nocite{WatHauRobHor2014}’s
view that caps on recursion/complexity must be understood in terms of
conditions external to narrow syntax resonates with our findings, though it is
not at all clear to us how external (or arbitrary) these really are. Our
studies point to the need for future work to determine and articulate the
nature of \isi{complexity} at the interface.

\printchapterglossary{}

\section*{Acknowledgements}

Authors are listed in alphabetical order. We are grateful to Anny P.
Castilla-Earls, Erin Hall, Gabrielle Klassen, Erin Pettibone, Tom Roeper, Petra
Schulz, Ian Roberts and two anonymous reviewers for helpful comments and
discussion. We also gratefully acknowledge funding from the Social Sciences and
Humanities Research Council of Canada (IG 435-2014-2000 \enquote{Development of
NP complexity in children} to Ana T. Pérez-Leroux \& Yves Roberge).

{\sloppy
\printbibliography[heading=subbibliography,notkeyword=this]
}

\end{document}
