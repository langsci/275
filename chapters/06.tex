\documentclass[output=paper]{langsci/langscibook}
\ChapterDOI{10.5281/zenodo.3972836}

\author{Anna Roussou\affiliation{University of Patras}}
\title{Some (new) thoughts on grammaticalization: Complementizers}

% \chapterDOI{} %will be filled in at production

\abstract{Grammaticalization creates new grammatical exponents out of existing
(lexical) ones. The standard assumption is that this gives rise to categorial
reanalysis and lexical splits. The present paper argues that categorial
reanalysis may not be so pervasive and that lexical splits may also be
epiphenomenal. The set of empirical data involves the development of
(Indo-European) complementizers out of pronouns. The main claim is that the
innovative element (the complementizer) retains its nominal feature; thus
strictly speaking, there is no categorial\is{syntactic categories} \isi{reanalysis}, but a change in function
and selectional requirements, allowing for an IP complement as well. As a
complementizer, the pronoun is semantically weakened (the nominal core), and
phonologically reduced (no prosodic unit). In its pronominal use, it may bind a
variable (interrogative/relative) and defines a prosodic unit. What is
understood as a lexical split then reduces to a case of different selectional
requirements, followed by different \gls{LF} and \gls{PF} effects.}

\maketitle

\begin{document}\glsresetall

%\textbf{Keywords.} complementizer\is{complementizers}, pronoun, \isi{reanalysis}, syntactic category

\section{An overview}

According to \citet{Meillet:1912}, the two basic mechanisms for \isi{language
change} are \emph{grammaticalization} and \emph{analogy}. While
\isi{grammaticalization} creates new grammatical material out of
\enquote{autonomous} words, analogy develops new paradigms by formal
resemblance to existing ones. Grammaticalization\is{grammaticalization} has
received great attention in the literature (for an overview
see~\citealt{NarHei2011}), raising the question whether it is a mechanism of
change or an epiphenomenon. The answers provided mainly depend on the
theoretical framework adopted and the view on how \enquote{grammar} is to be
defined.  Thus, in functional approaches, \isi{grammaticalization} is a
mechanism that leads to the formation of \enquote{grammar} (or of grammatical
structures), while in formal approaches, \isi{grammaticalization} is either
denied altogether
\parencite{Newmeyer1998,Lightfoot1998,Lightfoot2006,Janda2001,Joseph2011} or
considered an epiphenomenon \parencite{RobRou2003,vanGelderen2004}.

Despite the different views on the topic, it is generally accepted that
\isi{grammaticalization} (or whatever it reduces to) has a visible effect
cross{}-linguistically. There are common tendencies and patterns in how the
lexical to functional change may take place (see \citealt{HeiKut2002} for a
wide range of examples). For example, \isi{complementizers} can have their
origin in pronouns (in\-ter\-rog\-a\-tives, demon\-stra\-tives, relatives),
verbs (\emph{say}, \emph{like}, etc.), nouns (\emph{thing}, \emph{matter}), or
prepositions (allative). Within functional/typological perspectives,
\isi{grammaticalization} is pri\-mar\-i\-ly viewed as a \enquote{semantic}
process where concepts are transferred into constructions (for an overview, see
\citealt{HoppTrau2003}). Once the relevant elements are used as grammatical
markers, they show semantic \enquote{bleaching} (loss of primary meaning) and
phonological reduction. According to \citet{Traugott2010},
\isi{grammaticalization} is not only a matter of reduction, but also of
pragmatic expansion in terms of content. At the same time, a typological
account shows that some lexical items are more amenable than others to give
rise to grammatical markers, although this is not deterministic in any respect.
Still, this observation points towards an interesting direction with respect to
how the lexicon interacts with (morpho-)syntax.

Within the formal approach to \isi{grammaticalization}, the basic assumption is that
it is an epiphenomenon. More precisely, \isi{grammaticalization} is argued to derive
through the loss of \isi{movement} steps. In more technical terms, it is a change
from internal to external merge. This change gives rise to the creation of new
exponents of functional heads\is{functional items}, along with structural simplification (see for
example \citealt{RobRou2003,vanGelderen2004}). The notion of simplification is
built on the idea that external merge draws directly on the lexicon, while
internal merge draws on lexical items already present in the
structure.\footnote{The change from internal to external merge is rather
    simplified here. As \citet{RobRou2003} point out, this change may involve
    additional steps, including the \enquote{movement} of features from a lower to a
    higher position; this is, for example, the case with the development of the
    subjunctive marker \emph{na} in \ili{Greek}, where the expression of mood changes
    from being an inflectional/affixal feature to being a modal marker
    (\emph{na}) in the left periphery, p.\ 73--87). In all cases though, the
    change known as \isi{grammaticalization} is selective, affecting a subset of
lexical items, as also pointed out by an anonymous reviewer; a more thorough
discussion is provided in \textcite[Ch.\ 5]{RobRou2003}.}  Thus internal merge
gives rise to displacement (movement) and requires at least two copies of the
same lexical item in different structural positions. The change from internal
to external merge implies a single copy in the higher position and elimination
of the lower one. This single copy becomes the new exponent of the higher
(functional) head. Since merge is bottom-up, it follows that internal merge
will also follow this upward (and leftward) path, and the change from internal
to external merge will also affect the upper parts of the structure.

In standard terms, irrespectively of the framework adopted, a basic tenet is
that \isi{grammaticalization} involves a change from lexical to functional, or from
functional to functional, as in~\eqref{ex:key:6.1}:

\ea\label{ex:key:6.1}
     Content word > grammatical word > clitic > inflectional affix
\z

In~\eqref{ex:key:6.1} above, what appears on the left hand side of the arrow
“>” indicates a preceding stage. Assuming that lexical categories (content
words) are embedded under functional projections (grammatical morphemes), the
order in~\eqref{ex:key:6.1} is consistent with the view that
\enquote{grammaticalization}\is{grammaticalization} is accounted for in a
bottom-up fashion. More precisely, a lexical item \emph{α} can start as part of
a lexical projection, and by internal merge occur in a higher functional
position \emph{β}. The loss of \isi{movement} steps has an effect in the categorial
status of \emph{α}, which now becomes the exponent of \emph{β}. The change from
grammatical word to clitic does not affect the functional status but affects
the morphosyntactic status of \emph{α}. The same holds for the final stage
(from clitic to an inflectional affix), where \emph{α} becomes part of the
morphological structure, as best summarized in \citegen[413]{Givon1971} quote
\enquote{today’s morphology is yesterday’s syntax}.

In the present paper I retain the basic view of \citet{RobRou2003} on
\isi{grammaticalization}, namely that it is an epiphenomenon; I also use the
term \enquote{grammaticalization} in a rather loose way, as the development of
grammatical elements out of existing ones. I take \isi{complementizers} with a
pronominal source as the exemplary case, a pattern which is very typical of the
Indo\-Eu\-ro\-pean languages. The primary question raised is whether the change
from pronoun to complementizer implies categorial\is{syntactic categories}
\isi{reanalysis}. The secondary question is whether this \isi{reanalysis} gives
rise to a lexical split that ends up creating homophonous lexical items (i.e.,
pronoun vs complementizer). The claim put forward here is that the
grammaticalized element retains (or at least may retain) its categorial core,
thus eliminating homonymy in the lexicon.  In~\Cref{sec:comp-pron}, I discuss
the dual status of some lexical items as pronouns and \isi{complementizers},
arguing that to a large extent the distinction is functional and not really
formal. In~\Cref{sec:double-pron}, I consider the properties of \ili{Greek}
declarative \isi{complementizers} in connection with their pronominal uses,
showing that we can account for the differences in terms of their \gls{LF} and
\gls{PF} properties. In~\Cref{sec:gram-cat}, I consider the implications of
this distribution for \isi{grammaticalization}, and argue that what looks like
a change from pronoun to complementizer\is{complementizers} indicates a change
in selection (expansion) and scope, with visible \gls{PF} effects also.
\Cref{sec:21-conclusions} concludes the discussion.

\section{On \isi{complementizers} and pronouns}\label{sec:comp-pron}

\citet{Kiparsky1995} argued that the development of \isi{complementizers} in
Indo-Eu\-ro\-pean shows a change from \emph{parataxis} to \emph{hypotaxis}: a
previously independent clause becomes dependent on a preceding matrix
predicate. This change is linked to a previous one, namely the development of
the C position as manifested by V2-phenomena. Another way to view this change
is as an anaphoric relation between a pronoun in the first clause which refers
to the second (paratactic) clause. \citet{RobRou2003} and
\citet{vanGelderen2004} argue that in this configuration, the pronoun is
reanalyzed as part of the second clause, with the latter becoming part
(hypotaxis) of the now main clause since it is embedded under the matrix
predicate. This can happen in two steps: first, the pronoun retains its
pronominal status and qualifies as a phrase (in a Spec position), and second,
it is reanalyzed as a C head, as in~\eqref{ex:key:6.2}:

\ea\label{ex:key:6.2}
   [\textsubscript{IP} [\textsubscript{VP} V pronoun]] [\textsubscript{IP} ] > [\textsubscript{IP} [\textsubscript{VP} V [pronoun [\textsubscript{IP} ]]]]
\z

\citet[118]{RobRou2003} argue that although this looks like \enquote{lowering},
the reanalyzed structure can still be construed in an upward fashion, since the
boundary of the second clause shifts to the left (hence upward) to include the
pronoun. In their terms, this kind of change is both categorial\is{syntactic
categories} (pronoun > complementizer) and structural (creating a complement
clause headed by the reanalyzed pronoun).\footnote{\citet[238]{Kayne2005b}
    argues that as a complementizer\is{complementizers} \emph{that} merges
    above the VP, while as a pronoun it merges inside the VP, accounting for
    the fact that as a pronoun it may inflect (in Germanic) for case, while as
    a complementizer\is{complementizers} it cannot.  Within this framework the
    change from parataxis to hypotaxis would involve merger of \emph{that} in
    different positions, signalling embedding under the Kayne’s requirement
that \enquote{For an IP to function as the argument of a higher predicate, it
must be nominalized} \parencite[236]{Kayne2005b}.  The
complementizer\is{complementizers} status further implies a silent
N.\label{fn:4.2}} A
further aspect of this change is that it has created a new exponent for the C
head.

The use of pronouns as \isi{complementizers} is quite pervasive in Indo-European
languages. \ili{English} \emph{that} is related to the demonstrative\is{demonstratives} \emph{that}
(\emph{that book}), \ili{Romance} \emph{que/che} is related to the interrogative
pronoun ‘what’ (\emph{che fai?} ‘what are you doing?’), \ili{Greek} \emph{oti} is
related to a relative pronoun, while \emph{pos} is related to the interrogative
‘how’, to mention just a few examples (see also \citealt{Rooryck2013} on \ili{French}
\emph{que} as a single element). In recent approaches to complementation, the
relation between pronouns and \isi{complementizers} is argued to hold synchronically
as well. There are basically two ways of analyzing sentential complementation:
either to reduce complement clauses to some form of relatives
\parencite[e.g.,][]{Arsenijevic2009,Kayne2010b,ManSav2011}, or to reduce
relative clauses to an instance of complementation (e.g., \citealt{Kayne1994}).
Either way, the link between the two types of clauses is evident. If indeed
there is structural similarity between relatives and complement clauses and the
assumption is that \isi{complementizers} somehow retain their (pro)nominal feature,
then what has been considered as categorial\is{syntactic categories} \isi{reanalysis} in the context of
\isi{grammaticalization} may have to be reconsidered.

In their discussion, \citet{RobRou2003} point out that one of the differences
between D \emph{that} and C \emph{that} has to do with the different
complements they embed. In particular, demonstrative\is{demonstratives} \emph{that} takes an NP
complement (a set of individuals), while complementizer\is{complementizers} \emph{that} takes an IP
complement (a set of situations/worlds).  \textcite{ManSav2007,ManSav2011} and
\citet{Roussou2010} argue that \isi{complementizers} of this sort are (pro-)nominal.
They merge as arguments of the (matrix) selecting predicate and take the CP/IP
as their complement; strictly speaking then, they are outside the complement
clause. This kind of approach maintains the view that there is embedding,
mediated by the \enquote{complementizer}, but essentially the relevant element, being a
pronominal of some sort (demonstrative, relative/interrogative) occurs as the
argument of the predicate. On this basis, it is arguable whether the pronoun
changes formally or just functionally. To be more precise, the question is how
\enquote{real} the D > C \isi{reanalysis} is. The alternative is to assume that
the new element classified as a complementizer\is{complementizers} retains its
nominal property, but expands in terms of selection, allowing not only for an
NP but for an IP complement as well.

The approach just outlined regarding complementation is very close to
\citegen[828--829]{Davidson1968} view according to which \blockquote{sentences
    in indirect discourse, as it happens, wear their logical form on their
    sleeves (except for one small point). They consist of an expression
referring to a speaker, the two-place predicate \enquote{said}, and a
demonstrative referring to an utterance.} So the sentence in~(\ref{ex:key:6.3}a) has the
logical structure in~(\ref{ex:key:6.3}b):

\ea\label{ex:key:6.3}
	\ea Galileo said that the earth was round.
	\ex Galileo said that: the earth is round.
	\ex Galileo [\textsubscript{v/VP} said [that [\textsubscript{CP/}\textsubscript{IP} the earth is round]]]
	\z
\z

The logical structure in~(\ref{ex:key:6.3}b) can translate to the syntactic structure in~(\ref{ex:key:6.3}c)
where the complementizer\is{complementizers} is the argument of the selecting predicate. If
\emph{that} is construed as a pronoun in~(\ref{ex:key:6.3}c), then it retains its nominal
feature. This is reminiscent of \citegen{Kayne1982} claim that complementizers
have the role of turning the proposition to an argument (also
\citealt{Kayne2005b}; see fn.~\ref{fn:4.2}). It also recalls \citegen[25]{Rosenbaum1967}
analysis, according to which \isi{complementizers} \blockquote{are a function of
    predicate complementation and not the property of any particular sentence
or set of sentences}.  In Rosenbaum’s analysis, \isi{complementizers} are introduced
transformationally, and complement clauses are sentences dominated by an NP
node.

Leaving many details aside, the next question that arises is to what extent the
complementizer splits apart from the pronoun it originates from, leading a life
of its own. Is this a lexical split that diachronically yields two homophonous
elements, e.g. demonstrative\is{demonstratives} \emph{that}-com\-ple\-men\-tiz\-er \emph{that},
interrogative \emph{che}-com\-ple\-men\-tiz\-er \emph{che}, interrogative
\emph{pos}-complementizer \emph{pos}, and so on? Homonymy is an instance of
accidental overlap in form with clearly distinct meanings. However, the
phenomenon here is very systematic within and across grammars and as such it
cannot be treated as accidental. If we exclude homonymy (synchronically), we
still need to account for the differences between the original pronoun and the
derived complementizer\is{complementizers}. Note that while \emph{che} as an interrogative
requires a Q operator, \emph{che} as a complementizer\is{complementizers} is declarative and
incompatible with a Q-selecting predicate. The same holds for \ili{Greek}
\emph{pos}, which shows a split between an interrogative and a
non-interrogative use, as we will see in the following section.

Interestingly, \ili{English} \emph{how} shows a similar distribution. Consider the
following examples from \citet[122]{Legate2010}:

\ea\label{ex:key:6.4}
    \ea[]{They told me how the tooth fairy doesn’t really exist.}
    \ex[]{And don’t you start in on how I really ought to be in law enforcement
        or something proper
        (\href{http://www.ealasaid.com/writing/shorts/nightchild.html}{www.ealasaid.com/writing/shorts/nightchild.html}).}
    \ex[]{They told me about how the tooth fairy doesn’t really exist.}
    \ex[*]{They told me about that the tooth fairy doesn’t really exist.}
	\z
\z

A clear difference between \emph{that} and \emph{how} is that \emph{how}
can be embedded under a preposition, while this is not the case with
\emph{that}, as in~(\ref{ex:key:6.4}c).\footnote{One of the reviewers points out that
\emph{how} can be embedded under a preposition because it has a degree
feature, which \emph{that} lacks. More precisely, \emph{about} refers to
properties which can be provided by the adverbial \emph{how}; \emph{that}
refers to truth values, so embedding under \emph{about} results in an empty
intersection, hence the ungrammaticality.}  Legate argues that
\emph{how}{}-declarative complements are associated with factivity (see also
\citealt{Nye2013}); structurally, they have an abstract DP-layer (c-selection),
and semantically they qualify as propositions (s-selection). Unlike
\emph{that}, \emph{how} is excluded in \isi{relative clauses}. The use of
\emph{how} as a complementizer\is{complementizers} does not affect the use of \emph{how} as an
interrogative manner adverbial though, as in ‘\emph{how} did you fix the
car?’ (= in what manner/way). The question then is whether complementizer
\emph{how} is a grammaticalized\is{grammaticalization} version of the manner interrogative and a
separate entry in the lexicon.

In relation to the above, note that \citet{vanGelderen2015} discusses another
use of \emph{how} in matrix yes/no questions, where it remains interrogative
(i.e. restricted to questions) but has no adverbial manner interpretation.
Consider the following examples \parencite[164--165]{vanGelderen2015}:

\ea\label{ex:key:6.5}
	\ea How would you like to go to the park?
	\ex How would you mind clearing a blocking path for Brando Jacobs, eh?
		(\url{https://twitter.com/jimshearer/status/178244064238514177})
	\z
\z

As \citeauthor{vanGelderen2015} argues, this \emph{how} occurs in matrix yes/no
questions, and is neither a manner adverbial nor a complementizer\is{complementizers}. She further
shows that throughout the history of \ili{English}, \emph{how} was not just
restricted to a \emph{wh}-manner adverbial, but also conveyed an exclamative or
an emphatic reading.  In the latter use it emphatically modifies the modal. Let
us illustrate this with the example in~(\ref{ex:key:6.5}a). In the manner reading, \emph{how}
gives rise to the interpretation “in what way would you like to go to the
park?”. In the non-manner reading it expresses the degree to which something
may hold, giving emphasis on the modal; the reading is something like “Can it
be the case that you’d like to go to the park?”, that is an epistemic one. As
van Gelderen shows, the emphatic interpretation is already attested in Old
English \emph{hu}, so this is not an innovation. What is an innovation though
is the yes/no reading of the question introduced by \emph{how}.

On the basis of the empirical data, \citeauthor{vanGelderen2015} argues for the
following steps in the development of \emph{how} in yes/no questions
(emphatic/epistemic) and complement clauses (complementizer):

\ea\label{ex:key:6.6}
	\ea Adverbial \emph{how}: \isi{movement} to Spec,CP as a manner adverb modifying vP >
	\ex \isi{Merge} in C [i-degree] (interpretable feature) >
	\ex Spec to Head (not complete for \emph{how}).
	\z
\z

The step in (\ref{ex:key:6.6}b) involves a change from internal to external
merge with an interpretable feature. The step in (\ref{ex:key:6.6}c) eliminates
specifiers in favor of heads. According to her analysis, this state is not
complete for \emph{how}, while it is for \emph{whether}. The steps in the
\isi{reanalysis} of \emph{how} affect the features associated with it. More
precisely, van Gelderen argues that, as a \emph{wh}-element, \emph{how} has the
feature bundle \{i-wh, manner/quantity/degree\}. The formal \emph{wh}-feature
is interpretable and agrees with the uninterpretable \emph{wh}-feature of C in
questions, triggering a \emph{wh}-question reading. If the interpretable
feature of \emph{how} is that of [i-degree], as opposed to [wh], then it is
emphatic (i.e., to such a great degree). If this latter feature becomes
uninterpretable, then \emph{how} is merged directly in C and \emph{how}
qualifies as a complementizer\is{complementizers}. In yes/no questions, as in
\eqref{ex:key:6.5}, \emph{how} has an interpretable polar feature. In this
latter context, according to van Gelderen, the Spec-to-Head \isi{reanalysis} is
not complete.\footnote{One of the reviewers suggests that the degree feature of
    \emph{how} combined with the pragmatics of verbs like \emph{mind},
    \emph{like}, etc., as in “I would SO like to be there”, is maintained in
    yes/no questions as well. Thus~(\ref{ex:key:6.5}b) could get the answer “Well, not very
    much”. Van Gelderen models this change in terms of interpretability (an
    interpretable feature becomes uninterpretable in its new position); I agree
    with the reviewer, however, that the degree feature remains interpretable.
    What is crucial is that yes/no questions introduced by \emph{how} implicate
    an epistemic (or evidential) reading, which shifts the degree feature from
    the predicate to the proposition.  Note also that in all the examples with
matrix \emph{how}, the modal \emph{would} is present.}

The account provided by \citet{vanGelderen2015} highlights different stages of
\emph{how} both diachronically and synchronically, by manipulating the
repertoire of features associated with \emph{how} and its structural position.
Synchronically, this allows for different functions associated with \emph{how}
(from \emph{wh}-interrogative to polar interrogative to declarative factive). One way
to account for this is to assume that activating different features gives rise
to different interpretations. Instead of treating the different \emph{hows} as
distinct elements (homonyms), we can treat all instances of \emph{how} as a
single but polysemous element, where polysemy is structurally-conditioned. For
example, in the context of a Q operator, the only available reading is that of
an interrogative, either as a \emph{wh}-element, or as an epistemic (yes/no
questions). If there is no Q operator, then no interrogative reading arises and
therefore \emph{how} can only be compatible (\emph{modulo} its degree feature)
with a declarative context under selection by a certain class of predicates
(hence its factive reading). The distinction between a specifier and a head
(complementizer) is a function of the syntactic position of \emph{how} and the
dependency it forms either with a constituent or a proposition. Note that the
interpretation of \emph{how} seems to be affected by the presence or the
absence of an operator in the clause-structure, in a way that is reminiscent of
polarity-item licensing. The pronoun then acquires its quantificational force,
as a \emph{wh}-phrase, through the presence of a Q operator. Once Q is absent, there
is no \emph{wh}-reading either, allowing for a declarative use as a complementizer\is{complementizers}. We
come back to this issue in the following section.

In what follows, I will expand the empirical base by considering similar data
in \ili{Greek} which has a range of declarative \isi{complementizers} with a
pronominal (interrogative, relative) counterpart. As will be shown below, this
\enquote{duality} can give rise to ambiguity in some contexts (recall
\emph{how} in~(\ref{ex:key:6.5}a)).

\section{The double behavior of pronouns}\label{sec:double-pron}

In the discussion that precedes we saw that a clear-cut distinction between
pronouns and \isi{complementizers} is not so obvious. To put it differently, as the
discussion in \citet{vanGelderen2015} shows, the non-manner uses of \emph{how}
are also attested in earlier stages of \ili{English}, so this is not an innovation.
One way to understand this is as follows: the non-manner readings are
compatible with a core interpretation of \emph{how} that allows it to modify
manner in qualitative terms as well (degree > emphasis). The interrogative use
depends on the activation of the \emph{wh}-feature in the scope of a Q operator; in
fact, it only arises in the scope of Q. The issue of categorial\is{syntactic categories} \isi{reanalysis} now
emerges in clearer terms: does it really exist, and if so to what extent? It is
interesting to mention that in a framework where lexical items are considered
as feature bundles in the lexicon \citep{Chomsky1995}, categorial
classification can be viewed in a different perspective, as will be shown
below.

Bearing the above in mind, let us now consider \ili{Greek} which has a range of
declarative \isi{complementizers}. Along with \emph{oti} (‘that’), we also find
\emph{pos} (‘how’). This looks very much like \ili{English} \emph{that} and
\emph{how}. There is a crucial difference though: \emph{oti} and \emph{pos}
seem to be in free variation and are selected by the same predicates (note that
some dialects may show a strong preference for \emph{pos} instead of
\emph{oti}). \ili{Greek} possesses a third declarative complementizer\is{complementizers}, namely
\emph{pu} (lit.\ ‘where’) which is selected by factive predicates
(\citealt{Christidis1982}; \citealt{Roussou1994}; \citealt{Varlokosta1994}).
The complementizer\is{complementizers} \emph{pu} also introduces restrictive and non-restrictive
relative clauses, where \emph{oti} and \emph{pos} are
excluded:\footnote{\emph{How} can also be used in \isi{relative clauses}, in examples
like “The way how she walks”.  The equivalent construction in \ili{Greek} would use
the relative pronoun \emph{opos}, which has the prefix \emph{o-} and the
\emph{wh}-pronoun \emph{pos} (lit.\ ‘the how’).}

\ea\label{ex:key:6.7}Greek
	\ea
		\gll    nomiz-o   oti/pos  kerdhis-e  to  vravio\\
			    think-\Fsg{} that/that won-\Tsg{} the prize\\
		\glt    \enquote*{I think that she won the prize.}
	\ex
		\gll    thimame  pu kerdhis-e to vravio\\
                remember-\Fsg{} that won-\Tsg{} the prize\\
		\glt    \enquote*{I remember that she won the prize.}
	\ex
		\gll    o fititis pu sinandis-es ine filos mu\\
			    the student that met-\Ssg{} is friend mine\\
		\glt    \enquote*{The student that you met is my friend.}
	\z
\z

\ili{Greek} then has a two-way distinction of three \isi{complementizers}: \emph{oti/pos}
and \emph{pu}. The two-way distinction involves factivity and relativization.
Specifically, \emph{pu}{}-complements are associated with a factive
interpretation, while \emph{oti/pos}{}-comple\-ments are selected mainly by
non-factives and only some factives \parencite{Christidis1982,Roussou1994}. So
there is a one-way implication between sentential complementation and
factivity, since not all factive complements are introduced by \emph{pu}. With
respect to relativization, \emph{pu} is the only complementizer\is{complementizers} available; as
will be shown immediately below, free relatives behave differently (and exclude
\emph{pu}).

Considering \emph{pos} and \emph{pu} in more detail, we observe that they also
correspond to \emph{wh}-pronouns, as in the following examples:

\ea\label{ex:key:6.8}Greek
	\ea
		\gll pos  tha   fij-is?\\
			how \Fut{} leave-\Ssg{}\\
		\glt \enquote*{How will you leave?}
	\ex
		\gll pu  tha  pa-s?\\
			where  \Fut{} go-\Ssg{}\\
		\glt \enquote*{Where will you go?}
	\ex
		\gll pu  to=edhos-es to  vivlio?\\
			where  it=gave-\Ssg{}  the book\\
		\glt \enquote*{Where/to whom did you give the book?}
	\z
\z

In~\eqref{ex:key:6.8} both \emph{pos} and \emph{pu} occur in matrix questions. They may also
introduce embedded \emph{wh}-interrogatives. Both sentences have a \emph{wh}-question
(rising) intonation. As~(\ref{ex:key:6.8}c) shows, \emph{pu} as an interrogative apart from
the locative reading, it may also realize an indirect (oblique) \emph{wh}-argument.
Finally, it is crucial to mention that although \emph{oti} does not have a
\emph{wh}-counterpart, it has a relative pronoun one, which in orthographic terms is
spelled as \emph{o,ti} (lit.\ ‘the what’). As a relative pronoun, it is found in
free relatives with an inanimate\is{animacy} referent, and is excluded from restrictive and
non-restrictive \isi{relative clauses} (the relevant examples are given below).

The picture we have so far with respect to the distribution of \ili{English} and
\ili{Greek} \isi{complementizers} and their pronominal counterparts can be
summarized as in~\Cref{tab:key:06.1}.

\begin{table}
\begin{tabular}{lcccc}
\lsptoprule
             & \emph{Demonstrative} & \emph{Complementizer} & \emph{Relative} & \emph{Interrogative}\\
\midrule
 \emph{oti}  & no                   & yes                   & yes             & no\\
 \emph{pos}  & no                   & yes                   & no              & yes\\
 \emph{pu}   & no                   & yes                   & yes             & yes\\
 \emph{that} & yes                  & yes                   & yes             & no\\
 \emph{how}  & no                   & yes                   & yes             & yes\\
\lspbottomrule
\end{tabular}
\caption{Pronoun and \isi{complementizers} (Greek and \ili{English})\label{tab:key:06.1}}
\end{table}

A quick look at~\Cref{tab:key:06.1} shows that all five elements qualify as
declarative \isi{complementizers}, despite their different feature
specifications. It further shows that all of them have a pronominal use as
well, despite differences again. Based on this pattern, I will assume that
their core defining property is that of N, i.e., they are essentially nominal
elements (see also \citealt{Franco2012}), which can be construed with different
features (D, wh, etc.) or different functional layers \citep{Baunaz2015}. In
this respect, their core (minimal) categorial\is{syntactic categories} content
is N – very much like indefinites; this property can account for the fact that
they may also distribute like indefinites, subject to Operator licensing
(polarity-like). I leave this issue open for the time being.

Let us now consider the following sentence (I leave \emph{oti} unglossed on
purpose in the following example):

\ea\label{ex:key:6.9} \ili{Greek}\\
    \gll    pistev-i oti dhjavas-e\\
	        believe-\Tsg{} \textsc{oti} read-\Tsg{}\\
	\glt    i.  \enquote*{He believes that he has studied.}\\
            ii.  \enquote*{He believes whatever he has read.}
\z

The two sentences above exemplify two different readings. In (i) \emph{oti} is
a complementizer\is{complementizers} that introduces the complement clause of \emph{pistevi}. In
(9b) it is a relative pronoun construed as the argument (object) of
\emph{dhjavase}. The whole clause introduced by \emph{oti} (or \emph{o,ti}) is
the internal argument (object) of \emph{pistevi}. What is responsible for this
ambiguity? First, a verb like \emph{pistevi} \enquote*{believe} can take either a noun
or a clause as its complement; second, the embedded verb \emph{dhjavazi}
\enquote*{read} can take an implicit argument.\is{implicit arguments} So in~(\ref{ex:key:6.9}a), the matrix verb
selects a sentential complement, and the embedded verb has an implicit
argument.\is{implicit arguments} In~(\ref{ex:key:6.9}b), on the other hand, the matrix predicate selects a free
relative (akin to a noun phrase) while the argument of the embedded verb is not
implicit\is{implicit arguments} but present in the form of the displaced pronoun. The string of words
in \emph{pistevi oti dhjavase} is ambiguous between a complement clause (where
\emph{oti} functions as a complementizer) and a relative clause\is{relative clauses} (where
\emph{oti} functions as a free relative pronoun). Thus the surface string of
words in this case corresponds to two different syntactic configurations.

Similar examples hold with the other two elements, namely \emph{pos} and
\emph{pu}, as below (again left unglossed):

\ea\label{ex:key:6.10} \ili{Greek}\\
	\gll paratiris-a  pos jiriz-i o troxos\\
		observed-\Fsg{} \textsc{pos} spin-\Tsg{} the wheel\\
	\glt i.  \enquote*{I observed that the wheel was spinning.}\\
		ii.  \enquote*{I observed how the wheel was spinning.}
\ex\label{ex:key:6.11} \ili{Greek}\\
    \gll    emath-a pu perpatis-e\\
            learnt-\Fsg{} \textsc{pu} walked-\Tsg{}  \\
	\glt    i.  \enquote*{I learnt/found out that he had walked.}\\
		    ii.  \enquote*{I learnt/found out where he had walked.}
\z

In the absence of any PF-indication (prosody), each of these lexical items can
be construed as a declarative complementizer\is{complementizers} (i), or a
pronoun (ii).

In all examples (9–11) so far, the complementizer\is{complementizers} construal is possible to the
extent that a declarative complement is selected by the matrix predicate. In
(10–11) for example, if instead of \emph{paratirisa} and \emph{ematha}
accordingly we have an interrogative predicate, such as \emph{rotisa}
(‘asked’), then only a \emph{wh}-interrogative reading is available, as expected. So
ambiguity arises in certain contexts only. The second property we need to point
out is that the interrogative reading in these examples (and accordingly, the
free relative in~\eqref{ex:key:6.9}) depends on the availability of a variable in the
complement clause, that is an open position that modifies the predicate for
manner, place, etc. (or as an implicit argument\is{implicit arguments} in~\eqref{ex:key:6.9} for the free relative
reading).

I will assume, in line with work in the recent literature
\parencite{ManSav2007,ManSav2011,Roussou2010,Franco2012}, that as a
complementizer each of these elements merges as the argument of the verb and
takes the CP/IP as its complement. On the other hand, as a pronoun it is
internal to the embedded clause, at least to the extent that it has a copy
inside the clause (at the v/VP level), as illustrated below:\largerpage

\ea\label{ex:key:6.12}
    \ea V \emph{oti}/\emph{pos}/\emph{pu} [\textsubscript{CP/}\textsubscript{IP} \dots{}]
    \ex V [\emph{o,ti}/\emph{pos}/\emph{pu} [\textsubscript{CP/}\textsubscript{IP} \dots{} \sout{\emph{o,ti}/\emph{pos}/\emph{pu}} ]]
	\z
\z

The different configurations map onto different \glspl{PF}; thus the ambiguity is
resolved prosodically. As \isi{complementizers}, \emph{oti}, \emph{pos}, and
\emph{pu}, are unstressed, i.e., they do not form a prosodic unit. As pronouns,
however, they are stressed, in a manner typical of \emph{wh}-questions; i.e. the
pronoun defines an L*+H prosodic unit. This holds for all three cases,
including the \emph{o,ti} relative function. The pattern with \emph{pu} as a
complementizer has one more interesting angle: as expected, \emph{pu} is
unstressed, but the preceding predicate is stressed (an L*+H prosodic unit). In
other words, selection of a \emph{pu}{}-complement in this context is
associated not only with the semantics of the selecting predicate but also with
\isi{focus}. As expected, focus on the predicate turns the \emph{pu}{}-complement to
the presupposed part, hence its association with factivity (on the interaction
of \isi{focus} with factivity, see \citealt{Kallulli2006}).

What we observe so far is that the lexical items under consideration have two
phonological variants: a stressed one (pronominal) and an unstressed one
(complementizer). This kind of alternation is quite common in the pronominal
system. For example, in Classical \ili{Greek} the indefinite pronoun \emph{tis} has
an accented variant (\emph{tís}) as an interrogative (also \ili{Latin} \emph{quis});
in (Modern) \ili{Greek} negative polarity items like \emph{kanenas} (‘anyone’) and
\emph{tipota} (‘anything’) acquire a status of universal quantifiers (negative
quantifiers) when focused.\is{focus} So the different categorizations of \emph{oti},
\emph{pos} and \emph{pu} as pronouns vs \isi{complementizers} in relation to their
phonological properties comes as no surprise in this respect. But does this
property suffice to classify them as distinct lexical items synchronically? The
answer seems to be negative, given that their differences can be accounted for
independently.

Assuming that the use of \emph{pos} and \emph{pu} as \isi{complementizers} is
an innovation in the path of their diachronic development, the question is
whether \isi{grammaticalization} is at stake or not. So far, I have argued that,
strictly speaking, there is no categorial\is{syntactic categories}
\isi{reanalysis} as such, in the sense that in either function, these elements
retain their nominal core. The activation of the \emph{wh}-feature depends on
the presence of a Q operator and involves focusing\is{focus} of the item in question. If
this is correct, the interrogative reading is syntactically defined and is read
off at the two interfaces (it introduces a variable at \gls{LF}, it defines a
prosodic unit at \gls{PF}). On the other hand, as complementizers, they are
selected by designated predicates and in turn they select a proposition. The
complementizer\is{complementizers} is externally merged, subject to selection
by the matrix predicate. This structure is accordingly read off at \gls{LF},
given that the complementizer\is{complementizers} turns the clause to an
argument, and at \gls{PF}, since it has not prosodic properties. This latter
characteristic is in accordance with the notion of phonological reduction
attested in \isi{grammaticalization}. What about semantic weakening (or bleaching)?
As a complementizer, the pronoun retains its nominal core, and has no
additional features (like \emph{wh}-). At the same time, under its
complementizer\is{complementizers} use, the lexical items under consideration
expand, on the assumption that they manifest a wider choice in terms of
selection; for example, i.e., selection of an NP or a clause (CP/IP). Note that
the complementizer\is{complementizers} status assigned to \emph{oti} is not an
innovation, since it is used as a complementizer\is{complementizers} throughout
the history of \ili{Greek}.

The above properties can be summarized as follows:

\ea\label{ex:key:6.13}
	\ea Pronoun: prosodic unit, internal merge/scope
	\ex complementizer\is{complementizers}: no prosodic unit, external merge.
	\z
\z

The development of the complementizer\is{complementizers} use for \emph{pos}
and \emph{pu} under the current approach is consistent with the
\enquote{change} from internal to external merge. As already pointed out, as
displaced pronouns due to internal merge, they bind a lower copy and take
scope. As \isi{complementizers} they merge externally and therefore do not bind
a copy.  Does this approach account for the idea of \enquote{upward reanalysis}
of \citet{RobRou2003}? Recall that in relation to the schema
in~\eqref{ex:key:6.2}, \citeauthor{RobRou2003} assume that this involves a
leftward shift of the clause boundary; this means that while the pronoun as a
complementizer literally lowers, since it becomes part of the embedded clause,
the boundaries of the embedded clause move upwards to include the reanalyzed
pronoun. This account though has some shortcomings, given that it takes
\enquote{upward} in linear and not structural terms. In terms of the claim made
in the present paper, the upward \isi{reanalysis} is accounted for
structurally: the pronoun as a complementizer\is{complementizers} merges as the
argument of the predicate (as was before), and the paratactic clause becomes
embedded under the pronoun, triggering the change from parataxis to hypotaxis.
The pronoun nominalizes the second clause, which now qualifies as an argument.
The relation between the pronoun and the clause changes from being anaphoric to
being an instance of complementation.

In short, the presentation of the data above points towards a unified account
of pronouns and \isi{complementizers}. The basic line of reasoning is the following:
if two grammatical lexical items look the same, they are (most probably) the
same. Further evidence is provided by the fact that this similarity is
diachronically and synchronically supported. Diachronically, because we can
trace the steps in the development of \isi{complementizers}, and synchronically,
because it is very systematic across grammars, but also within a given grammar,
to be simply treated as accidental (as is the case with homonymy).

\section{Grammaticalization and syntactic categories}\label{sec:gram-cat}

The above discussion on (Greek) \isi{complementizers} has concentrated on the
connection between the \enquote{new} functional item and its lexical source. So
the question has been whether \isi{complementizers} retain their core nominal
feature or not. So far, I have talked about \isi{complementizers} and pronouns,
assuming that the latter occur in the left periphery of the clause, while the
former (potentially) as arguments of the selecting predicate. I have made no
reference to the C head as such though. In fact, the approaches that assign a
nominal feature to \isi{complementizers} distinguish it from the C positions as
such. If C is a position retained for verbal elements that is part of the
(extended) projection of the verb \parencite{ManSav2011}, then it is not and
cannot be realized by nominal-type elements (such as pronouns or
\isi{complementizers}, unless the latter are verbal-like). This line of
reasoning allows us to maintain that the pronoun to
complementizer\is{complementizers} \isi{reanalysis} is not an instance of
categorial reanalysis. In other words, it is more of a functional change
(affecting the use of the pronoun) and less so of a formal one.

This issue of categorial\is{syntactic categories} \isi{reanalysis} arises in
all contexts of \isi{grammaticalization}. For example, when verbs become modals, do
they retain their verbal feature? Does \emph{for}, as a
complementizer\is{complementizers} in \ili{English}, retain its prepositional
feature? Is the infinitival marker \emph{to} in \ili{English} the same as the
preposition \emph{to}? This question can be obviously asked for every single
case of \isi{grammaticalization}, and it is related to the nature of syntactic
categories, their repertoire and feature specification. The answers to the
question just raised can vary. Consider the case of \emph{for} as in the
following example (for a historical account, see \citealt{vanGelderen2010}):

\ea\label{ex:key:6.14}
	\ea A present for Mary/her
	\ex I prefer for Mary/her to be late
	\z
\z

In descriptive terms, \emph{for} in~(\ref{ex:key:6.14}a) is a Preposition which takes a DP
complement (\emph{Mary}) or an accusative\is{accusative case} pronoun (\emph{her}). In~(\ref{ex:key:6.14}b), on
the other hand, \emph{for} introduces the infinitival complement, and is
usually analyzed as a C element. However, it can still assign accusative\is{accusative case} case
to the embedded subject (\emph{Mary/her}), at least under standard assumptions
in the generative grammar. If so, then it maintains its prepositional property
of being a case assigner. This has been further supported by the fact that in
Standard \ili{English} at least, \emph{for} forces the presence of an overt subject
and excludes a null one (that is a PRO subject), as in *\emph{I prefer for to
be late} vs.\ \emph{I prefer to be late}. Based on similarities of this sort,
\parencite{Kayne1984,Kayne2000b} pointed out the affinity between prepositions
(P) and \isi{complementizers} (C), but also determiners (D). This then turns out to
be a recurrent theme in the literature.

In the light of the present discussion, the link between apparently different
categories is not surprising. To be more precise, if \emph{for} is a
preposition in~(\ref{ex:key:6.14}a) there is no particular reason why it cannot
be a preposition in~(\ref{ex:key:6.14}b). The difference between the two
instances has to do with the different complements \emph{for} takes in either
case. That prepositions can introduce subordinate clauses is rather well
established, and can be further illustrated with the following examples:

\ea\label{ex:key:6.15}
	\ea We went for a walk after the dinner
	\ex After we had dinner, we went for a walk
	\z
\z

Once again, the same element can take different types of complements: a DP or a
clause (finite or non-finite). Unlike \emph{for}, \emph{after} can only select
for a finite clause, does not interfere with the realization of the embedded
subject, and can only introduce adverbial (non-complement) clauses.

Having provided a discussion of the relation between pronominals and
complementizers (potentially extending this to prepositions as well), let us
discuss a bit more the question that we first raised, namely that of lexical
splitting in the context of \isi{grammaticalization}. Related to this is the
categorial identification of the new lexical item. As discussed in the
literature, there are many cases where the limits between two categories are
not very obvious, or \enquote{fuzzy} \parencite[see][]{TrauTrou2010}. In
typological approaches to \isi{grammaticalization}, where grammatical
categories are under formation, the notion involved is that of
\emph{gradience}. However, in formal approaches where grammatical categories
are defined as bundles of features with a role in the syntactic computation,
the notion of gradience is problematic.  \citet{Roberts2010b} partly overcomes
this problem by assuming an elaborate functional hierarchy, along the lines of
\citet{Cinque2006}, allowing for the possibility for the same lexical item to
merge on different heads along this hierarchy. This has the advantage of
maintaining a core property, thus avoiding the issue of homonymy, while at the
same time it derives the different meanings by merger of the same item in
different positions. So in this respect, what looks like a lexical split has a
syntactic explanation: the same lexical item can realize different positions
along the functional hierarchy. One disadvantage of this approach is that it
requires every possible meaning to be syntactically encoded, introducing an
immense increase of functional categories, even for those cases where certain
readings can be derived pragmatically.

Another issue that opens up under the current approach concerns the lexical vs.
functional divide. If so-called grammaticalized\is{grammaticalization} elements can retain their
verbal or nominal (in a broad sense) features, then the question is what sort
of implications this has for the lexicon, the syntax and the view of parametric
variation, among other things. One possible answer is that this basic
distinction between two classes of lexical items is not primary but secondary.
This is consistent with the view that the lexicon consists of lexical items
with no a priori characterization (see \citealt{Marantz2001}). If something is
a predicate, i.e. assigns a property or expresses a relation, it has all the
typical characteristics to qualify as a core lexical category. Languages allow
these elements to generate quite freely in the lexicon. At the same time,
whether an element functions as a predicate or not is to a large extent
determined configurationally under current minimalist assumptions. Similarly,
whether the same element is \enquote{less lexical} or \enquote{more functional}
is also determined configurationally. This is indeed captured in
\citegen{Cinque2006} approach, and is to some extent implicit in
\citegen{Roberts2010b} account. So \enquote{more functional} in current terms
is understood as being associated with a high position (and scope) in the
clause structure. But if this is correct, it does not really tell us much about
this distinction as an aspect of the lexicon. As \citet[5]{ManSav2011} put it
\enquote{There is no separate functional lexicon – and no separate way of
accounting for its variation}.

Consider again the case of verbs, which are typical examples of predicates in
natural languages. The verb expresses its argument structure in connection with
certain positions, realized by nominals (giving rise to expressions of
transitivity, case, etc.). The typical I and C positions associated with the
verb are essentially scope positions (relating to the event, the proposition,
or various types of quantification over possible worlds, etc.). Nominals also
have a predicative base, carry inflectional properties and become arguments in
relation to a predicate. What actually lies in the heart of this discussion is
the categorization of concepts. Different choices give rise to different lexica
cross-linguistically. Interestingly, it is in this respect that
\isi{grammaticalization} in functionalist frameworks makes sense, since the idea is
that concepts acquire a grammatical form and consequently grammatical
categories are defined functionally. In formal approaches, syntactic
(grammatical) categories are meant to be well-defined, and languages differ as
to which concepts map onto which categories and how. This raises the question
of how well-defined categories are. Looking at \isi{complementizers} and how they
develop out of pronouns can shed some light into this question. So is
complementizer a formal category after all, or is it a functional
classification of a nominal (or a verbal in other languages) element?
\isi{grammaticalization} phenomena then allow us to have a better understanding of
how syntactic category\is{syntactic categories} could be defined, how they are realized
cross-linguistically and how they are manipulated by narrow syntax.

In short, \isi{grammaticalization} phenomena can tell us something about the
diachronic development of grammatical elements, especially with respect to
morphosyntax. At the same time, they force us to pay closer attention to what
actually a syntactic category\is{syntactic categories} is. The answers are not easy either way, but the
empirical data is there to be further explored. Taking a view towards
\isi{grammaticalization} along the lines suggested here, where its core property of
categorial \isi{reanalysis} is put into question, invites us to reconsider syntactic
change and focus more on how certain elements change the way they do, even if
they retain their categorial\is{syntactic categories} status (thus no categorial\is{syntactic categories} reanalysis).

\section{Concluding remarks}\label{sec:21-conclusions}

In the present paper I have mainly focused on the notion of categorial
reanalysis, and in this respect I have outlined an account which, at least to
some extent, casts some doubts on this standard view. The empirical set of data
was restricted to the development of \isi{complementizers} out of pronouns. The
basic argument has been that formally, the innovative element, namely the
complementizer, retains its nominal categorial\is{syntactic categories}
feature. In its new function as a complementizer\is{complementizers}, the
pronoun externally merges with the selecting predicate.  The change attested
involves properties that affect the interfaces, such as phonological reduction
and selectional/scope requirements.

\label{sec:abbreviations}

\printchapterglossary{}

\section*{Acknowledgements}

This paper is dedicated to Ian on the occasion of his 60th birthday. It
reflects on our joint work on \isi{grammaticalization}, adding a new angle on
categorial reanalysis. Working with Ian has been inspirational and a good
source of agreements and (productive) disagreements! The present version of the
paper has benefited from the constructive comments of two anonymous reviewers.
I thank them both.

{\sloppy
\printbibliography[heading=subbibliography,notkeyword=this]
}

\end{document}
