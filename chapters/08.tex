\documentclass[output=paper]{langsci/langscibook}
\ChapterDOI{10.5281/zenodo.3972840}

\author{Nigel Vincent\affiliation{The University of Manchester}\lastand Kersti Börjars\affiliation{St Catherine's College, University of Oxford}}
\title{Heads and history}

% \chapterDOI{} %will be filled in at production

\abstract{This paper considers and compares the status of the concept of head
    within different grammatical frameworks (Minimalism, LFG and HPSG) and its
    relevance to our understanding of the mechanisms of change involved in
    \isi{grammaticalization}. Our data is drawn from the developments of
    lexical prepositions into grammatical prepositions and
    complementisers\is{complementizers} in \ili{Romance} and \ili{Germanic}. We
    argue in favour of a non-derivational approach and in particular against
    accounts in which all developments are mediated through a chain of
functional heads\is{functional items} of the kind deployed in cartography and
nanosyntax.}

\maketitle

\begin{document}\glsresetall

\section{Introduction}

Heads come in two kinds: lexical and functional. While the former are treated
in a largely uniform way across theoretical frameworks, with the latter things
are different. Functional heads have been reified as a core theoretical
construct within Minimalism, where they abound particularly, but not
exclusively, in the cartographic version, but have much less presence in a
non-derivational framework like \gls{LFG}\is{Lexical Functional Grammar} and an even more reduced role in
\gls{HPSG}. The difference between the two kinds of heads also plays out in the
diachronic domain. Nouns, verbs and adjectives often have consistent historical
trajectories over centuries. Many of the nouns of modern \ili{English}, for example,
were also nouns a millennium ago in \ili{Old English} even if they have undergone
extensive phonological and semantic change in the meantime. The diachronic
profiles of items that realise functional heads\is{functional items} are very
different, since, typically, they start out as full lexical words before
developing into a grammatical item. \ili{English} \emph{will} is a good case in
point, having begun life as a lexical verb meaning \enquote*{want} before
becoming the temporal/modal marker that it is today and, in some approaches,
being assigned a structural position under a node such as T or I. The key
question then becomes: how do diachrony and synchrony interact, and in
particular how is the historical relation between lexical and functional
categories treated, in different grammatical frameworks?  In the present paper,
we seek to compare and contrast \gls{LFG}\is{Lexical Functional
    Grammar},\is{Lexical Functional Grammar} \gls{HPSG}\is{Head-Driven Phrase
Structure Grammar} and Minimalism as models of (morpho)syntactic change. Our
chosen dataset is the linked evolution of prepositions and complementisers in a
range of \ili{Romance} and \ili{Germanic} languages, but we hope and believe
that the conclusions we will draw on the basis of this evidence will extend
both to other categories and to other languages and
families.\is{LFG|see{Lexical Functional Grammar}}

\section{Grammaticalisation and category change}

The phenomena that we will examine in this paper fall under the general heading
of grammaticalisation\is{grammaticalization}, classically defined by \citet[131]{Meillet12} as
\enquote{l'attribution du caract\`ere grammatical \`a un mot jadis autonome
[the attribution of a grammatical value to a formerly autonomous word]} and by
\textcite[69]{Kurylowicz65} as \enquote{the increase of the range of a morpheme
advancing from a lexical to a grammatical or from a less to a more grammatical
status}.  We should be clear at the outset that such definitions seek to
identify a phenomenon or a mechanism of change. Grammaticalisation is a
descriptive label and not a theoretical construct, \emph{pace} the locution
\enquote{grammaticalisation theory} that is to be found from time to time in
the literature, for instance in the positive reference by
\textcite[318]{Haspelmath89} to the \enquote{explanatory standards of
grammaticalization theory}. There are two properties which characterise such
changes: the first is the fact that they recur within the histories of
unrelated languages. In our introduction, for example, we cited the case of the
English future auxiliary\is{auxiliaries} \emph{will}, which derives from the \ili{Old English}
\emph{willan} \enquote*{want}. A similar shift is to be seen in the use of the
\ili{Romanian} verb \emph{a vrea}, etymologically the reflex of \ili{Latin}
\emph{velle} \enquote*{want}, to signal futurity, in similar uses of the
\enquote*{want} verb elsewhere in the Balkans (\ili{Albanian}, \ili{Croatian},
\ili{Greek}), in the \ili{Swahili} future prefix -\emph{ta}- originating in the
verb \emph{taka} \enquote*{want}, and in parallel developments in a number of
other languages \citep{HeinKute02}.  The second property is the
unidirectionality -- or at least overwhelming asymmetry in direction -- of such
changes; thus, we find many instances of volition verbs becoming future tense
markers, but none of futures turning into verbs of volition (see
\citealt{BorjVinc11} for further discussion and exemplification).

To the claimed existence of grammaticalisation\is{grammaticalization} there have been two broad
classes of response. One is to deny its place as a special and separately
identifiable category among the general processes of reanalysis that
characterise morphosyntactic change (see amongst others \citealt{Campbell01,
Joseph01, Newmeyer01}). The alternative is to accept that grammaticalisation\is{grammaticalization}
exists and to seek to model it in theoretical terms. This, in very different
ways, is what has been done by \textcite{HeinClauHunn91},
\textcite{RobRou2003}, \textcite{vanGelderen11} and \textcite{TrauTrou13}, and
it is within this latter class of approaches that the present paper also falls.
A central issue then becomes the nature of the theoretical constructs that are
assumed. \textcite{RobRou2003}, for example, operate within a framework which
permits synchronic analyses involving \isi{movement} upwards from a lexical head to a
functional head but not downwards from functional to lexical -- a principle of
\gls{UG} which appears to mimic, and has been argued to explain, the
directionality of change from lexical to grammatical but not vice versa
implicit in Meillet's and Kury\l{}owicz's definitions. \gls{LFG}\is{Lexical
Functional Grammar} and \gls{HPSG}, by contrast, do not include \isi{movement}
within their theoretical inventories.

\section{Prepositions and complementisers in diachrony}

When it comes to categories and category change, prepositions are distinctive
in two complementary, but as we will suggest connected, ways. From a synchronic
point of view they appear to straddle the boundary between lexical items with
their own semantic content -- as in contrasting pairs such as \emph{on} and
\emph{off}, \emph{under} and \emph{over}, \emph{to} and \emph{from} -- and
functional items such as the various ways of marking arguments of adjectives
and verbs: \emph{proud of, convince someone of}, \emph{keen on, rely
on}, \emph{similar to, give to} or \emph{different from, differ from}. (For
more discussion in relation to a variety of languages, see the papers in
\citealt{SaintDizier06, Asburyetal08, Francoisetal09, CinqRizz10}.) At the same
time there is also evidence that they all behave in ways akin to other
functional items in acquisitional\is{language acquisition} and pathological contexts. In this
connection, the results of Froud's \citeyearpar{Froud01} study of an aphasic
patient are particularly striking and have led some to conclude that all
prepositions should be treated as functional heads\is{functional items}. A different but related
contrast is that between open and closed classes. Many languages are like
English in having a group of typically monosyllabic items that have high
textual frequencies, plus a more open class of polysyllabic and syntactically
complex items such as \emph{across}, \emph{behind}, \emph{against}, \emph{in
front of}, \emph{by virtue of}  and the like which share the distribution of,
and may alternate with, the monosyllabic items.

Diachronic considerations complicate the picture even further: polysyllables
may shorten into monosyllables as a result of sound change (\emph{over} >
\emph{o'er} in some dialects); simple and complex forms may contrast
(\emph{for} vs \emph{against}, \emph{behind} vs \emph{in front of}) and once
independent forms may fuse or lose syntactic and semantic content
(\emph{because} < \emph{by cause}, \emph{beside} < \emph{by side}, \emph{in
light of}, \emph{by virtue of}). In the historical context, prepositions are
also remarkable because of the sheer variety of their etymological origins.
Whereas temporal and aspectual markers are, for the most part, derived from
independent verbs, prepositions can emerge from a variety of categorial
sources. Thus, among the items that we will consider in more detail below, the
Swedish and \ili{Danish} prepositions \emph{till} and \emph{til} `to, towards',
are descended from a noun meaning `goal' and are cognate with the \ili{German}
noun \emph{Ziel} `goal, target'. As such, in origin they were accompanied by
nouns in the genitive\is{genitive case} as the case which typically marks
nominal dependents. A trace of this can be seen in the final -\emph{s} which
survives in such fixed expressions as \ili{Danish} \emph{til sengs} `to bed'
and \ili{Swedish} \emph{till sj\"{o}ss} `at sea'. A similar effect is to be
seen with the \ili{Latin} items \emph{causa} `because of' and \emph{gratia}
`thanks to', which have clear nominal origins and are the only \ili{Latin}
adpositions to govern the genitive\is{genitive case} case.  And with
prepositions too, we find recurrent patterns developing independently within
different languages. For example, the items \emph{hos} `at, with' in Swedish
and \ili{Danish} and \ili{French} \emph{chez} `at, with' are both descended
from nouns meaning `house, household' \citep{Plank15}, and are often contrasted
with the \ili{Swedish}/\ili{Danish} noun \emph{hus} and the fact that
\ili{Latin} \emph{casa} `hut' has stayed as the usual word for `house, home' in
\ili{Italian} and \ili{Spanish}.

In other instances, prepositions may stem from independent adverbial particles
which acted as specifiers for particular case forms. This is particularly
relevant for the items on which we focus below. Thus, \ili{Latin} \emph{ad}
`to, towards', and the infinitival markers in \ili{Swedish} \emph{att} and
\ili{Danish} \emph{at}, all descend from a Proto-Indo-European particle *ad
`at, near', hence the fact that the \ili{Latin} preposition takes the
accusative\is{accusative case} case, in origin used in a directional sense. By
contrast \ili{Latin} \emph{de} comes from a particle meaning `down, away from'
and so occurs with the ablative, where the latter fuses earlier distinct
locative and ablative cases \citep{Vincent99,Vincent17}.

In addition to nouns, particles and reduced complex structures of the
\emph{behind} type, prepositions may also derive from a range of non-finite
verb forms, as with \ili{French} \emph{pendant} `during' < \ili{Latin}
\emph{pendentem} `hanging', pres participle of \emph{pendeo}, \ili{English}
\emph{including}, \ili{Italian} \emph{presso} `near' < \ili{Latin}
\emph{prehensus}, past participle of \emph{prehendere} `take', \ili{Danish}
\emph{blandt} `among' < \emph{blandet}, past participle of \emph{blande}
`mix', Sicilian \emph{agghiri} `towards' < \emph{ad jiri} `to go.\Inf{}'.
Similar in function to participles and also possible etyma for prepositions are
adjectives as in \ili{Italian} \emph{vicino} `near' < \ili{Latin}
\emph{vicinum}, or \ili{English} \emph{near}.

Complementisers exhibit a similar diversity of etymological sources including
demonstrative pronouns as with \ili{English} \emph{that}, \ili{Swedish} finite
\emph{att} and Estonian \emph{et}, interrogative/relative pronouns  as with
\ili{French} \emph{que} (< \ili{Latin} \emph{quid} `what') and \ili{Greek}
\emph{oti}; nouns as for instance \ili{Korean} \emph{kεs} <
`thing' used with finite clauses; and verbs, especially verbs of saying, e.g.
Yoruba \emph{kp\'e}, Uzbek \emph{deb} and \ili{Turkish} \emph{diye}
\citep[870--874]{KehaBoye16}. As we shall see in what follows, they may also
evolve from prepositions as in the case of French \emph{\`a} and be linked to
infinitives, and corresponding patterns elsewhere in \ili{Romance},
\ili{Swedish} infinitival \emph{att} and \ili{Danish} \emph{at}, English
\emph{to} and \ili{German} \emph{zu}, \ili{Irish} \emph{go} and \ili{Basque}
-\emph{ela}; with the exception of \emph{de}  and its cognates, all of these
are derived from allative prepositions. Within the literature such patterns
have led some scholars to postulate an intermediate category of
\enquote{prepositional complementiser} (\citealt{Borsley86,Borsley01,Kayne99}
and see \Cref{sec:08.7} below). In this context, too, the directionality
property is evident in that, while a preposition may over time acquire
complementizing functions, the reverse development is not attested.

\section{Heads and diachrony across frameworks}

The evidence of diachrony has figured very differently within the frameworks
under consideration here. The fact of \isi{language change} and its implications for
general linguistic theory have figured as core issues within the Chomskyan
tradition ever since the seminal work of \cite{Lightfoot79}. By contrast, there
has to date been relatively little work from a diachronic perspective within
\gls{LFG} -- but see the contributions to \cite{ButtKing01} for some examples
and \cite{BorjVinc17} for a general overview -- and virtually nothing within
HPSG. And yet in different ways both these last-mentioned approaches have much
to offer historical linguists. In the first place, the absence of an assumption
of an innate UG makes them easier to reconcile with the historical datasets
derived from usage-based approaches without giving up on the commitment to
formal modelling.\footnote{As one of our reviewers reminds us, there is no
inherent incompatibility between a belief in the existence of an innate UG and
the assumptions of \gls{LFG}\is{Lexical Functional Grammar} and \gls{HPSG}. And there are also a range of views
within the Minimalist community as to what exactly is to be ascribed to UG.
However, the fact remains that, as far as we are aware, no variant of
Minimalism abandons UG in its entirety whereas within the \gls{HPSG}\is{Head-Driven Phrase Structure Grammar} and
\gls{LFG} communities there is general agreement that grammatical descriptions
and explanations do not require the postulation of any innate components of
language.}  Secondly, their less rigid approach to phrase structure and their
readiness to recognise other dimensions of linguistic information makes them
able more readily to accommodate linguistic diversity, including that which is
the result of change \citep[475]{EvaLev2009}.

Let us begin then by comparing the types of category that are available within
the different frameworks, with an eye particularly to the differences between
the sub-types of non-lexical category since it is at that point that they most
obviously diverge from each other. In this respect, Minimalism is in principle
the most straightforward, since it presupposes a simple contrast between
lexical heads (at least N, V, A; \citealt[303--325]{Baker03}) and functional
heads\is{functional items}.  Constituency trees are always binary and consist
of a head (lexical or functional) plus its complement; lexical heads are always
dominated by one or more functional projections and typically move from a lower
base-generated position to a higher functional one in the course of a
derivation. The system is thus apparently strictly constrained, but in fact the
restrictions in one part of the tree lead to considerable analytical freedom
elsewhere, since the inventory of functional heads\is{functional items} is
large and seemingly unconstrained, particularly in the cartographic variant of
the approach. And while some such heads have names at least which suggest a
semantic basis -- T(ense), Mod(al), D(et), etc. -- others seem to be there only
to facilitate the necessary movements or to provide an intermediate location
for arguments but which do not have any overt phonological exponence, as with
so-called \enquote{small} vP and nP. Moreover, all heads can in principle be
empty or be occupied by silent items, so the possible analytical space is in
practice quite unconstrained.\footnote{A more constrained approach to
    categorial structure within a derivational framework is the Universal Spine
explored in \cite{Wiltschko14}. Lack of space forbids further consideration of
this approach in the present context but for some discussion see
\cite{Vincent18}.}

When it comes to \gls{LFG}\is{Lexical Functional Grammar}, the opposite state
of affairs obtains. More basic types of category are available and there are no
constraints barring non-binary or non-headed configurations. On the other hand
the inventory of functional heads deployed is generally assumed to be very
limited and null heads are wherever possible avoided. \Cref{Fig1LFGCats} sets
out in tabular form the categories recognised within this framework.

\begin{table}
    \caption{Types of category in \gls{LFG}}
\label{Fig1LFGCats}
 \begin{tabular}{lll}
  \lsptoprule
       Lexical &Functional & Non-projecting (\^{P})\\
  \midrule
       \addlinespace[1ex] \enquote{full} semantics:  &    \enquote{weak} semantics:  &    may have     \\
      have \textsc{pred} feature &  no \textsc{pred} feature &  \enquote{full} semantics\\
     \addlinespace[1ex] projects to XP &   projects to XP &    does not project    \\ & \enquote{extension} of lexical category; & adjoins to X$^0$\\
         &  functional co-head & \\
  \lspbottomrule
 \end{tabular}
\end{table}

In the most constrained versions of \gls{LFG}\is{Lexical Functional Grammar}, a functional category is
postulated only when a feature comes to be associated with a structural
position within a particular language, but there is no expectation that such
categories are of universal validity (\citealt[6--7]{Kroeger:1993};
\citealt{BoPaChi99}). Much of the work that is done by such categories in a
model like Minimalism -- for example in the domains of tense and modality --
is instead handled within the f-structure (where \enquote{f-} stands for
functional in a different sense!), which is parallel to the
c(onstituent)-structure. The functional categories most commonly assumed are C,
I and D, and on such a view the natural diachronic trajectory is for a
structure like DP to gradually emerge or ``grow'';  definiteness first becomes
associated with a category D and in due course with a particular structural
position and hence as heading a DP where formerly there was an autonomous NP
\citep{Borjarsetal16}. A different kind of construct within
\gls{LFG}\is{Lexical Functional Grammar} is what, following \cite{Toivonen03},
have come to be known as non-projecting words (notated \^{X}). Items in this
class are of category X$^0$ but do not project to X$'$ or XP, they are marked
as such in the lexicon and are head-adjoined\is{adjunction} to an associated
and projecting X$^0$. \citegen{Toivonen03} case study focuses on \ili{Swedish}
particles such as \emph{ihj\"al} `to death' in the string \emph{sl\aa{}
ihj\"al} `kill', lit.\ `beat to death', where \emph{sl\aa} is of the
category V$^0$, as is the whole string, but where \emph{ihj\"al} is a
non-projecting P.  As she demonstrates, the items that fall within the class of
particles belong to a number of different categories -- verbal, nominal,
adjectival and prepositional --  but what they have in common is that they
adjoin to another item, to which in effect they cede head status.  What
Toivonen does not observe, but which is striking once the diachronic
perspective is adopted, is that most if not all the items she categorises as
non-projecting in this sense are themselves historically derived from full
projecting categories or even phrases. The form \emph{ihj\"al}, for example, is
a frozen version of the original PP \emph{i hel} `in the land of the
dead'.\footnote{A reviewer points out that some recent work within Minimalism
    has adopted a similar notion of non-projecting words as a way of dealing
with particles (see for example \citealt{Biberauer2017c}).}

When we come to \gls{HPSG}, beside full lexical heads stands the category of
transparent head \citep{Flickinger08}, that is to say an item which determines
the overall category of the phrase it heads but does not add any semantic
content (in the sense defined below) of its own. A case in point is the \ili{English}
complementiser \emph{that}, which heads and defines a CP, but does not
contribute to the semantic representation of the clause of which it is a part.
Such a concept is close to if not identical with the status the same item would
have in an \gls{LFG}\is{Lexical Functional Grammar} or Minimalist account. More radical, however, was the
suggestion by \cite[44--46]{PollSag94} that such items belong to a separate
category of \enquote{markers}. In their account, a marker is ``a word that is
`functional' or `grammatical' as opposed to substantive, in the sense that its
semantic content is purely logical in nature (perhaps even vacuous)''.
Crucially, a marker is not a head. This concept, which conforms in many
respects to traditional intuitions about such items, is not, however, the
preferred option. Rather, there has developed within recent
\gls{HPSG}\is{Head-Driven Phrase Structure Grammar} work the notion of a
\enquote{weak} head, defined by \cite[156]{Abeilleetal06} as ``a lexical head
that shares its syntactic category\is{syntactic categories} and other
\textsc{head} information with its complement''. \Cref{Fig2HPSGCats} below
summarises the various notions of head within \gls{HPSG}\is{Head-Driven Phrase
Structure Grammar}, and \Cref{Fig3CatsLFGHPSG} compares the inventory of
category types and their properties within \gls{LFG}\is{Lexical Functional
Grammar} and \gls{HPSG}.

\begin{table}\small
\caption{Types of category in HPSG\label{Fig2HPSGCats}}
 \begin{tabular}{lll}
  \lsptoprule
       Full head &Transparent head & Weak head\\
  \midrule
     \addlinespace[1ex]\enquote{full} semantics:  &    \enquote{weak} semantics:  &    \enquote{weak} semantics:     \\
     \textsc{content} feature &  no \textsc{content} feature &  no \textsc{content} feature\\
     \addlinespace[1ex]projects to XP &   projects to XP &    does not project    \\
     \addlinespace[1ex] combines with XP   & combines with XP & combines with XP (or X$'$)\\
         &  contributes all but  & contributes only \textsc{marking} feature; \\
      & the \textsc{content} feature & shares \textsc{head}\\
  \lspbottomrule
 \end{tabular}
\end{table}

\begin{table}\small
\caption{Heads in \gls{LFG}\is{Lexical Functional Grammar} and \gls{HPSG}\label{Fig3CatsLFGHPSG}}
\resizebox{\textwidth}{!}{\begin{tabular}{l  ccc  ccc}
  \lsptoprule
        &\multicolumn{3}{c}{LFG} & \multicolumn{3}{c}{HPSG}\\\cmidrule(lr){2-4}\cmidrule(lr){5-7}
     & {Lexical}  &    {Funct}  &  {Non-proj} &  {Full} & {Full\tss{\emph{transp}}} & {Weak}     \\
     \midrule
{lexical semantics} & $+$ &  $-$ &  $+$/$-$ & $+$ & $-$ & $-$ \\
    {\enquote{borrows} lexical semantics} &  $-$ & $+$ & $-$ & $-$ & $+$ & $+$   \\
      {projects}   & $+$ & $+$ & $-$ & $+$ & $+$ & $-$ \\
    {combines with} & XP & XP & X & XP & XP & XP (X$'$) \\
%         & the \textsc{content} feature & shares \textsc{head}\\
  \lspbottomrule
 \end{tabular}}
\end{table}

With these concepts and categories in mind we can now ask what kinds of
diachronic trajectories are predicted within the various systems and how these
stack up against the empirical evidence.

\section{Prepositions in the nominal domain}

We start with the example of \ili{Swedish} \emph{till} and compare the way it
can be analysed within the three frameworks under consideration in this paper.
As noted above, this item begins life as a noun, so the categorial shift in the
first instance is N~>~P. However, as the examples in (\ref{ex1}) demonstrate,
in the modern language it has acquired a range of functions.

\begin{exe}
\ex\label{ex1} \ili{Swedish}
\begin{xlist}
\ex\label{ex1a}
\gll Oscar tog t\aa get till Stockholm.\\
	Oscar take.\Pst{} train.\Def{} to Stockholm\\
\trans \enquote*{Oscar took the train to Stockholm.}
\ex\label{ex1b}
\gll Oscar gav boken till l\"araren.\\
	Oscar give.\Pst{} book.\Def{} to teacher.\Def{}\\
\trans \enquote*{Oscar gave the book to the teacher.}
\ex\label{ex1c}
\gll Oscar sparkade till d\"acket.\\
	Oscar kick.\Pst{} to tyre.\Def{}\\
\trans \enquote*{Oscar gave the tyre a kick.}
\end{xlist}
\end{exe}

In (\ref{ex1a}), we have the directional sense consistent with its etymological
source in a noun meaning `goal', in (\ref{ex1b}) it marks a grammatical
relation, and in (\ref{ex1c}) it behaves as an adverbial particle. Within
\gls{LFG}, these three uses can be modelled as in (\ref{ex2}). Here
(\ref{ex2a}) simply states that  \emph{till} is a full preposition with its own
semantic content expressed via the \textsc{pred} feature and that it
subcategorises for an item having the function \Obj{}(ect). The representation
in (\ref{ex2b}), by contrast, indicates its use to mark the grammatical relation
of an oblique recipient, and  (\ref{ex2c}) is an example of a non-projecting
word serving as a marker of dynamic aspect \citep[142]{Toivonen03}.

\begin{exe}
\ex\label{ex2}
\begin{xlist}
\ex\label{ex2a}
\emph{till}\hspace{0.3cm} P\hspace{0.3cm} \makebox[0pt][l]{(\emph{f}
\textsc{pred})}\phantom{(\emph{f} \textsc{pcase})} = `till <\Obj{}>'
\ex\label{ex2b}
\emph{till}\hspace{0.3cm} P\hspace{0.3cm} \makebox[0pt][l]{(\emph{f}
\textsc{pcase})}\phantom{(\emph{f} \textsc{pcase})} =
\Obl{}\tss{\emph{Recipient}}
\ex\label{ex2c}
\emph{till}\hspace{0.3cm} \^P\hspace{0.3cm} \makebox[0pt][l]{(\emph{f}
\textsc{aspect telic})}\phantom{(\emph{f} \textsc{aspect durative})} = $-$\\
\hphantom{\emph{till}\hspace{0.3cm} \^P\hspace{0.3cm}}
\makebox[0pt][l]{(\emph{f} \textsc{aspect
dynamic})}\phantom{(\emph{f} \textsc{aspect durative})} = $+$\\
\hphantom{\emph{till}\hspace{0.3cm} \^P\hspace{0.3cm}} (\emph{f} \textsc{aspect durative}) = $-$
\end{xlist}
\end{exe}

Neither of the developments in (\ref{ex2b}) and (\ref{ex2c}), which are
logically independent of each other, are possible until after the use of
\emph{till} as a preposition with a full semantics has emerged, so the
diachronic sequence is N > P$_{till}$ > P$_{\Obl{}}$/\^P. In other words,
on this view, once we reach the P stage the change is not reflected in the
categorial head status of the item but in the kinds of f-structure that are
associated with it and its projectability.

A complaint that is sometimes made about formal models by proponents of
grammaticalisation theory is that these formal models cannot capture what is
described as the ``gradualness'' of change because all they have at their
disposal is a set of discrete categories (see for instance
\citealt[330]{Haspelmath89}). The gradualness is more appropriately described
as change in small steps, as argued by \cite{Roberts10}. The analyses which we
describe here do exactly that; they provide ways of capturing those stages
between the prototypical categories that are characteristic of
grammaticalisation, though as we will see, the steps here are described in
functional and/or feature terms rather than through the use of a larger
inventory of syntactic heads in the way that is characteristic of cartographic
and nanosyntactic\is{nanosyntax} approaches.\footnote{For further discussion of the
gradualness question in the verbal domain, see \cite{BorjVinc19}.}

Within \gls{HPSG}, the full semantic use, or what \cite{PollSag94} call a
\enquote{predicative preposition}, is modelled as in (\ref{ex3}).\footnote{The
    authors we refer to here use slightly different versions of the \gls{HPSG}
    formalism without this affecting the general principles of the solutions.
    Our aim here has been to illustrate the points made by the different
authors in a unified way rather than to side with any one of them on detail.}

\begin{exe}
\ex\label{ex3}
\avm{
    [\type*{prep-word}
	cat & [ head & prep\\
          subj &  < NP \textsubscript{\1}  >\\
          comps & < NP \textsubscript{\2} & !{\ob}acc{\cb}! > ]\\
    cont & [\type*{allative-till}
		  figure & \1\\
		  ground & \2]
    ]
}
\end{exe}

That is to say, it is a full independent head of the type \emph{prep-word} with
an NP complement, where the \textsc{cont} feature is defined in terms of the
semantic concepts of \textsc{figure} and \textsc{ground} \citep{Tseng00, Tseng02}.
The grammatical use is also of type \emph{prep-word}, but in contrast to the
allative preposition, it has no independent  content value; the value for the
whole phrase is instead derived from that of the NP complement (this use is
referred to as \enquote{non-predicative} by \citealt{PollSag94}, and as
\enquote{transparent} by \citealt{Flickinger08}, whereas
\citealt{Abeilleetal06} describe it as a full head with \enquote{weak}
semantics). This is illustrated in (\ref{ex4}), where the values for the two
\textsc{cont} features are shared.

\begin{exe}
\ex\label{ex4}
\avm{
    [\type*{prep-word}
	cat & [ head & prep\\
	      marking & till\\
	      comps & < [cat & [ head &  noun ]\\
				     cont & \1 ] > ]\\
	cont & \1]
}
\end{exe}

In that sense, the preposition is semantically \enquote{transparent} but
preserves its head status and the constituent is accordingly still a PP. As
\Cref{Fig3CatsLFGHPSG} illustrates, in \gls{HPSG}, there is also a third
analysis possible, namely that of a \enquote{weak head}. This is the analysis
proposed for the use of the preposition in \ili{French} illustrated in (\ref{ex5})
\citep[150]{Abeilleetal06}, but it is not clear whether it would also be
applicable to the \ili{Swedish} example in (\ref{ex1c}). The relevant feature matrix
is provided in (\ref{ex6}).

\begin{exe}
\ex\label{ex5} \ili{French}
\begin{xlist}
\ex\label{ex5a}
\gll {\itshape Des} bijoux ont \'et\'e vol\'es.\\
    \textsc{de}.\Def.\Pl{} jewel.\Pl{} have.\Prs.\Tpl{} be.\Pst.\Ptcp{} steal.\Pst.\Ptcp.\Pl{}\\
\trans \enquote*{Jewels were stolen.}
\ex\label{ex5b}
\gll {\itshape De} sortir un peu plus te ferait  du bien.\\
	\textsc{de} {go out.\Inf{}} a little more you do.\Cond.\Tsg{}
    \textsc{de}.\Def.\M.\Sg{} good.\Sg{}	\\
\trans \enquote*{Getting out a bit more would do you good.}
\end{xlist}
\end{exe}

\begin{exe}
\ex\label{ex6}
\avm{
    [\type*{weak-head}
	  cat & [ head & \2\\
	        marking & de\\
	        comps & < [cat & \2 [ head & noun $\lor$ verb ]\\
				       cont & \1 ] > ]\\
	cont & \1]
}
\end{exe}

In (\ref{ex6}), \emph{de(s)} is no longer of type \emph{prep-word}, but of a
separate type \emph{weak-head}. Characteristic of this type is that it shares
the value for its \textsc{head} feature with its complement, which means that
these features, such as \Inf{} on the VP complement in (\ref{ex5b}), are
visible for external selection. This in turn means that it transmits nominal
properties if attached to a noun and verbal properties if attached to a verb.
Such prepositions are dubbed \enquote{minor} by \cite{vanEynde04} and
\enquote{non-oblique} by \cite{Abeilleetal06}. This is also the analysis
\cite{Tseng02} proposes for the complementiser\is{complementizers} \emph{that} in \ili{English}. The role
of weak heads within the overall descriptive apparatus of \gls{HPSG}\is{Head-Driven Phrase Structure Grammar} is similar
to that of non-projecting words in \gls{LFG}\is{Lexical Functional Grammar} in that they do not project,
though as shown in \Cref{Fig3CatsLFGHPSG}, they differ with respect to
semantic content.  Both these systems are thus significantly different from
Minimalism, where heads must always project. In diachronic terms, the
development is then captured in \gls{HPSG}\is{Head-Driven Phrase Structure Grammar} from N to \enquote{full} P head and
thence to either a transparent or a weak head or indeed, as here, to both.

The examples in (\ref{ex1}) instantiate a well-known difficulty in synchronic
descriptions of prepositions, namely how to model the formal identity beside
the functional differences, and accounts such as those set out in (\ref{ex2}),
(\ref{ex3}) and (\ref{ex4}) achieve this goal by retaining the syntactic category P\is{syntactic categories} while associating
it with different sets of morphosyntactic and semantic content. An alternative
way to proceed is to postulate a separate category for the grammatical marker,
in particular the functional head\is{functional items} K, which licenses the
associated NP or DP. K in turn can be realised either as a case-inflection or
as a preposition. This solution has been strongly advocated in recent work
within the nanosyntactic\is{nanosyntax} variant of Minimalism -- see for example
\textcite{Svenonius2008} and \textcite{RoySven09}.  Such an approach offers a
way to capture the functional equivalence of \emph{till} in an example like
(\ref{ex1b}) and the dative\is{dative case} case in the equivalent in a
language like \ili{Latin}, through the structural difference between the
preposition and the case marker is not as straightforwardly captured. In the
present context, it is to be noted that this case-marking function of
prepositions is itself the outcome of historical change. Items like
\ili{Swedish} \emph{till}, \ili{English} \emph{to} and \ili{French} \emph{à}
start out as semantically full expressions of direction and acquire this
secondary role over time. The same goes for prepositions like \ili{English}
\emph{of} and \ili{French} \emph{de} in their role as marking the argument of
nominal head in expressions like \emph{the king of England} or \emph{le roi de
France}. Within Minimalism such shifts can be seen as involving a change from P
to K, whereas once again, in \gls{HPSG}\is{Head-Driven Phrase Structure Grammar}
and \gls{LFG}, the change is in the information associated with the argument of
P rather than in the category itself.\footnote{For some discussion of the use
    of K in the analysis of complex prepositions like \emph{in spite of},
    \ili{Danish} \emph{p\aa {} grund af} `because', lit.\ `on ground of'
and \ili{French} \emph{à c\^ot\'e de} `beside', lit.\ `at side of', see
\textcite{RoySven09} and \textcite{VincInPress}.}

\section{Prepositions in the verbal domain}\label{sec:08.6}

Prepositional items may also develop in the direction of taking verbal
complements. In this section we examine three contrasting circumstances within
Germanic and one further one in \ili{Romance}. The \ili{Germanic} developments are
summarised in (\ref{ex7}).

\begin{exe}
\ex\label{ex7}
\begin{xlist}
\ex\label{ex7a}
\ili{English}: \emph{to} develops both as a preposition and as an infinitival marker.
\ex\label{ex7b}
\ili{German}: \emph{zu} derives from the same etymon as \ili{English} \emph{to}
(< PIE *do `to', `toward') and also has both prepositional and infinitival functions.
\ex\label{ex7c}
\ili{Swedish} and \ili{Danish}: the infinitive marker \emph{att}/\emph{at} also derives from a PIE locative particle *ad `to' but in this instance, unlike \ili{English} and \ili{German}, there is no homophony between infinitive marker and preposition, either because, as with \ili{Swedish} \emph{\aa t},  the preposition has an independent phonetic development or because, as in \ili{Danish}, the prepositional usage does not survive.
\end{xlist}
\end{exe}

All these developments are instances of the cross-linguistically recurrent
diachronic cline (\ref{ex8}) identified in \cite{Haspelmath89}.

\begin{exe}
\ex\label{ex8}
allative preposition > purposive marker > infinitival marker
\end{exe}

At the same time, there are significant structural differences between the
individual \ili{Germanic} languages under consideration here. \ili{German} \emph{zu}
cannot be separated from the verb and hence the grammaticality difference
between (\ref{ex9a}) and (\ref{ex9b}).

\begin{exe}
\ex\label{ex9} \ili{German}
\begin{xlist}
\ex[]{
\gll Er hat versprochen, bald zu kommen.\\
	he have.\Pst{} promise.\Pst.\Ptcp{} soon \textsc{zu} come.\Inf{}\\
\trans \enquote*{He had promised to come soon.}}\label{ex9a}
\ex[*]{
    Er hat versprochen, zu bald kommen.}\label{ex9b}
\end{xlist}
\end{exe}

Indeed \emph{zu} can, in certain circumstances, be part of the verb, as in the
infinitive \emph{aufzustehen} `to stand up' beside the finite \emph{ich stehe
    auf} `I stand up'. In the words of \cite[296]{Haspelmath89}. \enquote{Modern
    \ili{German} \emph{zu} is probably a bound prefix although the spelling
    treats it as a non-bound element} (compare \citealt{Giusti91} for a similar
    conclusion).

In \ili{English}, some separability is permitted, as in the Star Trek
introduction: \emph{To boldly go where no man has gone before} or in examples
like (\ref{ex10}), which are frequent despite the prescriptive prohibition of
the split infinitive, not least because there is no obvious alternative to
placing the adverb between \emph{to} and \emph{understand}.

\begin{exe}
\ex\label{ex10}
To really understand the situation you need to be an experienced politician.
\end{exe}

The grammatical category to be assigned to \ili{English} \emph{to} is more
controversial. \cite{Pullum82} argues that it behaves like an auxiliary\is{auxiliaries}, and
\cite{KostMay82} place it in I on the grounds that it expresses the feature
value [$-$finite] and that finiteness in \ili{English} is, in general, a
property of items that fall under I.  As \cite{Falk01to} observes, this
conclusion only follows if functional properties and categorial status have to
be aligned, as indeed they do in the GB framework adopted by Koster \& May, but
Falk is operating within \gls{LFG}\is{Lexical Functional Grammar} and, having
separated function and category, concludes that \emph{to} is in C. We will not
seek to resolve the matter here; it suffices for us to note that all are agreed
that its status in this construction is no longer prepositional. Moreover, it
is clear that the distribution of \emph{to} in earlier stages of the language
implies a different status from that which it has in the modern language
\citep{vanGelderen98}.  \cite{Haspelmath89} adduces similar evidence for the
separation of \emph{zu} from V in earlier stages of \ili{German}. Putting this
evidence together, therefore, we can postulate a diachronic trajectory from P
to an intermediate functional head such as C or I followed by
\isi{incorporation} under V.

When we come to North \ili{Germanic}, however, things look rather different.
Not only is the etymological source of the infinitival marker different but so
is its distribution (\citealt{Platzack86, BeukDikk89, Christensen07}). The
examples in (\ref{ex11}) show that \ili{Swedish} \emph{att}, for example, can
be separated from the verb even by whole phrases and clauses.

\begin{exe}
\ex\label{ex11} \ili{Swedish}
\begin{xlist}
\ex
\gll Hon nj\"ot av {\itshape  att} efter m\aa nga	\aa r \aa ter {\itshape  k\"anna} fast mark under f\"otterna.\\
she enjoy.\Pst{} of \textsc{att} after many year again feel.\Inf{} solid ground under foot.\textsc{pl.def}\\\trans
\enquote*{She enjoyed feeling solid ground under her feet again after many years.}
\ex\gll {\itshape  Att} fast\"an hon bara kunde ha st\"angt d\"orren efter sig {\itshape  stanna}	och {\itshape  lyssna} {p\aa} vad han hade att s\"aga visade sig vara ett d\aa ligt beslut.\\	\textsc{att}	although she only could ha.\Inf{} close.\Pst.\Ptcp{} door.\Def{} after \Refl{} stay.\Inf{} and listen.\Inf{} on  what he have.\Pst{} \textsc{att} say.\Inf{} show.\Pst{} \Refl{} be.\Inf{} a poor decision\\\trans \enquote*{To stay and listen to what he had to say, even though she could have simply closed the door behind her, turned out to have been a poor decision.}
\end{xlist}
\end{exe}

It is also the case that, in \ili{Swedish}, negation and negated objects
obligatorily occur between \emph{att} and the verb as in (\ref{ex12}).\largerpage[1]

\begin{exe}
\ex\label{ex12} \ili{Swedish}
\begin{xlist}
\ex
\gll Hon gjorde sitt b\"asta f\"or (*inte) att inte somna (*inte).\\
she do.\Pst{} \Refl.\Poss{} best for {} \textsc{att} not {fall asleep.\Inf{}} {}\\
\trans \enquote*{She did her best not to fall asleep.}\\
\ex
\gll K\"anslan av att ingenting kunna g\"ora (*ingenting) skr\"ammer mig.\\
feeling.\Def{} of \textsc{att} nothing {be able.\Inf{}} do.\Inf{} {} frighten.\Prs{} me\\
\trans \enquote*{The feeling of not being able to do anything about it frightens me.}
\end{xlist}
\end{exe}

Given this distribution it is natural to see \ili{Swedish} infinitival
\emph{att} and the corresponding forms in other Scandinavian languages as
occupying the complementiser position and hence as instantiating a change from
P to C. At the same time, it is of interest that these languages also display a
separate form, usually spelled the same but pronounced differently, that is,
the complementiser\is{complementizers} for finite clauses as in (\ref{ex13})
(examples (\ref{ex13b}) and (\ref{ex13c}) taken from \citealt{NordBoye16}).

\begin{exe}
\ex\label{ex13}
\begin{xlist}
\ex\label{ex13a} \ili{Swedish}\\
\gll Olle vet att han f\aa r komma {p\aa} festen.\\
Olle know.\Prs{} \Comp{} he {is allowed.\Prs{}} come.\Inf{} on party.\Def{}\\
\trans \enquote*{Olle knows that he is allowed to come to the party.}
\ex\label{ex13b} \ili{Danish}\\
\gll Hun tvivler {p\aa} at han er der.\\
she doubt.\Prs{} on \Comp{} he be.\Prs{} there\\
\trans \enquote*{She doubts that he is there.}
\ex\label{ex13c} \ili{Faroese}\\
\gll Hon fortelur at hann fer at koma i dag.\\
she tell.\Prs{} \Comp{} he go.\Prs{} \textsc{at} come.\Inf{} in day\\
\trans \enquote*{She says that he is going to come today.}
\end{xlist}	\end{exe}

Thus, in (\ref{ex13c}) for example, the first occurrence of \emph{at} is a
finite complementiser\is{complementizers} derived from a demonstrative\is{demonstratives} pronoun and cognate with
English \emph{that}, while the second occurrence in the future periphrasis
\emph{fer at koma} is cognate with \ili{Swedish} infinitival \emph{att} and has a
prepositional source.

What we have seen in this section, then, is how prepositional items, which are
traditionally defined as taking nominal complements may also over time come to
be associated with verbal complements. We now turn now to consider the
consequences of this alternative pattern of development.

\section{From the nominal to the verbal domain}\label{sec:08.7}

We have characterised the changes in the previous section in terms of a
historical shift from P to C and/or I, and this is indeed what would have to be
said within both Minimalism and \gls{LFG}. However, the \gls{HPSG}\is{Head-Driven Phrase Structure Grammar} concept of
\enquote{weak head} will allow us to generalise across all the developments by
simply saying that the original full head status of the prepositions in
question weakens over time. Recall that the definition of a weak head is one
that contributes only the value for the \textsc{marking} feature but yields its
{\sc head} value, that is, its syntactic category,\is{syntactic categories} to the item with which it
combines. Thus, if it combines with a verb, as with \ili{German} \emph{zu}, its
external distribution is determined by that verb; if it is an independent
constituent, as is the claim made in assigning an item the status of I or C,
then it will pattern with that larger constituent, be it finite or non-finite
as the context requires. We will consider now some evidence from \ili{Romance} where
the items in question do indeed yield their distributional power to the item
with which they co-occur but, unlike the \ili{Germanic} examples we have been
considering, they nonetheless retain their own value as prepositions. In other
terminology, they are prepositional complementisers\is{complementizers} \citep{Kayne99, Borsley01}.

Compare the two \ili{French} examples in (\ref{ex14}) as discussed by
\cite{Abeilleetal06}.

\begin{exe}
\ex\label{ex14} \ili{French}
\begin{xlist}
\ex\label{ex14a}
\gll Il	est all\'e \`a la gare.\\	he be.\Prs.\Tsg{} go.\Pst.\Ptcp{} to the station\\\trans \enquote*{He went to the station.}
\ex\label{ex14b}
\gll Il m'a invit\'e \`a venir demain.\\	he me-have.\Prs.\Tsg{} invite.\Pst.\Ptcp{} to come.\Inf{} tomorrow\\\trans \enquote*{He invited me to come tomorrow.}
\end{xlist}
\end{exe}

(\ref{ex14a}) is a clear case of the full lexical preposition {\it\`a} with the
directional meaning `to', akin therefore to \ili{Swedish} \emph{till} in
(\ref{ex1a}). (\ref{ex14b}), on the other hand, is another instance of an
allative preposition coming to introduce an infinitival complement of a higher
verb. The difference in the \ili{Romance} case is that the pattern with \emph{\`a}
(and its cognates in the other languages) exists and develops side by side with
another such pattern using the preposition \emph{de} `of, from' as in the
examples in (\ref{ex15}).

\begin{exe}
\ex\label{ex15} \ili{French}
\begin{xlist}
\ex\label{ex15a}
\gll Il	vient de Paris.\\	he come.\Prs.\Tsg{} \textsc{de} Paris\\\trans \enquote*{He comes from Paris.}
\ex\label{ex15b}
\gll Il a d\'ecid\'e de venir demain.\\	he have.\Prs.\Tsg{} decide.\Pst.\Ptcp{} \textsc{de} come.\Inf{} tomorrow\\\trans \enquote*{He has decided to come tomorrow.}\end{xlist}
\end{exe}

\citeauthor{Abeilleetal06} represent the lexical prepositions in (\ref{ex14a})
and (\ref{ex15a}) in much the same way as they would be represented in other
frameworks: they are of the type \emph{prep-word} and take an N-headed
complement. The difference between frameworks is rather to be seen in the
treatment of the grammaticalised\is{grammaticalization} use of the preposition to introduce an
infinitive. For \citeauthor{Abeilleetal06}, the weak heads \emph{\`a} and
\emph{de} in (\ref{ex14b}) and (\ref{ex15b}) are  heads in the sense that they
select a complement, viz.\ the infinitival VP \emph{venir demain}, and they add a
value for the feature \textsc{marking} to the phrases they head, but they remain
weak in the sense that they inherit the valence list of the complement. This
last point is crucial since the matrix verb, on the one hand, determines the
form of the complement -- \emph{inviter} in (\ref{ex14b}) selects an infinitive
marked with \emph{\`a} and \emph{d\'ecider} in (\ref{ex15b}) one with \emph{de}
-- and on the other contracts argument relations via \isi{control}, or in other
circumstances \isi{raising}, with the embedded infinitive.\footnote{Unlike
    either \gls{LFG}\is{Lexical Functional Grammar} or \gls{HPSG}, or indeed
    some versions of Minimalism, \cite[50]{Kayne99} takes the alternative tack
    of arguing with respect to precisely this kind of \ili{Romance} data that
``prepositional complementisers\is{complementizers} do not form a constituent
with the infinitival IP they are associated with''. For a detailed response to
Kayne's position, see \cite{Borsley01}.}

At first sight it might appear that this is no different from saying that the
items in question have become functional heads\is{functional items}. However, \cite[note
12]{Abeilleetal06} are at pains to stress that, in their words, \enquote{weak heads
differ from functional heads\is{functional items} in \gls{LFG}\is{Lexical
Functional Grammar} or GB}. In particular, a weak head is not a new type of
category. As they go on to say: ``Although a weak head's category is
underspecified in the lexicon, in any given syntactic context, it has a
completely ordinary syntactic category\is{syntactic categories} (e.g. N or V).
It is important to emphasise that when a weak head inherits a value of type
verb or noun, it does not actually ``become'' a verb or a noun (i.e., a lexical
object of type \emph{noun-word} or \emph{verb-word}).'' Rather, in our present
case, it maintains its status as a \emph{prep-word}, which it shares with the
full lexical preposition. In other words, the change is not a matter of
grammatical category but of the manner in which elements of this kind integrate
with the other parts of the sentence.\footnote{There is one significant respect
    in which the infinitival markers differ from ordinary prepositions, namely
    that they do not combine with the preverbal clitics\is{clitics} in the same
    way a preposition combines with the prenominal article. Thus,
    \emph{\`a}/\emph{de les voir} `\Comp{} them see.\Inf{}' does not become
    *\emph{aux}/\emph{des voir} in the way that underlying \emph{\`a}/\emph{de
    les gar\c{c}ons} obligatorily becomes \emph{aux}/\emph{des gar\c{c}ons}.
    Standard accounts explain this by treating the clitic and the article as
    belonging to the category D and attributing the differential behaviour to a
    categorial distinction between a P and C/I, whereas
    \citeauthor{Abeilleetal06}  follow traditional grammar and treat pronouns
    and articles as distinct categories with the phonological merger only
applying to the sequence P $+$ Art. However, as they observe in their footnote 9,
decisive evidence one way or the other is hard to come by.}\largerpage

Within \gls{LFG}\is{Lexical Functional Grammar}, a framework in which, as we have said, the distinction
between category and function is built into the basic architecture via the
distinction be\-tween f-structure and c-structure, an example like (\ref{ex14a})
can be treated in the same way as our \ili{Swedish} example (\ref{ex1a}). For the
infinitival construction, one option is to maintain the prepositional analysis,
which entails a c-structure of the form in (\ref{ex16}).

% \begin{exe}
% \ex \label{ex16}
% [$_{\textrm {PP}}$ [$_{\textrm {P}}$ \emph{\`a}] [$\textrm _{VP}}$ \emph{venir demain}]]
% \end{exe}

\begin{exe}
\ex\label{ex16} [$_{\textrm {PP}}$ [$_{\textrm{P}}$ \emph{\`a} ] [$_{\textrm{VP}}$ \emph{venir demain}] ]
\end{exe}

This in turn would imply that diachronically the shift is not in the
prepositional head but rather in an expansion of its f-structure to include
\textsc{xcomp} as well as \Obl{}, so that there is a single lexical item with
two alternate functional values depending on context. Alternatively, we have an
IP with \emph{\`a} defined as the value for the \textsc{compform} feature
within its associated f-structure. The latter solution comes back to saying
that there has been a diachronic shift at the categorial level, viz.\ P > C,
and hence two distinct items.

The empirical evidence here is split. \ili{Latin} prepositions did not govern
infinitives, but there was a construction in which \emph{ad} took a gerund as
complement, thus \emph{ad dicendum} `towards, for speaking'. The change seems
to have involved the loss of the gerund (in this function at least) and its
replacement by the infinitive, itself also a verbal noun in origin. While this
argues for \emph{ad} and its \ili{Romance} reflexes having retained the status of
prepositions, the fact that there are in the modern languages alternations
between prepositional infinitives and finite complements introduced by
\emph{que} `that' argues for the shift from P to C. Thus, if the complement of
the preposition \emph{avant} `before' is infinitival, it is introduced by
\emph{de}, and if it is a finite clause we have \emph{que}, as in
(\ref{ex17}).

\begin{exe}
\ex\label{ex17} \ili{French}
\begin{xlist}
\ex\label{ex17a}
\gll Pierre \'ecrira  la lettre avant de partir.\\
Pierre write.\Fut.\Tsg{} the letter before \textsc{de} leave.\Inf{}\\
\trans \enquote*{Pierre will write the letter before leaving.}
\ex\label{ex17b}
\gll Pierre \'ecrira la 	lettre avant que sa soeur ne parte.\\
Pierre write.\Fut.\Tsg{} the letter before \Comp{} his sister not leave.\Sbjv.\Sg{}\\
\trans \enquote*{Pierre will write the letter before his sister leaves.}
\end{xlist}
\end{exe}

Whichever solution is in the end adopted, there is a further difference between
the use of functional heads\is{functional items} in \gls{LFG}\is{Lexical
Functional Grammar} and Minimalism that needs to be emphasised. In the remark
quoted above, \citeauthor{Abeilleetal06} refer to \enquote{LFG and
\glsunset{GB}\gls{GB}}. While it is true that in the latter, functional heads
were for the most part restricted to C, T, I and D, at least one strand of
Minimalism, the so-called cartographic approach developed by Cinque and others,
takes the further step of  decomposing heads like C into a set of subsidiary
functional heads\is{functional items} \citep{Rizzi97}. Within such an approach,
the original simple functional head\is{functional items} C is split into a
series of separate heads, of which Force is the highest and Fin the
lowest.\footnote{In Rizzi's original account there were three intermediate
    heads between Force and Fin, namely two different Top(ic) heads ranged
    respectively above and below an intermediate Foc(us) head. In subsequent
    work within the framework, the number of such heads has expanded
considerably but, for the purposes of our argument, consideration of Rizzi's
original proposal is sufficient.}  The item \emph{de} in an example like
(\ref{ex15b}) or (\ref{ex17a}) would be assigned to the Fin head whereas a
finite complementiser\is{complementizers} like \emph{que} in (\ref{ex17b}) is
located in Force.  There are, however, two problems with moves of this kind.
First, there is the obvious danger that, as the number of such heads expands,
explanation is replaced by enumeration. The set of functional heads simply
becomes an ever more fine-grained taxonomy. To take a recent example,
(\ref{ex18}) sets out the structure proposed in \cite{MunaPole14} for items
meaning `where' (construed as a PP `at/to wh-place') in a range of Italian
dialects (= their (7)).

{\sloppy
\begin{exe}
\ex\label{ex18}
{}[\tss{\emph{{PPDirSource}}}  da/di [\tss{\emph{{PPDirGoal}}}  in
[\tss{\emph{{PPDirPath}}}  d [\tss{\emph{{DisjP}}}  o/u [\tss{\emph{{StatP}}}
[\tss{\emph{{DegreeP}}}  [\tss{\emph{{ModeDirP}}} [\tss{\emph{{AbsViewP}}}
[\tss{\emph{{RelViewP}}}  [\tss{\emph{{DeicticP/ExistP}}}  l\`a/v/nd
[\tss{\emph{{AxPartP}}}  [\tss{\emph{PP}}  [\tss{\emph{P⁰}}] [\tss{\emph{{NPplace/Restrictor}}} e [PLACE]]]]]]]]]]]
\end{exe}
}

As they go on to note, \blockquote{we assume that the whole extended projection
in (7) is active even when a single lexically realized morpheme is present,
irrespective of whether it occupies a high or low position}
\citeyearpar[292]{MunaPole14}. When the constituent structures reach this order
of complexity, it is reasonable to ask whether alternative approaches, in which
not all aspects of meaning have to be driven through the syntax, are not worth
considering. Moreover, diachrony adds a further difficulty: if, as we have seen
and as also emerges in the \citeauthor{MunaPole14} study and in related
nanosyntactic research such as \cite{RoySven09}, the source of such heads lies
in what were originally full lexical items, then the number of possible
diachronic intermediate steps is potentially infinite, since there are no
universally definable intermediate steps on the cline from lexical to
grammatical.

\section{Conclusions}

We are now in a position to draw some conclusions from the case studies we have
been considering and in particular to consider the relevance of diachronic data
for theory construction. Let us begin with the key point that this data set
reinforces the standard conclusion that grammaticalisation\is{grammaticalization} has a clear
directionality. Lexical items of various categories may become prepositions
with a range of functions and they move on from there to become
complementisers, thereby shifting from the domain of nominal marking to verbal
marking. A natural question to ask therefore is whether such directionality
follows from any independent properties of the frameworks we have been
exploring. And in the case of both \gls{LFG}\is{Lexical Functional Grammar}
and \gls{HPSG}\is{Head-Driven Phrase Structure Grammar} the answer is a clear
no. There are no  internal principles within their architectures which predict
the direction of change. This is a notable difference when compared to
Minimalism, where, as we noted at the outset, the fact that
grammaticalisation\is{grammaticalization} changes show a directionality  can be
argued -- and indeed has been argued, not least by Ian Roberts in a number of
studies -- to follow from the fact that Universal Grammar allows raising but
not lowering as a derivational operation. However, even this principle would
not account for our observation that prepositions become
complementisers\is{complementizers} but not vice versa since PP and CP are
typically different projections rather than one being the extension of the
other.

Two other types of diachronic pattern that have been considered from a
Minimalist perspective are so-called lateral \isi{grammaticalization} and
downwards \isi{grammaticalization}. The classic instance of the former is the
development of deictic markers into copular verbs (see \citealt{BorjVinc17} for
discussion and references), where an item appears to jump across from the
nominal to the verbal domain. Downward grammaticalization, by contrast, is to
be seen when an item starts its grammatical existence in a higher position and
evolves into something which occupies a lower position in the tree. A case in
point is the discussion by \cite{Munaro16} of the development of
complementisers\is{complementizers} in some Italo-Romance dialects, where an
item that was originally in the higher Force head position comes to occupy the
lower Fin position. The evidence of changes such as these suggests that
directionality of derivation is not the key to the directionality of change.

The alternatives, therefore, are either to find other internal mechanisms of
grammar, such as the Late \isi{Merge} and Economy principles proposed by van
Gelderen (\citeyear{vanGelderen2009,vanGelderen11}), or to consider the driving
force of change to be the external circumstances of language use, but to deploy
the devices of formal syntax in order to model such changes as and when they
are attested.  Thus, if, over time, we find evidence of nouns evolving into
prepositions, prepositions evolving into complementisers\is{complementizers}
and prepositions evolving from lexical (\enquote{full semantics}) to
grammatical (\enquote{weak} semantics), but we do not have any attested cases
of the reverse,  we may reasonably ask: why not? The answer, we suggest, lies
in the fact that non-finite forms start out as nominal and shift to verbal as
they are incorporated into the verbal paradigm. There is, by contrast, no
corresponding nominalisation of finite forms. In other words, the
directionality follows from the content and contextual function of the
constructions at issue and does not need to be ascribed to any principle of
\gls{UG}.

The constructions we have reviewed here also demonstrate that large scale
categorial changes can -- and given the diachronic evidence should -- be
broken down into smaller steps which in turn can be modelled using such formal
constructs as weak and transparent heads and non-projecting words. Within
frameworks like \gls{LFG}\is{Lexical Functional Grammar} and \gls{HPSG}, however, such constructs are not
required to respect universal principles of categorial hierarchy. And in
particular within a parallel correspondence architecture such as that provided
by \gls{LFG}\is{Lexical Functional Grammar}, changes in the different dimensions do not necessarily proceed at
the same pace. This, of course, is a familiar result when it comes to
(morpho)syntax and phonology, but even within the former dimension we can now
see that an item may cease to co-occur with nominals without necessarily losing
the marking properties of a preposition. What, on the other hand, all three
systems discussed here share is a commitment to the formal modelling of
linguistic structure. The relation between any formal account and a functional
explanation for the existence or development of that account remains, by
contrast, an open question.

\printchapterglossary{}

\section*{Acknowledgements}

An earlier version of this paper was presented at the HeadLex16 conference in
Warsaw in July 2016. Our thanks to those who commented on that occasion and to
the anonymous reviewers of the present version.

{\sloppy
\printbibliography[heading=subbibliography,notkeyword=this]
}

\end{document}
