\documentclass[output=paper]{langsci/langscibook}
\author{Eric Fuß\affiliation{Ruhr University Bochum}\lastand Carola
Trips\affiliation{University of Mannheim}}
\title{“Them’s the men that does their work best”: The Northern Subject Rule
revisited}

% \chapterDOI{} %will be filled in at production

\abstract{This paper addresses a set of issues concerning the analysis and
    historical development of the so-called Northern Subject Rule (NSR), which
    characterises many northern varieties of English. Based on an investigation
    of NSR effects in the Northern Middle English \emph{York Plays}, we present
    a new account of the NSR that combines a \glsunset{DM}\gls{DM} analysis of
    the relevant \isi{agreement} markers with the idea that inflectional heads
    lacking phi-features (“blank generation”, \citealt{Roberts:2010}) may
    acquire \isi{agreement} features via the \isi{incorporation} of adjacent subject
    pronouns.  Based on this analysis, we suggest a new scenario for the
    historical development of the NSR, arguing that after the breakdown of the
    \ili{Old English} \isi{agreement} system, the NSR developed via dialect contact between
    northern and southern varieties. More precisely, we propose that syncopated
    verb forms (resulting from southern Agr-weakening) were integrated into the
northern grammar as marked \isi{agreement} formatives that contrasted with the
generalized \emph{-s}-ending.}

\maketitle

\rohead{\thechapter\hspace{0.5em}The Northern Subject Rule revisited}

\begin{document}\glsresetall

\section{Introduction}\label{sec:intro}

This paper deals both with (i) synchronic properties and (ii) the diachronic
development of a peculiar \isi{agreement} phenomenon that characterizes many northern
dialects of (British) English. In varieties spoken in (central) northern
England (in particular, Northumberland, Cumberland, Durham, and Westmorland),
Scotland and northern Ireland (see \citealt{Pietsch:2005a,Pietsch:2005b} for
details concerning the geographical distribution), the distribution of the
verbal \isi{agreement} formative \emph{-s} is governed by what is today commonly
called the \emph{\glsdesc{NSR}} (\glsunset{NSR}\gls{NSR},
\citealt[221]{Ihalainen:1994}; in earlier work, the same phenomenon has also
been dubbed the ``Personal Pronoun Rule'', \citealt{McIntosh:1988}, or ``Northern
Present Tense Rule'', \citealt{Montgomery:1994}).\footnote{See
\cite{GodfreyTagliamonte:1999} for a similar pattern in Devon English spoken in
the southwest of England.} Many northern English dialects have in common that
the \emph{s}-inflection, which is confined to \Tsg{} present tense indicative in
Standard English, has a wider distribution and may (variably) occur in other
contexts as well (with plural subjects, in particular, but in certain varieties
also with \Fsg{} and \Ssg{}; see \citealt{Pietsch:2005a,Pietsch:2005b} for \gls{NSR}
dialects with different inventories of inflections). Crucially, however, the
realization of verbal \isi{agreement} is subject to further conditions in the \gls{NSR}
dialects. The relevant varieties typically show the standard \isi{agreement} pattern
(\Tsg{} -\emph{s}, zero ending elsewhere\is{Elsewhere Condition}) in cases where the finite verb is
directly adjacent to a pronominal subject, but whenever this configuration is
not given, the generalized -\emph{s} form occurs (cf.\ \citealt{Murray:1873},
\citealt{Berndt1956}, \citealt{McIntosh:1988}, \citealt{Montgomery:1994},
\citealt{Schendl:1996b}, \citealt{Corrigan:1997},
\citealt{Boerjarschapman:1998}, \citealt{Klemola:2000},
\citealt{Pietsch:2005a,Pietsch:2005b}, \citealt{deHaas:2011}, amongst others).
In other words, the realization of verbal \isi{agreement} is sensitive to (i) the
type of subject (pronouns vs.\ full DP subjects) and (ii) the position of the
subject.

\ea \textbf{\glsdesc{NSR}} (\textbf{\gls{NSR}}): A finite verb
  (in the present indicative) takes the ending \emph{-s} except when it is
  directly adjacent to a non-\Tsg{} pronominal subject
  (\emph{I/you.\Sg/we/you.\Pl/they}).
\z

As a result, the \gls{NSR}\is{Northern Subject Rule} dialects exhibit a three-way distinction dependent
on type and position of subject: if the subject is a full DP, the finite verb
takes the -\emph{s} and adjacency is no determining factor (see (1a)). If the
subject is a non-\Tsg{} pronoun and adjacent to the finite verb, the finite
verb doesn't take the -\emph{s} ending (see (1b)) and instead appears without
overt inflection; if the subject pronoun is not adjacent to the verb, the
-\emph{s} occurs again. The adjacency effect is triggered by \isi{adverbs} that
intervene between the subject and the finite verb as shown in (1c) and in cases
of VP \isi{coordination}, as in (1d). A related effect can be observed in relative
clauses such as (1e), where the relativizer intervenes between the pronominal
head and the finite verb.\largerpage

\ea
	\ea the birds (only) sing\textbf{s}
	\ex they sing
	\ex they only sing\textbf{s}
	\ex they sing and dance\textbf{s}
	\ex they that sing\textbf{s} (`they who sing')
    \label{nsrexs}
    \z
\z

The \gls{NSR}\is{Northern Subject Rule} also applies in cases where the pronoun is right-adjacent to the finite verb, i.e., in cases of subject-verb inversion:

\ea
	\ea \textbf{Do} they sing?
	\ex \textbf{Does} the birds sing?
    \z
\z

The differences between the Standard English \isi{agreement} system and the \gls{NSR}
dialects are schematically summarized in \Cref{tab:key:10.1}.\footnote{As
indicated in \Cref{tab:key:10.1}, in those dialects that have retained some reflex
of the original \Ssg{} pronoun \emph{thou}, the \Ssg{} pronouns typically
behave on a par with \Tsg{} forms in that they always trigger \emph{s}-marking
on the verb, \textcite[76]{Pietsch:2005b}. This observation will be addressed in more
detail below.}

\begin{table}[ht]
  \centering
 \renewcommand{\arraystretch}{1.1}
\begin{smaller}
    \begin{tabularx}{\textwidth}{@{}lllll}
    \lsptoprule
    &  Standard English &  \multicolumn{3}{c}{\gls{NSR} varieties of English}\\
    \midrule
    & & pronominal subjects & pronominal subjects & nominal subjects \\
    & & (adjacent to V) & (non-adjacent to V)\\
    \midrule
    \Fsg{} & sing & sing & sing-s & -\\
    \Ssg{} & sing & sing (thou sing-s)  & sing-s & -\\
    \Tsg{} & sing-s & sing-s & sing-s & sing-s \\
    \Fpl{} & sing & sing & sing-s & -\\
    \Spl{} & sing & sing & sing-s & -\\
    \Tpl{} & sing & sing & sing-s & sing-s \\
    \bottomrule
  \end{tabularx}
  \caption{Verbal inflection (present tense), Standard English
  vs.\ Northern varieties + \gls{NSR}}\label{tab:key:10.1}
\end{smaller}
\end{table}

The kind of \gls{NSR}\is{Northern Subject Rule} as defined in (1) above and illustrated in
\Cref{tab:key:10.1} has been reported for historical stages of Northern varieties
of English (cf.\ e.g.\ \citealt{Cowling:1915} on the dialect of Hackness in
North-Yorkshire; \citealt{Montgomery:1994} on Old Scots and northern ME/EModE), but does
not seem to exist in this `pure' form anymore today. Present-day varieties
typically exhibit some amount of variation concerning the distribution of
-\emph{s} (cf.\ \citealt{Montgomery:1994,Britain:2002,Pietsch:2005a,Pietsch:2005b,AdgerSmith2010,Buchstaller_etal:2013,Childs:2013}): With the exception of (i) \Tsg{} subjects (which
    invariably trigger -\emph{s}) and (ii) non-\Tsg{} pronouns adjacent to the
    verb (which strongly disfavour -\emph{s}), the use of the -\emph{s}-ending
    may vary with both nominal and pronominal subjects. To account for this
    kind of variation, it is often assumed that the constraints concerning type
    and position of subject are two separate (and competing) conditions
    \parencite{Montgomery:1994,Pietsch:2005a,Pietsch:2005b}: Little or no
    variation obtains when there is no conflict between the constraints (i.e.,
    with (i) \Tsg{} subjects and (ii) non-\Tsg{} pronouns adjacent to the
    verb), while variable \isi{agreement} patterns emerge in other contexts
    (e.g., with non-\Tsg{} pronouns that fail to be adjacent to the verb; more
    generally, non-adjacency of subject and verb generally seems to favour the
    use of -\emph{s}, cf.\ \citealt{Pietsch:2005b} for details). Still, we think that it
    is important to understand the somewhat idealized system in
    \Cref{tab:key:10.1}, which can be taken to represent the historical basis
    from which the present-day dialects developed.

In the literature, a number of analyses have been put forward to explain the
synchronic (and diachronic) facts (cf.\ \citealt{Henry:1995} on Belfast
English, \citealt{Boerjarschapman:1998}, \citealt{Hudson:1999},
\citealt{Pietsch:2005a}, \citealt{deHaas2008,deHaas:2011},
\citealt{deHaasandvanKemenade:2015}, \citealt{TortoraDenDikken:2010} on related
phenomena in Appalachian English, \citealt{AdgerSmith2010} on the variety of
Buckie in North-East Scotland).  However, as pointed out by
\textcite[180]{Pietsch:2005a}, most of these proposals focus on either the type
of subject or position of subject constraint and therefore typically miss a
subset of the relevant descriptive generalizations (cf.
\citealt{Pietsch:2005a} and \citealt{deHaas:2011} for extensive
discussion).\footnote{Pietsch himself proposes a usage-based account of the
    data which captures the variable \isi{agreement} facts in present-day \gls{NSR}
    varieties in terms of competing lexicalized constructions but misses the
morphological generalization that -\emph{s} is the underspecified exponent in
the relevant systems. See also \textcite[1122f.]{AdgerSmith2010} for critical
discussion.} This can be illustrated with the analysis proposed by
\textcite{Henry:1995} for so-called ``singular concord'' in \ili{Belfast English}
(basically the same account is adopted by \citealt{deHaas2008} to analyze
\gls{NSR} effects in the northern varieties more generally). Henry assumes that
there is a link between morphological case\is{case!morphological case} marking and the subject's ability to
trigger \isi{agreement} on the verb. More precisely, she claims that only elements
that are clearly marked as nominative\is{nominative case} (the pronouns \emph{I}, \emph{we},
\emph{he}, \emph{she}, \emph{they}; \emph{you} is treated as an exception) move
to SpecAgrsP and trigger ``standard'' \isi{agreement} on the verb (i.e., \Tsg{}
-\emph{s} vs.\ zero in all other contexts). In contrast, full DP subjects
occupy SpecTP, from which they cannot trigger verbal \isi{agreement}, leading to
insertion of the default ending -\emph{s}, which is analyzed as a pure
(present) tense marker:\footnote{\citeauthor{Henry:1995} seems to assume that
    \Tsg{} -\emph{s} and default -\emph{s} are separate markers, which happen
    to be homophonous.  To account for variable -\emph{s}-marking with phrasal
    subjects, she assumes that full DP subjects may optionally carry nominative\is{nominative case}
    (instead of default) Case, which licenses movement to SpecAgrsP.}

\ea
	\ea \mbox [\tief{CP} [\tief{AgrsP} They [\tief{Agrsʹ} are [\tief{TP} [\tief{Tʹ} T [\tief{VP} going]]]]]]\\
	\ex \mbox [\tief{CP} [\tief{AgrsP} [\tief{Agrsʹ} [\tief{TP} The teachers [\tief{Tʹ} is [\tief{VP} busy]]]]]]\\
    \z
\z

This approach accounts for the type-of-subject condition, but it does not seem
to have much to say about the adjacency condition that characterizes all other
\gls{NSR} varieties.\footnote{Note that the distribution of -\emph{s} is also
    subject to an adjacency effect in \ili{Belfast English}. However, the outcome of
    the adjacency condition seems to differ from what we have seen so far in
    that -\emph{s}-marking is blocked when an adverb intervenes between a
    phrasal \Tpl{} subject and a finite auxiliary\is{auxiliaries} (see
    \citealt[1116ff.]{AdgerSmith2010} for discussion of the difference between
    \ili{Belfast English} and other (Scottish/Northern English) \gls{NSR}\is{Northern Subject Rule} varieties).

\begin{exe}
    \exi{(i)}[]{The children really are late.}
    \exi{(ii)}[*]{The children really is late.}
\end{exe}\vspace{-.5\baselineskip}} Moreover, Henry's account makes use of a
number of non-stan\-dard
assumptions and stipulations (e.g.\ concerning the optional presence of
nominative Case on phrasal subjects), which does not seem to be particularly
attractive on conceptual grounds. Recently, \textcite{deHaas:2011} and
\textcite{deHaasandvanKemenade:2015} have put forward an update of Henry's
analysis that includes a set of extra assumptions that take care of the
adjacency condition. \Citeauthor{deHaas:2011} and
\citeauthor{deHaasandvanKemenade:2015} maintain the idea
that only pronominal subjects occupy the specifier of a functional \isi{agreement}
head located above TP (\citealt{deHaas:2011}: SpecFP;
\citealt{deHaasandvanKemenade:2015}: \mbox{SpecAgrsP}) whereas nominal subjects
occur in a lower position (SpecTP) from where they cannot induce \isi{agreement}. The
adjacency effect is then captured by assuming that the (post-syntactic)
realization of \isi{agreement} on the finite verb (situated in T in ME, but
presumably in an even lower position in the present-day varieties) is blocked
by material that intervenes between AgrS/F and T and interrupts the transfer of
agreement features from AgrS/F to T (which \citealt[166]{deHaas:2011} analyzes
as an instance of morphological merger, basically following
\citealt{Bobaljik:2002}).\footnote {The authors further assume that this
    additional condition has been dropped in a number of varieties which
    exhibit the subject condition only (i.e., where pronominal subjects
generally trigger a special form of agreement).} In all cases where the finite
verb cannot acquire a set of valued \isi{agreement} features, the resulting
non-inflected verb is repaired by the (post-syntactic) insertion of the default
inflection -\emph{s}.

While this kind of mixed approach successfully describes the basic facts
pertaining to the \gls{NSR}\is{Northern Subject Rule}, it still misses a couple of generalizations and
raises certain issues from the perspective of more recent developments in the
theory of syntax. First of all, it is based on the traditional assumption that
subject-verb \isi{agreement} is established in a spec-head relation and therefore
does not translate easily into more recent models where \isi{agreement} is taken to
result from the operation \isi{Agree}, that is, a configuration where a functional
head with unvalued Agr-features c-commands the \isi{agreement} controller (i.e., the
subject in the case at hand). Second, an approach that maintains that there is
a close connection between the \gls{NSR}\is{Northern Subject Rule} and multiple subject positions has to
assume that there are still two different subject positions in the present-day
\gls{NSR} varieties. However, it is far from clear whether this consequence is
supported by the facts. At least at first sight (abstracting away from the
\gls{NSR}), there does not seem to be a huge difference between Northern
dialects and Standard English with regard to the structural position of
pronominal and nominal subjects. In addition, the analysis raises the question
of why \isi{adverbs} intervening between the subject and the verb trigger an
adjacency effect in Northern English but not in Standard English. To account
for this empirical fact, \textcite{deHaas:2011} assumes that \isi{adverbs} have a
completely different syntax in the \gls{NSR}\is{Northern Subject Rule} varieties: According to her
analysis, \isi{adverbs} occupy specifiers of separate functional projections in the
Northern varieties (the heads of which block morphological merger of Agr and
the finite verb in T) while they are merely adjuncts in Standard English.
Again, this seems to be unwarranted. Moreover, as already pointed out by
\textcite{deHaas:2011} herself, the idea that default inflection is another
repair strategy (in addition to \emph{do}-support) that rescues an otherwise
uninflected verb by attaching -\emph{s} to it invites the question of why the
relevant varieties do not resort to \emph{do}-support instead (note that
\emph{do}-support is regularly used in other such contexts such as negation
etc. in the present-day \gls{NSR}\is{Northern Subject Rule} varieties).

In the literature dealing with the historical development of the \gls{NSR}\is{Northern Subject Rule},
basically three different lines of thinking can be discerned (in addition to
traditional accounts that typically invoke some form of analogical extension,
cf.\ e.g.\ \citealt{Sweet:1871} for the idea that the zero/vocalic plural ending
was generalized from the present subjunctive to the present indicative; see
\citealt{Pietsch:2005a,Pietsch:2005b} and \citealt{deHaas:2011} for comprehensive overviews and
critical discussion). First, it has been proposed that the \gls{NSR}\is{Northern Subject Rule} reflects
an \glsdesc{OE} (\glsunset{OE}\gls{OE}) pattern where \Fpl{} and \Spl{}
agreement endings are reduced to schwa in inversion contexts (\gls{OE}
agreement weakening, cf.\ \citealt{Rodeffer:1903}; see below for further details and
discussion). Second, several authors have put forward the claim that the
\gls{NSR} results from language contact with Celtic/Brythonic (cf.\ e.g.\ Klemola
2000), where similar differences between pronouns and DP subjects can be
observed (e.g., in \ili{Welsh}). Finally, the rise of the \gls{NSR}\is{Northern Subject Rule} is sometimes
attributed to dialect contact with southern varieties (cf.\ e.g.
\citealt{Pietsch:2005a,Pietsch:2005b}). It seems fair to conclude, however,
that no commonly accepted single explanation for the development of the
\gls{NSR} has hitherto been proposed. More recently, \textcite{deHaas:2011} and
\textcite{deHaasandvanKemenade:2015} (partially based on findings of
\citealt{Cole2014}) have put forward a multi-factorial approach to the rise of
the \gls{NSR}\is{Northern Subject Rule} which incorporates aspects of both language-internal and
language-external modes of explanation. They argue that the \gls{NSR}\is{Northern Subject Rule} developed
when learners reanalyzed extensive variation in the plural endings of the
present tense paradigm (-$\varnothing$/\emph{-e}, \emph{-s}, \emph{-th}, \emph{-n}) as
morphological marking of differential subject positions (i.e., a high position
for pronouns linked to \isi{agreement}, and a low position for other subjects giving
rise to non-agreement/default inflection). According to the authors, this
change was promoted by a conspiracy of factors, including \isi{agreement} weakening
in \gls{OE} (-$\varnothing$/\emph{-e} instead of \emph{-a\dh} with \Fpl{}, \Spl{} pronouns
in inversion contexts), language contact with \ili{Brythonic Celtic} (which
presumably had an \isi{agreement} system similar to present-day \ili{Welsh}, which makes a
systematic difference between pronominal and nominal subjects, see also
\citealt{Benskin2011}), language contact with Old Norse (which led to the
erosion of the \isi{agreement} morphology and presumably introduced the generalized
\emph{-s} marker), and the observation that pronominal subjects were
particularly frequent in the context of (present) subjunctive forms of the
verb, where the reduced ending -$\varnothing$/\emph{-e} had already become the norm (due
to loss of final \emph{-n}).\footnote{The connection between the subjunctive
    mood and pronominal subjects can be traced back to the fact that both tend
to be used in embedded clauses, cf.\ \citet{deHaas:2011}.} While the
scenario envisaged by \textcite{deHaas:2011} and
\textcite{deHaasandvanKemenade:2015} represents the most comprehensive
explanation of the historical development of the \gls{NSR}\is{Northern Subject Rule} so far, some
problems and open questions remain. In particular, the authors' decision to
focus solely on the plural part of the paradigm (cf.
\citealt[60]{deHaas:2011}) is somewhat unfortunate since it excludes the
possibility that a given morphological change is sensitive to properties of
the paradigm as a whole.  This applies to all other (diachronic) studies,
which usually ignore the first and second person singular.\footnote{An
    exception is Fernández-Cuesta's (\citeyear{Fernandez-Cuesta:2011})
    study of the \gls{NSR}\is{Northern Subject Rule} in first person singular contexts in Early
    Modern English. She shows that in 15th and 16th century wills
    from Yorkshire the adjacency constraint was still operative, especially
    in the period between 1450 and 1499. Further,
    \citeauthor{Fernandez-Cuesta:2011} cites evidence from the
    \emph{Linguistic Atlas of Early Middle English} (LAEME) which shows
    that the adjacency constraint was operative in Early \gls{ME}\il{Middle English} (although, it
    must be said that the numbers are very small). Overall, she comes to
    the conclusion that the emergence of the \emph{-s}/-\emph{eth} ending
    in the first person singular context should be seen as an extension of
the adjacency constraint of the \gls{NSR}\is{Northern Subject Rule}.}

In this paper, we attempt to narrow the empirical gap concerning the first and
second person singular by taking a look at the behavior of relevant forms in a
late Northern \gls{ME}\il{Middle English} text (the \emph{York (Corpus Christi)
Plays}) that is also affected by the \gls{NSR}\is{Northern Subject Rule}. In
addition, we will explore the synchronic and diachronic implications of an
alternative theoretical approach to the \gls{NSR} sketched in
\textcite{Roberts:2010}. \citeauthor{Roberts:2010} suggests a new analysis of
the \gls{NSR}\is{Northern Subject Rule} which is based on his notion of ``blank
generation'': He assumes that inflectional heads can enter the syntactic
derivation without content/phi-features.\is{φ-features} The \gls{NSR}\is{Northern Subject
Rule} is then attributed to the idea that subject pronouns incorporate into the
relevant Agr-head, endowing it with features that trigger the marked (zero)
agreement ending on the verb (while -\emph{s} signals the absence of \isi{agreement}
features). As a result, the verb can only appear in its inflected form (marked
by $\varnothing$) when it is string-adjacent to a weak/clitic subject
pronoun.\is{clitics}

The paper is structured as follows. In \Cref{sec:post-synt-appr} we
briefly highlight a set of morphological issues relating to the proper analysis
of the \gls{NSR}\is{Northern Subject Rule} (and singular forms, in particular) that are at least in part
only rarely discussed in theoretical approaches to the \gls{NSR}\is{Northern Subject Rule}. Section
\ref{sec:stages} deals with the historical development of the \gls{NSR}\is{Northern Subject Rule} and
shows that, although in \gls{OE} times there is unfortunately no direct textual
evidence for the rule (but see \citealt{Cole2014} on possible early traces of
the \gls{NSR}\is{Northern Subject Rule} in Northumbrian \gls{OE}), there are some indications that
\gls{OE} \isi{agreement weakening} in inversion patterns might have played a role in
its development. Further, we will take a closer look at (late) Northern ME,
focusing on the status of the \gls{NSR}\is{Northern Subject Rule} in the \emph{York (Corpus Christi)
Plays}, which exhibit an intermediate version of the \gls{NSR}\is{Northern Subject Rule} with a set of
special and interesting properties. \Cref{sec:towards-an-analysis}
presents an analysis of the \gls{NSR}\is{Northern Subject Rule} based on
\textcite{Roberts:2010} \nocite{Roberts:2010} in terms of ``blank generation''.
\Cref{sec:historical-origin} brings together our theoretical claims and
diachronic observations and shows that our analysis can shed new light on both
the inner mechanics of the \gls{NSR}\is{Northern Subject Rule} and its
historical development.  \Cref{sec:10.summary} provides a brief concluding
summary.

\section{Unfinished business: Morphology problems}
\label{sec:post-synt-appr}

\largerpage[-1]
The general morphological problem concerning the differences between Standard
English and the northern varieties is what \citet{Pietsch:2005b} refers to as
the ``markedness paradox'': while \emph{-s} appears to be the marked inflection
in Standard English, the situation in the \gls{NSR}\is{Northern Subject Rule}
dialects is more complex,
since with full DPs and non-adjacent subjects the \emph{-s} affix seems to
function as a default marker, whereas with subject pronouns adjacent to the verb
the \emph{-s} ending seems to mark the feature combination [-speaker, -pl] (at
least in the conservative \gls{NSR}\is{Northern Subject Rule} varieties that have retained the original
\Ssg{} pronoun \emph{thou}, compare the somewhat idealized system in \Cref{tab:key:10.1}
above). The ``markedness paradox'' presents certain problems for morphological
analysis which are rarely (if at all) addressed in the existing literature on
the \gls{NSR}\is{Northern Subject Rule}. In particular, it appears that the widespread assumption that
\emph{-s} is an underspecified default marker (possibly signalling tense and/or
mood, cf.\ e.g.\
\citealt{Henry:1995,Pietsch:2005b,deHaas:2011,deHaasandvanKemenade:2015}) does not suffice to capture its distribution in the above
paradigm: If the \emph{s}-marker represents the elsewhere\is{Elsewhere Condition} case, then the
zero marker must be specified for a certain combination of values for the
features [person] and [number]. However, assuming standard (binary) feature
systems such as [$\pm$speaker], [$\pm$hearer]/[$\pm$author in speech event],
[$\pm$participant in speech event] for [person] and [$\pm$plural] for
[number],\footnote{And excluding further options such as accidental homophony,
    or the possibility of disjunctive feature specifications  (e.g., [+plural OR
    \Fsg]), which we consider to be less attractive theoretically. However, see
    \textcite{AdgerSmith2010} for an account of variable \isi{agreement} marking in a
    present-day dialect based on the idea that a particular surface form may be
linked to different feature specifications.} it turns out that it does not seem
to be possible to describe the distribution of the zero marker in terms of a
specific set of feature values: As the zero marker occurs in the singular
(\Fsg{}) as well as in the plural, and with all three persons, it does not
signal any person or number distinctions (compare \Cref{tab:key:10.1} above). So
we seem to face a (impossible) situation where a paradigm is made up by two
seemingly equally underspecified markers. Note that this dilemma cannot be
resolved by treating \emph{s}-marking with nominal and non-adjacent subjects
separately (e.g.\ by assuming that verbs with nominal/non-adjacent subjects fail
to acquire a set of \isi{agreement} features in the syntax), at least as long as we
want to maintain the idea that there is only a single \emph{s}-affix in the
\gls{NSR} varieties. Such an approach merely restates the ``markedness
paradox'': Again, it would seem that while \emph{-s} is the unmarked/default
marker with nominal/non-adjacent subjects, it appears to be more specified than
the zero ending in cases where a pronominal subject is adjacent to the verb
(cf.\ the second column in \Cref{tab:key:10.1}). Without additional assumptions,
this state of affairs also seems to be incompatible with the proposal of
\textcite{deHaas:2011} and \textcite{deHaasandvanKemenade:2015} that in the
\gls{NSR} dialects, the relevant inflectional markers are not linked to
specific phi-feature values, but are used instead to realize a minimal binary
distinction between ``real'' subject-verb \isi{agreement} (signaled by $\varnothing$) and default
inflection (via insertion of \emph{-s}).

In what follows, we will outline a new approach to the distribution of markers
in the `classic' \gls{NSR}\is{Northern Subject Rule} varieties (cf.\ \Cref{tab:key:10.1}) that maintains the basic
insight that the relevant dialects have only a single \emph{-s} affix with a
uniform specification. More precisely, we agree with previous work that
\emph{-s} is a completely underspecified default marker, which represents the
elsewhere case. We take it that the zero marker (\emph{sing-$\varnothing$}), on the
other hand, signals the presence of positive values for person or number
features.\footnote{Alternatively, we might assume that the \emph{-s} ending
    marks the absence of positive specifications for person or number. While this
    analysis seems to be a technical possibility, it fails to capture the
    elsewhere/default character of \emph{-s} is the relevant varieties
(e.g., its use under non-adjacency etc.).} The resulting (binary) inventory of
agreement markers can be described as follows:

\ea
\ea \mbox{[+phi]} $\leftrightarrow$ $\varnothing$
\ex elsewhere\is{Elsewhere Condition} $\leftrightarrow$ /-z/
\z
\z

Thus, if the process of Vocabulary Insertion detects a positive phi-feature
value for person or number (which is only possible in connection with adjacent
subject pronouns, see \Cref{sub:10.4.2} for a syntactic analysis), the verbal
agreement morpheme will be realized by the zero affix, while in all other cases
the default marker \emph{-s} is inserted.

As concerns the presence of the \emph{-s} affix with \Tsg{} pronouns, we follow
the common idea that \Tsg{} forms are characterized by the absence of
(positive) specifications for [person] and [number] (cf.\ e.g.\
\citealt{Benveniste:1966}, \citealt{Halle:1997}, \citealt{Noyer:1997},
\citealt{Harleyritter:2002}). As a result, the elsewhere\is{Elsewhere Condition} marker \emph{-s} is
inserted in all \Tsg{} contexts.

But note that this morphological analysis faces a similar problem as previous
approaches in that it apparently fails to account for the use of the
\emph{s}-affix in the context of \Ssg{} (note that (5) should lead us to expect
that the zero marker is used in \Ssg{} contexts in connection with
\emph{thou}). To solve this puzzle, we would like to propose that the relevant
agreement morphemes are subject to the following Impoverishment\is{impoverishment} rule that
operates on the output of the syntactic derivation and reduces the feature
content of \isi{agreement} morphemes (on T) under adjacency with subject pronouns
prior to the insertion of Vocabulary Items (NOM = nominative\is{nominative case}):\footnote{See
    \cite{HalleMarantz1993}, \cite{Halle:1997}, and \cite{Noyer:1997} on the
    workings of Impoverishment\is{impoverishment} rules, which typically lead to an extension of the
contexts where underspecified markers can be used.}

\ea
\mbox{[+hearer]} {$\rightarrow$ $\varnothing$} / \_\_\ pronoun\tief{[NOM]}
\z

As a result of (6), the feature [+hearer] is deleted when the finite verb is adjacent to a subject pronoun (i.e., part of the same phonological phrase/word). This serves to block insertion of the zero marker in the context of \Ssg{} due to the absence of positively valued feature values, leading to systematic syncretism of \Ssg{} and \Tsg{}. In all other contexts, a positively valued feature remains ([+speaker] with \Fsg{}, [+pl] with all plural forms), which triggers insertion of the zero marker.

This analysis not only accounts for the basic facts in the
\gls{NSR}\is{Northern Subject Rule} dialects but also makes available a new
perspective on \Tsg{} \emph{-s} in the present tense of Standard English.
Similar to the \gls{NSR}\is{Northern Subject Rule} dialects, we might assume
that this affix is not explicitly specified for [person] and [number]; rather,
the distribution of \emph{-s} and the zero form is sensitive to the
presence/absence of positive feature values for [person] or [number] in the
following way: The zero marker is inserted in all cases where a positive value
for person or number can be detected (that is, in all contexts apart from
\Tsg{}); in the remaining context, \emph{-s} is used (see Haeberli
\citeyear{Haeberli:2002b}, \citealt{Roberts:2010} for a related analysis).

\section{The historical development of the NSR}
\label{sec:stages}

\subsection{Historical stages in the rise of the NSR}

In this section, we take a look at the historical development of the
\gls{NSR}\is{Northern Subject Rule}.  Before we deal with possible \gls{OE}
origins of the \gls{NSR}\is{Northern Subject Rule} in some more detail, we
first outline its historical development from \gls{OE} via \gls{ME}\il{Middle
English} to ModE (basically following \citealt{Pietsch:2005a,Pietsch:2005b},
\citealt{deHaas:2011}, and \citealt{Cole2014}).

It is a well-known fact that during the transition from \gls{OE} to \gls{ME}
nominal and verbal affixes became drastically reduced. The loss of inflections
is particularly apparent in northern varieties. As shown by \cite{Berndt1956}
and \cite{Cole2014}, the erosion of the inflectional system first led to
variation between several competing \isi{agreement} markers, as evidenced in the
\emph{Lindisfarne Gospels}, where \Tsg{} and \Fpl{}/\Spl{}/\Tpl{} subjects may
be cross-referenced on the verb variably by \emph{-es}, \emph{-as},
\emph{-e\dh}, or \emph{-a\dh}. The default ending for \Ssg{} is \emph{-st} in
\gls{OE}; variants include \emph{-est}, \emph{-as}. In early Northern ME
(\gls{NME}), the \gls{OE} \Ssg{}  \emph{-est}, \Tsg{} \emph{-e/{\dh}e} and plural
forms \emph{-a/{\dh}e}/\emph{-as} had already fallen together in the form
\emph{-e(s)}, which could be interpreted as an underspecified inflectional
marker. Further, after the loss of vowels in the final syllable, Northern ME
started to exhibit an opposition between \Fsg{} -$\varnothing$\ and all other contexts
(\emph{-s}). At this point, new zero markers were introduced in the Northern ME
varieties, eventually giving rise to the \gls{NSR}. First, the zero marker was
introduced in plural contexts where a finite lexical verb was adjacent to a
subject pronoun, initially with \Fpl/\Spl{} and somewhat later with \Tpl{}. In
a further step, the \emph{-s} affix was extended to \Fsg{} pronouns
(non-adjacent to the verb), presumably as a result of analogical pressure
(\citealt{Holmqvist:1922} assumes that the inherited null \Fsg{} ending came to
be perceived as being subject to the same mechanism that governed the
alternation between \emph{-s} and -$\varnothing$\ with plural forms).  Finally, again
probably via processes of analogy, the \gls{NSR}\is{Northern Subject Rule} was extended to forms of
\emph{be}, including \emph{was/were}.\footnote{Apparently, the use of \emph{is}
    and \emph{was} in the plural was never as categorical as the use of
    \emph{-s} with lexical verbs (cf.\ e.g.\ \citealt{Montgomery:1994}). However, it
    seems that present-day dialects exhibit a different tendency, in that they
    preserve the \gls{NSR}\is{Northern Subject Rule} more strongly with forms of \emph{be}
    (\citealt[12--13]{Pietsch:2005b}; but see \citealt{Buchstaller_etal:2013} for
different findings).} In some Northern dialects, \Ssg{} \emph{thou} was
replaced with \emph{you} (the original plural form) in the \gls{EModE} period,
which further broadened the scope of the \gls{NSR}\is{Northern Subject Rule}.\footnote{Concerning the
    empirical gap in studies of the \gls{NSR}\is{Northern Subject Rule}, \textcite[46]{Pietsch:2005b}
    notes that the \emph{LALME} \parencite{McIntosh2013} ``[...] does not give
    detailed accounts or statistics regarding [...] any information about the
    first and second persons in the documents studied. The only information
    given per document is whether -s forms were used regularly or rarely.''}
    Somewhat idealised, these stages of development are schematised and
    summarised in \Cref{tab:key:10.2}.\glsunset{NME}

\begin{table}[h]
    \centering
    \renewcommand{\arraystretch}{1.1}
    \begin{smaller}
        \begin{tabularx}{\textwidth}{lllllll}
            \lsptoprule
    & \gls{OE} & Northumbrian \gls{OE} & \gls{NME} I & \gls{NME} II & \gls{NME} III/NSR\\
    \midrule
    \Fsg{} & sing-e & sing-e/-$\varnothing$ & sing-e/-$\varnothing$ & sing-$\varnothing$ & sing-$\varnothing$/-s & I sing-$\varnothing$\\
    \Ssg{} & sing-es(t) & sing-es/-as & sing-es & sing-s & sing-s & thou sing-s\\
    \Tsg{} & sing-e\dh & sing-es/-as/-e\dh/-a\dh & sing-es & sing-s & sing-s & he sing-s\\
    \Fpl{} & sing-a\dh & sing-es/-as/-e\dh/-a\dh & sing-es & sing-s & sing-s  & we sing-$\varnothing$\\
    \Spl{} & sing-a\dh & sing-es/-as/-e\dh/-a\dh & sing-es & sing-s & sing-s  & you sing-$\varnothing$\\
    \Tpl{} & sing-a\dh & sing-es/-as/-e\dh/-a\dh & sing-es & sing-s & sing-s  & they sing-$\varnothing$\\
    \lspbottomrule
\end{tabularx}

\caption{Historical development of verbal inflection, Northern
varieties}\label{tab:key:10.2}
\end{smaller}
\end{table}

\subsection{Old English}
\label{sec:oe}

\citet{Berndt1956} makes the observation that a group of late Northumbrian
texts, including the \emph{Lindisfarne Gospels}, the \emph{Rushworth Gloss}, and
the \emph{Durham Ritual}, which are all dated to the mid-10th century, are the
first \gls{OE} texts showing the \emph{-s} form variably with the
\emph{-\dh}-ending. Berndt assumes that the triggering factor for the
occurrence of this form are subject pronouns which could take over the function
of person marking. What is implied in his comment is the special role subject
pronouns play as opposed to full DP subjects, and his observations and
assumptions hence foreshadow part of the \gls{NSR}\is{Northern Subject Rule}. Berndt's finding is
corroborated by \textcite{Cole2014}, the most comprehensive study of the
earliest (Northumbrian \gls{OE}) stages of the \gls{NSR}\is{Northern Subject Rule} so far. Cole provides
an in-depth textual and linguistic analysis of the \emph{Lindisfarne Gospels},
focusing on the \isi{agreement} system and early traces of the \gls{NSR}\is{Northern Subject Rule}, in
particular. Using statistical methods, she is able to identify a set of factors
that govern the variation between the various \isi{agreement} endings. One of her
most intriguing results is the observation that adjacency between the finite
verb and a (plural) subject pronoun (usually cases of inversion) clearly
favours \emph{-s} over \emph{-\dh}. For the 1/2\Pl{} subject pronouns \emph{we}
and \emph{ge} she finds that they occur 57\% and 59\% of the time with an
\emph{-s} ending on the finite lexical verb \parencite[112]{Cole2014}. Two
examples are given here (cf.\ \citealt[93]{Cole2014}):

\ea
\ea
\label{exwege1}
\gll þæt ue gesegun \textbf{we} \textbf{getrymes}.\\
that we seen we testify\\
\glt`What we have seen we testify.'\\
(JnGl(Li) 3.11)
\ex
\gll huu minum wordum \textbf{gelefes} \textbf{ge}.\\
how my words believe you.pl\\
\glt `How will you believe my words?'\\
(JnGl(Li) 5.47)
\z
\z

Thus, at first sight it seems that in late Northumbrian \gls{OE}, there is
already an early form of the \gls{NSR}\is{Northern Subject Rule} that differs from its later installments
in that the (innovative) \emph{s}-ending plays the role later assumed by the
zero/vocalic endings. However, this conclusion is misleading, since the
relevant markers have a different status in their respective paradigms. While
zero represents the marked inflection in the \gls{NSR}\is{Northern Subject Rule} varieties, \emph{-s} is
clearly the elsewhere\is{Elsewhere Condition} case in the Northumbrian \isi{agreement} system (cf.\
\Cref{tab:key:10.2} above). At least from a morphological point of view, the
Northumbrian facts are more similar to southern \gls{OE} \isi{agreement weakening},
in that a less distinctive \isi{agreement} marker is used in connection with adjacent
pronominal subjects.\footnote{This can perhaps be analyzed as an instance of
    featural haplology \citep{Nevins:2012}, where the verb's phi-set is deleted
    in cases where the verb is adjacent to another pure phi-set, i.e., a
subject pronoun.}  Recall that (late) southern \gls{OE} exhibits an
agreement alternation that is sensitive to subject type and the position of
the finite verb (\citealt[15]{Jespersenpart4:1949};
\citealt[42]{QuirkWrenn:1955}; \citealt[296]{Campbell:1959};
\citealt{vanGelderen:2000}). In cases where the \Fpl/\Spl{} subject
pronouns \emph{we} or \emph{ge} directly follow the inverted finite verb,
the regular \isi{agreement} endings (present tense indicative/subjunctive
\emph{-að}, \emph{-on},\emph{-en}) are replaced by schwa:\footnote{Similar
    observations hold for early \gls{OHG} (\Fpl{}), cf.\
    \citet[262]{BrauneReiffenstein:2004}, and present-day \ili{Dutch}
    \citep[193]{AckNel:2004}:

    \begin{exe}
        \exi{(i)}
        \gll  Jij loop-t  dagelijks met een hondje over straat.\\
        you walk-\Ssg{} daily   with a  doggy over street\\
        \glt

        \exi{(ii)}
        \gll Dagelijks loop-{$\varnothing$} jij  met een hondje over straat.\\
        daily   walk  you.\Pl{} with a  doggy over street\\
\end{exe}}

\ea
\ea\label{exwege2}
\gll Ne \textbf{sceole} \textbf{ge} swa softe sinc gegangen.\\
\Neg{} must you so easily treasure obtain\\
\glt `You must not obtain treasure so easily.'\\
(Battle of Maldon, p. 244, 1.59)
\ex
\gll Hwæt \textbf{secge} \textbf{we} be  þæm coc?\\
what say we about the cook\\
\glt `What do we say about the cook?'\\
(AElfric's Colloquy on the Occupations, p. 188, 1.68)
\z
\z

As noted above, \citet{Rodeffer:1903} explicitly assumes that these syncopated
forms were the direct source of the later affixless forms in the \gls{NSR}
varieties. Although there is no direct equivalent of the \gls{NSR}\is{Northern Subject Rule} in \gls{OE},
the finding that the reduced \emph{-e} affix occurs in inversion contexts might
have contributed to the development of the \gls{NSR}\is{Northern Subject Rule} (see
\Cref{sec:historical-origin} for further discussion).

In \Cref{sec:intro} we have noted that in the studies hitherto presented, there
is an empirical gap concerning the \Fsg{} and \Ssg{} forms.  Since we are
interested in the development of the full paradigm, we are going to include
these two forms in the empirical study that we will present in the following
section.

\subsection{Middle English}
\label{sec:me}

In a recent study of the \gls{NSR}\is{Northern Subject Rule} in ME, \cite{deHaasandvanKemenade:2015}
investigate the \isi{agreement} properties of full verbs, focussing on present tense
indicative plural forms. The study is based on 36 texts dated between 1150 and
1350 taken from the \nocite{LAEME} LAEME corpus, as well as the sample of the
\emph{Northern Prose Rule of St. Benet} from the PPCME2 and a digitized version
of a Lancaster romance. They identify 15 texts which display variation between
\emph{-$\varnothing$/-e/-n} and \emph{-s/-th} endings and show the strongest effects for
the adjacency and type-of-subject condition in their corpus. Further, they
locate a core area of the \gls{NSR}\is{Northern Subject Rule} in Yorkshire and note that in texts from
more peripheral areas the adjacency condition is often weaker or even absent.
They interpret this finding as evidence for an analysis that is based on
different subject positions, as mentioned above in \Cref{sec:post-synt-appr}. A
short glance at the sample of Richard Rolle's Epistles in the PPCME2
(\citealt{KroTay2000})\footnote{Richard Rolle of Hampole (ca.\ 1290--1349),
    Yorkshire, English hermit and mystic, was one of the first religious writers to
    use the vernacular. He was very well known at his time, and his writings were
widely read during the 14th and 15th century.} confirms that both the adjacency
and the type-of-subject condition seem to be quite well established:

\ea
\ea
\gll Some þe devell deceyves þurgh vayne glory, þat es ydil joy: when any has pryde and delyte in þamself of þe penance þat \textbf{þai} \textbf{suffer}, of gode dedes þat \textbf{þai} \textbf{do}. of any vertu 	þat \textbf{þai} \textbf{have}; es glad when \textbf{men} \textbf{loves} þam, sari when \textbf{men} \textbf{lackes} þam, \textbf{haves} \textbf{envy} to þam þat es spokyn mare gode of þan of þam;\\
some the devil deceives through vain glory that is idle joy when any has pride and delight in themselves of the penance that they suffer of good deeds that they do of any virtue that they have is glad when men loves them sorry when men lacks them haves envy to them that is spoken more good of than of them\\
\glt (ROLLEP,86.368)
\ex
\gll He says þat `he lufes þam þat lufes hym, and \textbf{þai} \textbf{þat} \textbf{arely} \textbf{wakes} til hym sal fynde him'. \\
he says that he loves them that loves him and they that early wakes till him shall find him\\
\glt (ROLLEP,76.212)
\z
\z

As the Yorkshire area seems to have played an important role in the historical
development of the \gls{NSR}\is{Northern Subject Rule}, it might be worthwhile to take a closer look at
texts from that region to complement \citegen{deHaasandvanKemenade:2015}
findings on plural forms with relevant data from the singular part of the
agreement paradigm (with a focus on \Fsg{} and \Ssg{}; recall that \Tsg{}
usually does not take part in the \gls{NSR}\is{Northern Subject Rule}). Under the assumption that first
and second singular pronouns are likely to occur in dialogues, we decided to
survey the \emph{York Plays}, a \gls{ME}\il{Middle English} cycle of 47 mystery plays dated between the
mid-fourteenth century and 1463--1477, when the manuscript (MS. Add. 35290,
British Library, London) was copied.\footnote{For our study we tagged the
    collection of plays which are part of \emph{The corpus of Middle English Prose
    and Verse}. In addition, we conducted a full text analysis of all plays and
looked through them manually, see references below.}

As has been repeatedly pointed out in the literature (cf.\ e.g.\
\citealt{Smith:1885,Cawley:1952,Beadle:1982,BurrowTurville-Petre:2005,Johnston:2011}),
the \emph{York Plays} (even if they are the work of different authors) display
an identifiably northern variety interspersed with some southern/Midlands
influences (in particular concerning loanwords, spellings including
combinations of southern spelling and a northern rhyme etc.).\footnote{It is
    commonly assumed that dialectal features of south-east Midland and London
    varieties were introduced when the \emph{York Plays} were copied in the
mid/late 15th century, cf.\ e.g.\ \citet{BeadleKing:1984}.} In what follows, we
will report our findings on properties of the \isi{agreement} system as found in the
\emph{York Plays}, focusing on \Ssg{} (and \Fsg{}) forms, and the distribution
of the \gls{NSR}\is{Northern Subject Rule}.
%
As already briefly mentioned above, the make-up of the \isi{agreement} paradigm and
the scope of the \gls{NSR}\is{Northern Subject Rule} depend in part on the inventory of pronominal forms.
The pronominal system found in the individual plays is remarkably uniform, with
variation being confined to differences in spelling. \Cref{tab:key:10.3} gives
an overview of the relevant subject forms (cf.\ \citealt[ixxii]{Smith:1885};
\citealt[272]{BurrowTurville-Petre:2005}, \citealt{Johnston:2011}):

\begin{table}[ht]
    \centering
    \renewcommand{\arraystretch}{1.1}
    \begin{small}
        \begin{tabular}{ll}
            \lsptoprule
    & Subject pronouns\\
    \midrule
            \Fsg{} & I\\
            \Ssg{} & þou, þow(e), thou, thow\\
            \Tsg{} & he (masc.), scho (fem.), it (neut.)\\
            \Fpl{} & we\\
            \Spl{} & ye, ge\\
            \Tpl{} & þei, þai, þey, þay\\
            \lspbottomrule
        \end{tabular}

        \caption{Subject pronouns as found in the \emph{York
        Plays}}\label{tab:key:10.3}

    \end{small}

\end{table}

As can be gathered from \Cref{tab:key:10.3}, the pronominal system of the
\emph{York Plays} features the inherited \Ssg{} subject pronoun \emph{thou} in
combination with the \Tpl{} form \emph{they} borrowed from Old Norse. We thus
expect full verbs to take \emph{-s} in \Ssg{} contexts (in the present tense
indicative).

The system of verbal \isi{agreement} endings is characterized by a higher
amount of linguistic variation, although it should be pointed out that the
inventory of endings is quite limited.\footnote{It is very likely that the
linguistic variation found in the \emph{York Plays} is at least partially the
result of the fact that the plays were composed by different authors. However,
an in-depth investigation of the impact of authorship on the type of
\gls{NSR}\is{Northern Subject Rule} found in the individual plays is well
beyond the scope of the present paper.} In the present tense, the only
significant residue of the formerly more elaborate \gls{OE}/ME \isi{agreement}
paradigm is \emph{-s}, which appears in a variety of different surface
manifestations dependent on factors such as spelling preferences and phonetic
context (e.g.~\emph{-s}, \emph{-is}, \emph{-es}, \emph{-ys} etc.).\footnote{In
addition, there are few \Tsg{} forms ending in \emph{-th} such as \emph{haith}
`have-\Tsg{}', which clearly reflect Midlands/southern influence.} In addition
to the variants of the \emph{-s}-marker, present tense verbs appear with zero
inflection, or \emph{‑e}. However, there are reasons to believe (e.g.\ evidence
from rhymes) that the latter is usually not pronounced, representing the
residue of a former contrast which by the time the \emph{York Plays} were
composed was confined to the writing (cf.\ e.g.\ Johnston 2011). This leaves us
with a basically binary contrast between variants of \emph{-s} and variants of
the zero marker (-$\varnothing$, \emph{-e}). The situation is made more complex
by the workings of the \gls{NSR}\is{Northern Subject Rule} (which widens the
scope of the \emph{-s}-marker) and the fact that there are cases where the
\emph{-s}-marker and the zero marker seem to vary freely. \Cref{tab:key:10.4}
gives a rough overview of the distribution of markers in the present tense (for
the time abstracting away from variants of \emph{-s} and -$\varnothing$). Each
cell of the paradigm contains the dominant (i.e.\ most frequent) marker, while
competing variants are added in parentheses.\footnote{Table 4 is based on the
descriptions in \cite[lxxii]{Smith:1885}, \cite[272]{BurrowTurville-Petre:2005},
and \cite{Johnston:2011}, which we have cross-checked with our own corpus-based
studies.}

\begin{table}[ht]
    \centering
    \renewcommand{\arraystretch}{1.1}
    \begin{small}
        \begin{tabularx}{\textwidth}{lXXX}
            \lsptoprule
    & pronominal subjects & pronominal subjects & nominal \\
    & (adjacent to V) & (non-adjacent to V) & subjects\\
    \midrule
            \Fsg{} & -$\varnothing$ & -$\varnothing$ & -\\
            \Ssg{} & -s (-$\varnothing$) & -s (-$\varnothing$) & -\\
            \Tsg{} & -s & -s & -s \\
            \Fpl{} & -$\varnothing$ & -s (-$\varnothing$)  & -\\
            \Spl{} & -$\varnothing$ & -s (-$\varnothing$) & -\\
            \Tpl{} & -$\varnothing$ & -s (-$\varnothing$) & -s \\
            \lspbottomrule
        \end{tabularx}
        \caption{Verbal inflection in the \emph{York Plays} (present tense
        indicative)}\label{tab:key:10.4}
    \end{small}
\end{table}

As can be seen from \Cref{tab:key:10.4}, the \isi{agreement} system found in the \emph{York
Plays} exhibits some special properties that possibly shed some light on the
historical development of the \gls{NSR}\is{Northern Subject Rule}. First of all, the \gls{NSR}\is{Northern Subject Rule} seems to
be restricted to the plural part of the paradigm, whereas the realization of
singular forms is not influenced by the position or (in the case of \Tsg{})
type of subject. According to the standard view of the historical development
of the \gls{NSR}\is{Northern Subject Rule}, this seems to be indicative of an
early stage of the \gls{NSR}, where the \isi{agreement} alternation had not yet
spread to singular forms (see~\Cref{sec:stages}).\footnote{But note that there
    are few examples where \emph{-s} seems to appear with a \Fsg{} subject
    under non-adjacency, as shown in (11).} Second, it appears that while
    non-adjacency may license \emph{-s}-inflection in connection with
    pronominal subjects, zero-marked forms or forms marked with \emph{-e} do
    also occasionally turn up in this context.\footnote{The fact that the zero
        ending co-varies with \emph{-s} under non-adjacency might be taken to
        represent an early stage of a development in which the type of subject
        constraint gradually gains more importance, eventually leading to zero
    marking of pronominal subjects independently of their position relative to
the verb (contrasting with \emph{-s}-marking of nominal subjects).} The
variation between \emph{-s} and zero in connection with pronominal subjects
non-adjacent to the verb seems to suggest a tripartite agreement system with a
distinction between pronominal subjects adjacent to the verb (which invariably
trigger zero marking), pronominal subjects non-adjacent to the verb (which
trigger either \emph{-s} or -$\varnothing$), and nominal subjects (which always
trigger \emph{-s}). In what follows, we will first add more data and examples,
including some quantitative findings resulting from our corpus study, before we
address the question of how the \isi{agreement} system should be analysed in
\Cref{sec:towards-an-analysis}. As noted above, we will focus on forms which
have been neglected in previous work on the \gls{NSR}\is{Northern Subject
Rule}, i.e.\ \Ssg{} in particular.\largerpage[-2]

In contrast to later \gls{NSR}-varieties, \emph{-s} is only rarely found with
\Fsg{} forms, which strongly tend to exhibit \emph{-e}/zero marking in the
present tense independently of their position (adjacency/non-adjacency)
relative to the subject. This is shown in (\ref{york1}). However, there are few
examples where \gls{NSR}\is{Northern Subject Rule} effects do show up in
connection with \Fsg{} subjects (the majority of which in connection with
`have'), as in (\ref{york2}):\footnote{Examples taken from
    \citegen{Davidson:2011} edition of the \emph{York Plays} are referenced in
the format ``play number, line''. All other examples are taken from the edition
by \textcite{Beadle:1982}.}

\ea
\label{york1}
\ea \gll For thowe art my Savyour, \textbf{I} \textbf{say},\\
for you are my savour I say \\
\glt `For you are my saviour, I say.'\\
(York Plays, 17, 404)
\ex \gll A, lorde, to the \textbf{I} \textbf{love} and \textbf{lowte}.\\
a lord  to thee I love and bow\\
\glt `Ah, Lord, I love and venerate you.'\\
(York Plays, 9, 189)
\ex \gll And so I schall fulfille / That I before \textbf{haue} highte.\\
and so I shall fulfil {} that I before have promised \\
\glt `And so I shall fulfil what I have promised before.'\\
(York Plays, 37, 396)
\z
\z

\ea\label{york2}
\ea
\gll For \textbf{I} am lame as men may se / And \textbf{has} ben lang. \\
Because I am crippled as man may see and have been long.\\
\glt `Because I am crippled as one may see and have been long so.'\\
(York Plays, 25, 369)\\
\ex \gll A, sir, a blynde man am \textbf{I} / And ay \textbf{has} bene of tendyr yoere Sen
I was borne.\\
A sir, a blind man am I and always has been of tender year since I was borne.\\
\glt `Ah, Sir, I am a blind man and always have been of tender year since I was born.'\\
(York Plays, 25, 297)\\
\ex \gll \textbf{I} here the lorde and \textbf{seys} the nought.\\
I hear thee lord and sees thee not\\
\glt `I hear the Lord and do not see you.'\\
(York Plays, 5, 139)\\
\z
\z

This finding corroborates the findings of \textcite{Fernandez-Cuesta:2011}. In
her study of the \emph{LAEME} data only two non-adjacent \Fsg{} verbs occur
with an -\emph{s} ending (of six unambiguous cases of non-adjacency).

We will now take a closer look at \Ssg{} forms, which present a set of interesting properties that are directly relevant for the analysis of the \isi{agreement} system and the type of \gls{NSR}\is{Northern Subject Rule} found in the \emph{York Plays}. The following discussion is based on a data set of 852 clauses with \Ssg{} subjects that we extracted from \cite{Davidson:2011}'s edition of the \emph{York Plays}.
With \Ssg{} subjects (variants of the ‘old' form \emph{thou}), \emph{-s} is the
dominant ending in the present tense (indicative), independently of whether the
subject is adjacent to the verb or not (in general, non-adjacency between
subject pronoun and finite verb is much less frequently found in the corpus
than adjacency). In other words, there are no clear \gls{NSR}\is{Northern
    Subject Rule} effects in the context of \Ssg{}. Alternative forms of the
    \emph{-s} inflection include markers extended by \emph{-t(e)} (particularly
    frequent with forms of ‘have’, e.g.\ \emph{hast(e)}, see
    \Cref{table1-2ndps-lexical-verbs} below),\footnote{Apart from verbs that
        are made up by only a single CV-pattern (e.g.\ \emph{se} `see'), we
        have counted here all verbs ending in \emph{-e}, including forms such
        as \emph{come}, \emph{take} etc. There are seven instances (all under
        adjacency of subject and verb) where \emph{-e} attaches to the
        \emph{s}-ending as in (i).

    \begin{exe}
        \exi{(i)} And sen thou \textbf{dose} not as I thee tell,
        \glt (York Plays, 22, 169).
    \end{exe}
    These are counted as instances of \emph{-s}. In addition, there are four
    examples where the enlarged ending \emph{-st} combines with \emph{-e} (e.g.
    \emph{saiste} `say-\Ssg{}', 30, 477). Modals such as `can', `must', `shall'
    always appear without \emph{-s} (due to their origin as preterite-presents) and are
    therefore not considered here (there are a few instances of \emph{moste}
‘must-\Ssg{}', though).} and by pre-consonantic vowels (\emph{-es},
\emph{-is}, \emph{-ys}). See \Cref{table1-2ndps-lexical-verbs} below for the quantitative
distribution of the \Ssg{} endings with lexical verbs and (12)--(13) for a
selection of relevant examples.

The subject and the finite verb are adjacent:
\ea
\ea \gll And \textbf{thou} \textbf{sais} \textbf{thou} \textbf{hast} insight\\
and thou says thou hast insight\\
\glt (York Plays, 20, 99)
\ex \gll \textbf{Heris} \textbf{thou} not what I saie thee?\\
hers thou not what I say thee\\
\glt ‘Don't you hear what I say to you?'\\
(York Plays, 31, 317)
\ex \gll \textbf{Thou} \textbf{makist} her herte full sare\\
thou makest her heart full sore\\
\glt ‘You make her heart fully sore.'\\
(York Plays, 13, 251)
\ex \gll Fro thens \textbf{come} \textbf{thou}, Lorde, as I gesse\\
from thence come thou lorde as I guess\\
\glt ‘From thence thou come, Lord, as I guess.'\\
(York Plays, 21, 114)
\z
\z

The subject and the finite verb are not adjacent:

\ea
\ea \gll For tho that \textbf{thou} to wittenesse \textbf{drawes} / Full even agaynste thee will begynne.\\
for though that thou to witness draws {} full even against thee will begin\\
\glt `For those whom you cite as witnesses are equally against you.'\\
\glt (York Plays, 37, 279)
%\item And all for synne that thou done hast,
%\glt 'And all for the sin that you have done.'
%\glt (York Plays, 25,478)
\ex \gll \textbf{Thou} arte combered in curstnesse / and \textbf{caris} to this coste.\\
thou are troubled in cursedness {} and cares to this cost\\
\glt `You are troubled with sin and dread this price.'\\
\glt (York Plays, 26, 171)
\z
\z

At first sight, it appears that the use of reduced markers (zero or \emph{-e})
is also widespread. However, upon closer inspection it turns out that the vast
majority of reduced endings represent subjunctive or imperative/optative forms
(as illustrated in (\ref{exsopt})). The latter are conspicuously frequent,
which can be attributed to the religious character of the plays, which include
many prayers, or passages where the characters directly address Jesus or God.
If subjunctive (and adhortative/optative) forms are filtered out, it appears
that around 80\% of \Ssg{} lexical verbs carry some form of the
\emph{s}-inflection in the present tense indicative; see
\Cref{table1-2ndps-lexical-verbs} for a summary of our quantitative
findings.\footnote{In quite a number of cases it is hard to tell whether we
    are dealing with a subjunctive or indicative form. This seems to support the
    hypothesis (cf.\ e.g.\ Sweet 1871) that the spread of the reduced ending involved
    a \isi{reanalysis} of originally subjunctive forms as indicative (most likely in
subordinate clauses).} Furthermore, it turns out that of the 31 cases with
\emph{-e} 16 are forms of the preterite-present verb \emph{witen} `know' that
usually does not inflect for \Ssg{}. That is, the share of \emph{s}-marked
forms is probably even larger than 80\%.

\ea
\label{exsopt}
\ea \gll Luk nowe that \textbf{thou} \textbf{wirke} noght wrang ...\\
look now that thou act not wrong \\
\glt `Look now, that you do not wrong.'\\
(York Plays, s439)
\ex \gll Iff I haue fastid oute of skill, \textbf{Wytte} \textbf{thou} me hungris not so ill\\
if I have fasted out of skill know you me hungers not so ill\\
\glt `If I have fasted unreasonably, you should know that I'm not so hungry.'\\
(York Plays, s1962)
\z
\z

\begin{table}[ht]
    \centering
    \renewcommand{\arraystretch}{1.1}
    \begin{small}
        \begin{tabularx}{\textwidth}{lcccccccc}
            \lsptoprule
& \multicolumn{4}{c}{S and V are adjacent} & \multicolumn{4}{c}{S and V are non-adjacent} \\
\midrule
Verb endings  & -s & -st & -e & -$\varnothing$ & -s & -st & -e & -$\varnothing$\\
uninverted & 68 & 15 & 13 & 5 & 13 & 0 & 2 & 0\\
inverted & 49 & 8 & 16 & 1  & 0 & 0 & 0 &  0\\
\textbf{Total} & 117  & 23  & 29  & 6  & 13  & 0 & 2  & 0\\
& (66.9\%) & (13.1\%) & (16.6\%) & (3.4\%) & (86.7\%) & 0 & (13.3\%) & 0\\
\lspbottomrule
\end{tabularx}
\caption{Verbal endings of the second person present tense indicative in the
\emph{York Plays} (lexical verbs only)}\label{table1-2ndps-lexical-verbs}
 \end{small}
\end{table}

In what follows, we take a closer look at the behaviour of the auxiliaries
`have' and `be'. As shown in \Cref{table1-2ndps-lexical-verbs}, variants of the
\emph{s}-ending (especially \emph{-st}) are particularly frequent with `have’
in its use as a perfect tense auxiliary\is{auxiliaries} (almost obligatory, in
fact).\footnote{In connection with the perfect auxiliary\is{auxiliaries} ‘have'
    \Ssg{} \emph{st}-forms are frequently extended with \emph{e}: In cases
    where the subject is adjacent to the verb, we have found 9 instances of
\emph{haste} in inversion contexts, and 11 instances of \emph{haste} without
inversion.}

\begin{table}[ht]
    \centering
    \renewcommand{\arraystretch}{1.1}
    \begin{small}
        \begin{tabularx}{\textwidth}{lcccccccccc}
            \lsptoprule
& \multicolumn{4}{c}{S and V are adjacent} & \multicolumn{4}{c}{S and V are non-adjacent} \\
\midrule
Verb endings  & -s & -st & -e & -$\varnothing$ & -s & -st & -e & -$\varnothing$\\
uninverted & 19 & 21 & 1 & 0 & 2 & 3 & 1 & 0\\
inverted & 8 & 14 & 0 & 0  & 0 & 0 & 0 &  0\\
\textbf{Total} & 27  & 35  & 1  & 0 & 2  & 3  & 1  & 0\\
 & (42.8\%) &  (55.6\%) & (1.6\%) & 0 &  (33.3\%) & (50\%) & (16.7\%) & 0\\
 \bottomrule
\end{tabularx}

\caption{Verbal endings of the second person perfect auxiliary\is{auxiliaries} `have' in the \emph{York Plays}}
 \end{small}
 \label{table1-2ndps-have}
\end{table}

\largerpage[1]
So it appears that forms ending in \emph{-s/st} are highly grammaticalized\is{grammaticalization} as
realizations of the \Ssg{} perfect auxiliary\is{auxiliaries} `have'. Furthermore, note that the
extended \Ssg{} marker \emph{-st} has been better preserved in connection
with `have', which can presumably be attributed to the fact that auxiliary\is{auxiliaries}
`have' is a highly frequent element. A similar frequency-related preservative
effect can be observed with \Ssg{} forms of `be', albeit with a different
effect on the distribution of \emph{s}-marked forms, as illustrated in Table
7.

\begin{table}[ht]
    \centering
    \renewcommand{\arraystretch}{1.1}
    \begin{small}
        \begin{tabularx}{\textwidth}{lllllll}
            \lsptoprule
& \multicolumn{3}{c}{S and V are adjacent} & \multicolumn{3}{c}{S and V are non-adjacent} \\
\midrule
Verb forms  & is & art & arte & is & art & arte\\
uninverted & 2 & 13 & 29 & 3 & 0 & 0\\
inverted & 2 & 5 & 10 & 0  & 0 & 0\\
\textbf{Total} & 4 (6.6\%) & 18 (29.5\%) & 39 (63.9\%) & 3 (100\%) & 0 & 0\\
\lspbottomrule
\end{tabularx}

\caption{\Ssg{} forms of the auxiliary\is{auxiliaries} `be' (present tense) in the \emph{York
Plays}}
 \end{small}
 \label{table1-2ndps-be}
\end{table}

`be' differs significantly from the other verbs surveyed so far, and its
special behaviour is of particular theoretical interest, as will become clear
shortly. First and foremost, the \emph{s}-marked form \emph{is} (which is also
standardly used in connection with all kinds of \Tsg{} subjects) is quite
rare;\footnote{The \emph{s}-ending also appears on preterite forms of ‘be’
(\emph{was}).} in around 90\% of all cases, the \Ssg{} of `be' is realized
by a variant of \emph{art}, with the extended form \emph{arte} being twice
as frequent as the short alternative. Again, the fact that the suppletive
form of \Ssg{} `be' has been preserved in the \emph{York Plays} can be
attributed to the high token frequency of \emph{art(e)}, which in this case
has blocked the spreading of the \emph{s}-marked alternative \emph{is}.
However, \emph{art(e)} seems to be confined to contexts where the subject
is adjacent to the finite auxiliary\is{auxiliaries}. In any case, the absence of
\emph{art(e)} in non-adjacent contexts seems to be noteworthy. It
might well be that non-adjacent instances of \emph{art(e)} are simply by
chance absent from the records (recall that there is a strong tendency for
pronominal subjects to be adjacent to the verb). Moreover, examples like
(15) suggest that the use of \emph{is} is not necessarily a reflex of the
\gls{NSR} in \Ssg{} contexts, since \emph{is} is used both under adjacency
and non-adjacency with the subject pronoun.\footnote{Note that despite
    appearances, cases like (i) and (ii) are not to the point, since both
    \emph{arte} and \emph{haste} as well as \emph{arte} and \emph{caris}
    are the regular (fully inflected) \Ssg{} forms of the relevant verbs.

    \begin{exe}
        \exi{(i)}
        \gll Why, \textbf{arte} thou a pilgryme and \textbf{haste} bene at Jerusalem \\
        why are thou a pilgrim and has been at Jerusalem\\
        \glt `Why, are you a pilgrim who has been in Jerusalem?'\\
        (York Plays, 40, 70)
        \exi{(ii)}
        \gll Thou \textbf{arte} combered in curstnesse / and \textbf{caris} to this coste.\\
        thou are troubled in cursedness {} and cares to this cost\\
        \glt `You are troubled with sin and dread this price.'\\
        (York Plays, 26, 171)
\end{exe}}

\ea
\gll For thou \textbf{is} one and \textbf{is} abill and aught to be nere.\\
for thou is one and is able and ought to be near\\
\glt (York Plays, 32, 33)
\z

\Tsg{} subjects always trigger \emph{s}-forms (V+\emph{s}, \emph{has},
\emph{is}); there is no trace of the \gls{NSR}\is{Northern Subject Rule}, that
is, type of subject and position of the subject relative to the verb do not
matter, as shown in (\ref{york4}):

\ea
\label{york4}
\ea \gll Or ellis \textbf{this} \textbf{brande} in youre braynes sone \textbf{brestis} and \textbf{brekis}.\\
Or else this fire in your brain soon bursts and breaks\\
\glt `Or else this wrath in your brain soon bursts and breaks out.'\\
(York Plays, s2941)
\ex \gll Here sirs howe \textbf{he} \textbf{sais}, and \textbf{has} forsaken His maistir to this woman here twyes \\
Hear sirs how he says and has forsaken his master to this woman here twice\\
\glt `Hear sirs, what he says and how he has betrayed his master twice with this woman here.'\\
(York Plays, s2793)
\z
\z

As already briefly mentioned above, the effects of the \gls{NSR}\is{Northern Subject Rule} can be most readily observed with plural (pronominal) subjects. While nominal \Tpl{} subjects usually require \emph{s}-marking in the present tense, as shown in (\ref{york5}), the verb appears in its bare form when the subject is a pronoun adjacent to the verb. This is illustrated in (\ref{york6}).

\ea
\label{york5}
\ea
\gll Say, Jesu, \textbf{the} \textbf{juges} \textbf{and} \textbf{the} \textbf{Jewes} \textbf{hase} me enioyned\\
say Jesus the judges and the Jews has man pleased\\
\glt `Say, Jesus, the judges and the Jews have pleased man.'\\*
(York Plays, s3120)
\ex
\gll To mischeue hym, with malis in there mynde haue thei menyd,
And to accuse hym of cursednesse \textbf{the} \textbf{caistiffis} \textbf{has} caste. \\
To harm him with malice in their mind have they meant and to accuse him of sinfulness the captives has uttered.\\
\glt `To harm him with malice in their mind they  complained and to accuse him of sinfulness the captives have uttered.'\\
(York Plays, s5243)
\ex
\gll This matter that thowe moves to me is for \textbf{all} \textbf{these} \textbf{women} bedene That \textbf{hais} conceyved with syn fleshely \\
    This matter that though moves to me is for all these women bidding that has conceived with sin fleshly\\
\glt `This matter that though moves to me if for all these bidding women that have conceived in fleshly sin.'\\
(York Plays, s1511)
\z
\z

\ea
\label{york6}
\ea \gll Sir kyng, \textbf{we} all \textbf{accorde} / And \textbf{sais} a barne is borne \\
Sir king we all accord {} and says a bairn is born\\
\glt `Sir king, we all accord and say a child is born.'\\
(York Plays, 16, 209--210)
\ex \gll Therfore some of my peyne \textbf{ye} \textbf{taste} / And \textbf{spekis} now nowhare my worde waste, \\
Therefore some of my pain you\tief{[2pl]} taste {} and speaks now nowhere my word waste\\
\glt `Therefore some of my pain you taste and speak now nowhere my word waste.'\\
(York Plays, 41, 87)
\ex
\gll Howe these folke spekes of our chylde. \textbf{They} \textbf{say} \textbf{and} \textbf{tells} of great maistry \\
How this folk speaks of our child. They say and tells of great authority. \\
\glt `How this folk speaks of our child. They say and tell of great authority.'\\
(York Plays, 15, 79)
\z
\z

\largerpage[1]
In traditional descriptions of the inflectional system of the \emph{York Plays}
it is sometimes taken for granted that \gls{NSR}\is{Northern Subject Rule}
effects as in (\ref{york6}) are the norm with plural subject pronouns that are
not adjacent to the verb (cf.\ e.g.\ \citealt[272]{BurrowTurville-Petre:2005}).
However, it seems that the \isi{agreement} system is more variable. For
example, there are also cases where the verb fails to be adjacent to the
subject and still lacks \emph{s}-marking as illustrated in
(\ref{york7}).\footnote{In general, cases where pronouns are not adjacent to
the verb are quite rare. It is therefore difficult to estimate the status of
patterns such as (\ref{york6}) and (\ref{york7}). One might speculate that in
at least some of those cases, the zero ending is used to facilitate rhyming as
in (19c). Alternatively, cases of zero marking under non-adjacency might be
taken to foreshadow the loss of the position-of-subject constraint, eventually
leading to general verbal zero marking with pronominal subjects (as in many
present-day dialects). The extension of the zero marker could then perhaps be
analysed as an analogical change made available by the overall rarity of cases
where the pronoun fails to be adjacent to the verb.}

\ea
\label{york7}
\ea \gll Wherefore \textbf{we} dresse vs furth oure way / And \textbf{make} offerand to God this day \\
wherefore we dress us forth our way {} and make offerings to God this day\\
\glt `Wherefore we go on our way and make offerings to God this day.'\\
(York Plays, 17, 228)
\ex  \gll Yhe comaunded me to care, als \textbf{ye} kenne wele and \textbf{knawe}, To Jerusalem on a journay, with seele; \\
You commanded me to care as you\tief{[2pl]} know well and know to Jerusalem on a journey with good-fortune\\
\glt `You commanded me to come, as you well understand and know, to Jerusalem on a journey, with good fortune.' \\
(York Plays, 30, 336)
\ex \gll That lurdayne that \textbf{thei} loue and \textbf{lowte} / To wildirnesse he is wente owte\\
that rascal that they love and venerate {} to wilderness he is gone out\\
\glt `That rascal that they love and venerate he went out to the wilderness.'\\
(York Plays, 22, 32)
%\item We haue done his bidding/How so they wraste or wryng,
%\glt We have done his bidding how so they wrest or wryng
%\glt  (York Plays, 19,239)
%\item They speke oure speche als wele as we,/And in ilke a steede it vndirstande.
%\glt (York Plays, 43,159)
\z
\z

Particularly interesting in this regard is the behaviour of the plural of `be',
which is realized by variants of \emph{are}. It turns out that independently of
the category (nominal/pronominal) and the position of the subject
(adjacent/non-adjacent to the verb), the plural form of `be' is almost always
\emph{are}, that is, forms of `be' are usually not subject to the
\gls{NSR}\is{Northern Subject Rule}.\footnote{It should be pointed out,
    however, that there are few examples, such as (i), where \gls{NSR}\is{Northern
    Subject Rule} effects do show up with non-adjacent forms of `be'. At least
    with plural subjects, these are vastly outnumbered by cases where the
    regular plural form \emph{are} appears under non-adjacency.

    \begin{exe}
        \exi{(i)} \gll Thei that \textbf{is} comen of my kynde [...] \\
        they that is come of my kind \\
        \glt (York Plays, 44, 128)
    \end{exe}
} The different behaviour of `be' and lexical verbs is illustrated by the examples in (\ref{rule-be}).

\ea
\label{rule-be}
\ea \gll Sir knyghtis that \textbf{are} in dedis dowty, \textbf{Takes} tente to vs,\\
Sir knights that are in deeds doughty takes entente to us\\
\glt `Sir knights who are doughty in deeds devote themselves to us.'\\
(York Plays, 38, 417)
\ex \gll Men that \textbf{are} stedde stiffely in stormes or in see And \textbf{are} in will wittirly my worschippe to awake, And thanne \textbf{nevenes} my name in that nede, \\
men that are placed unwavering in storms or in sea And are in will fully my worship to awake And then call my name in that need \\
\glt `Men who are unwavering in storms or at sea and are fully in will to awake my worship and then call my name in that need.' \\
(York Plays, 44, 137--139)
\ex \gll All that \textbf{are} in newe or in nede and \textbf{nevenes} me be name,\\
All that are in harm or in need and call me by name\\
\glt `All of them are in harm or in need and call me by my name.'\\
(York Plays, 44, 144)
\z
\z

So there is a major difference between the plural forms of lexical verbs and
`be': With lexical verbs, nominal subjects differ from pronominal subjects in
that the former always trigger \emph{s}-marking on the verb (both in the
singular and the plural), while the latter take part in the \gls{NSR}\is{Northern Subject Rule}. With
`be', however, nominal and pronominal subjects behave alike: Singular forms
trigger \emph{is}, while plural subjects invariably trigger \emph{are}. In the
following section, we will discuss the theoretical relevance of this asymmetry.
We would also like to point out that \Fsg{} forms seem to play a special role
in that they are by and large (see above for some exceptions) exempt from the
\gls{NSR}, in contrast to the system listed in \Cref{tab:key:10.2} above.
\Cref{tab:key:8} summarises our findings regarding the inventory of inflectional
endings found with present tense verbs in the \emph{York Plays} (``pron.'' stands
for ``pronoun'', ``adjac.'' stands for ``adjacent''; recall that
``-$\varnothing$'' is a shortcut
for zero marking and forms that end in \emph{-e}).

\begin{table}[ht]
    \centering
    \renewcommand{\arraystretch}{1.1}
    \begin{footnotesize}
        \begin{tabularx}{\textwidth}{@{}l@{ }lllllllll}
            \lsptoprule
    & \multicolumn{3}{c}{+pron, +adjac.} & \multicolumn{3}{c}{+pron, −adjac.} & \multicolumn{3}{c}{−pron, −adjac.}\\
    & lexical V & have & be & lex.\ V & have & be & lex.\ V & have & be\\
    \midrule
    \Fsg{} & -$\varnothing$ & have & am & -$\varnothing$ (-s) & have (has) & am (is) & - & - & -\\
    \Ssg{} & -s (-st, -$\varnothing$) & has, hast(e)  & art, arte & -s (-$\varnothing$) & has, hast(e) & is? & - & - & -\\
    \Tsg{} & -s & has & is & -s & has & is & -s & has & is\\
    \Fpl{} & -$\varnothing$ & have & are & -s (-$\varnothing$) & has (have) & are & - & - & -\\
    \Spl{} & -$\varnothing$ & have & are & -s (-$\varnothing$) & has (have) & are & - & - & -\\
    \Tpl{} & -$\varnothing$ & have & are & -s (-$\varnothing$) & has (have) & are & -s & has & are\\
    \lspbottomrule
\end{tabularx}
\caption{Verbal inflection in the \emph{York Plays} (present tense
indicative)}\label{tab:key:8}
\end{footnotesize}
\end{table}

\section{The NSR in the \emph{York Plays}: Towards an analysis}
\label{sec:towards-an-analysis}
An adequate analysis of the type of \gls{NSR}\is{Northern Subject Rule} as
exhibited by the \emph{York Plays} should capture the following basic
system-defining characteristics: (i) the effect of subject type/position of the
subject on verbal \isi{agreement} marking; (ii) the fact that apart from some minor
exceptions (which probably reflect differences in authorship, or language
change in progress), the \gls{NSR}\is{Northern Subject Rule} seems to be
confined to plural forms; (iii) the observed differences between `be' and other
verbs (only `be' signals regular number agreement\is{agreement!number
agreement} independently of type and position of the subject). In what follows,
we will present a syntactic analysis of these findings that makes use of the
notion that inflectional heads may lack phi-content when they enter the
syntactic derivation, which \textcite{Roberts:2010} calls ``blank generation''.
The basic idea is that the absence of \isi{agreement} features on the T-head
may be repaired in different ways, either via insertion of default inflection
(i.e., \emph{-s} in many \gls{NSR}\is{Northern Subject Rule} varieties), or by
\isi{incorporation} of adjacent subject pronouns, leading to the presence of
phi-features\is{φ-features} on T, which can then be spelled out by (marked/more
specified) zero \isi{agreement}.

However, before we turn to the specifics of that approach to the
\gls{NSR}\is{Northern Subject Rule}, we would like to discuss in some more
detail a set of morphological aspects pertaining to the \isi{agreement} system
as found in the \emph{York Plays}, including the inventory of markers and their
featural specifications (see \Cref{sub:10.4.2} for the question of how richness
of inflection might be linked to the featural content of the relevant
underlying inflectional heads in the syntax).

\subsection{Morphological aspects}

\largerpage[1]
The \emph{York Plays} exhibit a mixed system, where the \gls{NSR}\is{Northern
Subject Rule} is more or less confined to the plural part of the paradigm (with
some few exceptions with \Fsg{}) and has not yet spread to `be'. In the
inventory of present tense markers we still find \Ssg{} forms extended by
\emph{t}, similar to earlier stages of English. The extended forms are rare
with lexical verbs, but are the dominant pattern with auxiliary\is{auxiliaries}
verbs (\emph{hast(e)}, and in particular \emph{art(e)}). With auxiliaries, they
serve to preserve the distinction between \Ssg{} and plural (and \Tsg{}) forms,
which is blurred with lexical verbs (due to the loss of final \emph{t} in the
\Ssg{}).\footnote{Note that it is not entirely clear whether the \Ssg{} forms
extended by \emph{t} represent a retention or are the result of dialect contact
(e.g., the MED lists \emph{hæfes} as the \Ssg{} of `have' in Northumbrian
\gls{OE}).} The evidence for distinctive \Ssg{} forms provided by auxiliaries
precludes the development of a general Impoverishment\is{impoverishment} rule
suggested above (here repeated in (21)), which leads to system-wide syncretism
of \Ssg{} and \Tsg{} forms (in varieties that have preserved \emph{thou}).

\ea
\mbox{[+hearer]} {$\rightarrow$ $\varnothing$} / \_\_\ pronoun\tief{[NOM]}
\z

To capture the fact that syncretism of \Ssg{} and \Tsg{} is confined to lexical
verbs, we propose the following slightly modified version of (21), which
applies only to lexical verbs and deletes the verbal \isi{agreement} feature
[+hearer] when the finite verb is adjacent to a \Ssg{} subject pronoun. As a
result of (22), finite verbs that agree with \Ssg{} subjects in the syntax lack
positive values for [person] and [number] at the point of Vocabulary Insertion
(assuming a realisational model of grammar, where phonological exponents of
abstract morphosyntactic features are inserted postsyntactically, cf.\ e.g.\
\citealt{HalleMarantz1993}).\footnote{Alternatively, one might assume that the
\emph{-s}-marker found with \Ssg{} lexical verbs is still a genuine \Ssg{}
form, which only happens to be accidentally homophonous with the default
\emph{-s} found in other contexts (i.e., in the \Tsg{} and plural).}

\ea
\mbox{[+hearer]} {$\rightarrow$ $\varnothing$} / \tief{V}\_\_\ pronoun\tief{[NOM]}
\z

The system of present tense indicative markers for lexical verbs can thus be
described by basically the same set of Vocabulary Items that we posited for the
system in \Cref{tab:key:10.2} above (following standard assumptions, more
specified exponents/markers take precedence over less specified exponents due
to the \isi{Elsewhere Condition}, \citealt{Kiparsky:1973}):

\ea
%\item \mbox{[+hearer, -pl]} $\leftrightarrow$ \emph{-s}
\ea \mbox{[+phi]} $\leftrightarrow$ -$\varnothing$
\ex elsewhere\is{Elsewhere Condition} $\leftrightarrow$ \emph{-s}
\z
\z

After deletion of [+hearer] (due to the Impoverishment\is{impoverishment} rule in (22)), both
\Ssg{} and \Tsg{} forms are spelled out by the default inflection \emph{-s}
(recall that we assume that ``\Tsg{}'' corresponds to the absence of (positive)
specifications for [person] and [number]). In this way, (22), in combination
with the inventory of \isi{agreement} markers in (23), accounts for the lack of
\gls{NSR} effects with \Ssg{} (and \Tsg{}) subjects.

A slightly different set of Vocabulary Items is used for present tense
indicative forms of `have'. We take it that the extended form \emph{hast(e)}
still signals \Ssg{}. To account for the fact that \emph{hast(e)} covaries with
the reduced and ambiguous form \emph{has}, we assume that the same feature set
can be spelled out by \emph{has} (probably as a result of phonological erosion
(reduction of the final consonant cluster \emph{st}), which happens to be
homophonous with the elsewhere\is{Elsewhere Condition} marker.%\footnote{Note that after replacement of \emph{thou} by \emph{you} a simpler systems becomes available that makes do with a simple [+phi] vs.\ [-phi] opposition, similar to lexical verbs}

\ea
\ea \mbox{[+hearer, -pl]} $\leftrightarrow$ \emph{hast(e), has}
\ex \mbox{[+phi]} $\leftrightarrow$ \emph{have}
\ex elsewhere\is{Elsewhere Condition} $\leftrightarrow$ \emph{has}
\z
\z

The present tense paradigm of `be' has preserved even more distinctions (three
persons in the singular, and the distinctive plural form \emph{are}). Moreover,
\gls{NSR} effects are virtually non-existent with `be',\footnote{Recall that
    there are very few instances where \emph{is} occurs with (non-adjacent)
    \Fsg{} and \Ssg{} subjects.} and it is the only verb that exhibits proper
    number agreement with nominal subjects.\is{agreement!subject agreement} The
    inventory can thus be described as follows:

\ea
\ea \mbox{[+speaker, -pl]} $\leftrightarrow$ \emph{am}
\ex \mbox{[+hearer, -pl]} $\leftrightarrow$ \emph{art(e)}
\ex \mbox{[+pl]} $\leftrightarrow$ \emph{are}
\ex elsewhere\is{Elsewhere Condition} $\leftrightarrow$ \emph{is}
\z
\z

\largerpage[-1]
Note that the inventory in (23)--(25) single out \emph{s}-marked forms as the
elsewhere case. In the next subsection, we will address the question of why
\emph{s}-marked forms gain a wider distribution in contexts where the verb
fails to be adjacent to a pronominal subject. In addition, we will argue that
the absence of \gls{NSR}\is{Northern Subject Rule} effects in connection with \Fsg{} (in contrast to
\Ssg{} and \Tsg{}) and all forms of `be' cannot be attributed to morphological
properties, i.e., the inventory of Vocabulary Items plus Impoverishment, and
should thus receive a syntactic explanation.\footnote{An anonymous reviewer
    raised the question whether the asymmetry between `be' and other verbs
    could not simply be analysed as a lexical difference, in the sense that the
    paradigm of inflected forms of `be' is richer than the paradigms of other
    verbs.  However, a lexical solution fails to account for the fact that the
    difference between `be' and lexical verbs is syntactic in nature: With
    lexical verbs, the \isi{agreement} alternation (that is, the
\gls{NSR}\is{Northern Subject Rule}) is governed by syntactic factors (type and
position of the subject), while no such effects are observed with `be'.}

\subsection{Syntactic aspects}\label{sub:10.4.2}

In this section, we will present an analysis of the \isi{agreement} system
displayed by the \emph{York Plays} that is based on \citegen{Roberts:2010}
proposal that functional heads may enter the syntactic derivation without
featural content (so-called ``blank generation''). We will argue that a
slightly modified version of Roberts's approach to the \gls{NSR}\is{Northern
    Subject Rule} provides enough leeway to account for the mixed or hybrid
    character of the \isi{agreement} system found in the \emph{York Plays} (in
    particular, the special behaviour of `be'), in contrast to previous
    theoretical analyses. We take it that the lack of \gls{NSR}\is{Northern
    Subject Rule} effects with `be' and \Fsg{} subjects reflects a genuine
    syntactic difference and should not be captured by purely
    post-syntactic/morpho-phonological mechanisms (in contrast to what we have
    proposed for the absence of relevant effects with \Ssg{} and \Tsg{}). More
    precisely, the facts suggest that in these cases, subject-verb agreement is
    established by a syntactic operation (e.g., \isi{Agree};
    \citealt{Chomsky2000}) that leads to feature matching between the
    phi-content of a relevant functional head\is{functional items} (T/INFL) and
    the subject, independently of type and position of the latter.

\largerpage[-1]
\textcite{Roberts:2010} outlines an analysis of the \gls{NSR}\is{Northern Subject Rule} that is based on
the idea that in the relevant varieties, T/INFL lacks a phi-set of its own
(blank generation). As a result, T/INFL enters the syntactic derivation without
agreement features. it can only acquire such features via \isi{incorporation} of
(clitic)\is{clitics} subject pronouns.\footnote{A related, but purely
    post-syntactic, analysis of the \gls{NSR}\is{Northern Subject Rule} is
    proposed by \citealp{TripsFuss:2010}, who posit the following
    \isi{agreement} rule that operates on the output of the syntactic
    derivation:

    \begin{exe}
        \exi{(i)} \emph{-$\varnothing$} marks the presence of positive specifications for [person] or [number] in the minimal phonological domain the finite verb is part of; \emph{-s} is inserted elsewhere\is{Elsewhere Condition}.
    \end{exe}
    Similar to an approach in terms of blank generation, (i) assumes that the
    relevant \isi{agreement} features are provided by weak subject pronouns
    under adjacency with the verb. However, notice that the special behaviour
    of `be' seems to call for a (partially) syntactic treatment of subject-verb
    \isi{agreement} in the \emph{York Plays}. See below for further discussion
    and a synthesis of the two accounts.} The presence of (positively
    specified) \isi{agreement} features in T/INFL (resulting from the
    \isi{incorporation} of clitic pronouns) is then signalled by zero marking
    on the verb, while \emph{-s} is inserted as a default inflection when
    T/INFL lacks \isi{agreement} features (cf.\ (23)).\footnote{Recall that we
        assume that `\Tsg{}' corresponds to the absence of (positively
        specified) person and number features, cf.\ e.g.\
        \cite{Harleyritter:2002}.\label{fn:10.38}} To account for the adjacency effect, Roberts assumes that \isi{incorporation} must go hand in hand with phonological cliticisation of the subject pronoun to the verb.\footnote{Interestingly, it seems that the only elements that may regularly intervene between a subject pronoun and a zero-marked verb are (weak) object pronouns as in (i).

    \begin{exe}
        \exi{(i)}
        \gll That we \textbf{hym} tharne sore may vs rewe,\\
        that we him lose sure may us regret\\
        \glt `We will certainly regret that we lost him.'\\
        (York Plays, 42, 14)
    \end{exe}
This can be accounted for if we assume that both the subject and object pronoun
are part of a clitic cluster that attaches to the verb.} In other words, a
T/INFL head without an inherent phi-set may acquire \isi{agreement} features in
the course of the derivation when the conditions in (26) are met.

\ea
\ea \isi{incorporation} of the subject pronoun: [\tief{T} T D\tief{[+phi]}]
\ex phonological cliticisation: (pronoun - X - V) (where X is null or another clitic)\is{clitics}
\z
\z

This account provides a straightforward description of ``pure''
\gls{NSR}\is{Northern Subject Rule} systems similar to the one given in
\Cref{tab:key:10.2} where all verbs (including auxiliaries) take the
marked (zero) ending only in connection with adjacent non-\Tsg{} (clitic)
pronouns, while \emph{-s} occurs elsewhere\is{Elsewhere Condition}. In addition
to the adjacency condition, Roberts' analysis also correctly predicts that
stressed, coordinated and modified forms (which are not \isi{clitics}) trigger
default inflection on the verb:

\ea
\ea They've recently comed, \textbf{has them}.
(Yorkshire English; \citealt[88]{Pietsch:2005b})
\ex Him and me \textbf{drinks} nought but water.
\parencite[6]{Roberts:2010}
\ex Us students \textbf{is} going.
(\ili{Belfast English}; \citealt[24]{Henry:1995})
\z
\z

However, something more must be said to capture (a) the fact that the pronoun's
phi-set is spelled out twice (as the pronoun itself and as zero marking on the
verb), and (b) the observation that in many \gls{NSR}\is{Northern Subject Rule}
dialects, the marked zero inflection also appears in inversion contexts, where
the finite verb precedes an adjacent subject pronoun:

\ea
\gll So sir, \textbf{slepe} ye, and \textbf{saies} no more.\\
so sir sleep you.pl and say no more\\
\glt (York Plays, 30, 148)
\z

Under the assumption that \isi{incorporation} of the pronoun is a purely syntactic
process, the fact that it may precede (compare (18) above) or follow the
zero-marked verb (as in examples like (28)) does not seem to receive a
satisfying explanation. If \isi{incorporation} is analysed as an instance of head
movement, we would expect that the relative order of pronoun and finite verb is
not variable. As a possible solution, one might suggest that the linearisation
of the incorporated pronoun is sensitive to the syntactic position of the
finite verb in the sense of a second position/Wackernagel effect that is only
triggered when the verb has moved to C\hoch{0}. However, such an account would
be quite stipulative. In what follows, we would like to argue that a more
principled explanation becomes available  if we take a closer look at the
nature and cause of the assumed \isi{incorporation} process. What we would like to
propose is that in the \gls{NSR}\is{Northern Subject Rule} varieties, \isi{incorporation} of the pronoun is in
fact a postsyntactic repair operation that is triggered to patch up a T head
that enters the morpho-phonological component without phi-content. The
rationale behind this idea is that in a language with at least some
morphological \isi{agreement}, a phi-less T-head creates a problem at the interface
to the morpho-phonological component.\footnote{Arguably, no such repair is
    needed in languages that completely lack \isi{agreement} features (e.g.,
Indonesian).} This problem can be repaired either by the insertion of default
inflection (a Last Resort prior or during Vocabulary Insertion), or by
``incorporation'' of an adjacent phi-set that can then be spelled out by an
appropriately marked \isi{agreement} formative. The latter option is arguably more
specific/complex and therefore preempts repair via default inflection (due to
the Elsewhere Condition). To account for the fact that both the pronoun and the
phi-set on T are spelled out (the latter usually via the zero marker in the
\gls{NSR} varieties), we assume that the pronoun's phi set is copied onto the
finite verb/T under adjacency (i.e. when both elements are part of the same
minimal prosodic domain).\is{prosodic domains} Crucially, this repair operation (giving rise to zero
inflection) can apply in both inversion and non-inversion contexts as long as
the pronoun is directly adjacent to the finite verb (note that this
modification of Roberts' original account combines the idea of blank generation
with certain aspects of the postsyntactic approach proposed by
\citealt{TripsFuss:2010}, cf.~\cref{fn:10.38}).

Some additional tweaking is needed to account for the intricacies of the
version of the \gls{NSR}\is{Northern Subject Rule} that is found in the
\emph{York Plays}. First of all, it is evident that in contrast to other verbs,
`be' cannot be subject to blank generation. Rather, `be' is the phonetic
realization of a special T/INFL node that comes with its own
phi-features\is{φ-features} (in
contrast to T/INFL linked to other verbs). As a result, `be' may agree with
non-pronominal subjects as well. Note that the special behaviour of `be' is a
major challenge for theoretical approaches that analyse agreement/non-agreement
as the result of different subject positions (as e.g.\
\citealt{deHaasandvanKemenade:2015}). The fact that regular number
agreement\is{agreement!number agreement} occurs with nominal subjects (which
otherwise do not trigger agreement) shows that the structural position of the
subject is not relevant. Rather, it seems that `be' (in contrast to other
verbs) can detect the phi-features\is{φ-features} of any kind of subject
(independent of its position and categorial nature) due to the fact that
T\tief{be} always carries an unvalued set of phi-features\is{φ-features} that
triggers a syntactic \isi{Agree} operation. Thus, we take the asymmetry between
`be' and other verbs to suggest that blank generation may be parameterized so
that it affects only certain types of inflectional heads.

Basically the same approach can be used to account for the absence of \gls{NSR}\is{Northern Subject Rule} effects with \Fsg{} subjects. Again, we assume that T is not subject to blank generation in this case. Of course, this raises the more general question of how and why blank generation of inflectional heads is triggered.
What we would like to propose is that the absence of \isi{agreement} features on T is
intimately linked to the breakdown of the (morphological) \isi{agreement} system in
Northern Old/\ili{Middle English}. Recall that as a result of phonological
erosion (and probably language contact with Scandinavian), \emph{-s} (or
rather, variants of it) became the only overt \isi{agreement} marker in Northern
varieties, eventually leading to a binary \isi{agreement} system that does not any
longer signal featural distinctions apart from [+/-phi]. We take it  that this
is the prototypical situation that brings about wholesale ``blank generation''
of T/INFL.\footnote{On a more technical note, one might assume that blank
    generation of T/INFL results from another type of Impoverishment\is{impoverishment} rule that
    deletes person and number features from T/INFL before the latter enters the
    syntactic derivation (cf.\ e.g.\ \citealt{Mueller:2006} on the notion that
    Impoverishment\is{impoverishment} rules may also operate presyntactically):

    \begin{exe}
        \exi{(i)} \mbox{[Person, Number]} {$\rightarrow$ $\varnothing$} / T\_\_\
    \end{exe}

    However, note that such an approach raises a number of questions concerning the
    interplay between presyntactic and postsyntactic Impoverishment\is{impoverishment} that we cannot
    discuss here. We leave this issue for future research.} In the \emph{York
Plays}, however, we still find a slightly richer system of endings. In addition
to the fact that `be' has preserved more inflectional distinctions than other
verbs (including a systematic distinction between \Ssg{} and \Spl{}), the zero
ending is still closely linked to \Fsg{}, in that it unambiguously signals
[+speaker, -plural] with singular subjects and in cases where the subject fails
to be adjacent to the verb (presumably reflecting an earlier pre-\gls{NSR}
stage where \Fsg{} was the only feature combination that was clearly marked on
the verb, by zero marking; cf.\ \Cref{tab:key:10.2}). It seems thus plausible to
assume that blank generation of T is blocked in contexts where \isi{agreement}
marking can still be linked to featural distinctions that are more specific
than a binary [+/-phi] contrast. Our approach to the \gls{NSR}\is{Northern Subject Rule} in the
\emph{York Plays} is summarized in (29):

\ea
\ea \textbf{\gls{NSR} effects (plural forms)}: blank generation of T, repair via (a) default inflection ($\rightarrow$ \emph{-s}), (b) \isi{incorporation} of adjacent subject pronouns ($\rightarrow$ $\varnothing$);
\ex \textbf{no \gls{NSR}\is{Northern Subject Rule} effects/`be' \& \Fsg{}}: no blank generation of T, regular syntactic \isi{agreement};
\ex \textbf{no \gls{NSR}\is{Northern Subject Rule} effects/\Ssg{} \& \Tsg{}}:
Impoverishment\is{impoverishment} and underspecification of markers
($\rightarrow$ \emph{-s}).
\z
\z

This approach captures basic properties of the \isi{agreement} system exhibited by
the \emph{York Plays}. However, note that in addition to these general
patterns, we have also observed a number of alternative \isi{agreement} options. Some
of these are presumably residues of a former system (such as the few cases of
\Ssg{} \emph{-st} on lexical verbs), while others represent innovations that
compete with some of the options in (29), such as \gls{NSR}\is{Northern Subject Rule} effects in
connection with \Fsg{} (which can perhaps be analyzed as extensions of blank
generation to \Fsg{} contexts), and cases where the position of subject
constraint seems to be neutralized, leading to general zero marking with
pronominal subjects (which foreshadows a development that has taken place in a
number of \gls{NSR}\is{Northern Subject Rule} dialects).\footnote{A fuller description and quantitative
    analysis of the \isi{agreement} options in the \emph{York Plays} is beyond the scope
of this paper. We leave it for future investigation.} The existence of this
type of linguistic variation suggests that the particular version of the
\gls{NSR} that is found in the \emph{York Plays} represents an intermediate
stage that eventually gave way to a more balanced \isi{agreement} system where blank
generation of T/INFL is not (lexically) confined to certain contexts.

%\footnote{Recall that `be' must be exempt from blank generation, which is possibly linked to the fact that \emph{are} is clearly a plural form. Still, the fact that `be' fails to signal person distinctions in the plural suggests that it is affected by another Impoverishment\is{impoverishment} rule that deletes person features in the context of [+pl].}



\section{Some remarks on the historical origin of the NSR}
\label{sec:historical-origin}
So far, we have presented a theoretical analysis of the \gls{NSR}\is{Northern Subject Rule} in terms of
``blank generation'' of inflectional heads. From a diachronic point of view, we
have seen that in \gls{OE}, special inversion contexts show an unexpected
-\emph{e} affix which can be interpreted as foreshadowing the \gls{NSR}\is{Northern Subject Rule}, and
that this rule actually occurred in some \gls{ME}\il{Middle English} texts. In this section, we
will bring these observations together and argue that after the breakdown of
the \gls{OE} \isi{agreement} system, the \gls{NSR}\is{Northern Subject Rule} developed via a combination of
generalized V2 in the northern varieties and \isi{agreement weakening} in inversion
contexts (which turned into the \gls{NSR}\is{Northern Subject Rule} after the loss of V2).\footnote{Some
    authors (cf.\
    \citealt{Hamp:1976,Klemola:2000,Filppulaetal:2002,deHaas2008}) have
    claimed that the rise of the \gls{NSR}\is{Northern Subject Rule} was promoted by language contact
    with the \ili{Brythonic Celtic} languages, which exhibit a similar distinction
    between pronouns and non-pronouns. See e.g.\ \textcite{Pietsch:2005a},
\textcite{deHaas:2011} and \textcite{Benskin2011} for critical discussion.}

The starting point for our diachronic analysis is Northumbrian \gls{OE}, where
only \Fsg{} is unambiguously marked by verbal \isi{agreement} (via \emph{-e}/$\varnothing$).
Elsewhere, we find some form of \emph{-s} marking, which alternates with the
dental markers in \Tsg{} contexts and in the plural part of the paradigm. The
question then is how and why new zero markers were introduced into the northern
paradigm. We believe that the rise of new zero-marked plural forms is closely
related to the phenomenon of \isi{agreement weakening} in \gls{OE}. Following
\citet{Roberts:1996}, we analyze \gls{OE} \isi{agreement weakening} in terms of
contextual allomorphy of \Fpl/\Spl{} forms which can be attributed to syntactic
factors, namely the structural position of the finite verb (similar to
complementizer agreement\is{agreement!complementizer agreement} in present-day West Germanic dialects): (i) The
reduced form is used only when the verb moves to C\is{verb movement} (in contexts with fronted
operators such as \emph{wh}, negation etc.). In contrast, full \isi{agreement} obtains in
all other contexts, where the verb occupies a lower inflectional head (Infl/T)
(cf.\ e.g.\
\citealt{CardinalettiRoberts:1991,Pintzuk:1999,HulkKemenade:1995,Krochtaylor:1997,Haeberli:1999,Fischeretal:2000},
and many others). As a result, \isi{agreement weakening} is confined to
inversion contexts where the finite verb immediately precedes a \Fpl/\Spl{}
subject pronoun. In the other cases where the finite verb is in a lower
position we find regular \isi{agreement} with both subject pronouns and full
subject DPs. This is illustrated with the following structures:

\ea Original (southern) \gls{OE} pattern
\ea \mbox[\tief{CP} Op [\tief{Cʹ} C+V\tief{fin} [\tief{TP} subj.pron. [\tief{Tʹ} T [\tief{VP} \ldots{} ]]]]]\\ $\rightarrow$ \isi{agreement} weakening
\ex \mbox[\tief{CP} XP [\tief{Cʹ} C [\tief{TP} [\tief{Tʹ} T+V\tief{fin} [\tief{VP} DP subject \ldots{}]]]]]\\
$\rightarrow$ regular \isi{agreement}
\ex \mbox[\tief{CP} XP [\tief{Cʹ} C [\tief{TP} subj.pron. [\tief{Tʹ} T+V\tief{fin} [\tief{VP}
\ldots{}]]]]]\\ $\rightarrow$ regular \isi{agreement}
\z
\z

The evidence available suggests that this kind of systematic (syntactic)
agreement weakening was originally confined to southern varieties of \gls{OE},
while northern texts show only occasional examples of reduced \isi{agreement}
endings (i.e., schwa or -$\varnothing$) in inversion contexts (cf.\ e.g.\
\citealt{Berndt1956,Cole2014} on Northumbrian \gls{OE}). In other
words, it does not seem to be possible to analyze the \gls{NSR}\is{Northern
Subject Rule} as a direct continuation of \gls{OE} agreement weakening (but
recall that Northumbrian \gls{OE} exhibits a related pattern where the
\emph{s}-marker appears under adjacency with a subject pronoun).  However, it
seems likely that the \isi{agreement} patterns that eventually turned into the
\gls{NSR}\is{Northern Subject Rule} entered northern grammars via dialect
contact with southern varieties (cf.\ \citealt[53f.]{Pietsch:2005b} for
discussion). In the northern varieties the original \gls{OE} pattern shown in
(30) was then generalized to all contexts with adjacent plural subject pronouns
(cf.\ \citealt{Rodeffer:1903,Pietsch:2005b}).\footnote{Rodeffer's proposal is
    criticized by \textcite{Berndt1956}, who argues that quantitative data from
    Northumbrian \gls{OE} texts indicate that there is no direct link between
    \isi{agreement weakening} in \gls{OE} and the \gls{NSR}\is{Northern Subject
    Rule} (more precisely, Berndt argues that the evidence available to us
    suggests that \isi{agreement} weakening had already been in decline in the
    northern varieties before \emph{-s} was generalized to all persons and
    numbers; see \citealt[50ff.]{Pietsch:2005b} for comprehensive discussion
and a critical assessment of Berndt's arguments).} But why did this only happen
in the northern varieties? To answer this question, let us take a closer look
at grammatical factors that shaped the impact of dialect contact and possibly
led to the rise of the \gls{NSR}\is{Northern Subject Rule} in the northern
varieties. It has been claimed by a number of authors (cf.\ e.g.
\citealt{Krochtaylor:1997,Trips:2002}) that there are major syntactic
differences between northern and southern early ME
varieties.\footnote{Moreover, the \gls{NSR}\is{Northern Subject Rule} could not
    have developed in the southern varieties for purely morphological reasons:
    the loss of plural /-n/ in the \gls{ME}\il{Middle English} period served to
    neutralize the contrast between full and syncopated forms formerly
    introduced by \gls{OE} Agr-weakening.} In particular, the northern
    varieties had developed generalized V2 which means that the finite verb
    consistently occurred in C regardless of the nature of the initial
    constituent.  As a result of this change, the syntactic differences between
    subject pronouns and phrasal subjects seem to be less clear-cut than in
    \gls{OE} (the only remaining diagnostic is the placement of the subject
    relative to certain high \isi{adverbs}, cf.\ \citealt{deHaas:2011},
    \citealt{deHaasandvanKemenade:2015} for details):

\ea
\ea \mbox[\tief{CP} XP [\tief{Cʹ} C+V\tief{fin} [\tief{TP} subject [\tief{Tʹ} T [\tief{VP} \ldots{}]]]]]\\
\ex \mbox[\tief{CP} subject [\tief{Cʹ} C+V\tief{fin} [\tief{TP} t\tief{subj} [\tief{Tʹ} T [\tief{VP} \ldots{} ]]]]]
\z
\z

So as soon as the northern learners were confronted with southern
\isi{agreement weakening}, they could neither attribute it to a special position
of the verb (due to generalized V2) nor, arguably, to a special position for
subject pronouns since the evidence for differential subject positions had
become blurred. What we would like to propose is that, at this point, learners
did not discard the pattern (presumably because it was too robustly attested in
the input), but rather reanalysed it in terms of a structure where the
radically impoverished\is{impoverishment} inflectional head was endowed with
phi-features\is{φ-features} via \isi{incorporation} of the subject pronoun.
This gave rise to an early version of the \gls{NSR} that initially
distinguished between \Fpl/\Spl{} pronouns and all other subjects. The
\isi{reanalysis} of southern \isi{agreement weakening} as \isi{incorporation}
of subject \isi{clitics} led to the loss of syntactic restrictions on the
distribution of reduced endings, and \isi{agreement weakening} could be
extended to all contexts with adjacent subject pronouns (VS and SV). The result
was that the syncopated \Fpl/\Spl{} forms were not any longer confined to
operator contexts, which widened the scope of \isi{agreement weakening} to all
\Fpl/\Spl{} contexts, including preverbal pronouns in both main and embedded
clauses:

\ea
\gll \ldots{} \textbf{we} \textbf{go{-$\varnothing$}} by trouthe, noghte by syghte, þat es, \textbf{we} \textbf{lyff{-$\varnothing$}} in trouthe, noghte in bodily felynge;\\
{} we go by truth not by sight that is we live in truth not in bodily feeling\\
\glt (ROLLTR,36.752)
\z

\noindent
A further result was that the rule was extended to \Tpl{} contexts:

\ea
\gll \ldots{}  þe   penance    þat    \textbf{þai} \textbf{suffer} \ldots{}\\
{} the  penance  that  they  suffer {}\\
\glt (ROLLEP, 86.368)

\z

This extension can possibly be attributed to the fact that in the Northern ME
varieties the original \gls{OE} \Tpl{} pronoun \emph{hio/heo} was replaced by the
Scandinavian form \emph{\dh{}ai} (which later spread to all varieties). In
inversion contexts, this innovation led to cluster reduction of [s + \dh{}] to
[\dh{}] for phonetic reasons (which was possibly promoted by analogical
pressure, \Fpl{}/\Spl{}, cf.\ \citealt[56]{Pietsch:2005a}).

A closer look at morphological aspects of this change reveals that we can
indeed talk about a ``markedness reversal'' (\citealt{Pietsch:2005a}) since the
``weak'' syncopated southern \gls{OE} forms turned into the marked inflections in
the \gls{NSR}\is{Northern Subject Rule} dialects. When the zero affix entered the northern grammars via
dialect contact with the southern varieties, it was pressed into service as a
marked \isi{agreement} formative on the model of the zero inflection that occurred
with \Fsg{} subjects.  The observation that \gls{NSR}\is{Northern Subject Rule} effects appeared first in
connection with lexical verbs is perhaps related to the fact that the
underspecified \emph{s}-marker had already gained a wider distribution here,
which facilitated a reinterpretation of the zero inflection as a marked
agreement formative that contrasted with default \emph{-s}.

After the initial \isi{reanalysis}, independent changes led to the extension of the
zero affix first from \Fpl/\Spl{} to \Tpl{}, then to \Fsg{} and – in some varieties –
\Ssg{}, when the former \Spl{} \emph{you} replaced the original \Ssg{} form
\emph{thou}. Note that the latter changes led to a more balanced and less
complex \isi{agreement} system combining general ``blank generation'' of T/INFL with
a binary inventory of \isi{agreement} markers ([+phi] $\varnothing$\ vs.\ [-phi]
\emph{-s}).\footnote{See \textcite{Fuss:2010} for an analysis of relevant analogical
    changes in terms of a learning strategy that favours a minimal inventory of
    inflectional markers/features (based on the notion of \emph{Minimize Feature
Content}, \citealt{Halle:1997}).} The evidence from the \emph{York Plays}
suggests that the development of this system, which corresponds to \Cref{tab:key:10.1}, proceeded via a set of intermediate stages where blank generation of
T/INFL was restricted to certain verbs or verb classes and parts of the verbal
paradigm that had ceased to show distinctive \isi{agreement} marking.

\section{Conclusions}\label{sec:10.summary}

In this paper, we have discussed a set of open questions concerning the
synchronic analysis and diachronic development of the \gls{NSR}\is{Northern Subject Rule} in northern varieties
of English. We have presented a set of new data from the Northern ME
\emph{York Plays}, which exhibit an early stage of the \gls{NSR}\is{Northern Subject Rule} where its effects
are confined to plural forms of lexical verbs and `have', while `be' shows
regular number agreement\is{agreement!number agreement} with all kinds of subjects. We have argued that the
agreement system found in the \emph{York Plays} suggests a theoretical
analysis of the \gls{NSR}\is{Northern Subject Rule} in which inflectional heads enter the syntactic derivation
without a phi-set (due to pre-syntactic Impoverishment\is{impoverishment} leading to ``blank
generation'', \citealt{Roberts:2010}) and acquire \isi{agreement} features ([person] and
[number]) via the \isi{incorporation} of clitic subject pronouns.\is{clitics} Heads that have
been endowed with positive specifications for [person] and/or [number] in the
course of the syntactic derivation are spelled out by the zero marker.
Elswehere, the underspecified form \emph{-s} is used. Based on this account,
we have then suggested a new scenario for the historical development
of the \gls{NSR}\is{Northern Subject Rule}, arguing that, after the breakdown of the \gls{OE} \isi{agreement} system, the
\gls{NSR} developed via dialect contact between northern and southern varieties. More
precisely, we have proposed that syncopated verb forms (resulting from
Agr-weakening in the southern dialect) were integrated into the northern
grammar as marked \isi{agreement} formatives that contrasted with \emph{-s}. We
have linked the rise of the \gls{NSR}\is{Northern Subject Rule} to the interplay of a set of morphosyntactic
properties of Northern \gls{ME}\il{Middle English} (including generalized V2 and the advanced loss of
inflections), which made available a \isi{reanalysis} where southern Agr-weakening
was attributed to syntactic \isi{incorporation} of subject pronouns, which supplied a
radically impoverished\is{impoverishment} T/INFL-head with \isi{agreement} features.
This contact-induced change paved the way for an extension of the
\gls{NSR}\is{Northern Subject Rule} to adjacent pronouns more generally,
including preverbal and singular forms.

\printchapterglossary{}

\section*{Acknowledgements}

We would like to use the opportunity to express our gratitude and indebtedness
to Ian for support, linguistic insight and advice over the years (including a
set of invaluable dos and don'ts for giving an academic presentation like
``never \dots{} during your own talk''). Instead of 60 candles on a birthday
cake, we originally planned to give him 60 linguistic examples on this special
occasion, but must admit that we have fallen a bit short of that (if you want
to know the exact number, you're welcome to count!). This is when we changed
our plans and decided to present Ian with a neat analysis of the
\gls{NSR}\is{Northern Subject Rule} by using his notion of ``blank
generation''. And voilà, it works!

Earlier versions of this paper were presented at DiGS 2010 in Cambridge and
WOTM 2010 in Wittenberg. We are very grateful to the audiences for helpful
comments and suggestions. In particular, we want to thank Patrick Brandt, Nynke
de Haas, Fabian Heck, Roland Hinterhölzl, and Ans van Kemenade. In addition, we
benefited from comments by two anonymous reviewers for this volume, which led
to a number of improvements.

{\sloppy
\printbibliography[heading=subbibliography,notkeyword=this]
}

\end{document}
