\documentclass[output=paper]{langsci/langscibook}
\author{Larry M. Hyman\affiliation{University of California, Berkeley}}
\title{In search of prosodic domains in Lusoga}

% \chapterDOI{} %will be filled in at production

\epigram{%
“\dots{} the very types of prosodic category above the foot and syllable are
syntactically grounded and universal.”
(\citealt[3]{SelkirkLee2015})\\[1\baselineskip]
%
“\dots{} the prosodic phonology of \ili{Luganda} is among the most intricate and
complex of any language.” (\citealt[69]{HymanKatamba2010})}

\abstract{In this paper I raise the question of whether Lusoga, a \ili{Bantu}
    language of Uganda, recognizes syntactically determined prosodic domains,
    which have been extensively described in near-mutually intelligible
    \ili{Luganda}. I first briefly recapitulate the syntactic constructions that give
    rise to the tone group (TG) and tone\is{tone} phrase (TP) domains in
    \ili{Luganda} and then consider the same constructions in Lusoga. Whereas the
    expectation is that pre-verbal constituents will be treated prosodically
    differently than post-verbal constituents in SVO \ili{Bantu} languages, Lusoga
    treats both pre- and post-verbal constituents the same, including both
    left- and right-dislocations. While certain clitics do form a TG with the
    preceding word, perhaps forming a recursive phonological word, there is
    nothing corresponding to the multiword TG or TP of \ili{Luganda}. \ili{Lusoga} either
    fails to distinguish phonological phrases or if they do exist in the
    language (as universally claimed), \ili{Lusoga} fails to mark them. I conclude
    that linguistic typology should not only determine how universal linguistic
properties can be reflected in the grammar of a language, but also in how well
a grammar can get along without signaling them at all.}

\maketitle

\begin{document}\glsresetall

\section{Introduction}\label{sec:13.1}

The purpose of this paper is to raise the question whether the phrasal tonology\is{tone}
of \ili{Lusoga} (Bantu; Uganda), the most closely related language to \ili{Luganda}, is
syntactically grounded—or is free to apply without respect to syntax. Outside
of \ili{Bantu}, cases have been reported where phrasal or post-lexical tonology\is{tone}
applies whenever two words meet within a clause, independently of the syntax,
and hence without the need of prosodic domains\is{prosodic domains}. This
includes the VSO Chatino languages of Mexico
\parencite{Cruz2011,Campbell2014,McIntosh2015,Sullivant2015,Villard2015} and
the SOV Kuki-Thaadow language (Kuki-Chin; NE India, Myanmar) \citep{Hyman2010}.
In such languages appropriate tonal alternations occurring between words are
blocked only by pause or “sentence breaks”.

The story is considerably different in the \ili{Bantu} languages. Although there is
considerable variation, the expectation is that there will be extensive
interaction between the syntax and the prosodic phonology, specifically between
syntactic constituency and/or information structure (focus) with tone and/or
penultimate lengthening. Specifically, we expect the SVO syntax to be
prosodically reflected by an asymmetry between what precedes vs.\ follows the
verb. Thus, in a number of works on \ili{Luganda}, e.g.
\citet{HymanKatambaWalusimbi1987}, \citet{HymanKatamba2010}, we have recognized
the following postlexical domains within which tone rules act on the lexical
stem and word tones:\footnote{We also recognize an intersecting \gls{CG}, which
pertains mostly to vowel length alternations.}

\ea\label{ex:key:13.1}
    \ea a smaller \gls{TG}\is{tone}, within which H tone plateauing (\glsunset{HTP}\gls{HTP}) occurs
    \ex a larger \gls{TP}, within which H tone anticipation (\glsunset{HTA}\gls{HTA}) occurs
    \z
\z
One question is whether this sensitivity to syntax can be attributed, perhaps
universally, to the SVO syntax of \ili{Luganda} (and other \ili{Bantu} languages), or
whether the prosodic phonology of an SVO language can also apply across the
board, without any sensitivity to syntactic structure.

As I will show below, despite its near-mutual intelligibility with \ili{Luganda},
\ili{Lusoga} provides no evidence of prosodic domains\is{prosodic domains} above the phonological word.
In what follows I will first briefly identify the above \ili{Luganda} domains, then
consider the corresponding structures in \ili{Lusoga}, which show no empirical
evidence for either prosodic domain\is{prosodic domains}. I will then discuss what \ili{Lusoga} does have
and what this might mean for syntax--phonology\is{syntax--phonology interface} interactions and the quest for
universals.

\section{Prosodic domains in Luganda}\label{sec:13.2}

The analysis of \ili{Luganda} tone is given in \eqref{ex:key:13.2}, as summarized by
\citet[70]{HymanKatamba2010}:

\ea\label{ex:key:13.2}\hspace*{1.5em}\makebox[0pt][l]{\emph{level of representation}}\phantom{\hspace{4.5cm}}  \makebox[0pt][l]{\emph{tonal contrasts}}\phantom{\hspace{3cm}}   \emph{description} \\
    \ea \makebox[0pt][l]{underlying input (\glsunset{UR}\glspl{UR})}\phantom{\hspace{4.5cm}} \makebox[0pt][l]{/H, $\varnothing$/}\phantom{\hspace{3cm}}                  privative
    \ex \makebox[0pt][l]{intermediate}\phantom{\hspace{4.5cm}} \makebox[0pt][l]{H, L, $\varnothing$}\phantom{\hspace{3cm}}                 ternary
    \ex \makebox[0pt][l]{broad phonetic output}\phantom{\hspace{4.5cm}} \makebox[0pt][l]{H, L}\phantom{\hspace{3cm}}                    equipollent
    \z
\z
As indicated, moras are either marked by an underlying privative /H/ or are
toneless ($\varnothing$). Within the lexical (word-level) phonology, L tones arise in one
of two ways, illustrated in \eqref{ex:key:13.3}.

\ea\label{ex:key:13.3}
    \ea /\tn{h1}{ba}-\tn{h2}{lab}-a/  ${\rightarrow}$  \tn{h3}{bá}-\tn{l}{làb}-a  ‘they see’
        \begin{tikzpicture}[overlay, remember picture]
            \node [below=.2cm of h1.center, font=\small] {H};
            \node [below=.2cm of h2.center, font=\small] {H};
            \node [below=.2cm of h3.center, font=\small] {H};
            \node [below=.2cm of l.center, font=\small] {L};
        \end{tikzpicture}\vspace{.75\baselineskip}
    \ex /\tn{h1}{ba}-bal-a/  ${\rightarrow}$  \tn{h3}{bá}-\tn{l}{bàl}-a  ‘they count’
        \begin{tikzpicture}[overlay, remember picture]
            \node [below=.2cm of h1.center, font=\small] {H};
            \node [below=.2cm of h3.center, font=\small] {H};
            \node [below=.2cm of l.center, font=\small] {L};
        \end{tikzpicture}\vspace{.75\baselineskip}
    \z
\z
In (\ref{ex:key:13.3}a) Meeussen’s Rule converts a sequence of Hs on successive
moras to one H followed by all Ls. A sequence /H-H-H-H/ would thus become
H-L-L-L. In (\ref{ex:key:13.3}b) L tone insertion applies after a lone H which would
not be subject to Meeussen’s Rule. The result is an intermediate ternary
contrast between H, L, and $\varnothing$. Finally, after the phrasal phonology applies, the
$\varnothing$s are all filled in with either H or L, thereby bringing the system back to a
binary contrast, this time equipollent.\footnote{There also is a marginal
downstepped \ds{}H which arises when two phonological
phrases\is{phonological phrases} meet, the first ending in a HL falling tone,
the second beginning with H.}

\subsection{The TP}\label{sub:13.2.1}

We are now ready to consider the two prosodic domains\is{prosodic domains} mentioned in
\eqref{ex:key:13.1}. As illustrated in \eqref{ex:key:13.4}, within the \gls{TP},
H tone is anticipated across words onto any number of preceding toneless moras,
indicated here and in subsequent examples by underlining:\footnote{In
\eqref{ex:key:13.4} and subsequent examples \%L marks an initial boundary tone\is{phonological phrases} which
will be crucial to establishing the tone phrases in \ili{Luganda}. In
\Cref{sec:13.3} we will see that this \%L is restricted to post-pause
position in Lusoga.}

\ea\label{ex:key:13.4}
    \ea verb + object\\
        a-bal-a e-bi-k\tn{h1}{ó}p\tn{l1}{ò} ${\rightarrow}$ \tn{l2}{à}-b\underline{ál-á é-bí}-k\tn{h2}{ó}p\tn{l3}{ò} ‘s/he is counting cups’
        \begin{tikzpicture}[overlay, remember picture]
            \node [below=.2cm of h1.center, font=\small] {H};
            \node [below=.2cm of l1.center, font=\small] {L};
            \node [below=.2cm of l2.center, font=\small] {\%L};
            \node [below=.2cm of h2.center, font=\small] {H};
            \node [below=.2cm of l3.center, font=\small] {L};
        \end{tikzpicture}\vspace{.75\baselineskip}
    \ex object + object\\
        a-bal-ir-a  o-mu-limi  e-bi-k\tn{h1}{ó}p\tn{l1}{ò}  ${\rightarrow}$
            \tn{l2}{à}-b\underline{ál-ír-á ó-mú-límí  é-bí}-k\tn{h2}{ó}p\tn{l3}{ò}\\
        {\footnotesize \Tsg{}-count-\Appl-\Fv{}}\\
        ‘s/he is counting cups for the farmer’
        \begin{tikzpicture}[overlay, remember picture]
            \node [below=.2cm of h1.center, font=\small] {H};
            \node [below=.2cm of l1.center, font=\small] {L};
            \node [below=.2cm of l2.center, font=\small] {\%L};
            \node [below=.2cm of h2.center, font=\small] {H};
            \node [below=.2cm of l3.center, font=\small] {L};
        \end{tikzpicture}%\vspace{.75\baselineskip}
    \z
\z
The example in (\ref{ex:key:13.4}a) shows \gls{HTA} applying from the direct object
onto the verb, while (\ref{ex:key:13.4}b) shows \gls{HTA} from the second object
through the first object and, again, onto the verb (which is marked by the
applicative \emph{-ir-} suffix). In \eqref{ex:key:13.5} we see that \gls{HTA} also
applies between a right-dislocated element (\glsunset{RD}\gls{RD}) and the verb
and between \glspl{RD}\is{right dislocation}, again onto the verb:\footnote{Here and elsewhere it is
    important to note that, without exception, when two vowels meet across a
    word boundary, they coalesce with deletion or gliding of the first vowel
    and compensatory lengthening of the second. Thus, (\ref{ex:key:13.5}a) is
    pronounced [à-bí-bál éé-bí-kópò]. Thus, to answer one reviewer, there is no
    pause between a right dislocation and what precedes. For more on the
phonological processes involved, see \citet{Clements1986},
\citet{HymanKatamba1999}, and references cited therein.}

\ea\label{ex:key:13.5}
    \ea verb + RD\\
        a-bi-bal-a  e-bi-k\tn{h1}{ó}p\tn{l1}{ò}  ${\rightarrow}$
            \tn{l2}{à}-\underline{bí-bál-á é-bí}-k\tn{h2}{ó}p\tn{l3}{ò}\\
        {\footnotesize s/he{}-them{}-count}\\
        ‘s/he is counting them, the cups’
        \begin{tikzpicture}[overlay, remember picture]
            \node [below=.2cm of h1.center, font=\small] {H};
            \node [below=.2cm of l1.center, font=\small] {L};
            \node [below=.2cm of l2.center, font=\small] {\%L};
            \node [below=.2cm of h2.center, font=\small] {H};
            \node [below=.2cm of l3.center, font=\small] {L};
        \end{tikzpicture}%\vspace{.75\baselineskip}
    \ex \gls{RD} + RD\\
        {\footnotesize
        a-bí-mù-bal-ir-a  o-mu-limi  e-bi-k\tn{h1}{ó}p\tn{l1}{ò}
            ${\rightarrow}$ \tn{l2}{à}-\underline{bí-mù-bál-ír-á  ó-mú-límí
            é-bí}-k\tn{h2}{ó}p\tn{l3}{ò}} \\
        {\footnotesize s/he{}-them-him-count-\Appl-\Fv{}} \\
        ‘s/he is counting them for him, the farmer, the cups’
        \begin{tikzpicture}[overlay, remember picture]
            \node [below=.2cm of h1.center, font=\small] {H};
            \node [below=.2cm of l1.center, font=\small] {L};
            \node [below=.2cm of l2.center, font=\small] {\%L};
            \node [below=.2cm of h2.center, font=\small] {H};
            \node [below=.2cm of l3.center, font=\small] {L};
        \end{tikzpicture}%\vspace{.75\baselineskip}
    \z
\z

\gls{HTA} does not, however, apply from the verb onto a constituent that
precedes, whether the subject, an adverb, or a \gls{LD}:\footnote{Below in
    (\ref{ex:key:13.9}a) I will suggest that each such constituent is marked by an
    initial \%L boundary tone\is{phonological phrases} which is responsible for blocking \gls{HTA}.}

\ea\label{ex:key:13.6}
    \ea subj + verb\\
    o-mu-limi  a-bi-l\tn{h1}{á}b-\tn{l1}{à} ${\rightarrow}$
    \tn{l2}{ò}-\dashuline{mù-lìmì à-bì}-l\tn{h2}{á}b-\tn{l3}{à}  ‘the farmer sees them’
        \begin{tikzpicture}[overlay, remember picture]
            \node [below=.2cm of h1.center, font=\small] {H};
            \node [below=.2cm of l1.center, font=\small] {L};
            \node [below=.2cm of l2.center, font=\small] {\%L};
            \node [below=.2cm of h2.center, font=\small] {H};
            \node [below=.2cm of l3.center, font=\small] {L};
        \end{tikzpicture}\vspace{.75\baselineskip}
    \ex \gls{LD} + {LD}\\
        o-mu-limi e-bi-k\tn{h1}{ó}p\tn{l1}{ò} a-bi-l\tn{h2}{á}b-\tn{l2}{à}
            ${\rightarrow}$ \tn{l3}{ò}-\dashuline{mù-lìmì
                è-bì}-k\tn{h3}{ó}p\tn{l4}{ò}
                à-bì-l\tn{h4}{á}b-\tn{l5}{à}\\\vspace{1\baselineskip}
        \begin{tikzpicture}[overlay, remember picture]
            \node [below=.2cm of h1.center, font=\small] {H};
            \node [below=.2cm of l1.center, font=\small] {L};
            \node [below=.2cm of l2.center, font=\small] {\%L};
            \node [below=.2cm of h2.center, font=\small] {H};
            \node [below=.2cm of l3.center, font=\small] {L};
            \node [below=.2cm of h3.center, font=\small] {L};
            \node [below=.2cm of l4.center, font=\small] {\%L};
            \node [below=.2cm of h4.center, font=\small] {H};
            \node [below=.2cm of l5.center, font=\small] {L};
        \end{tikzpicture}%
        ‘the farmer, the cups, he sees them’
    \z
\z
As indicated by the dashed underlining, (\ref{ex:key:13.6}a) shows that \gls{HTA}
does not apply from the verb onto the subject \emph{ò-mù-lìmì}, which instead
receives default L tones. Nor is there \gls{HTA} from one \gls{LD} onto another in
(\ref{ex:key:13.6}b).\footnote{Note in (\ref{ex:key:13.6}b) that \gls{HTA} does not
    apply between \emph{e-bi-kópò} ‘cups’ and \emph{a-bi-láb-à} ‘he sees them’
because the former ends in a L tone. For \gls{HTA} to apply, the preceding word
must end with a toneless vowel.}  Instead, \glspl{LD}\is{left dislocation} and other pre-verbal
constituents are marked off in a way that post-verbal constituents including
RDs are not.\footnote{Again, not shown is the V\#V coalescence that
automatically applies between any words in sequence, including \glspl{LD}\is{left dislocation} and RDs, but
does not affect the tonal discussion.}

Before accounting for this fact let us consider the opposite marking of
dislocations in closely related Haya
(\citealt[201--2]{ByarushengoHymanTenenbaum1976};
\citealt[155]{HymanKatamba1999}). In this language a /H-$\varnothing$/ sequence is realized
[HL-L] at the end of a tone phrase, e.g.\ in isolation:

\ea\label{ex:key:13.7}
    \ea a-ba-k\tn{h1}{á}zi  ${\rightarrow}$  à-bà-k\tn{hl}{â}zì  ‘woman’ \\
        \begin{tikzpicture}[overlay, remember picture]
            \node [below=.2cm of h1.center, font=\small] {H};
            \node [below=.2cm of hl.center, font=\small] {HL};
        \end{tikzpicture}%
    \ex e-m-b\tn{h1}{ú}zi  ${\rightarrow}$  e-m-b\tn{hl}{û}zi  ‘goat(s)’ \\
        \begin{tikzpicture}[overlay, remember picture]
            \node [below=.2cm of h1.center, font=\small] {H};
            \node [below=.2cm of hl.center, font=\small] {HL};
        \end{tikzpicture}%
    \z
\z

Noting this, we now see in \eqref{ex:key:13.8} that \ili{Haya} presents a near
mirror-image of \ili{Luganda} (we can ignore the “augment” initial vowel H on the
nouns):

\ea\label{ex:key:13.8}
    \ea base sentence:\quad a-ba-k\tn{h1}{á}zi  ni-ba-bal-\tn{h2}{í}l-a
ó-mw-\tn{h3}{á}na  é-m-b\tn{hl}{û}zi\\\vspace{1\baselineskip}
        \begin{tikzpicture}[overlay, remember picture]
            \node [below=.2cm of h1.center, font=\small] {H};
            \node [below=.2cm of h2.center, font=\small] {H};
            \node [below=.2cm of h3.center, font=\small] {H};
            \node [below=.2cm of hl.center, font=\small] {HL};
        \end{tikzpicture}%
        ‘the women are counting the goats for the child’
    \ex \makebox[0pt][l]{three \glspl{LD}\is{left dislocation}:}\phantom{base sentence:\quad} a-ba-k\tn{h1}{á}zi  ó-mw-\tn{h2}{á}na
        é-m-b\tn{h3}{ú}zi ni-ba-zi-mu-bal-\tn{hl}{î}l-a\\\vspace{1\baselineskip}
        \begin{tikzpicture}[overlay, remember picture]
            \node [below=.2cm of h1.center, font=\small] {H};
            \node [below=.2cm of h2.center, font=\small] {H};
            \node [below=.2cm of h3.center, font=\small] {H};
            \node [below=.2cm of hl.center, font=\small] {HL};
        \end{tikzpicture}%
        ‘the women, the child, the goats, they are counting them for him’
    \ex \makebox[0pt][l]{three \glspl{RD}\is{right dislocation}:}\phantom{base sentence:\quad}  ni-ba-zi-mu-bal-\tn{hl1}{î}l-a
        á-ba-k\tn{hl2}{â}zi  ó-mw-\tn{hl3}{â}na é-m-b\tn{hl4}{û}zi\\\vspace{1\baselineskip}
        \begin{tikzpicture}[overlay, remember picture]
            \node [below=.2cm of hl1.center, font=\small] {HL};
            \node [below=.2cm of hl2.center, font=\small] {HL};
            \node [below=.2cm of hl3.center, font=\small] {HL};
            \node [below=.2cm of hl4.center, font=\small] {HL};
        \end{tikzpicture}%
        ‘they are counting them for him, the women, the child, the goats’
    \z
\z
The base sentence is given in (\ref{ex:key:13.8}a). In (\ref{ex:key:13.8}b) we see
that the /H/ of \glspl{LD}\is{left dislocation} is not affected, while in (\ref{ex:key:13.8}c), the /H/
of the verb and each \gls{RD} becomes HL. \glspl{RD}\is{right dislocation} are thus each marked off,
while \glspl{LD}\is{left dislocation} are not. The two languages are thus analyzed with the reverse
nested structures in \eqref{ex:key:13.9} (\citealt[84]{ByarushengoHymanTenenbaum1976};
\citealt{HymanKatamba2010}).

\newpage

\ea\label{ex:key:13.9}
    \ea \ili{Luganda} marks beginnings of complete Us\\
        \begin{tikzpicture}[baseline]

            \tikzset{frontier/.style={distance from root=100pt}}

            \Tree   [.U
                        \node (l1) {farmer\tss{i}};
                        [.U
                            \node (l2) {cups\tss{j}};
                            [.U
                                \edge [roof]; {she\tss{i}-them\tss{j}-count}
                            ]
                        ]
                    ]

            \node [below=.2cm of l1.center] {\%L};
            \node (cups) [below=.2cm of l2.center] {\%L};
            \node [right=.65cm of cups.center] {\%L};

        \end{tikzpicture}

    \ex \ili{Haya} marks ends of complete Us\\
        \begin{tikzpicture}[baseline]

            \tikzset{frontier/.style={distance from root=100pt}}

            \Tree   [.U
                        [.U
                            [.U
                                \edge [roof]; {she\tss{i}-them\tss{j}-count}
                            ]
                            \node (l1) {farmer\tss{i}};
                        ]
                        \node (l2) {cups\tss{j}};
                    ]

            \node (farmer) [below=.2cm of l1.center] {\%L};
            \node [below=.2cm of l2.center] {\%L};
            \node [left=1.00cm of farmer.center] {\%L};

        \end{tikzpicture}
    \z
\z
In \eqref{ex:key:13.9} I have labeled each complete syntactic utterance with U.
Luganda thus marks the beginning of each U with a \%L boundary tone\is{phonological phrases}, while Haya
marks the end of each U with a final L\% boundary tone\is{phonological phrases}, one of whose effects is
to convert a penultimate H into HL. As \citet{ByarushengoHymanTenenbaum1976} point out,
each L\% correlates with the end of a complete assertion.

Before moving on to the tone group, it should perhaps be pointed out that if
the TP correlates with the phonological (or even intonational)
phrase\is{phonological phrases} of
prosodic domain\is{prosodic domains} theory, we don’t expect to find a TP break
within a simple noun phrase. While this is largely the case, there is a problem
with \isi{numerals} in \ili{Luganda}:

\ea\label{ex:key:13.10}
    \ea noun + adjective:\\
        a-ba-limi  a-ba-n\tn{h1}{é}n\tn{l1}{è}  ${\rightarrow}$
            \tn{l2}{à}-\underline{bá-límí  á-bá}-n\tn{h2}{é}n\tn{l3}{è}  ‘big farmers’
        \begin{tikzpicture}[overlay, remember picture]
            \node [below=.2cm of h1.center, font=\small] {H};
            \node [below=.2cm of l1.center, font=\small] {L};
            \node [below=.2cm of l2.center, font=\small] {\%L};
            \node [below=.2cm of h2.center, font=\small] {H};
            \node [below=.2cm of l3.center, font=\small] {L};
        \end{tikzpicture}\vspace{.75\baselineskip}
    \ex noun + numeral:\\
        a-ba-limi ba-s\tn{h1}{á}t\tn{l1}{ù}  ${\rightarrow}$  \tn{l2}{à}-b\dashuline{à-lìmì
    bà}-s\tn{h2}{á}t\tn{l3}{ù}  ‘three farmers’
        \begin{tikzpicture}[overlay, remember picture]
            \node [below=.2cm of h1.center, font=\small] {H};
            \node [below=.2cm of l1.center, font=\small] {L};
            \node [below=.2cm of l2.center, font=\small] {\%L};
            \node [below=.2cm of h2.center, font=\small] {H};
            \node [below=.2cm of l3.center, font=\small] {L};
        \end{tikzpicture}\vspace{.75\baselineskip}
    \z
\z
As expected, \gls{HTA} applies in (\ref{ex:key:13.10}a) from an adjective onto a
preceding noun. However, \gls{HTA} does not apply in (\ref{ex:key:13.10}b) from
the numeral onto the noun. It is as if the noun is in a separate TP, as in the
case of a preverbal constituent. I don’t see any reason to think of numerals
as predicative, such that ‘farmers’ would be preposed to the numeral (as a
subject is to the verb marked by \%L). While it is hard to motivate
syntactically, the apparent need is for there to be an analogous \%L separating
the numeral from the preceding noun. This being said, \ili{Bantu} languages that
allow a subset of modifiers to be either pre- or post-nominal, e.g.
demonstratives \parencite{vandeVelde2005}, may also not phrase\is{phonological
phrases} them with the head noun.

\subsection{The TG}\label{sub:13.2.2}

The \gls{TG}\is{tone} is a smaller domain in which the head V or N of the corresponding
XP undergoes reduction when followed by an appropriate dependent with H tone.
In Haya, the V or N undergoes deletion of its one or more H tones, while in
Luganda, the V or N loses the L(s) of a H to L pitch drop, as the result of a
process of H tone plateauing (\glsunset{HTP}\gls{HTP}). For this to occur
several conditions must be met, as schematized in \eqref{ex:key:13.11}
(\citealt[75]{HymanKatamba2010}):

\begin{exe}
    \ex\label{ex:key:13.11} \begin{tikzpicture}[baseline]

        \Tree 	[.\node(XP){XP};
                    X
                    [.YP
                        \node (z) {Z};
                    ]
                ]

            \node (wh) [right=2cm of XP.south east, align=left]
                    {where: \begin{tabular}{ll}
                                (i) & X ≠ [+\textsc{focus}] \\
                                (ii) & Z ≠ [+\textsc{augment}]
                            \end{tabular}};

            \node at (z -| wh.west) [anchor=west, align=left]
                    {Z = a phonological word};

    \end{tikzpicture}
\end{exe}
In \eqref{ex:key:13.11}, Z stands for a phonological word (\glsunset{PW}\gls{PW})
which is not necessarily the head of YP (as when there is an empty head, e.g.
‘we saw two’).  The [±\textsc{focus}] feature refers to whether a verb
\gls{TAM}/polarity is inherently focused.\is{focus} The following pair of examples shows
that negation is inherently [+focus] (cf.\ \citealt{HymanWatters1984}):

\ea\label{ex:key:13.12}
    \ea tw-\tn{h1}{áá}-l\tn{l1}{à}b-\tn{l2}{à}  ${\rightarrow}$
        tw-\underline{\tn{h2}{áá}-l\tn{o1}{á}b-\tn{o2}{á}
        bí-k\tn{h3}{ó}}p\tn{l3}{ò}  ‘we saw \textsc{cups}’\hfill(Past\tss{2})
        \begin{tikzpicture}[overlay, remember picture]
            \node [below=.2cm of h1.center, font=\small] {H};
            \node [below=.2cm of l1.center, font=\small] {L};
            \node [below=.2cm of l2.center, font=\small] {L};
            \node [below=.2cm of h2.center, font=\small] {H};
            \node [below=.2cm of o1.center, font=\small] {$\varnothing$};
            \node [below=.2cm of o2.center, font=\small] {$\varnothing$};
            \node [below=.2cm of h3.center, font=\small] {H};
            \node [below=.2cm of l3.center, font=\small] {L};
        \end{tikzpicture}\vspace{.75\baselineskip}
    \ex te-tw-\tn{h1}{áá}-l\tn{l1}{à}b-\tn{l2}{à}  ${\rightarrow}$
        tè-tw-\tn{h2}{áá}-l\tn{l3}{à}b-\tn{l4}{à}  bì-k\tn{h3}{ó}p\tn{l5}{ò}
        ‘we didn’t see cups’\hfill(Past\tss{2})
        \begin{tikzpicture}[overlay, remember picture]
            \node [below=.2cm of h1.center, font=\small] {H};
            \node [below=.2cm of l1.center, font=\small] {L};
            \node [below=.2cm of l2.center, font=\small] {L};
            \node [below=.2cm of h2.center, font=\small] {H};
            \node [below=.2cm of l3.center, font=\small] {\underline{L}};
            \node [below=.2cm of l4.center, font=\small] {\underline{L}};
            \node [below=.2cm of h3.center, font=\small] {H};
            \node [below=.2cm of l5.center, font=\small] {L};
        \end{tikzpicture}\vspace{.75\baselineskip}
    \z
\z
In (\ref{ex:key:13.12}a) the Hs of the verb and object create an all-H plateau,
requiring the Ls of the verb to be deleted (indicated by $\varnothing$). (As
glossed, \isi{focus}
is on \emph{bí-kópò} ‘cups’, marked by the absence of the augment \emph{e-.})
However, H tone plateauing (\glsunset{HTP}\gls{HTP}) does not apply in
(\ref{ex:key:13.12}b), where the only grammatical difference is the negative
marking on the verb.\footnote{\gls{HTA} also does not apply since it must cross
a word boundary, but it cannot do so when the preceding word ends L (vs.\ $\varnothing$).}

The [±\textsc{augment}] feature refers to whether a noun has an augment,
usually an initial \emph{e-, o-} or \emph{a-}. As seen in
(\ref{ex:key:13.13}a), \gls{HTP} will not apply if the augment is present.
(\ref{ex:key:13.13}b) shows that the augment is obligatorily absent after a
negative verb (without any \isi{focus} effect), as it was in (\ref{ex:key:13.12}b)
above.

\ea\label{ex:key:13.13}
    \ea \makebox[0pt][l]{tw-\tn{h1}{áá}-l\tn{l1}{à}b-\tn{l2}{à}}\phantom{te-tw-áá-làb-à}  ${\rightarrow}$
        tw-\tn{h2}{áá}-l\tn{l3}{à}b-\tn{l4}{à}
        \underline{è}-bì-k\tn{h3}{ó}p\tn{l5}{ò}  ‘we saw cups’\hfill(Past\tss{2})
        \begin{tikzpicture}[overlay, remember picture]
            \node [below=.2cm of h1.center, font=\small] {H};
            \node [below=.2cm of l1.center, font=\small] {L};
            \node [below=.2cm of l2.center, font=\small] {L};
            \node [below=.2cm of h2.center, font=\small] {H};
            \node [below=.2cm of h3.center, font=\small] {H};
            \node [below=.2cm of l3.center, font=\small] {L};
            \node [below=.2cm of l4.center, font=\small] {L};
            \node [below=.2cm of l5.center, font=\small] {L};
        \end{tikzpicture}\vspace{.75\baselineskip}
        \ex te-tw-\tn{h1}{áá}-l\tn{l1}{à}b-\tn{l2}{à} ${\rightarrow}$
        \llap{*}tè-tw-\tn{h2}{áá}-l\tn{l3}{à}b-\tn{l4}{à}
        \underline{è}-bì-k\tn{h3}{ó}p\tn{l5}{ò}  ‘we didn’t see cups’\hfill(Past\tss{2})
        \begin{tikzpicture}[overlay, remember picture]
            \node [below=.2cm of h1.center, font=\small] {H};
            \node [below=.2cm of l1.center, font=\small] {L};
            \node [below=.2cm of l2.center, font=\small] {L};
            \node [below=.2cm of h2.center, font=\small] {H};
            \node [below=.2cm of h3.center, font=\small] {H};
            \node [below=.2cm of l3.center, font=\small] {L};
            \node [below=.2cm of l4.center, font=\small] {L};
            \node [below=.2cm of l5.center, font=\small] {L};
        \end{tikzpicture}\vspace{.75\baselineskip}
    \z
\z

Within the verb phrase the YP can be anything as long as it isn’t
[\textsc{+augment}] (or a RD). This includes an object NP, prepositional
phrase, adverb etc. Within the noun phrase, plateauing occurs only in (some)
\isi{compounding} (\citealt{HymanKatamba2005}) and before a possessive/genitive NP.
In \eqref{ex:key:13.14} we see that \gls{HTP} does not apply between a noun and following
adjective (possibly because adjectives are not YPs):

\ea\label{ex:key:13.14}
    \ea   N + A e-bi-k\tn{h1}{ó}p\tn{l1}{ò} ${\rightarrow}$
            e-bi-k\tn{h2}{ó}p\tn{l2}{ò} è-bì-n\tn{h3}{é}n\tn{l3}{è} ‘big cups’\\
        \begin{tikzpicture}[overlay, remember picture]
            \node [below=.2cm of h1.center, font=\small] {H};
            \node [below=.2cm of l1.center, font=\small] {L};
            \node [below=.2cm of l2.center, font=\small] {\underline{L}};
            \node [below=.2cm of h2.center, font=\small] {H};
            \node [below=.2cm of h3.center, font=\small] {H};
            \node [below=.2cm of l3.center, font=\small] {L};
        \end{tikzpicture}\vspace{.75\baselineskip}
    \ex   \hphantom{N + A}
        \makebox[0pt][l]{bi-k\tn{h1}{ó}p\tn{l1}{ò}}\phantom{e-bi-kópò}
        ${\rightarrow}$ te-tw-\tn{h2}{áá}-l\tn{l2}{à}b-\tn{l3}{à}
        bì-k\tn{h3}{ó}p\tn{l4}{ò}  bì-n\tn{h4}{é}n\tn{l5}{è}\\\vspace{1\baselineskip}
        \begin{tikzpicture}[overlay, remember picture]
            \node [below=.2cm of h1.center, font=\small] {H};
            \node [below=.2cm of l1.center, font=\small] {L};
            \node [below=.2cm of l2.center, font=\small] {L};
            \node [below=.2cm of h2.center, font=\small] {H};
            \node [below=.2cm of h3.center, font=\small] {H};
            \node [below=.2cm of l3.center, font=\small] {L};
            \node [below=.2cm of h4.center, font=\small] {H};
            \node [below=.2cm of l4.center, font=\small] {L};
            \node [below=.2cm of l5.center, font=\small] {L};
        \end{tikzpicture}%
        \hphantom{N + A e-bi-kópò $\rightarrow$} ‘we didn’t see big cups’
    \z
\z
While this could also be attributed to the augment on \emph{è-bì-nénè} ‘big’ in
(\ref{ex:key:13.14}a), the non-plateauing in the absence of the augment after the
negative verb in (\ref{ex:key:13.14}b) unambiguously shows that N + A fails to
become a \gls{TG}\is{tone}. The examples in \eqref{ex:key:13.15} show that a possessive pronoun and
genitive noun will form a \gls{TG}\is{tone} with the preceding head noun:\largerpage

\ea\label{ex:key:13.15}
    \ea {N + Poss}\\
        e-bi-k\tn{h1}{ó}p\tn{l1}{ò} ${\rightarrow}$
        e-bi-k\tn{h2}{ó}p\tn{o1}{\underline{ó}} by-\tn{hl}{ê}\hfill‘his/her cups’\\
        \begin{tikzpicture}[overlay, remember picture]
            \node [below=.2cm of h1.center, font=\small] {H};
            \node [below=.2cm of l1.center, font=\small] {L};
            \node [below=.2cm of h2.center, font=\small] {H};
            \node [below=.2cm of o1.center, font=\small] {$\varnothing$};
            \node [below=.2cm of hl.center, font=\small] {HL};
        \end{tikzpicture}\\\vspace{.75\baselineskip}
    \ex N + GenN\\
        e-bi-k\tn{h1}{ó}p\tn{l1}{ò} ${\rightarrow}$
        e-bi-k\tn{h2}{ó}p\underline{\tn{o1}{ó} by-áá=Kátáá}mb\tn{hl}{â}\hfill‘Katamba’s cups’\\
        \begin{tikzpicture}[overlay, remember picture]
            \node [below=.2cm of h1.center, font=\small] {H};
            \node [below=.2cm of l1.center, font=\small] {L};
            \node [below=.2cm of h2.center, font=\small] {H};
            \node [below=.2cm of o1.center, font=\small] {$\varnothing$};
            \node [below=.2cm of hl.center, font=\small] {HL};
        \end{tikzpicture}\\\vspace{.75\baselineskip}
    \z
\z
In (\ref{ex:key:13.15}a) the final L of ‘cups’ is deleted as a result of
plateauing with the HL of \emph{by-ê} ‘his/her’. The same occurs in
(\ref{ex:key:13.15}b), where there is plateauing with the HL of proper noun,
pronounced \emph{Kàtààmbâ} in isolation.

It is important to note that the \gls{TG}\is{tone} is a relation of the head and one
word (Z) to its right. That is, the full YP in \eqref{ex:key:13.11} does not join
the head X to form the \gls{TG}\is{tone}. This is illustrated in \eqref{ex:key:13.16}.

\ea\label{ex:key:13.16}
    \ea   tw-\tn{h1}{áá}-l\tn{l1}{à}b-\tn{l2}{à} ${\rightarrow}$
    tw-\underline{\tn{h2}{áá}-l\tn{o1}{á}b-\tn{o2}{á}
    bí-k\tn{h3}{ó}}p\tn{l3}{ò} bi-n\tn{h4}{énè}  ‘we saw \textsc{big cups}’
        (Past\tss{2})\\
        \begin{tikzpicture}[overlay, remember picture]
            \node [below=.2cm of h1.center, font=\small] {H};
            \node [below=.2cm of l1.center, font=\small] {L};
            \node [below=.2cm of l2.center, font=\small] {L};
            \node [below=.2cm of h2.center, font=\small] {H};
            \node [below=.2cm of o1.center, font=\small] {$\varnothing$};
            \node [below=.2cm of o2.center, font=\small] {$\varnothing$};
            \node [below=.2cm of h3.center, font=\small] {H};
            \node [below=.2cm of l3.center, font=\small] {\underline{L}};
            \node [below=.2cm of h4.center, font=\small] {H-L};
        \end{tikzpicture}\\\vspace{.75\baselineskip}
        \ex tw-\tn{h1}{áá}-l\tn{l1}{à}b-\tn{l2}{à} ${\rightarrow}$
        tw-\tn{h2}{áá}-l\tn{l3}{à}b-\tn{l4}{à} bì-tábó bí-n\tn{h3}{énè}  ‘we saw \textsc{big books}’  (Past\tss{2})\\
        \begin{tikzpicture}[overlay, remember picture]
            \node [below=.2cm of h1.center, font=\small] {H};
            \node [below=.2cm of l1.center, font=\small] {L};
            \node [below=.2cm of l2.center, font=\small] {L};
            \node [below=.2cm of h2.center, font=\small] {H};
            \node [below=.2cm of l3.center, font=\small] {\underline{L}};
            \node [below=.2cm of l4.center, font=\small] {\underline{L}};
            \node [below=.2cm of h3.center, font=\small] {H-L};
        \end{tikzpicture}\\\vspace{.75\baselineskip}
    \z
\z
In (\ref{ex:key:13.16}a) there is plateauing between the verb and ‘cups’, which
maintains its H-L pitch drop before the H-L of the adjective ‘big’. In
(\ref{ex:key:13.16}b) the verb joins with \emph{bi-tabo} ‘cups’, but since the
latter is underlyingly toneless there is no possibility of H tone plateauing.
Crucially, the verb cannot “see” the H of the adjective ‘big’. The Hs that are
observed on \emph{bì-tábó} result from \gls{HTA} within the larger TP domain.

However, there are cases where a H tone plateau can encompass several words.
The following examples show that \gls{HTP} can affect sequences of Head-Dependent
words without respect to bracketing \citep[159]{Hyman1988}:

\ea\label{ex:key:13.17}
    \ea e-bi-k\tn{h1}{ó}p\tn{o1}{ó}  by-áá  mú-g\tn{h2}{áá}nd\tn{o2}{á}
    w-áá=Kátáámb\tn{hl}{â}\hfill[ N\tss{1} [ N\tss{2} N\tss{3}] ]
    \\\vspace{1\baselineskip}
        \begin{tikzpicture}[overlay, remember picture]
            \node [below=.2cm of h1.center, font=\small] {H};
            \node [below=.2cm of h2.center, font=\small] {H};
            \node [below=.2cm of o1.center, font=\small] {$\varnothing$};
            \node [below=.2cm of o2.center, font=\small] {$\varnothing$};
            \node [below=.2cm of hl.center, font=\small] {HL};
        \end{tikzpicture}%
        ‘cups of brother of Katamba’
        \ex e-bi-k\tn{h1}{ó}p\tn{o1}{ó}  by-áá=k\tn{h2}{áá}w\tn{o2}{á} by-áá=Kátáámb\tn{hl}{â}\hfill[ [ N\tss{1} N\tss{2} ] N\tss{3} ]\\\vspace{1\baselineskip}
        \begin{tikzpicture}[overlay, remember picture]
            \node [below=.2cm of h1.center, font=\small] {H};
            \node [below=.2cm of h2.center, font=\small] {H};
            \node [below=.2cm of o1.center, font=\small] {$\varnothing$};
            \node [below=.2cm of o2.center, font=\small] {\underline{$\varnothing$}};
            \node [below=.2cm of hl.center, font=\small] {HL};
        \end{tikzpicture}%
        ‘cups of coffee of Katamba’
    \z
\z
The more common right-branching structure is observed in (\ref{ex:key:13.17}a). In
this case N\tss{2} + N\tss{3} form a constituent which then joins N\tss{1}. In
the less common left-branching structure in (\ref{ex:key:13.17}b), N\tss{1} +
N\tss{2} first form a constituent, which then joins N\tss{3}. Although a
single, three-word \gls{TG}\is{tone} is formed, \gls{HTP} does not apply to the whole
constituent all at once. This is seen from the fact that an intervening
toneless phonological word blocks \gls{HTP} \citep[157]{Hyman1988}. In the
following examples, underlined Hs are from the application of \gls{HTA}:

\ea\label{ex:key:13.18}
    \ea   e-bi-k\tn{h1}{ó}p\tn{l1}{ò}  by-àà  mù-\underline{túúndá +
    bí}-k\tn{h2}{ó}p\tn{l2}{ò}\hfill  [ N\tss{1} [ N\tss{2} N\tss{3}]
    ]\\\vspace{1\baselineskip}
        \begin{tikzpicture}[overlay, remember picture]
            \node [below=.2cm of h1.center, font=\small] {H};
            \node [below=.2cm of h2.center, font=\small] {H};
            \node [below=.2cm of l1.center, font=\small] {\underline{L}};
            \node [below=.2cm of l2.center, font=\small] {L};
        \end{tikzpicture}%
		 ‘cups of the cup-seller’ (literally, seller-cups)
         \ex mu-k\tn{h1}{ú}b\tn{l1}{à} + bà-l\underline{ímí  w-áá
             Kátáá}mb\tn{hl}{â}\hfill [ [ N\tss{1} N\tss{2} ] N\tss{3}
             ]\\\vspace{1\baselineskip}
        \begin{tikzpicture}[overlay, remember picture]
            \node [below=.2cm of h1.center, font=\small] {H};
            \node [below=.2cm of l1.center, font=\small] {\underline{L}};
            \node [below=.2cm of hl.center, font=\small] {HL};
        \end{tikzpicture}%
    ‘farmer-beater of Katamba” (literally, beater-farmers)
    \z
\z
Even though the same right- and left-branching complex \glspl{TG} are formed,
\gls{HTP} must progress on a word-by-word basis. For this reason I proposed
that \gls{HTP} be a domain-juncture rule of the following form
\citep[158]{Hyman1988}:

\ea\label{ex:key:13.19}
    L\textsuperscript{n}  ${\rightarrow}$  $\varnothing$  /  [  \tss{TG}[ \dots{} \tss{PW}[ \dots{} H \_\_ ] [ H \dots{} ]\tss{PW} \dots{} ]\tss{TG}
\z
Presented as a rule of L tone deletion followed by the fusion of the left and
right H tones, the conception is that \gls{HTP} occurs between \glspl{PW} which
are grouped together within a \gls{TG}\is{tone}.\footnote{A perhaps equivalent
    alternative is that \glspl{TG} are nested.}

In summary, the above and other \ili{Luganda} facts potentially bear on multiple
issues concerning prosodic domain\is{prosodic domains} theory vs.\ direct reference to syntax, the
nature and number of prosodic domains\is{prosodic domains} (\gls{TP}, \gls{TG}\is{tone}, and ultimately the
\gls{CG}), the potential interaction between domains (domain juncture effects,
nesting), and the interaction of prosodic domains\is{prosodic domains} with information structure
(\isi{focus}). With all of this hyper-activity in \ili{Luganda}, we now turn to consider
the equivalent structures in closely related Lusoga.

\section{Prosodic domains in \ili{Lusoga} (?)}\label{sec:13.3}

In \ili{Lusoga} the most striking property is a historical process of \gls{HTR} onto
the preceding mora. In the following examples \%L is an initial boundary tone\is{phonological phrases},
and H\% is the declarative phrase-final boundary tone\is{phonological phrases} (which also occurs, but
is variable in \ili{Luganda}):\is{phonological phrases}

\NumTabs{4}
\ea\label{ex:key:13.20}
    \ea \tn{l1}{ò}-kú-lágír-\tn{h1}{á} \tab{‘to command’} \tab{cf.\
            \ili{Luganda} \tn{l2}{ò}-kú-lágír-\tn{h2}{á}}\\
        \begin{tikzpicture}[overlay, remember picture]
            \node [below=.2cm of h1.center, font=\small] {H\%};
            \node [below=.2cm of h2.center, font=\small] {(H\%)};
            \node [below=.2cm of l1.center, font=\small] {\llap{\%}L};
            \node [below=.2cm of l2.center, font=\small] {\%L};
        \end{tikzpicture}\vspace{.25\baselineskip}
    \ex \tn{l1}{ò}-k\tn{h1}{ú}-gh\tn{l2}{ù}l\tn{l3}{ì}r-\tn{h2}{á} \tab{‘to
        hear’} \tab{cf.\ \ili{Luganda}
    \tn{l4}{ò}-kù-w\tn{h3}{ú}l\tn{l5}{ì}r-\tn{h4}{á}}\\
        \begin{tikzpicture}[overlay, remember picture]
            \node [below=.2cm of h1.center, font=\small] {\underline{H}};
            \node [below=.2cm of h2.center, font=\small] {H\%};
            \node [below=.2cm of h3.center, font=\small] {H};
            \node [below=.2cm of h4.center, font=\small] {\hspace{1mm}(H\%)};
            \node [below=.2cm of l1.center, font=\small] {\llap{\%}L};
            \node [below=.2cm of l2.center, font=\small] {L};
            \node [below=.2cm of l3.center, font=\small] {L};
            \node [below=.2cm of l4.center, font=\small] {\%L};
            \node [below=.2cm of l5.center, font=\small] {L};
        \end{tikzpicture}\vspace{.25\baselineskip}
    \z
\z
The infinitive in (\ref{ex:key:13.20}a) is lexically toneless, realized L-H-H-H-H by
mapping \%L to the first mora, and H\% to the remaining moras. The \ili{Luganda}
realization is either the same, or all L if the variable H\% is not chosen. In
contrast, the verb root has an underlying tone in (\ref{ex:key:13.20}b). In this
case the \ili{Luganda} form is more straightforward: The verb base \emph{-wúlir-}
‘hear’ has an underlying /H/ on its first mora, which as seen earlier in
(\ref{ex:key:13.3}b) then conditions L tone insertion on the second mora. The
remaining toneless moras receive L tone, unless H\% is realized, in which case
the output is \emph{ò-kù-wúlìr-á}, with a final H. In Lusoga, instead, the H is
realized on the preceding infinitive prefix \emph{{}-kú-} followed by two L
tone moras. The H tone of the verb root clearly has shifted onto the preceding
mora. The historical derivation is presented in \eqref{ex:key:13.21}.

\begin{exe}\label{ex:key:13.21}
    \ex
    {\small
    \begin{tabularx}{\textwidth}[t]{@{}llllllll@{}}
        \multicolumn{1}{c}{\emph{stage 1}}                   & \multicolumn{1}{c}{\emph{stage 2}}                          & \multicolumn{1}{c}{\emph{stage 3}}                                    & \multicolumn{1}{c}{\emph{stage 4}} \\
        \multicolumn{1}{@{}l@{ >}}{o-ku-gh\tn{h1}{ú}lir-a}   & \multicolumn{1}{l@{ >}}{o-ku-gh\tn{h2}{ú}l\tn{l1}{ì}r-a}    & \multicolumn{1}{l@{ >}}{o-k\tn{h3}{ú}-gh\tn{l2}{ù}l\tn{l3}{ì}r-a}     & \tn{l4}{ò}-k\tn{h4}{ú}-gh\tn{l5}{ù}l\tn{l6}{ì}r-\tn{h5}{á}       & \hspace{-.75em}‘to hear’\\
        \addlinespace[2.5ex]
        \multicolumn{1}{@{}l@{ >}}{o-ku-k\tn{h1a}{á}lakat-a} & \multicolumn{1}{l@{ >}}{o-ku-k\tn{h2a}{á}l\tn{l1a}{à}kat-a} & \multicolumn{1}{l@{ >}}{o-k\tn{h3a}{ú}-k\tn{l2a}{à}l\tn{l3a}{à}kat-a} & \tn{l4a}{ò}-k\tn{h4a}{ú}-k\tn{l5a}{à}l\tn{l6a}{à}kát-\tn{h5a}{á} & \hspace{-.75em}‘to scrape’\\
        \addlinespace[2.5ex]
    \end{tabularx}}
    \begin{tikzpicture}[overlay, remember picture]
        \node [below=.1cm of h1.center, font=\smaller] {H};
        \node [below=.1cm of h2.center, font=\smaller] {H};
        \node [below=.1cm of h3.center, font=\smaller] {H};
        \node [below=.1cm of h4.center, font=\smaller] {H};
        \node [below=.1cm of h5.center, font=\smaller] {H\%};
        \node [below=.1cm of l1.center, font=\smaller] {L};
        \node [below=.1cm of l2.center, font=\smaller] {L};
        \node [below=.1cm of l3.center, font=\smaller] {L};
        \node [below=.1cm of l4.center, font=\smaller] {\%L};
        \node [below=.1cm of l5.center, font=\smaller] {L};
        \node [below=.1cm of l6.center, font=\smaller] {L};

        \node [below=.1cm of h1a.center, font=\smaller] {H};
        \node [below=.1cm of h2a.center, font=\smaller] {H};
        \node [below=.1cm of h3a.center, font=\smaller] {H};
        \node [below=.1cm of h4a.center, font=\smaller] {H};
        \node [below=.1cm of h5a.center, font=\smaller] {H\%};
        \node [below=.1cm of l1a.center, font=\smaller] {L};
        \node [below=.1cm of l2a.center, font=\smaller] {L};
        \node [below=.1cm of l3a.center, font=\smaller] {L};
        \node [below=.1cm of l4a.center, font=\smaller] {\%L};
        \node [below=.1cm of l5a.center, font=\smaller] {L};
        \node [below=.1cm of l6a.center, font=\smaller] {L};

    \end{tikzpicture}
\end{exe}
At stage 1 we start with a H tone on the first mora of the verb base. Stage 2
represents the L tone insertion rule that was discussed with regard to \ili{Luganda},
but which characterizes both languages. Stage 3 is where H tone retraction
(\glsunset{HTR}\gls{HTR}) applies in \ili{Lusoga} only. As seen, I have indicated a L
tone phonological “trace” on the original root-initial H tone mora in stage 3.

While \eqref{ex:key:13.21} is historically correct, the proposed synchronic
analysis is that *H is now /L/. In other words, the \ili{Lusoga} tone contrast has
become /L/ vs.\ $\varnothing$ \citep{Hyman2018}:

\begin{exe}\ex\label{ex:key:13.22}
    \begin{minipage}[t]{.50\textwidth}
        \begin{xlist}
            \exi{a.} o-ku-gh\tn{l1}{ù}lir-a  \tab{‘to hear’}
        \end{xlist}
    \end{minipage}
    \begin{minipage}[t]{.50\textwidth}
        \exi{b.} o-ku-k\tn{l2}{à}lakat-a  \tab{‘to scrape’}
    \end{minipage}
    \begin{tikzpicture}[overlay, remember picture]
        \node [below=.2cm of l1.center, font=\small] {L};
        \node [below=.2cm of l2.center, font=\small] {L};
    \end{tikzpicture}
\end{exe}
Two rules are needed to derive the correct outputs. The first is L tone
spreading (LTS): an input L spreads one mora to the right:

\begin{exe}\ex\label{ex:key:13.23}
    \begin{minipage}[t]{.50\textwidth}
        \begin{xlist}
            \exi{a.} o-ku-gh\tn{l1}{ù}l\tn{v1}{ì}r-a  \tab{‘to hear’}
        \end{xlist}
    \end{minipage}
    \begin{minipage}[t]{.50\textwidth}
        \exi{b.} o-ku-k\tn{l2}{à}l\tn{v2}{à}kat-a  \tab{‘to scrape’}
    \end{minipage}
    \begin{tikzpicture}[overlay, remember picture]
        \node (1) [below=.5cm of l1.center, font=\small] {L};
        \node (2) [below=.5cm of l2.center, font=\small] {L};
        \draw [-] (l1.south) to (1.north);
        \draw [-, densely dashed] (v1.south) to (1.north);
        \draw [-] (l2.south) to (2.north);
        \draw [-, densely dashed] (v2.south) to (2.north);
    \end{tikzpicture}\vspace{.25\baselineskip}
\end{exe}
The second rule is H tone insertion (\glsunset{HTI}\gls{HTI}): a H is inserted
on a mora that precedes an input L:

\begin{exe}\ex\label{ex:key:13.24}
    \begin{minipage}[t]{.50\textwidth}
        \begin{xlist}
            \exi{a.} o-k\tn{h1}{ú}-gh\tn{l1}{ù}l\tn{v1}{ì}r-a  \tab{‘to hear’}
        \end{xlist}
    \end{minipage}
    \begin{minipage}[t]{.50\textwidth}
        \exi{b.} o-k\tn{h2}{ú}-k\tn{l2}{à}l\tn{v2}{à}kat-a  \tab{‘to scrape’}
    \end{minipage}
    \begin{tikzpicture}[overlay, remember picture]
        \node (1) [below=.5cm of l1.center, font=\small] {L};
        \node (2) [below=.5cm of l2.center, font=\small] {L};
        \node (3) [below=.5cm of h1.center, font=\small] {H};
        \node (4) [below=.5cm of h2.center, font=\small] {H};
        \draw [-] (l1.south) to (1.north);
        \draw [-, densely dashed] (v1.south) to (1.north);
        \draw [-] (l2.south) to (2.north);
        \draw [-, densely dashed] (v2.south) to (2.north);
        \draw [-, densely dashed] (h1.south) to (3.north);
        \draw [-, densely dashed] (h2.south) to (4.north);
    \end{tikzpicture}\vspace{.25\baselineskip}
\end{exe}
As seen in \eqref{ex:key:13.25} \gls{HTI} has to be specified to insert a single H
before a sequence of L morphemes (which we can assume to fuse into a single,
multilinked L):

\ea\label{ex:key:13.25}
    \Aug-\Inf{}-it-him-us-give-\Appl-\Fv{}\\
    ò-kú-c\tn{l1}{ì}-m\tn{l2}{ù}-t\tn{l3}{ù}-gh\tn{l4}{à}-\tn{v1}{è}r-á
        ${\rightarrow}$
        \tn{l5}{ò}-k\tn{h1}{ú}-c\tn{l6}{ì}-m\tn{l7}{ù}-t\tn{l8}{ù}-gh\tn{l9}{è}-\tn{l10}{è}r-\tn{h2}{á}\\\vspace{1.5\baselineskip}
    \begin{tikzpicture}[overlay, remember picture]
        \node (1) [below=.5cm of l1.center, font=\small] {L};
        \node (2) [below=.5cm of l2.center, font=\small] {L};
        \node (3) [below=.5cm of l3.center, font=\small] {L};
        \node (4) [below=.5cm of l4.center, font=\small] {L};
        \node (5) [below=.5cm of l5.center, font=\small] {\llap{\%}L};
        \node (6) [below=.5cm of h1.center, font=\small] {\underline{H}};
        \node (7) [below=.5cm of l8.center, font=\small] {L};
        \node (8) [below=.5cm of h2.center, font=\small] {H\rlap{\%}};
        \draw [-] (l1.south) to (1.north);
        \draw [-] (l2.south) to (2.north);
        \draw [-] (l3.south) to (3.north);
        \draw [-] (l4.south) to (4.north);
        \draw [-] (l6.south) to (7.north);
        \draw [-] (l7.south) to (7.north);
        \draw [-] (l8.south) to (7.north);
        \draw [-] (l9.south) to (7.north);
        \draw [-] (l10.south) to (7.north);
        \draw [-, densely dashed] (v1.south) to (4.north);
        \draw [-, densely dashed] (l5.south) to (5.north);
        \draw [-, densely dashed] (h1.south) to (6.north);
        \draw [-, densely dashed] (h2.south) to (8.north);
    \end{tikzpicture}%
    ‘to give it to him for us’
\z

With this established, we now have two relevant criteria to test for
postlexical domains in Lusoga: (i) \gls{HTI} conditioned by the initial /L/
syllable of one word onto the final syllable of the preceding word. The question is
whether a word-initial L will condition the insertion of a H onto the final
vowel of the preceding word. (ii) \gls{HTA} from one word onto toneless moras
of the preceding word(s), as in \ili{Luganda}. The question is whether there are any
syntactic configurations that block \gls{HTA} (as some do in \ili{Luganda}). To
anticipate the demonstration, the conclusion we will reach is that syntactic
constituency never blocks \gls{HTI} or \gls{HTA}, thereby raising two competing
hypotheses:

\ea\label{ex:key:13.26}
    Hypothesis 1: \ili{Lusoga} does not have the prosodic domains\is{prosodic domains} found in \ili{Luganda}.\\
    Hypothesis 2: \ili{Lusoga} has prosodic domains\is{prosodic domains}, but does not mark them the same
    as \ili{Luganda}.
\z
The significance of the first is that the mapping of syntactic structures into
prosodic domains\is{prosodic domains} would not be universal in the sense of
Selkirk \& Lee’s claim in the quote at the beginning of this paper. The problem
with the second is that there is no empirical evidence to justify the prosodic
domains\is{prosodic domains}. To see this we need to consider the \ili{Lusoga} facts
which correspond to Luganda’s \gls{TP} and \gls{TG}\is{tone}. We first consider
\gls{HTA}, then \gls{HTI}.

\subsection{H tone anticipation (HTA)}\label{sub:13.3.1}

Unlike \ili{Luganda}, the final H\% boundary tone\is{phonological phrases} can reach
the subject (as well as left-dislocations\is{left dislocation}):

\ea\label{ex:key:13.27}
    \NumTabs{7}
    \ea \ili{Luganda} \tab{\tn{l1}{ò}-mù-lìmì [ \tn{l2}{à}-\underline{lágír}-\tn{h1}{á}} \tab{‘the farmer commands’}\\
    \begin{tikzpicture}[overlay, remember picture]
        \node [below=.2cm of l1.center, font=\small] {\llap{\%}L};
        \node [below=.2cm of l2.center, font=\small] {\llap{\%}L};
        \node [below=.2cm of h1.center, font=\small] {H\rlap{\%}};
    \end{tikzpicture}%\vspace{.25\baselineskip}
    \ex \ili{Lusoga} \tab{\tn{l3}{ò}-m\underline{ú-límí [ á-lágí}r-\tn{h2}{á}} \tab{(idem)}\\
    \begin{tikzpicture}[overlay, remember picture]
        \node [below=.2cm of l3.center, font=\small] {\llap{\%}L};
        \node [below=.2cm of h2.center, font=\small] {H\rlap{\%}};
    \end{tikzpicture}%
    \z
\z
Similarly, unlike \ili{Luganda}, \gls{HTA} can spread a lexical or inserted H tone
onto the subject:

\ea\label{ex:key:13.28}
    \NumTabs{7}
    \ea \ili{Luganda} \tab{\tn{l1}{ò}-mù-lìmì [ \tn{l2}{à}-b\underline{ál-á é-mí}-s\tn{h1}{ó}t\tn{l3}{à}} \tab{‘the farmer counts snakes’}\\
    \begin{tikzpicture}[overlay, remember picture]
        \node (1) [below=.5cm of l1.center, font=\small] {\llap{\%}L};
        \node (2) [below=.5cm of l2.center, font=\small] {\llap{\%}L};
        \node (3) [below=.5cm of l3.center, font=\small] {L};
        \node (4) [below=.5cm of h1.center, font=\small] {H};
        \node (5) [right=.1cm of 3.center, font=\small] {H\rlap{\%}};
        \draw [-, densely dashed] (l1.south) to (1.north);
        \draw [-, densely dashed] (l2.south) to (2.north);
        \draw [-, densely dashed] (l3.south) to (3.north);
        \draw [-] (h1.south) to (4.north);
    \end{tikzpicture}\vspace{.5\baselineskip}
    \ex \ili{Lusoga} \tab{\tn{l4}{ò}-m\underline{ú-límí [ á-bál-á é}-m\tn{h3}{í}-s\tn{l5}{ò}t\tn{h4}{á}} \tab{(idem)}\\
    \begin{tikzpicture}[overlay, remember picture]
        \node (6) [below=.5cm of l4.center, font=\small] {\llap{\%}L};
        \node (7) [below=.5cm of l5.center, font=\small] {L};
        \node (8) [below=.5cm of h3.center, font=\small] {H};
        \node (9) [below=.5cm of h4.center, font=\small] {H\rlap{\%}};
        \draw [-, densely dashed] (l4.south) to (6.north);
        \draw [-] (l5.south) to (7.north);
        \draw [-, densely dashed] (h3.south) to (8.north);
        \draw [-, densely dashed] (h4.south) to (9.north);
    \end{tikzpicture}%
    \z
\z
The following examples show that H\% and \gls{HTA} can also reach
left-dislocations\is{left dislocation}:

\ea\label{ex:key:13.29}
    \NumTabs{7}
    \ea o-mu-limi e-bi-tabo a-bi-bal-a \tab{${\rightarrow}$
    \tn{l1}{ò}-m\underline{ú-límí é-bí-tábó  á-bí-bál}-\tn{h1}{á}}\\\vspace{1\baselineskip}
		 ‘the farmer, the books, he counts them’
        \begin{tikzpicture}[overlay, remember picture]
            \node [below=.2cm of l1.center, font=\small] {\llap{\%}L};
            \node [below=.2cm of h1.center, font=\small] {H\rlap{\%}};
        \end{tikzpicture}%\vspace{.25\baselineskip}
    \ex o-mu-limi e-bi-tabo a-bi-b\tn{l1}{o}n-a \tab{${\rightarrow}$
    \tn{l2}{ò}-m\underline{ú-límí é-bí-tábó á}-b\tn{h1}{í}-b\tn{l3}{ò}n-á}\\\vspace{1\baselineskip}
        ‘the farmer, the books, he sees them’
        \begin{tikzpicture}[overlay, remember picture]
            \node [below=.2cm of l1.center, font=\small] {L};
            \node [below=.2cm of l2.center, font=\small] {\llap{\%}L};
            \node [below=.2cm of h1.center, font=\small] {H};
            \node [below=.2cm of l3.center, font=\small] {L};
        \end{tikzpicture}%\vspace{.25\baselineskip}
    \z
\z
Spreading of H\% and \gls{HTA} can also start from a right-dislocated element:

\ea\label{ex:key:13.30}
    \NumTabs{7}
    \ea a-bi-bal-a o-mu-limi e-bi-tabo \tab{${\rightarrow}$
        \tn{l1}{à}-b\underline{í-bál-á  ó-mú-límí  é-bí-tá}b\tn{h1}{ó}}\\\vspace{1\baselineskip}
		‘he counts them, the farmer, the books’
        \begin{tikzpicture}[overlay, remember picture]
            \node [below=.2cm of l1.center, font=\small] {\llap{\%}L};
            \node [below=.2cm of h1.center, font=\small] {H\rlap{\%}};
        \end{tikzpicture}%\vspace{.25\baselineskip}
    \ex a-bi-bal-a o-mu-limi e-bi-kop\tn{l1}{o} \tab{${\rightarrow}$
        \tn{l2}{à}-b\underline{í-bál-á  ó-mú-límí  é-bí}-k\tn{h1}{ó}p\tn{l3}{ò}}\\\vspace{1\baselineskip}
        ‘he counts them, the farmer, the cups’
        \begin{tikzpicture}[overlay, remember picture]
            \node [below=.2cm of l1.center, font=\small] {L};
            \node [below=.2cm of l2.center, font=\small] {\llap{\%}L};
            \node [below=.2cm of h1.center, font=\small] {H};
            \node [below=.2cm of l3.center, font=\small] {L};
        \end{tikzpicture}%\vspace{.25\baselineskip}
    \z
\z
As in \ili{Luganda}, \gls{HTA} will apply only if the preceding word ends in at least
one toneless mora, as in (\ref{ex:key:13.31}a). It will not apply if the preceding
word ends in L, as in (\ref{ex:key:13.31}b).

\ea\label{ex:key:13.31}
    \NumTabs{9}
    \ea o-k\tn{h1}{ú}-gh\tn{l1}{ù}l\tn{v1}{ì}r-a e-m\tn{h2}{í}-s\tn{l2}{ò}t\tn{v2}{à}
        \tab{${\rightarrow}$
            \tn{l3}{ò}-k\tn{h3}{ú}-gh\tn{l4}{ù}l\tn{v3}{ì}r-\underline{á
        é}-m\tn{h4}{í}-s\tn{l5}{ò}t\tn{h5}{á}}\\\vspace{1.5\baselineskip}
        \enquote*{to hear snakes}
    \begin{tikzpicture}[overlay, remember picture]
        \node (1) [below=.5cm of h1.center, font=\small] {H};
        \node (2) [below=.5cm of l1.center, font=\small] {L};
        \node (3) [below=.5cm of h2.center, font=\small] {H};
        \node (4) [below=.5cm of l2.center, font=\small] {L};
        \node (5) [below=.5cm of l3.center, font=\small] {\llap{\%}L};
        \node (6) [below=.5cm of h3.center, font=\small] {H};
        \node (7) [below=.5cm of l4.center, font=\small] {L};
        \node (8) [below=.5cm of h4.center, font=\small] {H};
        \node (9) [below=.5cm of l5.center, font=\small] {L};
        \node (10) [below=.5cm of h5.center, font=\small] {H\rlap{\%}};
        \draw [-, densely dashed] (h1.south) to (1.north);
        \draw [-] (l1.south) to (2.north);
        \draw [-, densely dashed] (h2.south) to (3.north);
        \draw [-] (l2.south) to (4.north);
        \draw [-, densely dashed] (v2.south) to (4.north);
        \draw [-, densely dashed] (v1.south) to (2.north);
        \draw [-, densely dashed] (l3.south) to (5.north);
        \draw [-] (h3.south) to (6.north);
        \draw [-] (l4.south) to (7.north);
        \draw [-] (v3.south) to (7.north);
        \draw [-] (h4.south) to (8.north);
        \draw [-] (l5.south) to (9.north);
        \draw [-, densely dashed] (h5.south) to (10.north);
    \end{tikzpicture}%
    \ex o-k\tn{h1}{ú}-b\tn{l1}{ò}n-\tn{v1}{à}
            e-m\tn{h2}{í}-s\tn{l2}{ò}t\tn{v2}{à}
            \tab{${\rightarrow}$
                \tn{l3}{ò}-k\tn{h3}{ú}-b\tn{l4}{ò}n-\dashuline{\tn{v3}{à}
            è}-m\tn{h4}{í}-s\tn{l5}{ò}t\tn{h5}{á}}\\\vspace{1.5\baselineskip}
        \enquote*{to see snakes}
    \begin{tikzpicture}[overlay, remember picture]
        \node (1) [below=.5cm of h1.center, font=\small] {H};
        \node (2) [below=.5cm of l1.center, font=\small] {L};
        \node (3) [below=.5cm of h2.center, font=\small] {H};
        \node (4) [below=.5cm of l2.center, font=\small] {L};
        \node (5) [below=.5cm of l3.center, font=\small] {\llap{\%}L};
        \node (6) [below=.5cm of h3.center, font=\small] {H};
        \node (7) [below=.5cm of l4.center, font=\small] {L};
        \node (8) [below=.5cm of h4.center, font=\small] {H};
        \node (9) [below=.5cm of l5.center, font=\small] {L};
        \node (10) [below=.5cm of h5.center, font=\small] {H\rlap{\%}};
        \draw [-, densely dashed] (h1.south) to (1.north);
        \draw [-] (l1.south) to (2.north);
        \draw [-, densely dashed] (h2.south) to (3.north);
        \draw [-] (l2.south) to (4.north);
        \draw [-, densely dashed] (v2.south) to (4.north);
        \draw [-, densely dashed] (v1.south) to (2.north);
        \draw [-, densely dashed] (l3.south) to (5.north);
        \draw [-] (h3.south) to (6.north);
        \draw [-] (l4.south) to (7.north);
        \draw [-] (v3.south) to (7.north);
        \draw [-] (h4.south) to (8.north);
        \draw [-] (l5.south) to (9.north);
        \draw [-, densely dashed] (h5.south) to (10.north);
    \end{tikzpicture}%
    \z
\z

From the above we can safely assume that \gls{HTA} will apply no matter what
the syntactic configuration. As stated in \Cref{sec:13.1}, this is quite
surprising, given that almost all \ili{Bantu} languages treat pre-verbal constituents
differently from post-verbal ones. In the next section we will see that \gls{HTI}
leads to the same conclusion.

\subsection{H tone insertion (HTI)}\label{sub:13.3.2}

In this section it will be briefly demonstrated that \gls{HTI} can also apply across
any syntactic boundary. Because nouns have a prefix which is underlyingly
toneless, this will have to be demonstrated by means of other word classes,
e.g. verbs and \isi{demonstratives}. Consider first (\ref{ex:key:13.32}a), where the
subject prefix \emph{a-} is underlyingly toneless:

\ea\label{ex:key:13.32}
    \NumTabs{11}
    \ea o-mu-k\tn{l1}{à}zi a-sek-a \tab{${\rightarrow}$}
        \tab{\tn{l2}{ò}-m\tn{h1}{ú}-k\tn{l3}{à}z\tn{v1}{ì}
        à-s\tn{v2}{é}k-\tn{h2}{á}} \tab{‘the woman laughs’}\\
    \begin{tikzpicture}[overlay, remember picture]
        \node (1) [below=.5cm of l1.center, font=\small] {L};
        \node (2) [below=.5cm of l2.center, font=\small] {\llap{\%}L};
        \node (3) [below=.5cm of h1.center, font=\small] {H};
        \node (4) [below=.5cm of l3.center, font=\small] {L};
        \node (5) [below=.5cm of h2.center, font=\small] {H\rlap{\%}};
        \draw [-] (l1.south) to (1.north);
        \draw [-, densely dashed] (l2.south) to (2.north);
        \draw [-, densely dashed] (h1.south) to (3.north);
        \draw [-] (l3.south) to (4.north);
        \draw [-, densely dashed] (v1.south) to (4.north);
        \draw [-, densely dashed] (v2.south) to (5.north);
        \draw [-, densely dashed] (h2.south) to (5.north);
    \end{tikzpicture}\vspace{.5\baselineskip}
    \ex a-ba-k\tn{l1}{à}zi b\tn{l2}{à}-sek-a \tab{${\rightarrow}$}
        \tab{\tn{l3}{à}-b\tn{h1}{á}-k\tn{l4}{à}z\tn{h2}{í}
        b\tn{l5}{à}-s\tn{v1}{è}k-\tn{h3}{á}} \tab{‘the women laugh’}\\
    \begin{tikzpicture}[overlay, remember picture]
        \node (1) [below=.5cm of l1.center, font=\small] {L};
        \node (2) [below=.5cm of l2.center, font=\small] {L};
        \node (3) [below=.5cm of l3.center, font=\small] {\llap{\%}L};
        \node (4) [below=.5cm of h1.center, font=\small] {H};
        \node (5) [below=.5cm of l4.center, font=\small] {L};
        \node (6) [below=.5cm of h2.center, font=\small] {\underline{H}};
        \node (7) [below=.5cm of l5.center, font=\small] {L};
        \node (8) [below=.5cm of h3.center, font=\small] {H\rlap{\%}};
        \draw [-] (l1.south) to (1.north);
        \draw [-] (l2.south) to (2.north);
        \draw [-, densely dashed] (h1.south) to (4.north);
        \draw [-] (l4.south) to (5.north);
        \draw [-, densely dashed] (h2.south) to (6.north);
        \draw [-] (l5.south) to (7.north);
        \draw [-, densely dashed] (v1.south) to (7.north);
        \draw [-, densely dashed] (h3.south) to (8.north);
    \end{tikzpicture}\vspace{.5\baselineskip}
    \z
\z
In this case the subject noun ‘woman’ ends with a L tone by virtue of the
\gls{LTS} rule. Therefore, the final H\% cannot spread onto the subject noun.
Compare this now with (\ref{ex:key:13.32}b), where the subject prefix /bà-/ has an
underlying /L/. In this case \gls{HTI} overrides LTS onto the final mora of the
subject noun. In historical terms, the *H of \emph{*bá-} has been anticipated
from the verb onto the subject (cf.\ \ili{Luganda} \emph{à-bà-kázì}
\emph{bá{}-sèk-á}). The same facts are seen with left dislocations:

\ea\label{ex:key:13.33}
    \NumTabs{7}
    \ea e-bi-b\tn{l1}{à}la a-bi-bal-a \tab{${\rightarrow}$}
        \tab{\tn{l2}{è}-b\tn{h1}{í}-b\tn{l3}{à}l\tn{v1}{à}
        à-b\tn{v2}{í}-b\tn{v3}{á}l-\tn{h2}{á}}\\\vspace{1.5\baselineskip}
    \begin{tikzpicture}[overlay, remember picture]
        \node (1) [below=.5cm of l1.center, font=\small] {L};
        \node (2) [below=.5cm of l2.center, font=\small] {\llap{\%}L};
        \node (3) [below=.5cm of h1.center, font=\small] {H};
        \node (4) [below=.5cm of l3.center, font=\small] {L};
        \node (5) [below=.5cm of h2.center, font=\small] {H\rlap{\%}};
        \draw [-] (l1.south) to (1.north);
        \draw [-, densely dashed] (l2.south) to (2.north);
        \draw [-, densely dashed] (h1.south) to (3.north);
        \draw [-] (l3.south) to (4.north);
        \draw [-, densely dashed] (v1.south) to (4.north);
        \draw [-, densely dashed] (v2.south) to (5.north);
        \draw [-, densely dashed] (v3.south) to (5.north);
        \draw [-, densely dashed] (h2.south) to (5.north);
    \end{tikzpicture}%
        ‘the fruits, s/he counts them’
    \ex e-bi-b\tn{l1}{à}là  b\tn{l2}{à}-bi-bal-a \tab{${\rightarrow}$}
    \tab{\tn{l3}{è}-b\tn{h1}{í}-b\tn{l4}{à}l\tn{h2}{á}
    b\tn{l5}{à}-b\tn{v1}{ì}-b\tn{v2}{á}l-\tn{h3}{á}}\\\vspace{1.5\baselineskip}
    \begin{tikzpicture}[overlay, remember picture]
        \node (1) [below=.5cm of l1.center, font=\small] {L};
        \node (2) [below=.5cm of l2.center, font=\small] {L};
        \node (3) [below=.5cm of l3.center, font=\small] {\llap{\%}L};
        \node (4) [below=.5cm of h1.center, font=\small] {H};
        \node (5) [below=.5cm of l4.center, font=\small] {L};
        \node (6) [below=.5cm of h2.center, font=\small] {\underline{H}};
        \node (7) [below=.5cm of l5.center, font=\small] {L};
        \node (8) [below=.5cm of h3.center, font=\small] {H\rlap{\%}};
        \draw [-] (l1.south) to (1.north);
        \draw [-] (l2.south) to (2.north);
        \draw [-, densely dashed] (h1.south) to (4.north);
        \draw [-] (l4.south) to (5.north);
        \draw [-, densely dashed] (h2.south) to (6.north);
        \draw [-] (l5.south) to (7.north);
        \draw [-, densely dashed] (v1.south) to (7.north);
        \draw [-, densely dashed] (v2.south) to (8.north);
        \draw [-, densely dashed] (h3.south) to (8.north);
    \end{tikzpicture}%
         ‘the fruits, they count them’
    \z
\z
In (\ref{ex:key:13.33}a), H\% does not reach the
left-dislocated\is{left dislocation} noun /e-bi-bàla/ ‘fruits’, since its /L/
spreads onto the final mora. In (\ref{ex:key:13.33}b), however, where the
subject prefix /bà-/ has /L/ tone, \gls{HTI} applies, and the H links to the
final mora of the left-dislocated\is{left dislocation} noun. In fact,
\gls{HTI} will apply across any sequence of words, provided that the preceding
word does not end in a single /L/. This is illustrated in \eqref{ex:key:13.34}.

\ea\label{ex:key:13.34}
    \NumTabs{11}
    \ea e-bí-b\tn{l1}{à}là b\tn{l2}{ì}-no \tab{${\rightarrow}$}
    \tab{\tn{l3}{è}-b\tn{h1}{í}-b\tn{l4}{à}l\tn{h2}{á} b\tn{l5}{ì}-n\tn{h3}{ó}} \tab{‘these fruits’}\\
    \begin{tikzpicture}[overlay, remember picture]
        \node (1) [below=.5cm of l1.center, font=\small] {L};
        \node (2) [below=.5cm of l2.center, font=\small] {L};
        \node (3) [below=.5cm of l3.center, font=\small] {\llap{\%}L};
        \node (4) [below=.5cm of h1.center, font=\small] {H};
        \node (5) [below=.5cm of l4.center, font=\small] {L};
        \node (6) [below=.5cm of h2.center, font=\small] {H};
        \node (7) [below=.5cm of l5.center, font=\small] {L};
        \node (8) [below=.5cm of h3.center, font=\small] {H\rlap{\%}};
        \draw [-] (l1.south) to (1.north);
        \draw [-] (l2.south) to (2.north);
        \draw [-, densely dashed] (l3.south) to (3.north);
        \draw [-, densely dashed] (h1.south) to (4.north);
        \draw [-] (l4.south) to (5.north);
        \draw [-, densely dashed] (h2.south) to (6.north);
        \draw [-] (l5.south) to (7.north);
        \draw [-, densely dashed] (h3.south) to (8.north);
    \end{tikzpicture}\vspace{.5\baselineskip}
    \ex e-bí-kóp\tn{l1}{ò} b\tn{l2}{ì}-no \tab{${\rightarrow}$}
    \tab{\tn{l3}{è}-b\tn{v1}{í}-k\tn{h1}{ó}p\tn{l4}{ò} \tn{e}{\hphantom{ H }}b\tn{l5}{ì}-n\tn{h2}{ó}} \tab{‘these cups’}\\
    \begin{tikzpicture}[overlay, remember picture]
        \node (1) [below=.5cm of l1.center, font=\small] {L};
        \node (2) [below=.5cm of l2.center, font=\small] {L};
        \node (3) [below=.5cm of l3.center, font=\small] {\llap{\%}L};
        \node (4) [below=.5cm of h1.center, font=\small] {H};
        \node (5) [below=.5cm of l4.center, font=\small] {L};
        \node (6) [below=.5cm of e.center, font=\small] {\underline{H}};
        \node (7) [below=.5cm of l5.center, font=\small] {L};
        \node (8) [below=.5cm of h2.center, font=\small] {H\rlap{\%}};
        \draw [-] (l1.south) to (1.north);
        \draw [-] (l2.south) to (2.north);
        \draw [-, densely dashed] (l3.south) to (3.north);
        \draw [-, densely dashed] (v1.south) to (4.north);
        \draw [-, densely dashed] (h1.south) to (4.north);
        \draw [-] (l4.south) to (5.north);
        \draw [-] (l5.south) to (7.north);
        \draw [-, densely dashed] (h2.south) to (8.north);
    \end{tikzpicture}\vspace{.5\baselineskip}
    \z
\z
The proximate demonstrative\is{demonstratives} /-no/ ‘this, these’ requires a L tone noun class
agreement prefix, here /bì-/. As seen in (\ref{ex:key:13.34}a), the prefix
conditions \gls{HTI} on the final mora of ‘fruits’. In (\ref{ex:key:13.34}b), on the other
hand, the noun ‘cups’ ends in a single /L/ and hence \gls{HTI} is blocked.

We thus arrive at the conclusion that syntactic constituency never blocks \gls{HTI}
or \gls{HTA}. Returning to the two hypotheses in \eqref{ex:key:13.26}, we must address
whether \ili{Lusoga} recognizes prosodic domains\is{prosodic domains} at all—or whether it simply fails to
give evidence of the syntax-to-prosodic domain\is{prosodic domains} mapping that
\citegen{Selkirk2011} matching theory predicts. Favoring universality, let’s
tentatively entertain the latter theory-driven position, Hypothesis 2 in
\eqref{ex:key:13.26}: \ili{Lusoga} has prosodic domains\is{prosodic domains}, but does not mark them. As was
seen in \Cref{sec:13.2}, \ili{Luganda} marks TPs with an initial \%L, which can
be taken to block \gls{HTA} from the verb or between sentential preverbal
constituents, each one of which begins a TP with its own \%L. As Lisa Selkirk
puts it (email of March 18, 2016):

\begin{quotation}

In~Lusoga, if \gls{HTA} can extend from verb to subject and so on, it must be
that there is no such L at the left edge of TP/ip. In other words a
‘domain-less’ \gls{HTA} can spread its way leftward in~Lusoga~without a
problem, but it would be blocked by the boundary L in \ili{Luganda}.

\end{quotation}
Under this interpretation \ili{Lusoga} would not have \%L internal to the \gls{IP},
at most an \gls{IP}-initial \%L to predict the realization of post-pause
toneless words such as \emph{ò-kú-lágír-á} ‘to command’ in (\ref{ex:key:13.20}a).
Such words require an initial L to precede the multiple Hs from H\%. This could
either be the effect of an IP-initial \%L tone or is perhaps due to some kind
of constraint against initial H.

\subsection{The TG}\label{sub:13.3.3}

In \Cref{sec:13.2} we saw that \ili{Luganda} distinguishes two prosodic domains\is{prosodic domains}, the
\gls{TP} and the \gls{TG}\is{tone}. The preceding discussion of \gls{HTA} and \gls{HTI}
have both addressed the TP. In this section we show that \ili{Lusoga} provides
evidence for the \gls{TG}\is{tone} only at the phonological word (\glsunset{PW}\gls{PW})
level. Importantly, there is no “phrasal” \gls{TG}\is{tone} in Lusoga, i.e. no case of a
head (X) + phonological word (Z) producing H tone plateauing (\gls{HTP}). The
examples in \eqref{ex:key:13.35} show that the configurations that were seen to
produce \gls{HTP} in \ili{Luganda} in (\ref{ex:key:13.4}a) and (\ref{ex:key:13.15}b)
above fail to produce \gls{HTP} in Lusoga:

\ea\label{ex:key:13.35}
    \NumTabs{7}
	\ea verb + object\\
        tu-\tn{l1}{à}-b\tn{l2}{ò}n-\tn{l3}{à} + bi-s\tn{l4}{à}gho \tab{${\rightarrow}$}
            \tab{tw-\tn{hl}{áà}-b\tn{l5}{ò}n-\tn{l6}{à}
            b\tn{h1}{í}-s\tn{l7}{à}gh\tn{h2}{ó}}\\\vspace{1\baselineskip}
        \begin{tikzpicture}[overlay, remember picture]
            \node (1) [below=.2cm of l1.center, font=\small] {L};
            \node (2) [below=.2cm of l2.center, font=\small] {L};
            \node (3) [below=.2cm of l3.center, font=\small] {L};
            \node (4) [below=.2cm of l4.center, font=\small] {L};
            \node (5) [below=.2cm of hl.center, font=\small] {HL};
            \node (6) [below=.2cm of l5.center, font=\small] {\underline{L}};
            \node (7) [below=.2cm of l6.center, font=\small] {\underline{L}};
            \node (8) [below=.2cm of h1.center, font=\small] {H};
            \node (8) [below=.2cm of l7.center, font=\small] {L};
            \node (10)[below=.2cm of h2.center, font=\small] {\%H};
        \end{tikzpicture}%\vspace{.5\baselineskip}
        ‘we saw \textsc{bags}’
	\ex N + GenN\\
           e-bí-s\tn{l1}{à}gho + bi-a=jeeng\tn{l2}{a} \tab{${\rightarrow}$
               \tab{\tn{l3}{e}-b\tn{h1}{í}-s\tn{l4}{à}gh\tn{v1}{ò}
           by-\tn{v2}{à}à=j\tn{h2}{éé}ng\tn{l5}{à}}}\\\vspace{1.5\baselineskip}
        \begin{tikzpicture}[overlay, remember picture]
            \node (1) [below=.5cm of l1.center, font=\small] {L};
            \node (2) [below=.5cm of l2.center, font=\small] {L};
            \node (3) [below=.5cm of l3.center, font=\small] {\llap{\%}L};
            \node (4) [below=.5cm of h1.center, font=\small] {H};
            \node (5) [below=.5cm of l4.center, font=\small] {\underline{L}};
            \node (6) [below=.5cm of h2.center, font=\small] {H};
            \node (7) [below=.5cm of l5.center, font=\small] {L};
            \draw [-] (l1.south) to (1.north);
            \draw [-] (l2.south) to (2.north);
            \draw [-, densely dashed] (l3.south) to (3.north);
            \draw [-, densely dashed] (h1.south) to (4.north);
            \draw [-] (l4.south) to (5.north);
            \draw [-] (v1.south) to (5.north);
            \draw [-, densely dashed] (v2.south) to (5.north);
            \draw [-, densely dashed] (h2.south) to (6.north);
            \draw [-] (l5.south) to (7.north);
        \end{tikzpicture}\vspace{.5\baselineskip}
        ‘Jenga’s bags’
    \z
\z
In (\ref{ex:key:13.35}a) the distant past affirmative verb is followed by an
object noun which lacks the augment vowel since it is in focus, while
(\ref{ex:key:13.35}b) consists of a genitive\is{genitive case} construction marked by the proclitic\is{clitics}
/bi-a=/ on the second noun. In neither case is there \gls{HTP} as was observed
in \ili{Luganda} in (\ref{ex:key:13.12}a) and (\ref{ex:key:13.15}b), respectively.

While there is no case of a \gls{TG}\is{tone} consisting of two phonological words
(\glspl{PW}), \gls{HTP} does apply word-internally and between a \gls{PW} and
certain enclitics.\is{clitics} The first is seen in a process of noun reduplication which
introduces a derogatory meaning.  Thus, when \emph{ò-mú-pákàsí} ‘porter’ is
reduplicated to \emph{ò-mú-pákásí{}-pákàsì} ‘a lousy ol’ porter’ the portion I
have underlined shows \gls{HTP}. A full derivation is provided in
\eqref{ex:key:13.36}.

\ea\label{ex:key:13.36}
    \NumTabs{9}
    \ea reduplicated input +
    \tab{o-mu-pak\tn{l1}{à}s\tn{v1}{i}-pak\tn{l2}{à}s\tn{v2}{i}}
        \tab{‘a lousy ol’ porter’}\\
		L tone spreading:
        \begin{tikzpicture}[overlay, remember picture]
            \node (1) [below=.5cm of l1.center, font=\small] {L};
            \node (2) [below=.5cm of l2.center, font=\small] {L};
            \draw [-] (l1.south) to (1.north);
            \draw [-] (l2.south) to (2.north);
            \draw [-, densely dashed] (v1.south) to (1.north);
            \draw [-, densely dashed] (v2.south) to (2.north);
        \end{tikzpicture}\vspace{.5\baselineskip}
        \ex H tone insertion:
        \tab{o-m\tn{v1}{u}-p\tn{h1}{a}k\tn{l1}{a}s\tn{v2}{i}-p\tn{h2}{a}k\tn{l2}{a}s\tn{v3}{i}}\\
        \begin{tikzpicture}[overlay, remember picture]
            \node (1) [below=.5cm of h1.center, font=\small] {H};
            \node (2) [below=.5cm of l1.center, font=\small] {L};
            \node (3) [below=.5cm of h2.center, font=\small] {H};
            \node (4) [below=.5cm of l2.center, font=\small] {L};
            \draw [-] (h1.south) to (1.north);
            \draw [-] (l1.south) to (2.north);
            \draw [-, densely dashed] (h2.south) to (3.north);
            \draw [-] (l2.south) to (4.north);
            \draw [-, densely dashed] (v1.south) to (1.north);
            \draw [-] (v2.south) to (2.north);
            \draw [-] (v3.south) to (4.north);
        \end{tikzpicture}\vspace{.5\baselineskip}
        \ex H tone plateauing:
        \tab{o-m\tn{v1}{u}-p\tn{v2}{a}k\tn{v3}{a}s\tn{h1}{i}-p\tn{v4}{a}k\tn{l1}{a}s\tn{v5}{i}}\\
        \begin{tikzpicture}[overlay, remember picture]
            \node (1) [below=.5cm of h1.center, font=\small] {H};
            \node (2) [below=.5cm of l1.center, font=\small] {L};
            \draw [-] (h1.south) to (1.north);
            \draw [-] (l1.south) to (2.north);
            \draw [-] (v1.south) to (1.north);
            \draw [-] (v2.south) to (1.north);
            \draw [-] (v3.south) to (1.north);
            \draw [-] (v4.south) to (1.north);
            \draw [-] (v5.south) to (2.north);
        \end{tikzpicture}\vspace{.5\baselineskip}
    \ex Output with \%L\dots{}H\%:
    \tab{\tn{l1}{ò}-m\tn{v1}{ú}-p\tn{v2}{á}k\tn{v3}{à}s\tn{h1}{í}-p\tn{v4}{á}k\tn{l2}{à}s\tn{h2}{í}}\\\vspace{1.5\baselineskip}
        \begin{tikzpicture}[overlay, remember picture]
            \node (1) [below=.5cm of l1.center, font=\small] {\llap{\%}L};
            \node (2) [below=.5cm of h1.center, font=\small] {H};
            \node (3) [below=.5cm of l2.center, font=\small] {L};
            \node (4) [below=.5cm of h2.center, font=\small] {H\rlap{\%}};
            \draw [-] (l1.south) to (1.north);
            \draw [-] (h1.south) to (2.north);
            \draw [-] (l2.south) to (3.north);
            \draw [-] (h2.south) to (4.north);
            \draw [-] (v1.south) to (2.north);
            \draw [-] (v2.south) to (2.north);
            \draw [-] (v3.south) to (2.north);
            \draw [-] (v4.south) to (2.north);
        \end{tikzpicture}
    \z
\z
As seen, we begin with two identical stems /-pakàsi/, which both undergo LTS in
(\ref{ex:key:13.36}a). \gls{HTI} also applies twice in (\ref{ex:key:13.36}b). This is followed
by \gls{HTP} in (\ref{ex:key:13.36}c) and assignment of the boundary tones in
(\ref{ex:key:13.36}d).\footnote{Although not exemplified in \Cref{sec:13.2}, \gls{HTP}
also applies within a word in \ili{Luganda}.}

More significantly for our purposes, \eqref{ex:key:13.37} shows that \gls{HTP} also
applies between a possessive enclitic and the host noun:\is{clitics}

\begin{exe}\label{ex:key:13.37}
\ex\begin{small}
\begin{tabularx}{\textwidth}[t]{@{}lllll}
σ        & \multicolumn{1}{l@{:}}{/L/}     & \multicolumn{1}{l@{vs.}}{ò-mú-tì} & ò-mú-tíí=gwè     & ‘his/her tree’\\
         \addlinespace[1ex]
σ--σ     & \multicolumn{1}{l@{:}}{/L-$\varnothing$/}   & ò-mú-kàzì                         & ò-mú-kází=wè     & ‘his/her wife’\\
         & \multicolumn{1}{l@{:}}{/$\varnothing$-L/}   & è-kí-kópò                         & è-cí-kópó=cè     & ‘his/her cup’\\
         \addlinespace[1ex]
σ:--σ    & \multicolumn{1}{l@{:}}{/L$\varnothing$-$\varnothing$/}  & è-kí-wùùka                        & è-cí-wúúká=cè    & ‘his/her insect’\\
         & \multicolumn{1}{l@{:}}{/$\varnothing$L-$\varnothing$/}  & à-ká-sáàlè                        & à-ká-sáálé=kè    & ‘his/her arrow’\\
         & \multicolumn{1}{l@{:}}{/$\varnothing$$\varnothing$-L/}  & è-kí-déédè                        & è-cí-déédé=cè    & ‘his/her grasshopper’\\
         \addlinespace[1ex]
σ--σ--σ  & \multicolumn{1}{l@{:}}{/L-$\varnothing$-$\varnothing$/} & ò-bú-thùpùzi                      & ò-bú-thúpúzí=bwè & ‘his/her corruption’\\
         & \multicolumn{1}{l@{:}}{/$\varnothing$-L-$\varnothing$/} & ò-mú-pákàsì                       & ò-mú-pákásí=wè   & ‘his/her porter’\\
         & \multicolumn{1}{l@{:}}{/$\varnothing$-$\varnothing$-L/} & ò-bú-vúbúkà                       & ò-bú-vúbúká=bwè  & ‘his/her adolescence’\\
\end{tabularx}
\end{small}
\end{exe}
The tones of the unpossessed nouns in the first data column, all of which have
a H to L pitch drop, are shown after \gls{HTI} and LTS have applied, but
without a final phrasal H\%. As seen, the L tone possessive enclitic\is{clitics} /-è/
‘his/her’ fuses with a noun class agreement prefix. When \gls{HTI} applies to
the preceding noun, \gls{HTP} applies, and the H to L pitch drop is lost.
(There is no final H\%, since the forms end H-L.) As can be recalled from
(\ref{ex:key:13.15}a), noun+possessive is an environment where \gls{HTP} applies
in \ili{Luganda} as well. The examples in (38a,b) show that \gls{HTP} also applies in
verb+enclitic constructions:\is{clitics}

\ea\label{ex:key:13.38}
    \NumTabs{7}
    \ea tw-\tn{h1}{á}\tn{l1}{à}-gh\tn{l2}{ù}l\tn{v1}{ì}r-a
    \tab{${\rightarrow}$} \tab{tw-\tn{h2}{á}á-gh\tn{o1}{ú}l\tn{o2}{í}r-\tn{h3}{á}=k\tn{l3}{ù}} \\\vspace{1.5\baselineskip}
        ‘we heard’ \tab{} \tab{‘we heard a little’}
        \begin{tikzpicture}[overlay, remember picture]
            \node (1) [below=.5cm of h1.center, font=\small] {H\rlap{L}};
            \node (2) [below=.5cm of l1.center, font=\small] {};
            \node (3) [below=.5cm of l2.center, font=\small] {L};
            \draw [-, densely dashed] (h1.south) to (1.north);
            \draw [-] (l1.south) to (2.north);
            \draw [-] (l2.south) to (3.north);
            \draw [-, densely dashed] (v1.south) to (3.north);

            \node (4) [below=.5cm of h2.center, font=\small] {H\rlap{$\varnothing$}};
            \node (5) [below=.5cm of o1.center, font=\small] {$\varnothing$};
            \node (6) [below=.5cm of o2.center, font=\small] {$\varnothing$};
            \node (7) [below=.5cm of h3.center, font=\small] {H};
            \node (8) [below=.5cm of l3.center, font=\small] {L};

            \draw [-] (h2.south) to (4.north);
            \draw [-, densely dashed] (h3.south) to (7.north);
            \draw [-] (l3.south) to (8.north);

        \end{tikzpicture}\vspace{.5\baselineskip}%
	\ex tw-\tn{h1}{á}\tn{l1}{à}-gh\tn{l2}{ù}l\tn{v1}{ì}r-a
        \tab{${\rightarrow}$} \tab{tw-\tn{h2}{á}á-gh\tn{o1}{ú}l\tn{o2}{í}r-\tn{h3}{á}=c\tn{l3}{ì}} \\\vspace{1.5\baselineskip}
        ‘we heard’ \tab{} \tab{‘what did we hear?’}
        \begin{tikzpicture}[overlay, remember picture]
            \node (1) [below=.5cm of h1.center, font=\small] {H\rlap{L}};
            \node (2) [below=.5cm of l1.center, font=\small] {};
            \node (3) [below=.5cm of l2.center, font=\small] {L};
            \draw [-, densely dashed] (h1.south) to (1.north);
            \draw [-] (l1.south) to (2.north);
            \draw [-] (l2.south) to (3.north);
            \draw [-, densely dashed] (v1.south) to (3.north);

            \node (4) [below=.5cm of h2.center, font=\small] {H\rlap{$\varnothing$}};
            \node (5) [below=.5cm of o1.center, font=\small] {$\varnothing$};
            \node (6) [below=.5cm of o2.center, font=\small] {$\varnothing$};
            \node (7) [below=.5cm of h3.center, font=\small] {H};
            \node (8) [below=.5cm of l3.center, font=\small] {L};

            \draw [-] (h2.south) to (4.north);
            \draw [-, densely dashed] (h3.south) to (7.north);
            \draw [-] (l3.south) to (8.north);

        \end{tikzpicture}\vspace{.5\baselineskip}
    \ex ti-tw-\tn{h1}{á}\tn{l1}{à}-gh\tn{l2}{ù}l\tn{v1}{ì}r-a
        \tab{${\rightarrow}$}
        \tab{ti-tw-\tn{h2}{á}\tn{v2}{à}-gh\tn{l3}{ù}l\tn{v3}{ì}r-\tn{h3}{á}=k\tn{l4}{ù}}\\\vspace{1.5\baselineskip}
        ‘we didn’t hear’ \tab{} \tab{‘we didn’t hear a little’}
        \begin{tikzpicture}[overlay, remember picture]
            \node (1) [below=.5cm of h1.center, font=\small] {H\rlap{L}};
            \node (2) [below=.5cm of l1.center, font=\small] {};
            \node (3) [below=.5cm of l2.center, font=\small] {L};
            \draw [-, densely dashed] (h1.south) to (1.north);
            \draw [-] (l1.south) to (2.north);
            \draw [-] (l2.south) to (3.north);
            \draw [-, densely dashed] (v1.south) to (3.north);

            \node (4) [below=.5cm of h2.center, font=\small] {H};
            \node (5) [below=.5cm of l3.center, font=\small] {\underline{L}};
            \node (6) [below=.5cm of h3.center, font=\small] {H};
            \node (7) [below=.5cm of l4.center, font=\small] {L};

            \draw [-] (h2.south) to (4.north);
            \draw [-] (l3.south) to (5.north);
            \draw [-] (v2.south) to (5.north);
            \draw [-] (v3.south) to (5.north);
            \draw [-, densely dashed] (h3.south) to (6.north);
            \draw [-] (l4.south) to (7.north);

        \end{tikzpicture}\vspace{.5\baselineskip}
    \z
\z
In (\ref{ex:key:13.38}a), the locative noun class 17 enclitic\is{clitics} \emph{=kù} is used
also as an attenuative marker. As seen, \gls{HTI} applies followed by \gls{HTP} on the host
verb. The same is seen in (\ref{ex:key:13.38}b) with the interrogative enclitic\is{clitics}
\emph{=cì} ‘what’. However, for \gls{HTP} to apply, the verb must have the same
[\textsc{{}-focus}] status as was discussed in \ili{Luganda}. Recall that negative
verbs are [\textsc{+focus}], and hence although \gls{HTI} applies before \emph{=kù},
there is no \gls{HTP} in (\ref{ex:key:13.38}c). In addition, there is no \gls{HTP} with
the corresponding nominal interrogative \emph{=cì} ‘which’ (also paralleling
Luganda; cf.\ \emph{mù-kázì} \emph{=cí} ‘which woman?’):

\begin{exe}
\ex\label{ex:key:13.39}
\begin{small}
    \begin{tabularx}{\textwidth}[t]{lllll}
σ        & \multicolumn{1}{l@{:}}{/L/}     & \multicolumn{1}{l@{ $\rightarrow$}}{mú-tì}      & mú-\ds{}tíí=cì    & ‘which tree?’\\
σ--σ     & \multicolumn{1}{l@{:}}{/L-$\varnothing$/}   & \multicolumn{1}{l@{ $\rightarrow$}}{mú-kàzì}    & mú-kàzí=cì                      & ‘which woman?’\\
         & \multicolumn{1}{l@{:}}{/$\varnothing$-L/}   & \multicolumn{1}{l@{ $\rightarrow$}}{bí-kópò}    & bí-kó\ds{}pó=cì   & ‘which cups?’\\
σ:--σ    & \multicolumn{1}{l@{:}}{/L$\varnothing$-$\varnothing$/}  & \multicolumn{1}{l@{ $\rightarrow$}}{cí-wùùka}   & cí-wùùká=cì                     & ‘which insect?’\\
         & \multicolumn{1}{l@{:}}{/$\varnothing$L-$\varnothing$/}  & \multicolumn{1}{l@{ $\rightarrow$}}{ká-sáàlè}   & ká-sáàlé=cì                     & ‘which arrow?’\\
         & \multicolumn{1}{l@{:}}{/$\varnothing$$\varnothing$-L/}  & \multicolumn{1}{l@{ $\rightarrow$}}{cí-déédè}   & cí-déé\ds{}dé=cì  & ‘which grasshopper?’\\
σ--σ--σ  & \multicolumn{1}{l@{:}}{/L-$\varnothing$-$\varnothing$/} & \multicolumn{1}{l@{ $\rightarrow$}}{bú-thùpùzi} & bú-thùpùzí=cì                   & ‘which corruption?’\\
         & \multicolumn{1}{l@{:}}{/$\varnothing$-L-$\varnothing$/} & \multicolumn{1}{l@{ $\rightarrow$}}{mú-pákàsì}  & mú-pákàsí=cì                    & ‘which porter?’\\
         & \multicolumn{1}{l@{:}}{/$\varnothing$-$\varnothing$-L/} & \multicolumn{1}{l@{ $\rightarrow$}}{bú-vúbúkà}  & bú-vúbú\ds{}ká=cì & ‘which adolescence?’\\
\end{tabularx}
\end{small}
\end{exe}
As seen, the enclitic\is{clitics} \emph{=c}ì ‘which’ does not condition \gls{HTP} (perhaps
because it isn’t a YP), but always inserts a H, potentially combining with a
preceding L to create a downstepped \ds{}H.\footnote{Recall from
(\ref{ex:key:13.34}b) that the inserted H cannot be assigned to a single L when it
occurs between two phonological words.}

The above shows that clitics\is{clitics} work differently from full words in Lusoga. \gls{HTP}
occurs in the same environment as in \ili{Luganda}, except that Z must be an
enclitic.\is{clitics} Thus, compare \eqref{ex:key:13.40} with the corresponding \ili{Luganda}
configuration in \eqref{ex:key:13.11}.

\ea\label{ex:key:13.40}
    \begin{tikzpicture}[baseline]

        \Tree 	[.\node(XP){XP};
                    X
                    [.YP
                        \node (z) {Z};
                    ]
                ]

            \node (wh) [right=2cm of XP.south east, align=left]
                    {where: \begin{tabular}{ll}
                                (i) & X ≠ [+\textsc{focus}] \\
                                (ii) & Z ≠ [+\textsc{augment}]
                            \end{tabular}};

            \node at (z -| wh.west) [anchor=west, align=left]
                    {Z = an enclitic};

    \end{tikzpicture}
\z
We have seen that there are two kinds of X=cl: those which form a
\gls{TG}\is{tone} satisfying \eqref{ex:key:13.40}, hence \gls{HTP}, vs.\ those
which don’t satisfy \eqref{ex:key:13.40}, hence occurring without \gls{HTP}. I
propose that the first has the structure of a nested phonological word
[[~word~]\tss{PW}~=cl]\tss{PW}, while the second has the structure of a
clitic\is{clitics} group [[~word~]\tss{PW} =cl]\tss{CG}. If correct, this would mean that
\gls{HTP} only applies within a \gls{PW} whose definition, however, is subject
to the syntactic characterization in \eqref{ex:key:13.40}. A historical
conjecture would be that \gls{HTP} started out in individual words (X), then
expanded to X=Z, then X \# Z, always meeting the configuration and conditions
(i) and (ii) in \eqref{ex:key:13.40}. Note in this regard that enclitics\is{clitics} only
condition \gls{HTP} with their lexical host, not with each other:\largerpage

\ea\label{ex:key:13.41}
    \tn{h1}{a}-t\tn{l1}{a}-\tn{v1}{a}\tn{h2}{=}m\tn{l2}{u}\tn{v2}{u}\tn{h3}{=}k\tn{l3}{u}\tn{v3}{u}\tn{h4}{=}c\tn{l4}{i}\tn{v4}{i}
        \tn{h5}{ } b\tn{l5}{u}l\tn{v5}{i} l\tn{h6}{u}n\tn{l6}{a}k\tn{h7}{u} ${\rightarrow}$
        á-tá-á=\ds{}múú=\ds{}kúú=\ds{}cí
        bùlì lúnàkú\\\vspace{2.75\baselineskip}

    \begin{tikzpicture}[overlay, remember picture]
        \node (1) [below=.5cm of h1.center, font=\small] {H};
        \node (2) [below=.5cm of l1.center, font=\small] {\underline{L}};

        \node (she) [below=.2cm of 1.west, font=\footnotesize,
                    align=left, anchor=north west] {s/he-puts=in=a.little=what
                    every day};

        \node (3) [below=.5cm of h2.center, font=\small] {H};
        \node (4) [below=.5cm of l2.center, font=\small] {L};

        \node (5) [below=.5cm of h3.center, font=\small] {H};
        \node (6) [below=.5cm of l3.center, font=\small] {L};

        \node (7) [below=.5cm of h4.center, font=\small] {H};
        \node (8) [below=.5cm of l4.center, font=\small] {L};

        \node (9) [below=.5cm of h5.center, font=\small] {H};
        \node (10)[below=.5cm of l5.center, font=\small] {L};

        \node (11)[below=.5cm of h6.center, font=\small] {H};
        \node (12)[below=.5cm of l6.center, font=\small] {L};
        \node (13)[below=.5cm of h7.center, font=\small] {H\rlap{\%}};

        \draw [-, densely dashed] (h1.south) to (1.north);
        \draw [-] (l1.south) to (2.north);

        \draw [-, densely dashed] (v1.south) to (3.north);
        \draw [-] (l2.south) to (4.north);

        \draw [-, densely dashed] (v2.south) to (5.north);
        \draw [-] (l3.south) to (6.north);

        \draw [-, densely dashed] (v3.south) to (7.north);
        \draw [-] (l4.south) to (8.north);

        \draw [-, densely dashed] (v4.south) to (9.north);
        \draw [-] (l5.south) to (10.north);
        \draw [-, densely dashed] (v5.south) to (10.north);

        \draw [-, densely dashed] (h6.south) to (11.north);
        \draw [-] (l6.south) to (12.north);
        \draw [-, densely dashed] (h7.south) to (13.north);

    \end{tikzpicture}%\vspace{.5\baselineskip}
    ‘what does s/he put a little of in every day?
\z
In Lusoga, all enclitics\is{clitics} are /L/, requiring \gls{HTI} on the preceding mora.
They also differ from full words in preventing a preceding long vowel from
undergoing final vowel shortening (cf.\ ‘tree’ and ‘which tree?’ in
\eqref{ex:key:13.39}). The unavoidable conclusion is that Lusoga
tonology\is{tone} is not sensitive to prosodic domains\is{prosodic domains}
above the (nested) \gls{PW} level.

\section{Two outstanding problems}\label{sec:13.4}

I would like to end the coverage of tonal phenomena by considering two
outstanding problems. The first is a return to numerals, this time in Lusoga.
We saw in (\ref{ex:key:13.10}b) that \ili{Luganda} doesn’t allow \gls{HTA} from a
numeral onto the preceding noun. There is an analogous issue in Lusoga, which
is that \isi{numerals} which begin with /L/ do not condition \gls{HTI} (vs.
demonstratives, which do). This is seen in \eqref{ex:key:13.42}.

\ea\label{ex:key:13.42}
    \NumTabs{9}
    \ea \tn{l1}{è}-b\tn{h1}{í}-s\tn{l2}{à}gh\tn{v1}{ò}
            b\tn{l3}{ì}-b\tn{v2}{ì}r\tn{h2}{í}
        \tab{cf.} \tab{\tn{l4}{è}-b\tn{h3}{í}-s\tn{l5}{à}gh\tn{h4}{ó}
        b\tn{l6}{ì}-n\tn{h5}{ó}}\\\vspace{3.0\baselineskip}
    \begin{tikzpicture}[overlay, remember picture]
        \node (1) [below=.5cm of l1.center, font=\small] {\llap{\%}L};
        \node (2) [below=.5cm of h1.center, font=\small] {H};
        \node (3) [below=.5cm of l2.center, font=\small] {L};
        \node (4) [below=.5cm of l3.center, font=\small] {L};
        \node (5) [below=.5cm of h2.center, font=\small] {H\rlap{\%}};

        \node (6) [below=.5cm of l4.center, font=\small] {\llap{\%}L};
        \node (7) [below=.5cm of h3.center, font=\small] {H};
        \node (8) [below=.5cm of l5.center, font=\small] {L};
        \node (9) [below=.5cm of h4.center, font=\small] {\underline{H}};
        \node (10)[below=.5cm of l6.center, font=\small] {L};
        \node (11)[below=.5cm of h5.center, font=\small] {H\rlap{\%}};

        \draw [-, densely dashed] (l1.south) to (1.north);
        \draw [-, densely dashed] (h1.south) to (2.north);
        \draw [-] (l2.south) to (3.north);
        \draw [-] (l3.south) to (4.north);
        \draw [-, densely dashed] (h2.south) to (5.north);

        \draw [-, densely dashed] (l4.south) to (6.north);
        \draw [-, densely dashed] (h3.south) to (7.north);
        \draw [-] (l5.south) to (8.north);
        \draw [-, densely dashed] (h4.south) to (9.north);
        \draw [-] (l6.south) to (10.north);
        \draw [-, densely dashed] (h5.south) to (11.north);

        \draw [-, densely dashed] (v1.south) to (3.north);
        \draw [-, densely dashed] (v2.south) to (4.north);

        \node (o) [below=.5cm of v1.center, font=\small] {\hphantom{H}};
        \node (o1)[below=.3cm of o.center, font=\small] {\llap{no }H\rlap{
        here}};
        \draw [->] (o1.north) to (o.center);

    \end{tikzpicture}%\vspace{.5\baselineskip}
    ‘two bags’ \tab{} \tab{} \tab{‘these bags’}
    \ex \tn{l1}{t}w-\tn{h1}{á}\tn{v1}{à}-g\tn{l2}{ù}l-\tn{v2}{à}
        b\tn{l3}{ì}-b\tn{v3}{ì}r\tn{h2}{í}
        \tab{cf.}
        \tab{\tn{l4}{t}w-\tn{h3}{á}\tn{v4}{à}-\tn{l5}{g}\tn{v5}{ù}l-\tn{h4}{á}
        b\tn{l6}{ì}-n\tn{h5}{ó}}\\\vspace{3.0\baselineskip}
    \begin{tikzpicture}[overlay, remember picture]
        \node (1) [below=.5cm of l1.center, font=\small] {\llap{\%}L};
        \node (2) [below=.5cm of h1.center, font=\small] {H};
        \node (3) [below=.5cm of l2.center, font=\small] {L};
        \node (4) [below=.5cm of l3.center, font=\small] {L};
        \node (5) [below=.5cm of h2.center, font=\small] {H\rlap{\%}};

        \node (6) [below=.5cm of l4.center, font=\small] {\llap{\%}L};
        \node (7) [below=.5cm of h3.center, font=\small] {H};
        \node (8) [below=.5cm of l5.center, font=\small] {L};
        \node (9) [below=.5cm of h4.center, font=\small] {\underline{H}};
        \node (10)[below=.5cm of l6.center, font=\small] {L};
        \node (11)[below=.5cm of h5.center, font=\small] {H\rlap{\%}};

        \draw [-, densely dashed] (h1.south) to (2.north);
        \draw [-] (v1.south) to (3.north);
        \draw [-] (l2.south) to (3.north);
        \draw [-] (v2.south) to (3.north);
        \draw [-] (l3.south) to (4.north);
        \draw [-, densely dashed] (v3.south) to (4.north);
        \draw [-, densely dashed] (h2.south) to (5.north);

        \draw [-, densely dashed] (h3.south) to (7.north);
        \draw [-] (v4.south) to (8.north);
        \draw [-] (v5.south) to (8.north);
        \draw [-, densely dashed] (h4.south) to (9.north);
        \draw [-] (l6.south) to (10.north);
        \draw [-, densely dashed] (h5.south) to (11.north);

        \node (o) [below=.5cm of v2.center, font=\small] {\hphantom{H}};
        \node (o1)[below=.3cm of o.center, font=\small] {\llap{no }H\rlap{
        here}};
        \draw [->] (o1.north) to (o.center);

    \end{tikzpicture}%\vspace{.5\baselineskip}
    ‘we bought two’ \tab{} \tab{‘we bought these’}
    \z
\z
We see this between a numeral and noun in (\ref{ex:key:13.42}a) and between a
numeral and a preceding verb in (\ref{ex:key:13.42}b). We know that /bì-bìri/ has a
/L/ on its prefix because of the augmented form,  \emph{é{}-bì-bìrí} ‘(the)
two’, where the normally L augment receives a H from \gls{HTI}. Positing an initial
\%L was said to be unmotivated for \ili{Luganda}, but is even more so in Lusoga,
which otherwise doesn’t have clause-internal \%L. This is, however, the only
situation I have discovered to date where a /L/ does not trigger \gls{HTI}.

The second issue also characterizes both languages, this time in exactly the
same way. The question is why \gls{HTA} always has to leave at least one L tone
behind. This is seen in the \ili{Luganda} sentences in (\ref{ex:key:13.43}a,b).

\ea\label{ex:key:13.43}
    \NumTabs{7}
	\ea verb + object\\
        a-l\tn{h1}{á}b-\tn{l1}{à} bi-tabo \tab{${\rightarrow}$}
        \tab{\tn{l2}{à}-l\tn{h2}{á}b-\tn{l3}{à} \underline{bì}-táb\tn{h3}{ó}}
        \tab{‘s/he sees \textsc{books}’}\\
    \begin{tikzpicture}[overlay, remember picture]
        \node [below=.2cm of h1.center, font=\small] {H};
        \node [below=.2cm of l1.center, font=\small] {L};
        \node [below=.2cm of l2.center, font=\small] {\llap{\%}L};
        \node [below=.2cm of h2.center, font=\small] {H};
        \node [below=.2cm of l3.center, font=\small] {L};
        \node [below=.2cm of h3.center, font=\small] {H\rlap{\%}};
    \end{tikzpicture}\vspace{.5\baselineskip}
	\ex object + object\\
        te-y-a-b\tn{h1}{a}l-\tn{l1}{i}r-\tn{l2}{a} mu-límí
        bi-k\tn{h2}{ó}p\tn{l3}{ò} \hfill ${\rightarrow}$ \hfill
        t\tn{l4}{è}-y-à-b\tn{h3}{á}l-\tn{l5}{ì}r-\tn{l6}{à} \underline{mù}-límí
        bí-k\tn{h4}{ó}p\tn{l7}{ò}\\\vspace{1\baselineskip}
    \begin{tikzpicture}[overlay, remember picture]
        \node [below=.2cm of h1.center, font=\small] {H};
        \node [below=.2cm of l1.center, font=\small] {L};
        \node [below=.2cm of l2.center, font=\small] {L};
        \node [below=.2cm of h2.center, font=\small] {H};
        \node [below=.2cm of l3.center, font=\small] {L};
        \node [below=.2cm of l4.center, font=\small] {\llap{\%}L};
        \node [below=.2cm of h3.center, font=\small] {H};
        \node [below=.2cm of l5.center, font=\small] {L};
        \node [below=.2cm of l6.center, font=\small] {L};
        \node [below=.2cm of h4.center, font=\small] {H};
        \node [below=.2cm of l7.center, font=\small] {L};
    \end{tikzpicture}\vspace{.5\baselineskip}
        ‘s/he didn’t count cups for the farmer’
    \ex proclitics\\
        by-àà= [ bà= [ kàtààmb\tn{hl}{â}  \tab{‘(it’s) those of the Katambas’}
    \begin{tikzpicture}[overlay, remember picture]
        \node [below=.2cm of hl.center, font=\small] {HL};
    \end{tikzpicture}\vspace{.5\baselineskip}
    \z
\z
As seen, the H\% in (\ref{ex:key:13.43}a) is anticipated onto the preceding mora,
and yet the prefix \emph{bì-} stays L. In (\ref{ex:key:13.43}b), the H of /bi-kópo/
‘cups’ is anticipated up to the second syllable of toneless /mu-limi/ ‘farmer’,
leaving the prefix L. In addition, \gls{HTA} does not apply from the host onto
proclitics,\is{clitics} as seen in (\ref{ex:key:13.42}c). The question is: What’s wrong with
prohibited L to H sequences in the following corresponding outputs?

\ea\label{ex:key:13.44}
    \NumTabs{7}
    \ea[*]{tè-y-à-l\tn{h1}{á}b-\tn{l1}{à} ] [ b\tn{v1}{í}-k\tn{h2}{ó}p\tn{l2}{ò} \tab{‘s/he didn’t see cups’}}
        \begin{tikzpicture}[overlay, remember picture]
            \node (1) [below=.5cm of h1.center, font=\small] {H};
            \node (2) [below=.5cm of l1.center, font=\small] {L};
            \node (3) [below=.5cm of h2.center, font=\small] {H};
            \node (4) [below=.5cm of l2.center, font=\small] {L};

            \draw [-] (h1.south) to (1.north);
            \draw [-] (l1.south) to (2.north);
            \draw [-, densely dashed] (v1.south) to (3.north);
            \draw [-] (h2.south) to (3.north);
            \draw [-] (l2.south) to (4.north);
        \end{tikzpicture}\vspace{.5\baselineskip}
    \ex[*]{te-y-à-l\tn{h1}{á}b-\tn{l1}{à} ]  [ b\tn{v1}{í}-t\tn{v2}{á}b\tn{h2}{ó} \tab{‘s/he didn’t see books’}}
        \begin{tikzpicture}[overlay, remember picture]
            \node (1) [below=.5cm of h1.center, font=\small] {H};
            \node (2) [below=.5cm of l1.center, font=\small] {L};
            \node (3) [below=.5cm of h2.center, font=\small] {H};

            \draw [-] (h1.south) to (1.north);
            \draw [-] (l1.south) to (2.north);
            \draw [-, densely dashed] (v1.south) to (3.north);
            \draw [-, densely dashed] (v2.south) to (3.north);
            \draw [-, densely dashed] (h2.south) to (3.north);
        \end{tikzpicture}\vspace{.5\baselineskip}
    \ex[*]{by-\tn{l1}{à}\tn{v1}{à}= [ b\tn{v2}{á}= [
        k\tn{v3}{á}tá\tn{v4}{á}\tn{h1}{mb}\tn{hl}{â}  \tab{‘those of the Katambas’}}
        \begin{tikzpicture}[overlay, remember picture]
            \node (1) [below=.5cm of l1.center, font=\small] {\llap{\%}L};
            \node (2) [below=.5cm of h1.center, font=\small] {H};
            \node (3) [below=.5cm of hl.east, font=\small] {L};

            \draw [-, densely dashed] (l1.south) to (1.north);
            \draw [-, densely dashed] (v1.south) to (1.north);
            \draw [-, densely dashed] (v2.south) to (2.north);
            \draw [-, densely dashed] (v3.south) to (2.north);
            \draw [-, densely dashed] (v4.south) to (2.north);
            \draw [-] (hl.south) to (2.north);
            \draw [-] (hl.south) to (3.north);
        \end{tikzpicture}\vspace{.5\baselineskip}
    \z
\z
In (\ref{ex:key:13.44}a) we see that \gls{HTA} has applied word-internally. As
we have said, \gls{HTA} can only apply if it can cross a word boundary onto a $\varnothing$
mora. The problem in (\ref{ex:key:13.44}b) is that \gls{HTA} should leave one L
behind. ((\ref{ex:key:13.43}b) shows the same with a lexical /H/.) Finally,
(\ref{ex:key:13.44}c) shows that a proclitic\is{clitics} doesn’t count as “crossing a word
boundary”. Why should all of the above examples prohibit \gls{HTA} from hitting
every available toneless mora on its leftward path?

The answer is that the ungrammatical forms in \eqref{ex:key:13.43} have the
prohibited configuration in \eqref{ex:key:13.45}:

\begin{exe}
    \NumTabs{7}
    \ex[*]{\tn{m1}{µ} \tab{\tn{m2}{μ}} \tab{(\textsc{NoJump})}}\label{ex:key:13.45}
    \begin{tikzpicture}[remember picture, overlay]
        \node (1) [below=.5cm of m1.center, font=\small] {L};
        \node (p) [right=.35cm of 1.center, font=\small] {\tss{PW}[};
        \node (2) [below=.5cm of m2.center, font=\small] {H};

        \draw [-] (m1.south) to (1.north);
        \draw [-] (m2.south) to (2.north);
    \end{tikzpicture}
\end{exe}
The prohibited sequence is one where one would jump from a L to a H across a
\gls{PW} boundary. This \textsc{NoJump} constraint has the following
“conspiratorial” effects on \gls{HTA}: (i) It stops the H from reaching the
first mora of a word, which could then be preceded by a (\%)L; (ii) It stops
the H from reaching the first mora of a proclitic,\is{clitics} which would have be
PW-initial, preceded by a (\%)L.  \textsc{NoJump} is the kind of OT constraint
that can of course be dominated by another constraint, e.g. faithfulness to an
input /H/, as in \ili{Luganda} \emph{tè-y-à-láb-à} \emph{bí-bàlá} ‘s/he didn’t see
fruits’, where \emph{bí-bàlá} ‘fruits’ exceptionally has a /H/ prefix. The
constraint in \eqref{ex:key:13.45} can stop the creation of a L \tss{PW}[ H
output, but cannot remove a word-initial H tone. Of course the remaining
question is why \ili{Luganda} and \ili{Lusoga} bother to implement \gls{HTA} at all, since
the affected moras would otherwise have become L, presumably by default. For
this \citet{Selkirk2016} has proposed the constraint \textsc{HTS-left}: H tone
wants to spread to the left as far as it can go. The constraint in
\eqref{ex:key:13.45} puts a check on HTS-left: It spreads as far as it can, but
stops short if the result would be a L \tss{PW}[ H sequence.

\section{Conclusion}

To summarize the findings for \ili{Lusoga}, there is no empirical evidence for a
prosodic domain\is{prosodic domains} corresponding to the \gls{TP} in \ili{Luganda}.
Specifically, there is no evidence that what precedes the verb is treated
differently from what follows it. The domain corresponding to the
\gls{TG}\is{tone} in \ili{Luganda} does exist but is more restricted, being limited to
certain word=enclitic combinations.\footnote{As pointed out to me by Jenneke
    van der Wal (p.c.), it is possible to treat such word=enclitic\is{clitics} combinations
    as recursive phonological words, i.e.\ [[~word~]\tss{PW}~clitic]\tss{PW},
since they share the same tonal properties as the lexical phonological word.}
At this point one might ask what other evidence there might be for prosodic
domains\is{prosodic domains} in \ili{Lusoga}.  Two possibilities are intonation,
which has thus far not yielded anything concrete, and instrumental phonetic
studies, e.g. on segment durations, which I have not done—and which in any case
would take us beyond my question, which had to do with whether there are
discrete, categorical effects of prosodic domains in Lusoga.

I would like to conclude with some further thoughts about \ili{Lusoga} in terms of
linguistic typology, defined for our purposes as the study of how languages are
the same vs.\ different. First, since there is no known empirical evidence to
choose between the two hypotheses in \eqref{ex:key:13.26}, \ili{Lusoga} is not a
counterexample to the claim that syntax--phonology\is{syntax--phonology
interface} “matching” is universal.  Second, nothing looks syntactically or
prosodically aberrant in Lusoga. Rather, it is the lack of interest that Lusoga
shows for prosodic constituents that is striking, particularly from a \ili{Bantu}
point of view. In fact, \ili{Lusoga} provides the missing “cell” in the typology of
whether \glspl{LD}\is{left dislocation} and \glspl{RD}\is{right dislocation} phrase with the main clause in
Bantu:\is{phonological phrases}

\newpage

\ea\label{ex:key:13.46}
    \ea \begin{tabular}{|c|cc|}\hline \gls{LD} & S & RD\\\hline\end{tabular}\quad \ili{Luganda}
    \ex \begin{tabular}{|cc|c|}\hline \gls{LD} & S & RD\\\hline\end{tabular}\quad \ili{Haya}
    \ex \begin{tabular}{|c|c|c|}\hline \gls{LD} & S & RD\\\hline\end{tabular}\quad \ili{Chichewa}
    \ex \begin{tabular}{|ccc|}\hline \gls{LD} & S & RD\\\hline\end{tabular}\quad \ili{Lusoga}
    \z
\z

We have already seen that \ili{Luganda} and \ili{Haya} are mirror images of each other as
far as whether \glspl{LD}\is{left dislocation} (Luganda) or \glspl{RD}\is{right dislocation} (Haya) are marked off from
the main clause.  \ili{Chichewa} has been reported to mark off both \glspl{LD}\is{left dislocation} and
RDs (\citealt[1966]{DowningMtenje2011}-7). Finally \ili{Lusoga} provides the fourth
possibility: Neither \glspl{LD}\is{left dislocation} nor \glspl{RD}\is{right dislocation} are marked off.

The \ili{Lusoga} distinterest in marking prosodic domains\is{prosodic domains} is
remarkable from a Bantuist and perhaps universalist point of view. However, it
has long been known that languages vary in how much they “care” about some of
the “best bets” in phonology. \ili{Lusoga} can now be added to the list of languages
which have shown a disregard for one or another prosodic property:

\ea\label{ex:key:13.47}
    \NumTabs{7}
    \ea syllable structure: \ili{Gokana} cares very little if at all about grouping its Cs and Vs into syllables \citep{Hyman2011}
    \ex word stress: \ili{Bella Coola} cares very little if at all about highlighting one syllable per word \citep[132]{Newman1947}
    \ex prosodic domains:  \ili{Lusoga} cares very little if at all about reflecting syntactic constituency in the post-lexical phonology (this study)
    \z
\z
For me, typology should not only determine the different ways in which
universal linguistic properties can be reflected in the grammar of a language,
but also how well a grammar can get along without signaling them at all.



\printchapterglossary{}

\section*{Acknowledgements}

This article is a revision of a paper presented at the Workshop on the Effects
of Constituency on Sentence Phonology, University of Massachusetts, Amherst, on
July 30, 2016. I would like to thank the participants for their questions and
comments. I am especially endebted to extensive comments received from two
anonymous reviewers.

{\sloppy
\printbibliography[heading=subbibliography,notkeyword=this]
}

\end{document}
