\documentclass[output=paper]{langsci/langscibook}
\ChapterDOI{10.5281/zenodo.3972852}

\author{Enoch O. Aboh\affiliation{University of Amsterdam}}

\title[Apparent violations of the final-over-final constraint]
      {Apparent violations of the final-over-final constraint:\newlineCover The case of Gbe languages}
      [Apparent violations of the final-over-final constraint: The case of Gbe languages]


\abstract{In a series of recent talks and articles, Theresa Biberauer, Anders
    Holmberg, Ian Roberts, and Michelle Sheehan argue that the final-over-final
    condition (FOFC) is an absolute universal regulating structure building.
    Yet, many languages deviate from FOFC thus suggesting that this condition
    is not ``surface-true''. The question therefore arises what factors make
    languages violate FOFC on the surface. In order to answer this question, we
    need a typology of FOFC-violating languages, as well as a detailed
    description of such violations. In this short essay, I describe FOFC violations
    in Gbe and some creoles, while relating the observed phenomena to some
    theoretical questions they raise.}

\maketitle

\begin{document}\glsresetall

\section{Introduction}

In a series of recent talks and articles, Theresa Biberauer, Anders Holmberg,
Ian Roberts, and Michelle Sheehan, analyse a very strong tendency across human
languages which appears to be indicative of an absolute universal regulating
structure building: The \gls{FOFC}\is{final-over-final condition} defined as in \REF{ex:aboh:14.1}, and further
discussed in \textcite{SheeBibRobHol2017}, henceforth
SBRH.\is{FOFC|see{final-over-final condition}}

\ea\label{ex:aboh:14.1} \emph{The final-over-final condition (FOFC)}
    \ea A head-final phrase αP cannot immediately dominate a head-initial
    phrase βP if α and β are members of the same extended projection.
    \ex *[\textsubscript{αP} [\textsubscript{βP} β γ] α], where β and  γ are
    sisters and α and β are members of the same extended projection.
    \z
\z

FOFC is not bidirectional since the reverse does not hold: “a head-initial
phrase αP may dominate a phrase βP which is either head-initial or head-final,
where α and β are heads in the same extended projection”
\parencite[cf.][171]{BibHolRob2014}.

Accordingly, \gls{FOFC}\is{final-over-final condition} makes strict predictions both in terms of surface typological
variation and possible outcomes of \isi{language change}
\parencite[cf.][]{BibNewShee2009}. For instance, \gls{FOFC}\is{final-over-final condition} predicts the structures
in (\ref{ex:aboh:14.2}a--c) to exist with the exclusion of the pattern in (\ref{ex:aboh:14.2}d)
\parencite[cf.][171]{BibHolRob2014}.

\begin{exe}
\ex\label{ex:aboh:14.2} Harmonic structures\\
\begin{minipage}[t]{.5\linewidth}
    \exi{}  a. Consistent head-final\smallskip\\
            \begin{tikzpicture}[baseline=(root.base)]

                \Tree 	[.\node(root){β$'$};
                            [.αP
                                γP
                                \textbf{\textit{α}}
                            ]
                            \textbf{\textit{β}}
                        ]

            \end{tikzpicture}
\end{minipage}%
\begin{minipage}[t]{.5\linewidth}
    \exi{b.} Consistent head-initial\smallskip\\
            \begin{tikzpicture}[baseline=(root.base)]

                \Tree 	[.\node(root){β$'$};
                            \textbf{\textit{β}}
                            [.αP
                                \textbf{\textit{α}}
                                γP
                            ]
                        ]

            \end{tikzpicture}
\end{minipage}\smallskip\\
\sn Disharmonic structures\\
\begin{minipage}[t]{.5\linewidth}
    \exi{} c. Initial-over-final\smallskip\\
            \begin{tikzpicture}[baseline=(root.base)]

                \Tree 	[.\node(root){β$'$};
                            \textbf{\textit{β}}
                            [.αP
                                γP
                                \textbf{\textit{α}}
                            ]
                        ]

            \end{tikzpicture}
\end{minipage}%
\begin{minipage}[t]{.5\linewidth}
    \exi{d.} Final-over-initial\smallskip\\
            \begin{tikzpicture}[baseline=(root.base)]

                \Tree 	[.\node(root){\llap{*}β$'$};
                            [.αP
                                \textbf{\textit{α}}
                                γP
                            ]
                            \textbf{\textit{β}}
                        ]

            \end{tikzpicture}
\end{minipage}

\end{exe}

In its strong version, the generalisation in \REF{ex:aboh:14.2} could suggest that
the human mind \enquote{prefers} harmonic structures (\ref{ex:aboh:14.2}a,b), tolerates one type of
disharmonic structure in (\ref{ex:aboh:14.2}c), and totally excludes the disharmonic structure
in (\ref{ex:aboh:14.2}d). This view is obviously misleading since, looking at surface
form only, disharmonic structures abound in languages. This is, for instance,
the case in Kwa (see the discussion below), and in Sinitic (cf.
\citealt{HsiehSybesma2007}, \citealt{SybesmaLi2007}, \citealt{Chan2013} and
references therein). On the basis of his database, \citet{Dryer1992}
concludes that completely harmonic languages actually represent a minority.
Instead, the common cross-linguistic pattern seems to be that languages are
rigidly consistent in some domains, but less so in other domains. FOFC
therefore seems to strictly constrain certain core structures only. Given its
surface flexibility, one could consider the \gls{FOFC}\is{final-over-final condition} effect to derive from
processing constraints facilitating parsing. If one were to adopt
\citeapos{hawkins83} \textit{cross-category harmony}, defined in terms of head
dependent order preferences, or his 1990 \textit{early immediate constituent} principle
suggesting fast recognition of the immediate constituents of a mother node, its
seems intuitive that the parser would prefer orders in which heads and
dependents can be easily identified. In this regard, learning biases seem to
favour certain orders over others. Under this view, \gls{FOFC}\is{final-over-final condition} would be essentially a
third factor phenomenon, required by “principles of efficient computation” in
terms of \citet{Chomsky2005} (cf. \citealt{Walkden2009} for discussion).

SBRH (2017) argue for a different view. \gls{FOFC}\is{final-over-final condition} is a property of structure
building. At this point, the question arises how the notion of \enquote{harmony}
relates to structure building and computation. If \isi{Merge} applies to (categorial)
features only, and embeds no spell-out specification, how can we decide that
(\ref{ex:aboh:14.2}d) is computationally disharmonic compared to (\ref{ex:aboh:14.2}a)? If on the other hand, one
assumes \citegen{grimshaw91} extended projection and some version of
\citegen{Kayne1994} \gls{LCA},\is{linear correspondence axiom} as SBRH (2017) do, then disharmonic structures can be
understood as involving featural mismatches within a functional sequence. Under
this latter view, the bulk of apparent counterexamples to \gls{FOFC}\is{final-over-final condition} would derive
from movement: structures obey \gls{FOFC}\is{final-over-final condition} underlyingly, even though movement
operations may lead to apparent surface violations.\is{LCA|see{linear correspondence axiom}}

It seems to me that two fundamental questions arise here that merit further
investigation: The first question deals with the relation between the \gls{LCA}\is{linear correspondence axiom} and
FOFC, and why the language faculty (in the narrow sense,
cf.~\citealt{hauserchomskyfitch}) would involve such apparently competing
linearization mechanisms. The issue is not trivial as it relates to the
question of the place of linearization within the human faculty of language
(cf.~\citealt{ChoGalOtt2019} and \citealt{Kayne2018} for discussion). I will not
address this question any further in this essay. The second question I will be
concerned with instead is of a typological nature.  Why do some languages seem
to violate \gls{FOFC}\is{final-over-final condition} massively on the surface form? If \citet{Dryer1992} is right,
such violations would be the norm, while \gls{FOFC}\is{final-over-final condition} compliant languages would be the
exception. Why would this be if \gls{FOFC}\is{final-over-final condition} holds on structure building? Why would
languages systematically diverge from core principles imposed by the
computational system? For example, there does not seem to be such a massive
violation of the extended projection principle, a potential universal of
natural languages constraining structure building. In order to understand FOFC
apparent violations therefore, we need to take a closer look at the empirical
facts.

As I will show in the following paragraphs, the Gbe languages (and for that
matter many Niger-Congo languages) involve apparent violations of FOFC. I have
discussed many of these patterns in previous work and proposed an analysis in
terms of the \gls{LCA}.\is{linear correspondence axiom} Since its formulation in the early 2000s, the tenants
of \gls{FOFC}\is{final-over-final condition} have also reported similar patterns cross-linguistically and have
suggested various analyses to account for them (see SBRH 2017 and references
therein).  For instance, final negative markers, such as instantiated in the
Fongbe example in \REF{ex:aboh:14.3a}, can be analysed as not being merged
within the functional sequence of TP \parencite[cf.][]{BibHolRob2014}.  That
such a view is indeed adequate can be shown by the fact that the \ili{Fongbe} yes-no
question in \REF{ex:aboh:14.3b} displays a similar sentence-final particle,
which \textcite{Aboh2010a,Aboh2010b} shows interacts with final negation in
Gbe, as indicated by example \REF{ex:aboh:14.3c}. In this example, the negative
particle precedes a focus marker which in turn precedes the question particle.

\ea\label{ex:aboh:14.3} \ili{Fongbe}
    \ea\label{ex:aboh:14.3a}
        \gll    K\`ɔkú ná x\`ɔ às\'ɔn \'ɔ \v{a} \\
                Koku \Fut{} buy crab \Det{} \Neg{} \\
        \glt    \enquote*{Koku will not buy the crab.}
    \ex\label{ex:aboh:14.3b}
        \gll    Kòfí ɖù às\'ɔn \'ɔ à? \\
                Kofi eat crab \Det{} \glossQ{} \\
        \glt    \enquote*{Did Kofi eat the crab?}
    \ex\label{ex:aboh:14.3c}
        \gll    Kòfí ɖù às\'ɔn \'ɔ \v{a} w\`ɛ à? \\
                Kofi eat crab \Det{} \Neg{} \Foc{} \glossQ{}\\
        \glt    \enquote*{\textsc{Did Kofi not eat the crab?}}
    \z
\z

Facts like these led \citet{Aboh2010a} to propose that the sentence-final
negative particle belongs to the C-domain in Gbe. These data from the Gbe
languages, already show that \gls{FOFC}\is{final-over-final condition} as formulated in \REF{ex:aboh:14.1} is certainly
not “surface-true”. Can we, however, claim that \gls{FOFC}\is{final-over-final condition} constrain the underlying
structure? Given that SBRH (2017) adopt \citegen{grimshaw91} notion of extended
projection, we can answer this question only if we are able to characterize
precisely the featural bundle of the different heads within the functional
sequence of the left periphery in the Gbe languages. Though there is now a
significant body of literature on the complementizer\is{complementizers} system of the Gbe (and
other Kwa) languages, it is reasonable to say that we still do not have a
fine-grained map of the featural specifications of C-type heads in these
languages, and we do not know how learners acquire these features.\largerpage[-1]

This last question becomes even more critical when considering acquisition\is{language acquisition} in
contact situations. Indeed, if \gls{FOFC}\is{final-over-final condition} is an inviolable condition, as suggested by
SBRH (2017), one could imagine that the \gls{PLD} that learners are
exposed to would not generally contain systematic cues for them to derive
FOFC-violating grammars. Put differently, learners must have a way of deducing
underlying FOFC-compliant structures from massively FOFC-violating surface
forms. One would therefore expect superficial FOFC-violating orders (e.g.,
VO-Aux, VO-question particle, VO-Neg) to be unstable and eventually lost in
contact situations. This expectation, however, is not met in the case of
certain creole languages. Indeed, creole languages which emerged in colonial
settings involving enslaved Niger-Congo learners (i.e., speakers of Kwa and
Kikongo) inherited typical Niger-Congo disharmonic structural properties and
therefore display comparable \gls{FOFC}\is{final-over-final condition} surface violations.

Since the original formulations of FOFC, I have discussed some of these surface
FOFC violations with Ian Roberts and Theresa Biberauer. I was therefore only
partially surprised on June 3, 2016 at 3:45pm, when I received a mail from Ian,
which read as follows:\footnote{I am always excited by mails from Ian who also
    happens to be one of my favourite teachers and now very good colleague and
    friend. Ian introduced me to diachronic syntax at a time I had no idea such
    a thing existed. Actually, he has in various ways inspired my recent work
    on language contact and change. In addition, as his student, I liked his
    \ili{French} accent at a time when as a \emph{Béninois} trying to make sense of
    \emph{Français Genevois}, I wondered what \ili{French} and African politicians
    meant by \enquote{la francophonie}. What’s the point if I have hard times
    understanding both \emph{Genevois} and my \ili{French} L2 speaker teacher of
    diachronic syntax? How can we account for such a variation in a principled
    manner? These questions obviously led me to my current work on \emph{hybrid
    grammars}, a concept that is actually not very far from work that Ian has
    done in collaboration with Robin Clark in the early 90s. But let us return
to our current topic of discussion.}

\blockquote{I'm looking at languages with N-A-Num-Dem U20 order in the DP to
    see what (if any) clausal word orders they correlate with. Am I right in
    thinking that \ili{Gungbe} has head-initial order in the clause? According to
    WALS, it has head-final question particles though. Is that correct? In that
    case it looks like an apparent FOFC-violator.}\largerpage[-1]

As suggested in Ian’s message, the discussion on sentences under example
\REF{ex:aboh:14.3} already indicated that the Gbe languages involve clause-final
particles that encode negation \REF{ex:aboh:14.3a}, interrogation
\REF{ex:aboh:14.3b} or a combination thereof \REF{ex:aboh:14.3c}. The following
sentence further shows that these languages display
noun-adjective-numeral-demonstrative order as illustrated in \REF{ex:aboh:14.4}.
Further note that within the DP, the determiner and the plural marker occur to
the right edge (see \citealt{Aboh2004a,Aboh2004b} and references therein for
discussion):

\ea\label{ex:aboh:14.4} \ili{Gungbe}\\
    \gll    [ Òxwé kp\`ɛví àwè éhè l\'ɔ l\'ɛ ] jró mì. \\
            {} house small two \Dem{} \Det{} \Pl{} {} please \Fsg-\Acc{} \\
    \glt    \enquote*{I like these two houses.}, lit.\ \enquote*{These
            two houses please me.}
\z

With regard to Ian’s message therefore these examples indicate that Gbe
languages may constitute counter-examples to FOFC. \citet{Sheehan2013} claims
that the number of such FOFC-violating languages is rather restricted. Since
the Gbe languages exhibit right edge (or final) functional elements both in the
nominal and clausal domain, it is important to look at the facts closely in
order to determine whether these languages represent genuine \gls{FOFC}\is{final-over-final condition} violations or
not. Given the importance of \gls{FOFC}\is{final-over-final condition} in the literature, we need to better
understand such cases of apparent violations in order to find out whether the
principle holds of structure building or whether it relates to surface
phenomena deriving from processing (cf.
\citealt{hawkins83,Hawkins1990,Walkden2009}). In order to make this first step,
the following sections are meant to present more data from Gbe and some creoles
which appear to be \gls{FOFC}\is{final-over-final condition} violations.

Recall from the formulation of \gls{FOFC}\is{final-over-final condition} in \REF{ex:aboh:14.1} that it excludes
structure (\ref{ex:aboh:14.2}d): no language should exist in which a consistent head-initial
structure is dominated by a head-final structure. Under \gls{FOFC}\is{final-over-final condition} therefore a
structure like the one in \REF{ex:aboh:14.3b} cannot have the underlying
representation \REF{ex:aboh:14.5a}, but must be analysed as in
\REF{ex:aboh:14.5b} in which the complement of the Interrogative functional
projection InterP raises to its specifier position. In these representations,
the sentence-final floating low tone expresses a question particle that takes
the clause as complement. It is worth noting, however, that \citet{Aboh2004a},
\citet{AbohPfau2011} propose the same analysis under the \gls{LCA},\is{linear correspondence axiom} hence the
necessity to tease FOFC-related and LCA-related effects apart.

\ea\label{ex:aboh:14.5}
    {\setlength\multicolsep{0pt}
    \begin{multicols}{2}
    \ea\label{ex:aboh:14.5a}
        \begin{tikzpicture}[baseline=(root.base)]

            \Tree 	[.\node(root){InterP};
                        Spec
                        [.Inter$'$
                            [.FinP
                                \edge[roof]; {Kòfí ɖù às\'ɔn \'ɔ}
                            ]
                            [.Inter
                                à
                            ]
                        ]
                    ]

        \end{tikzpicture}
    \ex\label{ex:aboh:14.5b}
        \begin{tikzpicture}[baseline=(root.base)]

            \Tree 	[.\node(root){InterP};
                        [.\node(finp){FinP};
                            \edge[roof]; {Kòfí ɖù às\'ɔn \'ɔ}
                        ]
                        [.Inter$'$
                            [.Inter
                                à
                            ]
                            \node (t) {\sout{FinP}};
                        ]
                    ]

            %\draw [->] (t.south) -- +(0,-1.0) -- +(-2.875,-1.0) -- +(-2.875,0);
            \node (fin) [below=.5cm of finp] {};
            \draw [arrow] (t.south)..controls +(south:2.0)
                and +(south east:1.75)..(fin.south);

        \end{tikzpicture}
    \z
    \end{multicols}}
\z

It appears from the examples in \REF{ex:aboh:14.3} and \REF{ex:aboh:14.4} that
the Gbe languages, like many Niger-Congo, display disharmonic structures, as
represented in (\ref{ex:aboh:14.2}c) and (\ref{ex:aboh:14.2}d), in various components of their grammar (e.g., TP,
CP, PP). Likewise, studies on creole languages have shown that some creole
languages, which emerged from the contact between Gbe languages and \ili{French}
(e.g., \ili{Haitian Creole}), or Gbe languages and \ili{English} (e.g., Sranan,
Saramaccan), exhibit similar disharmonic structures in areas of their grammar.
Together these facts suggest that such apparent violations of \gls{FOFC}\is{final-over-final condition} are
not isolated phenomena, and therefore require some explanation. Such an
explanation can only be based on a precise description of the facts. In what
follows, I take this first step and illustrate the main contexts in which
Gungbe apparently violates FOFC, and provide comparable examples in
\ili{Haitian Creole} and Suriname creoles (e.g., Sranan and Saramaccan). These
creoles emerged in the 17th century colonial plantations in Suriname and Haiti
where thousands of enslaved African speakers of Niger-Congo languages were
deported to the Americas and came into contact with the languages of European
their colonists, namely \ili{French} in Haiti and \ili{English} and \ili{Dutch} in Suriname.

\section{Initial-over-final in Gbe}

\citet{Aboh2010c} reports that \ili{Gungbe} involves two types of adpositions
labelled P1 and P2. Elements of the type P1 generally derive from posture or
locative verbs, while items of the type P2 derive from nouns expressing
landmarks or body-parts. P1 projects a head-initial structure as indicated in
\REF{ex:aboh:14.6a}. P2 on the other hand projects an apparent head-final
structure as in \REF{ex:aboh:14.6b}. When P1 and P2 co-occur, P1 must precede
the phrase headed by P2, as indicated by example \REF{ex:aboh:14.6c} further
described in \REF{ex:aboh:14.6d}.

\ea\label{ex:aboh:14.6} \ili{Gungbe}
    \ea\label{ex:aboh:14.6a}
        \gll    Súrù zé kw\'ɛ [ xlán mì ]. \\
                Suru take money {} P1 \Fsg{} {} \\
        \glt    \enquote*{Suru sent me some money.}
    \ex\label{ex:aboh:14.6b}
        \gll    Súrù x\'ɛ [ só l\'ɔ jí ]. \\
                Suru climb {} hill \Det{} P2 {} \\
        \glt    \enquote*{Suru climbed on top of the hill.}
    \ex\label{ex:aboh:14.6c}
        \gll    Súrù nyìn àgán [ xlán [ só l\'ɔ jí ]]. \\
                Suru throw stone {} P1 {} hill \Det{} P2 {} \\
        \glt    \enquote*{Suru threw a stone on top of the hill.}
    \ex\label{ex:aboh:14.6d}
        \begin{tikzpicture}[baseline=(root.base)]

            \Tree 	[.\node(root){P1P};
                        [.P1 xlán ]
                        [.P2P
                            [.DP
                                \edge[roof]; {só l\'ɔ}
                            ]
                            [.P2 jí ]
                        ]
                    ]

        \end{tikzpicture}
    \z
\z

Note that in this example, both the DP inside P2P and P2P itself display a
head-final structure embedded under the head-initial P1P. \citet{Biberauer2016}
discusses these examples and concludes that the determining factors allowing
these apparent \gls{FOFC}\is{final-over-final condition} violations could be the lower structural position of P2
compared to P1 as represented in \REF{ex:aboh:14.6d}. Furthermore, P1 and P2 are
categorially distinct: the former developed from verbs, while the latter
developed from landmark nouns (cf. \citealt{Aboh2010c}). While this view is
plausible, one would need to find out how it squares with \citegen{Aboh2010c}
subsequent suggestions that elements of the type P2 should be analysed as
heading a predicate within a possessive phrase (which according to him is
typical of such locative expressions).  The idea being that a sequence like
\emph{só l\'ɔ jí} in \REF{ex:aboh:14.6b} should be analogised to \emph{the mountain
top} in \ili{English}, in which \emph{jí}, expressing P2, heads a possessive
predicate. If this view is correct and if we maintain the notion of extended
projection as argued for in SBRH (2017), then both P1 and P2 belong to the same
extended projection, and we would have to demonstrate how they are categorially
distinct.

\section{Final-over-initial in Gbe}\largerpage

The discussion above about the yes--no question particle already showed that Gbe
languages involve instances of final-over-initial disharmonic orders within the
clausal left periphery (cf. \citealt{Aboh2016a} for further discussion). In
what follows, I show that similar disharmonic orders are found within the TP
too. In \ili{Fongbe}, for instance, the so-called completive aspect can be expressed
by complex structures in which two apparent verbs circumvent an object
(cf.~\citealt{DaCruz1995,Aboh2009,VandenBergAboh2013}).

\ea\label{ex:aboh:14.7} \ili{Fongbe} \parencite[363]{DaCruz1995}
    \multicolsep=.25\baselineskip
    \begin{multicols}{2}
    \ea\label{ex:aboh:14.7a}
        \gll    K\`ɔkú wà àz\v{ɔ} \'ɔ fó \\
                Koku do work \Det{} finish\\
        \glt    \enquote*{Koku finished doing the work.}
    \ex\label{ex:aboh:14.7b}
        \gll    K\`ɔkú ɖù m\`ɔlìnkún \'ɔ v\`ɔ \\
                Koku eat rise \Det{} finish\\
        \glt    \enquote*{Koku finished eating the rice.}
    \z
    \end{multicols}
\z

Under the assumption that the final verb is comparable to an auxiliary\is{auxiliaries} or
aspect marker of some sort, these sequences would be akin to [VO]-Aux order
which is banned in \ili{Germanic} \parencite[cf.][173]{BibHolRob2014}.
\Citet{DaCruz1995} analysed these constructions as instances of serial verb
constructions arguing that, in these constructions, the final V is a
lexical verb with the same thematic properties as in the examples in
\REF{ex:aboh:14.8} in which these verbs select for an internal argument.

\ea\label{ex:aboh:14.8} \ili{Fongbe} \parencite[363]{DaCruz1995}
    \multicolsep=.25\baselineskip
    \begin{multicols}{2}
    \ea\label{ex:aboh:14.8a}
        \gll    K\`ɔkú fó àz\v{ɔ} \'ɔ \\
                Koku finish work \Det{} \\
        \glt    \enquote*{Koku finished the work.}
    \ex\label{ex:aboh:14.8b}
        \gll    K\`ɔkú v\`ɔ m\'ɔlìnkún \'ɔ \\
                Koku finish rice \Det{} \\
        \glt    \enquote*{Koku finished the rice.}
    \z
    \end{multicols}
\z

In recent work, however, \citet{VandenBergAboh2013} argue that these
constructions should be analysed similarly to equivalent constructions in
Gungbe which do not involve two apparent verbs and in which the final position
is realised by the quantifier meaning \textit{kpó} ‘all’.

\ea\label{ex:aboh:14.9} \ili{Gungbe}
    \multicolsep=.25\baselineskip
    \begin{multicols}{2}
    \ea\label{ex:aboh:14.9a}
        \gll    Dónà wà àz\'ɔn kpó \\
                Dona do work all \\
        \glt    \enquote*{Dona did the work completely.}, \enquote*{Dona did
                all the work.}
    \ex\label{ex:aboh:14.9b}
        \gll    Dónà ɖù l\'ɛsì l\'ɔ kpó \\
                Dona eat rice \Det{} all \\
        \glt    \enquote*{Dona ate the rice completely.}, \enquote*{Dona ate
                all the rice.}
    \z
    \end{multicols}
\z

In terms of this proposal, the Gbe languages involve a TP-internal functional
projection that expresses event quantification and may be spelled out by a verb
root or a quantifier root that merges in its head. Under this view therefore,
the \ili{Fongbe} and \ili{Gungbe} sentences in \REF{ex:aboh:14.7a} and \REF{ex:aboh:14.9a},
respectively, can be described as in \REF{ex:aboh:14.10} in which the event
quantifier merges under F and takes a head-initial VP.

\ea\label{ex:aboh:14.10}
    \begin{tikzpicture}[baseline=(root.base)]
        \Tree 	[.\node(root){FP};
                    {}
                    [.F$'$
                        [.VP
                            {}
                            [.V$'$
                                [.V wà ]
                                [.DP {àz\'ɔn / àz\v{ɔ}} ]
                            ]
                        ]
                        [.F {fó / kpó} ]
                    ]
                ]

    \end{tikzpicture}
\z

If representation \REF{ex:aboh:14.10} corresponded to the underlying structure
then this and similar examples would be genuine violations of FOFC.
Alternatively, however, one can argue along the lines of
\citet{VandenBergAboh2013} that the functional element heading event
quantification is head-initial, but its complement must move leftward,
presumably to its specifier position, as in \REF{ex:aboh:14.11}. In terms of
\textcite{Aboh2004a,Aboh2004b,Aboh2010a}, this event quantifier head belongs to
the class of markers in Gbe whose complements must raise to their specifier
position.

\ea\label{ex:aboh:14.11}
    \begin{tikzpicture}[baseline=(root.base)]
        \Tree 	[.\node(root){FP};
                    [.VP
                            {}
                            [.V$'$
                                [.V wà ]
                                [.DP {àz\'ɔn / àz\v{ɔ}} ]
                            ]
                        ]
                    [.F$'$
                        [.F {fó / kpó} ]
                        \sout{VP}
                    ]
                ]

    \end{tikzpicture}
\z

Under this view and assuming that Gbe languages are underlyingly head-ini\-tial
no issue arises, but this conclusion is not immediately obvious if we assume
FOFC and if linearization is not part of core syntax.

\section{FOFC in language contact and change}\largerpage

The examples discussed thus far indicate that Gbe languages involve the
disharmonic orders in (\ref{ex:aboh:14.2}c) and (\ref{ex:aboh:14.2}d). These languages therefore seem to violate
FOFC, on the surface. As suggested in previous paragraphs, one could
hypothesise that such apparent violations of \gls{FOFC}\is{final-over-final condition} are unstable in contact
situation because \gls{FOFC}\is{final-over-final condition} constrains structure building.  Alternatively, one could
also imagine that the process being so robust in Gbe (and other Kwa), prevails
in contact situations involving Gbe or similar Niger-Congo languages and
European languages such as \ili{French} or \ili{English}. It is the latter scenario that
characterizes certain Atlantic creoles. These new languages display disharmonic
orders in areas of their grammar in a way comparable to Gbe. This is the case
in Haitian Creole spoken in Haiti, Sranan and Saramaccan spoken in Suriname.
These languages developed in the Caribbean in the late 17th and early 18th
century during European colonial expansion (cf. \citealt{Aboh2015} and
references cited there).
%
We now face the crucial question of why, during acquisition\is{language acquisition} in such
multilingual contexts, disharmonic structures win over harmonic ones even
though the computational system favours the latter.

\subsection{Initial-over-final within PP: Sranan}

Just as Gbe languages exhibit P1 and P2 categories with apparent different
headness properties, one finds equivalent adpositions in Early Sranan
\REF{ex:aboh:14.11b}, as well as in other Suriname creoles (cf. \citealt{Bruyn2003}
and references cited there).

\ea\label{ex:aboh:14.11b} Sranan \parencite[32]{Bruyn2003}
    \sn\gll Sinsi a komm \textit{na} hosso \textit{inni} \dots{} \\
            since \Tsg{} come P1 house P2 {} \\
    \glt    \enquote*{Since she entered the house \dots{}}
\z

The surface string in \REF{ex:aboh:14.11b} indicates that like in Gbe, Sranan P1 is
head-initial and takes a complement which is head-final.
\citeauthor{Aboh2010c} (\citeyear{Aboh2010c}, \citeyear{Aboh2015},
\citeyear{Aboh2016b}, \citeyear{Aboh2017}) discusses these patterns as
well as other varying word orders found within the PP in these creoles and
shows how they derive from a recombination of syntactic features selected from
Gbe-languages and from \ili{English}.

\subsection{Final-over-initial within the DP: Haitian Creole}

Similar recombination is found within the DP in \ili{Haitian Creole}
\parencite{AbohDeGraff2014,Aboh2015}. This language exhibits both prenominal
and postnominal adjectives. The definite/specificity marker must follow the
noun phrase, while the indefinite marker \emph{yon} must precede:

\ea\label{ex:aboh:14.12} Haitian Creole \parencite[117--118]{DeGraff2007}
    \ea
        \gll    Nana vann gwo wòb la \\
                Nana sell big dress \Det{} \\
        \glt    \enquote*{Nana sold the big dress.}
    \ex
        \gll    Nana vann wòb jòn la \\
                Nana sell dress yellow \Det{} \\
        \glt    \enquote*{Nana sold the yellow dress.}
    \ex\label{ex:aboh:14.12c}
        \gll    Mwen te wè yon moun \\
                \Fsg{} \Ant{} see \Det{} person \\
        \glt    \enquote*{I saw someone.}
    \z
\z

Clearly, the distribution of adjectives in \ili{Haitian Creole} is similar to\largerpage
that of \ili{French} adjectives. Under \citet{Cinque2010}, \ili{French} and other \ili{Romance}
languages which exhibit similar distributive properties involve head-initial
structures and the relative position of adjectives (i.e., pre- vs post-nominal
adjective) is derived by N(P)-movement. Taking this as our starting point, it
must be the case that the post-nominal determiner-like element in Haitian
Creole dominates a head-initial structure. This view is further corroborated by
the fact that unlike adjectives, possessive pronouns, demonstratives as well as
the number marker follow the Gbe head-final order as illustrated by example
\REF{ex:aboh:14.13}.

\ea\label{ex:aboh:14.13}
    \ea Haitian \parencite[78]{Lefebvre1998}\\
    \sn
        \gll    krab mwen sa a yo\\
                crab \First{}.\Poss{} \Dem{} \Det{} \Pl{} \\
        \glt    \enquote*{these crabs of mine}
    \ex \ili{Gungbe} \parencite{Aboh2004a,Aboh2004b}
    \sn
        \gll    àgásá cè éhè l\'ɔ l\'ɛ \\
                crab \First{}.\Poss{} \Dem{} \Det{} \Pl{} \\
        \glt    \enquote*{these crabs of mine}
    \z
\z

Yet, example \REF{ex:aboh:14.12c} clearly shows that the indefinite determiner must
precede the noun, suggesting a head-initial pattern similar to \ili{French}
\textit{une} \textit{personne} ‘a person’. Again, what we see here is a
recombination of the Gbe disharmonic order with \ili{French} harmonic order with
mixed headness properties, leading to apparent FOFC-violations.

\subsection{Final-over-initial within TP: Sranan}

In the preceding paragraphs, I showed that \ili{Gungbe}, and Gbe languages in
general, involve event quantifiers which, on the surface, seem to exhibit a
head-final structure, even though they select a head-initial VP complement.
Similar constructions are found in the Suriname creoles as well. An example
from early Sranan is given in \REF{ex:aboh:14.14} in which the so-called completive
marker, \textit{keba}, follows the verb.

\ea\label{ex:aboh:14.14}Sranan
    \sn\gll yu syi tok, nownowdei mi leri \textit{keba} taki a \enquote*{oe} musu de ini wan lo geval wan \enquote*{u}. \\
            \Tsg{} see yet now.\Red{}-day \Fsg{} learn already that the \enquote*{oe} must be every one \textsc{lo} case a \enquote*{u} \\
    \glt    \textquoteleft{}You see, right, nowadays I have learned (I know)
            that the \enquote*{oe} must be (written) as \enquote*{u} in any
            case.\textquoteright{}
\z

These constructions are discussed in \citet{VandenBergAboh2013} who propose
an \gls{LCA}\is{linear correspondence axiom} account in the lines of representation \REF{ex:aboh:14.11} above. In terms
of this analysis, \textit{keba} (also realised sometimes as \textit{kba},
\textit{kaba}) is equivalent to the Gbe event quantifiers, in that it heads a
functional projection within TP that takes the VP preceding it as complement.
The latter must raise to [spec FP] to be licensed as described in
\REF{ex:aboh:14.11}.

The preceding paragraphs show that the Gbe languages and some creoles involve a
significant body of syntactic patterns which systematically violate \gls{FOFC}\is{final-over-final condition} on the
surface. These patterns are found within the determiner phrases, adpositional
phrases, tense or aspect phrases as well as within the complementizer\is{complementizers} system.
With regard to aspect phrases, for instance, the discussion on event
quantifiers suggests that these languages involve some event quantifier that
can project above the VP and surface as head-final structure even though the
embedded VP is head-initial. Assuming that these event quantifiers are
aspectual in nature (as commonly accepted in the literature), they are
comparable to aspect markers which, in many languages, are expressed by various
auxiliaries. Accordingly, we reach the description that these languages appear
to exhibit the order [VO]--Aux/Asp in which a head-initial VP precedes an
aspect marker or auxiliary\is{auxiliaries} which appears to be head-final. Since it is the
absence of the [VO]-Aux order in \ili{Germanic} which led to the postulation of
\gls{FOFC} (cf. SBRH 2017), one wonders why these languages display a sequence
in surface form that is banned in Germanic? If the ban in \ili{Germanic} holds on
surface form, why does it not apply to Gbe and similar languages as well? Given
such sharp discrepancies between Gbe languages (Niger-Congo), some creoles, and
Germanic, the question arises what fundamental aspect of Human Language
Capacity explains FOFC, and the observed cross-linguistic variation. Theresa
Biberauer’s chapter in SBRH (2017) addresses some of these questions, but I
hope that the data provided here will allow further research in this domain.

\printchapterglossary{}

%\section*{Acknowledgements}

{\sloppy
\printbibliography[heading=subbibliography,notkeyword=this]
}

\end{document}
