\documentclass[output=paper,modfonts,nonflat]{langsci/langscibook}

\usepackage{qtree}
\newcommand{\ro}[1]{$\sqrt{\mbox{\sc {#1}}}$}

\title{Past/passive participles and locality of attachment}
\author{Alison Biggs\affiliation{Georgetown University}}
% \chapterDOI{} %will be filled in at production

\abstract{%
In this short chapter I outline some properties of the structure “I’m done
writing Chapter 3”, which does not appear to have been formally analysed
before. Concentrating on the \emph{-en}/\emph{-ed} participle and the
structure’s semantics, I suggest that this is a kind of stative \isi{passive}, of a
kind not previously known. I offer a syntactic analysis, in which an aspectual
projection can stativize the eventive syntax it hierarchically embeds.}

\begin{document}\glsresetall

\maketitle

\section{Introduction}

English is traditionally described as having three \isi{participles} of the same
form: the stative \isi{passive}, verbal \isi{passive}, and perfect (\ref{sharedf}).

\begin{exe}
\ex\label{sharedf}
    \begin{xlist}
    \ex The letters are well written.
    \ex The letters were written by her.
    \ex She has written a letter.
    \end{xlist}
\end{exe}

Establishing points of difference and commonality in the syntax and
interpretation of the structures in (\ref{sharedf}) has played a central role
in the development of theories of syntax and word formation.

A particular pattern of interest for statives has been whether the states they
describe follow from a prior event. In this, \enquote{resulting state} statives
(\ref{priore}a), which do follow from a prior event, can be distinguished from
\enquote{pure} statives (\ref{priore}b), which lack event implications altogether
\citep{Parsons1990a, Embick2004a}.

\begin{exe}
\ex\label{priore}
    \begin{xlist}
    \ex The soup is cooled. \hfill \emph{Resulting state}
    \ex The soup is cool. \hfill \emph{Pure state}
    \end{xlist}
\end{exe}

It is often observed that in modern \ili{English}, the participle\is{participles} (potentially)
has a resulting state interpretation (as opposed to any other kind of state)
only where the past/passive morpheme \emph{-ed/-en} attaches to the item that
describes the event from which the state results~\citep[e.g.\ ][]{Parsons1990a,
Kratzer2001a, Alexiadou2008a, AlexiadouEtAl2015}.

As illustration, in (\ref{affix}a) \emph{-ed/-en} attaches to the main verb,
and a surface subject is interpreted as being in a state that results from a
(writing) event. In  contrast in (\ref{affix}b), \emph{-ed/-en} attaches to
non-main verb \emph{be}, with the present/active form \emph{-ing} attaching to
the main verb, and does not describe a resulting state.

\begin{exe}
\ex\label{affix}
    \begin{xlist}
    \ex Chapter 3 is written.
    \ex She has been writing Chapter 3 for days.\footnote{The present/active is often analysed as a state in temporal semantic terms~\citep[e.g.\ ][]{Parsons1990a}. Temporal semantic states are not usually analysed in the same way as resulting states of the kind of interest here.}
    \end{xlist}
\end{exe}

The contrast in (\ref{affix}) can be captured by some version of (\ref{16.gen}):

\begin{exe}
\ex\label{16.gen}  A resulting state interpretation requires an embedded lexical predicate in past/ \isi{passive} participle\is{participles} form.
\end{exe}

The structure in (\ref{d}) (`\emph{be done} VP-ing') seems to present an
exception to this generalization. (\ref{d}) can describe the object as in a
state resulting from the (writing Chapter 3) event, but the past/passive
morphology attaches to \emph{do}, with the embedded verb form a present/ active
participle.

\begin{exe}
\ex\label{d} She is done writing Chapter 3.
\end{exe}

As far as I can tell the structure in (\ref{d})  has not been analysed before,
and I label it the \emph{done}-state. (\ref{d}) has some unusual properties:
for example, morpho-semantically it is a stative \isi{passive}; syntactically,
however, (\ref{d}) is transitive and active, in the sense that it licenses a
direct object. The key point to be investigated in this paper is that (\ref{d})
describes a resulting state, even though the past/passive affix attaches to the
embedding item \emph{do}, apparently violating (\ref{16.gen}).

The paper is structured as follows. Section~\ref{ssem} makes precise that the
\enquote{resulting state} interpretation of the \emph{done}-state can be a
Target State. Section~\ref{ssyn} discusses the structure of the
\emph{done}-state, highlighting some implications for previous analyses of
Target State participles. Section~\ref{derivation} discusses and rejects an
alternative perfect analysis. Section~\ref{summary} concludes.

\section{The interpretation of the \emph{done}-state}\label{ssem}

\enquote{States} form a heterogeneous class~\citep[see
especially][]{Kratzer2001a}. Of interest in this chapter are Target States.

Target States are those which describe a temporary or reversible state, i.e.
the state held by the surface subject of (\ref{targetresult}a)
\citep{Parsons1990a, Kratzer2001a}; these are interpreted as being
characteristic of or resulting from the prior event. Target States are
typically contrasted with Resultant States, which simply describe the
post-state of an event; this post-state is interpreted as holding forever after
the prior event, e.g.\  the state held by the subject of
(\ref{targetresult}b).

\begin{exe}
\ex\label{targetresult}
\begin{xlist}
\ex The soup is cooled. \hfill \emph{Target State}
\ex I've eaten lunch. \hfill \emph{Resultant State}
\end{xlist}
\end{exe}

One surprising interpretation of the \emph{done}-state is a Target State of the
direct object. The Target State of the \emph{done} structure in (\ref{soup}a,b)
is the state resulting from the event described by the embedded VP\@. An
important point I will not address here is that the stateholder subject of the
\emph{done}-state is also interpreted as the agent of the embedded VP\@.

\begin{exe}
\ex\label{soup}
\begin{xlist}
\ex I'm done cooling the soup.
\ex She's done writing Chapter 3.
\end{xlist}
\end{exe}

Target and Resultant States describe states that follow a prior event, but
differ in the characterisation of the prior event.\footnote{Target States
always entail a Resultant State reading, e.g.\   (\ref{targetresult}a).  As
such the \emph{done}-state also has a Resultant State reading.}  Target States
refer to states that describe results of events, and the result is understood
as ongoing at the time of reference or evaluation~\citep{Kratzer2001a}, an
effect known as \enquote{current relevance}. Current relevance can be
demonstrated by certain kinds of modifiers, which are licit with Target State
interpretation only if the adverb can be construed as modifying a result of the
state. It is said to follow that Target States are not possible with \isi{adverbs} of
quantity or cardinality~\citep{Mittwoch2008a} (\ref{resst}a,b).
(Ungrammaticality refers to the Target State interpretation).

\begin{exe}
\ex\label{resst}
\begin{xlist}
\ex I'm done cutting his hair (*twice).  \hfill \emph{Done-state}
\ex The windows are closed (*each evening). \hfill \emph{Adjectival Passive}
\end{xlist}
\end{exe}

As the Resultant State describes the post-state of an event, the event may be over by the time of reference or evaluation, and the state does not require current relevance. Lack of current relevance (despite present tense) is illustrated by the perfect in (\ref{targetresult}b) and adjectival \isi{passive} in (\ref{nocr}a). (\ref{nocr}b) illustrates that quantity/cardinality \isi{adverbs} can modify Resultant States.

\begin{exe}
\ex\label{nocr}
\begin{xlist}
\ex The theory is proven.
\ex The windows are closed three times each evening.
\end{xlist}
\end{exe}

The Target State interpretation is also clearly distinct from a second
interpretation of the \emph{done}-state that I call the \enquote{cessation} or
\enquote{termination} reading, in which the surface subject is interpreted as
having ceased or terminated engagement in the activity described by the
embedded verb. The cessation reading of (\ref{writeprog}) is simply that Maria
is no longer \emph{writing Chapter 3}, i.e.\ it relates to her agency rather
than her (resulting) state. Cessation is therefore clearly different from the
Target State that results from the embedded VP\@.\footnote{Cessation bears a
    superficial similarity to the `{\em done-with}' construction (\emph{I'm
    done with baking cakes}), a structure which also, to the best of my
    knowledge, has not been analysed before. Like the \emph{done}-state, {\em
    done-with} is morpho-semantically a stative \isi{passive}; however, {\em
done-with} is syntactically intransitive, while the \emph{done result} is
transitive. The PP in {\em done-with} presumably has a nominal complement.
There are many syntactic and semantic differences between the constructions,
but for reasons of space I will point out just one: {\em done-with} requires an
agentive surface subject, while the {\em done}-result does not: \emph{The water
is done (*with) boiling /  The machine is done (*with) washing that load.}} For
reasons of space I leave to future work whether the cessation interpretation
derives from the same structure as that of the Target State.

\begin{exe}
\ex\label{writeprog} Maria's done writing Chapter 3 for the moment -- she has to run more subjects before writing more.
\end{exe}

One reason to analyze \emph{done} as a stative participle\is{participles} is that it only
occurs with the auxiliary\is{auxiliaries} \emph{be}.

\begin{exe}
\ex\label{donehave} *I've done baking the cake.\footnote{A reviewer accepts \emph{have} in (\ref{donehave}), and highlights that Google returns attested examples. I found: \emph{I've done watching the 6 seasons, I have watched the movie countless time [sic.], I've done reading the book.}, retrieved 10/11/2017 http://sachzca.blogspot.co.uk/2008/11/. This is ungrammatical for all speakers I consulted, but, judging from context, the \emph{have} variant does not seem to have a Target State reading, so I have left the asterisk in the main text. The observation of variation clearly requires further investigation.}
\end{exe}

This makes \emph{done} unlike aspectual predicates, which can appear with the
auxiliary \emph{have}.

\begin{exe}
\ex\label{stop}
\begin{xlist}
\ex I've finished/stopped running.
\ex I've finished/stopped writing Chapter 3.
\end{xlist}
\end{exe}

With the stative \emph{be} (\ref{finishbe}a), on the other hand, \emph{finish}
also has the Target State interpretation. Some speakers also accept \emph{be}
\emph{finish}-VP-ing (\ref{finishbe}b) (although not most of the British
speakers I consulted, including myself), apparently again with the Target State
interpretation.

\begin{exe}
\ex\label{finishbe}
\begin{xlist}
    \ex[\hphantom{\%}]{I'm finished.}
    \ex[\%]{I'm finished baking the cake.}
\end{xlist}
\end{exe}

Pending further investigation, I take the auxiliary\is{auxiliaries} \emph{be} to be
indicative of the structure that derives the Target State interpretation.

\section{The structure of the \emph{done}-state} \label{ssyn}

Different stative interpretations (such as the difference between Target and
Resultant States) are known to be built in different ways.

Target States are classically characterised by their having both an event and
(target) state argument \parencite{Kratzer2001a}:

\begin{exe}
\ex\label{tsk} λsλe [ cool(e) $\wedge$ event(e) $\wedge$ cooled(the soup)(s) $\wedge$ cause(s)(e) ]
\par `The soup is cooled' \hfill \parencite[391]{Kratzer2001a}
\end{exe}

Comparative investigation of the syntax of Target State \isi{participles} has shown
that this interpretation derives from a syntactic configuration where a
stativizer (labelled Asp) attaches to an eventive component (for example,
verbalising \emph{v}, or Root$_{event}$)~\citep{Alexiadou2008a, Embick2009a,
    Anagnostopoulou2013a, AlexiadouEtAl2015}.\footnote{(\ref{tsaas}) essentially
    derives the relevant aspect of the generalization in (\ref{16.gen}), that
    the past/passive morpheme attach directly to the lexical predicate: for
    Target States, this can be regarded as a reflex of the local attachment of
the aspectual and eventive components in the verbal structure.}


\begin{exe}
\ex\label{tsaas} Target States: \emph{v} attachment of Asp\\
\vspace{11pt}
\Tree  [  [  \ro{\dots}  {\dots} ].\emph{v}  Asp ].Asp
\end{exe}

Abstracting over (\ref{tsk}) and (\ref{tsaas}), Target States have a structure
defined by a local relation between an event and a
stativizer~\citep{Kratzer2001a, Alexiadou2008a, Embick2009a}:

\begin{exe}
\ex\label{es} {[ event, stative ]} $\rightarrow$ Target State interpretation
\end{exe}


At first blush, the \emph{done}-state seems to present an exception to
(\ref{es}), given that in the \emph{done}-state the stative \emph{(be) done}
clearly embeds the eventive VP\@. Evidence that \emph{done} has the stativizing
aspectual function is confirmed by the pair in (\ref{progdrs}), which show that
while the present/ active is aspectually unbounded or ongoing (\ref{progdrs}a),
the structure with \emph{done} has a result state (\ref{progdrs}b).

 \begin{exe}
\ex\label{progdrs}
\begin{xlist}
\ex  I'm writing Chapter 3.
\ex  I'm done writing Chapter 3.
\end{xlist}
\end{exe}



However, closer analysis of the \emph{done}-state structure indicates that the generalisation in (\ref{es}) can be retained.

I propose that the stativizer (-en) attaches to a semantically vacuous \emph{v}, and it is this local attachment that is responsible for deriving the Target State, in line with (\ref{tsaas} and (\ref{es}).

\begin{exe}
\ex\label{dsts}  {[ \emph{v}$_{vacuous}$-stative  [ event ] ]} $\rightarrow$ Target State interpretation in the \emph{done}-state
\end{exe}

This \emph{v} is realized as \emph{do}. As such, \emph{do} is a dummy item, a
form of \emph{do}-insertion that supports the aspectual morpheme. Dummy
\emph{do} can similarly appear in the participial form \emph{done} (rather than
\emph{do, did, does}, etc.) in the British varieties of \ili{English} that
allow \emph{do} to appear at the edge of a VP-\isi{ellipsis} site following a modal or auxiliary\is{auxiliaries} thanks to Dave Embick (p.c.) for this point.



\begin{exe}
\ex
\begin{xlist}
\ex He didn't eat it but he should have done.
\ex Have you looked up the scores yet? I haven't done, but will do.
\end{xlist}
\end{exe}



The intuition is then that because the eventive item that the stativizer
attaches to is semantically vacuous, the event that
\emph{v}\textsubscript{vacuous} describes is anaphoric with that described by
the embedded VP\@. This vacuity means that the \emph{done}-state only describes
one prior eventuality, and not two: (\ref{eventdo}) says that there was only a
\emph{cutting} event, for example.


\begin{exe}
\ex\label{eventdo} I'm done cutting his hair.
\end{exe}

For reasons of space I cannot address whether participial forms of \emph{do}
are eventive when they lack the VP complement (i.e., \emph{She's done}); for
observations that it may not (at least syntactically) see~\cite{Fruehwald2015a}
in connection with the dialectal form \emph{I'm done my homework}

Although on this account the Target State itself is created by the local
event-state relation, the non-local relation between the stativizer and the VP
event makes a prediction with respect to possible Target State interpretations.
It has often been observed that  local attachment in (\ref{es}) restricts sets
of possible interpretations in a way that non-local attachment does not (in the
context of \isi{participles}, see especially~\cite{Anagnostopoulou2013a}, and
references there). In particular, under local attachment of Asp to the eventive
component, root meaning interacts with Asp so that a Root that is not typically
a good property of states does not easily appear in the Target State structure
without significant context or coercion;~\cite{Kratzer2001a}
and~\cite{Embick2009a} give a range of examples of this of the type in
(\ref{tyre}).~\cite{Embick2004a} suggests the Target State reading of
\emph{kicked} can be coerced with a factory scenario where all of the tyres
have to be kicked before employees can leave; a similar factory scenario can
improve a Target State interpretation of hammered nails.


\begin{exe}
\ex\label{tyre}
\begin{xlist}
\ex ?The tyres are kicked.
\ex ?These nails are hammered.
\end{xlist}
\end{exe}

As the relation between the Target State component and the (lexical) event in
(\ref{dsts}) is non-local, Asp and eventive \emph{v} (or Root$_{event}$) should
not exhibit such restrictions, and \emph{done} should create a Target State
even with those verbs that do not easily form Target State interpretations via
direct attachment. This prediction is borne out. A Target State reading is
readily available with \emph{kick} and \emph{hammer} under \emph{done}
(\ref{nail}), even in out of the blue contexts.

\begin{exe}
\ex\label{nail}
\begin{xlist}
\ex I'm done kicking the tyres.
\ex I'm done hammering the nails.
\end{xlist}
\end{exe}

In sum, given the findings of the previous Section, I propose the structure of
the \emph{done}-state is as in (\ref{s1}).\newpage

\begin{exe}
\ex\label{s1} The \emph{done}-state structure: {\em I am done writing Chapter
3.}\\
\vspace{11pt}
    \Tree   [
                %\qroof{I}.NP$_{i}$
                [.NP$_i$ \edge[roof]; {I} ]
                [
                    be\textsubscript{[Pres]}
                    [
                        [ v -en ].Asp$_{1}$
                        [ {-ing}
                            [
                                %\qroof{PRO$_{i}$}.NP
                                [.NP \edge[roof]; {PRO$_i$} ]
                                [
                                    Voice\textsubscript{[Active]}
                                    %\qroof{write Chapter 3}.vP
                                    [.\emph{v}P \edge[roof]; {write Chapter 3} ]
                                ].Voice
                            ].VoiceP
                        ].AspP$_{2}$
                    ].AspP$_{1}$
                ].T
            ].TP
\end{exe}

The auxiliary\is{auxiliaries} \emph{be} is in T, and T takes a stativizing projection,
AspP$_{1}$ as its complement. This \enquote{top part} of the structure lacks an
argument introducing projection, such as Voice. This \enquote{top part} is, in
effect, a stative \isi{passive}.

A second aspectual projection is realised as the present/active morphological
form. In the lower component of (\ref{s1}), an active VoiceP has a transitive
syntax, introducing an argument in its specifier, and valuing Case on an
internal argument. It is the argument in the specifier of Voice that is the
agent of the embedded event; this argument is proposed to be PRO\@. The surface
subject is then interpreted as both the agent and state holder of the clause
via a \isi{control} relation.

The remainder of this chapter briefly discusses a possible alternative analysis
of the \emph{done}-state.

\section{Against a perfect syntax}\label{derivation}

An alternative analysis of the \emph{done}-state might draw a comparison with
the \ili{English} perfect.

Such \emph{be}-perfects are found in \ili{Bulgarian}, where a (Resultative)
perfect can be expressed with the perfective participle\is{participles}:

\begin{exe}
    \ex \ili{Bulgarian}
    \sn
    \gll  Ivan e postroil pjas\^{a}\v{c}na kula. \\
    Ivan be-\Tsg.\Prs{} build-\Prf.\M.\Sg{} sand castle \\
    \trans `Ivan has been building a sandcastle.' \hfill{\citep[296]{Pancheva2003a}}
\end{exe}

Perhaps, then, the \emph{done}-state has a syntactic structure like
(\ref{perfs1}), with \emph{done} a marker of perfectivity.

\begin{exe}
\ex A present perfect: {\em I am done writing Chapter 3.} (To be
rejected)\label{perfs1}\\
\vspace{12pt}
    \Tree
        [
            %\qroof{I}.NP
            [.NP \edge[roof]; {I} ]
            [
                be$_{[Pres]}$
                [
                    done
                    [
                        -ing
                        [
                            %\qroof{$<$I$>$}.NP
                            [.NP \edge[roof]; {$<$I$>$} ]
                            [
                                Voice$_{[Active]}$
                                %\qroof{write Chapter 3}.$v$P
                                [.\emph{v}P \edge[roof]; {write Chapter 3} ]
                            ].Voice
                        ].VoiceP
                    ].Asp
                ].Perf
            ].T
        ].TP
\end{exe}

Syntactically, though, the \emph{done}-state has different syntactic properties
to the \ili{English} \emph{have}-perfect. Building on tests discussed
in~\cite{Fruehwald2015a}, the \emph{done}-state is ok with {\em all}
modification (\ref{allmod}a), just like other stative passives (\ref{allmod}b).
The perfect is ungrammatical with {\em all} modification.

\begin{exe}
\ex\label{allmod}
\begin{xlist}
\ex I'm all done washing the dishes.
\ex I'm all ready.
\ex *I've all washed the dishes.
\end{xlist}
\end{exe}

Second, the \emph{done}-state can appear in a reduced relative clause, while the perfect of a transitive cannot.


\begin{exe}
\ex
\begin{xlist}
\ex Would all the students done signing the petition please leave?
\ex *Would all the students signed the petition please leave?
\ex Would all the students who have signed the petition please leave?
\end{xlist}
\end{exe}

I do not pursue a perfect analysis further; see~\cite{Fruehwald2015a} for a
similar conclusion for other forms of \emph{be done} based on study of the
dialectal \emph{done my homework} construction.\footnote{The \emph{I'm Done My
    Homework} construction (\glsunset{DMH}\gls{DMH}), found across
    Philadelphia, Canada, and Scotland, can also be syntactically and
    semantically distinguished from \emph{done}-state
    structures.~\cite{Fruehwald2015a} show at length that the state described
    by \gls{DMH} does not come about as a result of a semantically or
    syntactically identifiable prior event~\citep[154--7]{Fruehwald2015a}
    (thanks too to Meredith Tamminga and David Wilson for discussion). As
    such,~\cite{Fruehwald2015a} analyse the \gls{DMH} structure as a complex aP
    \emph{done} (which does not a have a VP component), an aP that Case
    licenses an NP complement in a \enquote{transitive adjectival passive}
    configuration.

Despite the syntactic and semantic differences between \gls{DMH} and
\emph{done}-VP-ing,~\cite{Fruehwald2015a} make the intriguing observation that
the availability of DMH across varieties of \ili{English} correlates with also having
the form \emph{X-en}-VP-ing. Some (Montreal) speakers, for example, have DMH
with \emph{start} (\emph{I'm started NP}), and this seems to correlate with
also having \emph{I'm started VP-ing}, ungrammatical in most varieties of UK
and US \ili{English}. I leave examination of possible structural parallels between
the two constructions to future work.}

\section{Summary}\label{summary}

An extensive body of work has shown that a Target State interpretation derives
from a structure in which an eventive and stative component are in a local
syntactic relationship. This paper investigated an apparent counter-example to
this analysis. It showed that statives of the form \emph{I'm done VP-ing}
(\emph{She's done writing Chapter 3}) have a Target State interpretation.
However, in this structure the stativizing past/passive morpheme attaches to
\emph{do}, so that it is in a non-local configuration with the event described
by an embedded verb phrase, the event from which the Target State is
interpreted as resulting.

I argued that the Target State interpretation of the \emph{done}-state is
nonetheless derived via a local relation between a state and eventive
component, as in previous work. However, in the \emph{done}-state, the eventive
component that the stativizer attaches to is semantically vacuous, so that the
prior event from which the Target State follows is understood to be that of the
embedded VP\@. The non-local relation between the stativizer and eventive VP
component permitted regular derivation of Target State interpretations of
events out of which Target States are not typically possible. Further research
is needed to address the general challenge of determining how the Target State
of the event is accessed by the stativizer, whether in a local or non-local
configuration.



\printchapterglossary{}

\section*{Acknowledgements}

Thanks to Ian Roberts for many conversations about \isi{passive} structures. The work
discussed in this chapter is an offshoot of a collaborative project on the
\emph{I'm done my homework} construction with Meredith Tamminga; particular
thanks are due to her and to Dave Embick for very helpful discussion. Thanks to
reviewers whose suggestions greatly improved exposition. Any errors are mine.

{\sloppy
\printbibliography[heading=subbibliography,notkeyword=this]
}
%\end{refcontext}
\end{document}

