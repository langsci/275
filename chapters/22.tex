\documentclass[output=paper]{langsci/langscibook}
\ChapterDOI{10.5281/zenodo.3972874}

\author{Marc Richards\affiliation{Queen's University Belfast}}
\title{Rethinking principles A and B from a Free Merge perspective}

% \chapterDOI{} %will be filled in at production

\abstract{This squib sketches out the beginnings of a bottom-up, minimalist
    rethinking of pronominal reference constraints (essentially, principles A
    and B of the binding theory\is{binding}) in terms of an approach to grammar-internal
    optionality originally pursued in \citealt{BibRob2005},
    \citealt{BibRich2006}. By combining a movement theory of binding
    (\citealt{Hornstein2001,Hornstein2013,Kayne2002,Abe2014})
    with phase\is{phases} theory (\citealt{Chomsky2000} et seq.), the essential difference
    between local \isi{binding} and local obviation reduces to the choice between
    Internal \isi{Merge} and External \isi{Merge} at the phase\is{phases} level, each yielding a
    distinct interpretive outcome at the \gls{CI} interface. Further, if the phase
    constitutes the maximal domain in which linguistic constraints can apply,
    then interpretive freedom is expected beyond the phase\is{phases} level. In this way,
    restrictions on the interpretation of pronouns turn out to be the \gls{CI}
equivalent of ordering restrictions at the sensorimotor interface (\glsunset{PF}\gls{PF}), which
likewise obtain up to the phase\is{phases} level but not beyond
\parencite{Richards2004,Richards2007b}.}

\maketitle

\begin{document}\glsresetall

\section{The price of freedom}\label{sec:22.1}

In its more recent developments, the Minimalist program has moved away from its
earlier emphasis on the formal features that trigger operations and the formal
constraints that restrict them. Accordingly, from the perspective of the
\gls{SMT},\is{strong Minimalist thesis} in which language-specific technology is expensive (i.e.\ adds to
the \enquote{first factor}; \citealt{Chomsky2005}), optionality should no
longer surprise us. The free application of operations is the default
expectation.\footnote{Cf.\ \citet[10--11]{Chomsky2015} on \blockquote{the
        lingering idea, carried over from earlier work, that each operation has
        to be motivated by satisfying some demand.  But there is no reason to
        retain this condition.  Operations can be free, with the outcome
evaluated at the phase\is{phases} level for transfer and interpretation at the
interfaces}.\label{fn:22.1}} Whereas earlier Minimalism \citep{Chomsky1995}
viewed optionality as problematic, with optional rules and operations
effectively excluded by a conspiracy of last resort and full interpretation, it
is in fact \enquote{obligatoriness} that is unexpected, as any limitation on
this freedom has to somehow be legislated for in the form of a
language-specific rule or constraint, thus departing from the \gls{SMT}\is{strong Minimalist thesis} (unless
this restriction can be reduced to more general, \enquote{third-factor}
considerations). By contrast, there is no need to legislate for optionality. A
maximally empty, minimally specified \glsunset{UG}\gls{UG} will necessarily
leave many options open, giving rise to operational indeterminacies, as
explored and exploited in \enquote{underspecification} models of (parametric)
variation (see, e.g.,
\citealt{Uriagereka1994,BibRich2006,BerCho2011,Richards2008b,Kandybowicz2009,Boeckx2011b,RobHol2010});
it also leads naturally to an \enquote{overgenerate and filter} view of the
syntax--interface relation (see, e.g., \citealt{Richards2004,Richards2007b} on the
syntax--\gls{PF} relation), perhaps based on \emph{Free Merge} (cf.
\citealt{Chomsky2007,Chomsky2008,Chomsky2013,Chomsky2015} --
see~\cref{fn:22.1}; also \citealt{Boeckx2011b}). Operative freedom itself now
comes for free; it is the restrictions on this freedom (rules, constraints: the
mechanisms of obligatoriness) that come at a price, carrying the burden of
explanation.

In this light, we need to reconsider how (and where) apparent strictures (or
their effect) might arise in this system. A simple way to curb the excesses of
a free syntax is to make it responsible to the interfaces, so that the choices
we make (in the syntax) have consequences (at the interface). From this
perspective, sometimes called \emph{interface economy} (cf.\
\citealt{Reinhart1995,Fox2000,Chomsky2001,BibRich2006}), the choice of applying
a syntactic operation like \isi{Merge} may itself be free, but this choice must be
cashed out at the interface in the form of an interpretive effect – an
\glsunset{EOO}\emph{\glsdesc{EOO}} (\gls{EOO}; \citealt[34]{Chomsky2001}).
Optional operations thus have an obligatory \gls{EOO}. Equally, where a
derivational option is independently excluded,\footnote{For example, the phase
    impenetrability condition might exclude the option of Internal \isi{Merge},
    where this would cross a phase boundary. See~\Cref{sub:22.2.2}
below.\label{fn:22.2}} we might expect the opposite pattern to obtain. These
two scenarios were summarized in \citet{BibRich2006} as in \REF{ex:22.1}.

\ea\label{ex:22.1}
    \ea\label{ex:22.1a} Optional operations feed obligatory interpretations;
    \ex\label{ex:22.1b} Obligatory operations feed optional interpretations.
    \z
\z

The refinement I would like to propose and pursue here is that an \gls{EOO}
will only be discernible up to a certain point in the derivation, namely the
phase level.  In terms of \citet{BibRich2006}, this means that the phase\is{phases} is the
level at which the system \enquote{minds} (i.e.\ the level at which the
derivational choices within a phase\is{phases} are made to count). Beyond the phase\is{phases} level,
the system stops caring,\footnote{This follows from the idea that phases are
    the units of computation, and that there is no memory of derivational
information beyond the phase\is{phases} level (cf. \citealt[8]{Chomsky2015}: \enquote{The
basic principle is that memory is phase-level -- as, e.g., in distinguishing
copies from repetitions}).\label{fn:22.3}} and interpretive freedom will
therefore result (i.e.\ a lack of \gls{EOO}, equivalent to~\ref{ex:22.1b}).
Let us refer to this as Claim 1, as in \REF{ex:22.2}.

\ea\label{ex:22.2}Claim 1\\
    The phase\is{phases} is the maximal domain in which syntactic\slash interpretive constraints
    can apply. Each choice within a phase\is{phases} registers a distinct \gls{EOO} at the
    interface.
\z

Effectively, the \gls{EOO} rationale in \REF{ex:22.1a}, in combination with
phases, will conspire to \emph{give the illusion} of local (syntactic)
constraints. In terms of (free) \isi{Merge}, the choice between applying Internal or
External \isi{Merge} at a given point in the derivation -- yielding copies versus
repetitions, respectively, at the interface -- can only make a difference
within a phase\is{phases}. The relevance of the copy/repetition distinction at the
interface is therefore predicted to break down beyond the phase\is{phases} level, as
\REF{ex:22.3} ostensibly confirms.

\ea\label{ex:22.3}
    \emph{He}\tss{i} thinks [\tss{CP} that \emph{he}\tss{i/j} can help Mary ]
\z

Here, due to the intervening CP phase\is{phases} boundary, the higher instance of
\emph{he} may be interpreted as either a copy of the lower \emph{he} (hence
referentially identical), or else as an independent repetition (hence with
independent reference). By contrast, where this choice is made within a phase\is{phases},
\glspl{EOO} are predicted to arise, as summarized in Claim 2.

\ea\label{ex:22.4}Claim 2\\
    \isi{Merge} \emph{within} a phase\is{phases} will be constrained (e.g.\ subject to
    particular interpretive restrictions) in a way that \isi{Merge} \emph{across}
    phases is not.
\z

At \gls{PF}, this yields order-preservation constraints on phase-internal \isi{movement}
\parencite{Richards2004,Richards2007b}, as I shall briefly review
in~\Cref{sub:22.2.1}. This then leads to my main claim,
in~\Cref{sub:22.2.2} --- namely, that \isi{binding} conditions (principles A and
B) can be rethought of, and made sense of, as the \gls{CI} equivalent of order
preservation at \gls{PF}.

\section{Escape to freedom}\label{sec:22.2}

The \enquote{obligatoriness} of local \isi{binding} and obviation constraints, as
captured by principles A and B of the binding theory\is{binding}, is unexpected from the
minimalist perspective set out in the previous section. A Roberts-style
\enquote{rethink} of this pervasive property of human language is therefore in order,
with the aim of reconciling it with the \gls{SMT}\is{strong Minimalist thesis}. If we can rationalize and
naturalize the \isi{binding} principles in terms of \REF{ex:22.2} and
\REF{ex:22.4}, i.e.\ as emergent \glspl{EOO}, we will have gone some way towards
achieving this aim. To see how this might work, it is worthwhile revisiting the
analysis of Holmberg’s generalization from \citealt{Richards2004}, in which
\REF{ex:22.2} and \REF{ex:22.4} conspire to constrain the interpretive output
of \isi{Merge} at the \gls{PF} interface.

\subsection{Phase-internal interpretive restrictions on Free Merge at PF: Order
preservation}\label{sub:22.2.1}

There is evidence to believe that local movements such as object shift are
subject to certain ordering restrictions that do not hold of longer-distance or
successive-cyclic \isi{movement}. For VO languages, this restriction is famously
captured under \glsunset{HG}\glsdesc{HG} (\gls{HG};
\citealt{Holmberg1986,Holmberg1999}); essentially, \enquote{VO in} implies \enquote{VO out},
thus excluding object shift in cases where the verb does not move to a position
above the object, as in \REF{ex:22.5b}.

\ea\label{ex:22.5} \ili{Icelandic}
    \ea\label{ex:22.5a}
        \gll    Nemandinn  las   [\emph{\tss{v}}\tss{*P} (\textit{bókina}) \emph{t}\tss{nemandinn} ekki [\tss{VP} \emph{t}\tss{las} (\textit{bókina})~]] \\
                the.student read {} \hphantom{(}the.book {} not {} {} \hphantom{(}the.book\\
        \glt \enquote{The student didn't read the book.}
    \ex\label{ex:22.5b}
        \gll    Nemandinn hefur [\emph{\tss{v}}\tss{*P} (\textit{*bókina}) \emph{t}\tss{nemandinn} ekki [\tss{VP} lesið (\textit{bókina})~]]\\
                the.student has {} \hphantom{(*}the.book {} not {} read \hphantom{(}the.book\\
        \glt \enquote{The student hasn't read the book.}
    \z
\z

Taking short-distance \isi{movement} of the object shift kind to be \emph{v}P- (and
thus phase-) internal, the relevant generalization seems to be that ordering
freedom arises only once the \emph{v}P phase\is{phases} is escaped. Thus longer-distance
(cross-phasal) \isi{movement} out of the \emph{v}P phase\is{phases} is free to invert the
original order, as in the case of A-movement/passivization, \emph{wh}-movement,
\isi{topicalization}, etc.\largerpage

\ea\label{ex:22.6}
\ea  A man [\emph{\tss{v}}\tss{P} arrived (\sout{\emph{a man}}) ]
\ex John was [\emph{\tss{v}}\tss{P} rescued (\sout{\emph{John}}) ]
\ex John, I [\emph{\tss{v*}}\tss{P} like (\sout{\emph{John}}) ]
    \ex Which book did you [\emph{\tss{v*}}\tss{P} read (\sout{\emph{which
    book}}) ]
    \z
\z

The constraint on short-distance \isi{movement} such that the derived order must
reinstate the base order is an unexpected limitation on Free \isi{Merge}; it is
another unexpected instance of \enquote{obligatoriness}. The phase-internal nature of
this constraint, combined with the assumption that linear order is imposed only
at the sensorimotor interface and is not a property of the syntactic structure
itself,\footnote{This long-standing insight is first elaborated in
    \textcite[334--340]{Chomsky1995}; more recently, it finds expression in the
    claim that \enquote{[o]rder is relegated to externalization}
\citep[4]{Chomsky2015}.} suggests an approach to \gls{HG}\is{Holmberg's generalization} in terms of \emph{cyclic}
\emph{linearization} (i.e.\ linearization by phase). Such a system is notably
proposed in \citet{FoxPes2005}, with the interesting property that
ordering freedom is allowed within a phase\is{phases} but not beyond, contra the claims in
\REF{ex:22.2} and \REF{ex:22.4} above. An alternative is offered in
\textcite{Richards2004,Richards2007b}, in which the same effects are delivered
by the opposite set of assumptions -- i.e., ordering freedom is allowed beyond
the phase\is{phases} but not within, in conformance with \REF{ex:22.2} and
\REF{ex:22.4}. This alternative follows from a Merge-based linearization
algorithm in which (symmetrical) \isi{Merge} overspecifies the word order
between \isi{Merge} pairs (sisters), giving \gls{PF} both options each time
(head-first, head-final); cf.\ \citealt{EpsGroKawKit1998}.  Then, at the
phase\is{phases} level, the interface simply discards one of these options,
consistently. Such an \enquote{overgenerate-and-filter} approach to
linearization may be expressed as in \REF{ex:22.7}.

\ea\label{ex:22.7}Parametrized desymmetrization\\
    Given Merge(α,β) → *\{\tuple{α,β}, \tuple{β,α}\}:
    \ea\label{ex:22.7a} Head-initial = delete all \emph{Comp $<$ Head}\\
    {}[i.e.\ \{\tuple{α,β}, \tuple{β,α}\} → \{\tuple{α,β}, \sout{\tuple{β,α}}\}]
    \ex\label{ex:22.7b} Head-final = delete all \emph{Head $<$ Comp}\\
    {}[i.e.\ \{\tuple{α,β}, \tuple{β,α}\} → \{\sout{\tuple{α,β}}, \tuple{β,α}\}]
    \z
\z

The contrast between \REF{ex:22.5} and \REF{ex:22.6} is a
straightforward consequence of this system. As depicted in \REF{ex:22.8},
short object displacement to spec-\emph{v}P across V is only orderable by
\REF{ex:22.7a} where further \isi{movement} of V across the displaced object
takes place, so that the latter becomes the tail of a V $<$ O chain, rather
than the head of an O $<$ V chain. (Any such O $<$ V instruction would be
deleted and thus \enquote{undone} at \gls{PF}, by
\ref{ex:22.7a}.)\largerpage[1.75]

\ea\label{ex:22.8}Object shift (phase-internal)\\
    \begin{tikzpicture}[baseline=(root.base)]
        \Tree 	[.\node(root){\emph{v}P};
                    \node (o) {O};
                    [.\node(VP){\makebox[0pt][r]{\dots{}}VP};
                        V
                        \node (o-t) {O};
                        ]
                    ]
                ]

        \draw [->, shorten <=.5mm, shorten >=.5mm] (o-t.base) -- +(0,-.25) -| (o.base);

        \node (text) [right=2cm of VP]
            {via External \isi{Merge}: \{V $<$ O, \sout{O $<$ V}\}};

        \node [above=1\baselineskip of text.west, anchor=west]
            {\emph{Precedence instructions}};

        \node at (o-t -| text.west) [align=left, anchor=west]
            {via Internal \isi{Merge}: \{\sout{O $<$ V}\}};

        \draw [->, shorten <=.5cm, shorten >=.5cm] (VP.east) to (text.west);

    \end{tikzpicture}
\z

The upshot is that \gls{HG}\is{Holmberg's generalization} is derived for
exactly that subset of languages in which it holds (i.e.\ those set to
\REF{ex:22.7a}: VO languages). Beyond the \emph{v}P phase\is{phases} level, however,
the information about the original ordering sister is lost, due to phase-level
memory (cf.~\cref{fn:22.3}), and the displaced DP is effectively relinearized
in the higher phase\is{phases} (hence the possibility of inverted orders, as in
\ref{ex:22.6}). Interpretive freedom at \gls{PF} is thus the result of
escaping the phase; the expected optionality re-emerges beyond the phase\is{phases} level.

\subsection{Phase-internal interpretive restrictions on Free Merge at \textsc{sem}:
Binding principles}\label{sub:22.2.2}

An obvious question is what the equivalent of \gls{PF} order preservation would
be at the CI-interface. Is there a similar basic pattern to the one in
(\ref{ex:22.5}--\ref{ex:22.6}) in which \isi{Merge} choices made locally (within the
\emph{v}P phase) are interpretively constrained at the interface, with
interpretive freedom again re-emerging once the phase\is{phases} is escaped? My
contention here is that principles\is{binding} A and B of the binding theory\is{binding}
instantiate just this pattern, and thus again implicate a minimalist system
based on \REF{ex:22.2} and \REF{ex:22.4}.

Clearly, in order to reconstruct the principles of \isi{binding} in terms of \isi{Merge}
choices, some version of a \gls{MTB} must be assumed
\parencite{Hornstein2001,Hornstein2009,Hornstein2013,Kayne2002,Abe2014}, with
anaphors and/or pronouns analysed as pronounced lower copies (cf.\ also
\citealt{Heinat2003}).  The present article is not the place to provide a full
justification of the \gls{MTB} or to pursue the technicalities of lower-copy
realization (see above references and related work); suffice it to say that I
take the \gls{MTB} to be the null hypothesis in a system of unconstrained
(\enquote{free}) \isi{Merge}, in which Internal \isi{Merge} to θ-positions cannot
(and should not) be excluded in the syntax, and in which Internal \isi{Merge}
provides the simplest possible mechanism by which to derive referentially
identified occurrences (tokens), in the form of copies. However, in a crucial
departure from earlier versions of \citeauthor{Hornstein2001}’s
\gls{MTB},\footnote{More recent versions, such as \textcite{Hornstein2013},
    come a lot closer to the present proposal.} it cannot be the case that
    anaphors and pronouns (principles\is{binding} A and B) stand in an \enquote{elsewhere}
    relation\is{elsewhere condition}, such that pronouns result wherever
    \isi{movement} is not possible.  Rather, the present system relies on there being
    a critical choice point (within the phase) where both options (Move and
    Merge) are equally available, with each choice then yielding a
    complementary outcome at the interface.

We restrict ourselves here to considering just the core facts of
principles\is{binding} A and B. Our aim is to simply derive the complementary
distribution of anaphors and pronouns within a given local domain, and thus the
fundamental difference between obligatory \isi{binding} and obligatory
obviation. These core facts are given in \REF{ex:22.9}.

\ea\label{ex:22.9}
    \ea\label{ex:22.9a} \makebox[0pt][l]{He\tss{i} likes himself\tss{i/*j}}%
        \phantom{He thinks that Mary likes him\qquad}%
        (principle A\is{binding}, local; obligatory \hphantom{He thinks that Mary likes
        him\qquad} coreference)
    \ex\label{ex:22.9b} \makebox[0pt][l]{He\tss{i} likes him\tss{*i/j}}%
        \phantom{He thinks that Mary likes him\qquad}%
        (principle B\is{binding}, local; obligatory \hphantom{He thinks that Mary likes
        him\qquad} obviation/disjoint reference)
    \ex\label{ex:22.9c} \makebox[0pt][l]{He\tss{i} thinks that Mary likes him\tss{i/j}}%
        \phantom{He thinks that Mary likes him\qquad}%
        (non-local; referential freedom)
    \ex\label{ex:22.9d} \makebox[0pt][l]{His\tss{i} mother likes him\tss{i/j}}%
        \phantom{He thinks that Mary likes him\qquad}%
        (no c-command; referential \hphantom{He thinks that Mary likes
        him\qquad} freedom)
    \z
\z

To derive the contrast between \REF{ex:22.9a} and \REF{ex:22.9b},
consider first the derivation at the point where the \gls{EA} is merged, after
\emph{v}* has been merged with its complement VP. At this point, there is a
free choice between \gls{IM}\is{Merge} or \gls{EM}: either option is in principle
possible here (and in practice too, as long as the VP and its contents have not
yet been transferred). Since this choice is made phase-internally,\is{phases}
the information as to which choice is made is available at the interface, upon
Transfer. Each option is therefore exploited at the interface in the form of a
different \gls{EOO} (cf.\ \ref{ex:22.2}).

According to the first option, the \gls{IA} may be raised to spec-\emph{v}P to
form the \gls{EA}, as in \REF{ex:22.10}.\footnote{The lower copy here is
    spelled out overtly, as an anaphor, and not deleted or left unpronounced,
    as it is in the case of passive/unaccusative \gls{IM}\is{Merge} of the \gls{IA}.  The
    salient difference between the two cases that accounts for this divergence
    is the nature of the \emph{v} head. The defective \emph{v} associated with
    passives/unaccusatives is unable to value Case on the \gls{IA} (cf.
    \citealt{Chomsky2001}). The \gls{IA} thus remains active, raising
    automatically to the phase\is{phases} edge to evade Transfer (cf.
    \citealt{Chomsky2000}). Since the lower (active) copy is not transferred,
    it cannot be realized at \gls{PF} (i.e.  pronounced). By contrast,
    \REF{ex:22.10} involves a transitive \emph{v}*, which values Case
    (accusative) on the \gls{IA}. Thus deactivated, the lower copy of the
    \gls{IA} is a candidate for Transfer and thus for PF-realization.}

\ea\label{ex:22.10}Option 1: Internal Merge of the \gls{IA} to form the \gls{EA}\\
    {}[\emph{\tss{v}}\tss{*P} \tn{he}{he} \emph{v}* [\tss{VP} likes \tn{him}{him} (→ himself) ]]
    \begin{tikzpicture}[remember picture, overlay]
        \draw [->, shorten <=.5mm, shorten >=.5mm] (him) -- +(0,-.5) -| (he);
    \end{tikzpicture}\vspace{1\baselineskip}
\z

Since \gls{IM}\is{Merge} is chosen and \gls{IM}\is{Merge} here is optional (given the availability
of another option, viz.\ \gls{EM}\is{Merge}), this choice must have an effect at the
CI-interface (cf.\ \ref{ex:22.2}). The two occurrences of the relevant lexical
item are detectable as copies at the phase\is{phases} level; therefore, the result
(\gls{EOO}) is obligatory referential identity at \gls{CI} (i.e.\ \emph{he} =
\emph{himself}, or a covariant/bound-variable reading with a quantificational
antecedent, as in \emph{Every boy likes himself}), in line with
\REF{ex:22.1a}.

Alternatively, the other option is for \gls{EM}\is{Merge} to apply at this stage, as in
\REF{ex:22.11}.

\ea\label{ex:22.11}Option 2: External Merge of \emph{he} to
    form the \gls{EA}\\
    {}[\emph{\tss{v}}\tss{*P} \tn{he}{he}  \emph{v}* [\tss{VP} likes \tn{him}{him} ]]
    \begin{tikzpicture}[remember picture, overlay]
        \node [below=.4cm of he] (em2) {EM\tss{2}};
        \node [below=.4cm of him] (em1) {EM\tss{1}};
        \draw [->, shorten >=.4mm] (em2) to (he);
        \draw [->, shorten >=.4mm] (em1) to (him);
    \end{tikzpicture}\vspace{1.5\baselineskip}
\z

Since \gls{EM}\is{Merge} is chosen and \gls{EM}\is{Merge} here is optional (given the availability
of another option, viz.\ \gls{IM}\is{Merge}), this choice must likewise have a distinct
effect at the CI-interface. The two occurrences of the relevant lexical item
are detectable as independent repetitions at the phase\is{phases} level; therefore, the
result (\glsunset{EOO}\gls{EOO}) is obligatory disjoint reference at \textsc{sem} (i.e.\ \emph{he} ${\neq}$
\emph{him}, or the absence of a bound-variable reading with a quantificational
antecedent, as in \emph{Every boy likes him}), again in line with
\REF{ex:22.1a}.

Turning finally to \REF{ex:22.9c} and \REF{ex:22.9d}, here the two indexed
positions cannot be derivationally related by \gls{IM}\is{Merge}. In the case of
\REF{ex:22.9c}, this is due to the presence of at least one intervening phase
boundary (the CP headed by \enquote*{that}). The embedded \gls{IA} is therefore rendered
inaccessible to the matrix subject position, in accordance with the phase
impenetrability condition. In \REF{ex:22.9d}, an interarboreal or sideward
dependency would be required to link the two positions. It is arguable that
such dependencies do not conform to the simplest conception of \isi{Merge} (cf.
\citealt{Chomsky2007}): in this case, \emph{him} is not contained in the sister
of \emph{his}, and thus \emph{his} cannot be the result of \gls{IM}\is{Merge} of
\emph{him}. In both cases, therefore, only \gls{EM}\is{Merge} is possible.\footnote{The
    same is true for those cases where the lower pronoun (bound or otherwise)
    is contained within an island,\is{islands} such as \emph{Every actor\tss{i}
    denied the rumour that the studio fired him}\tss{i/j}.} Since
    \gls{EM}\is{Merge} is now obligatory (there being no option of \gls{IM}\is{Merge}, unlike in
    (\ref{ex:22.10}--\ref{ex:22.11}) above), it will be associated with interpretive
    freedom, in line with \REF{ex:22.1b}.  Consequently, incidental
    coreference/covariance becomes a possible interpretation. As with the
    trans-phasal dependencies in \REF{ex:22.6}, crossing a phase\is{phases} results in
    liberation at the interface. This opening up of interpretive possibilities
    has the interesting consequence that there are two derivational sources for
    the same interpretation. Thus, for example, a bound variable may be derived
    either via the phase-internal, obligatory route (cf.\ \ref{ex:22.9a}), or
    via the cross-phasal, optional route (as in \ref{ex:22.9}c,d). I leave further
    exploration of this consequence for future
    research.\footnote{\citet{Hornstein2013} independently argues for a
        non-uniform approach to bound variables (i.e.\ those which are the
        product of \isi{movement} and those which are not), on compelling empirical
        grounds. The approach proposed here thus lends further support to
        Hornstein’s hunch. Note, too, that any c-command requirement on bound
        variables will only characterize the first kind (the local, IM-derived
        kind). Thus bound variables are readily available in
        \REF{ex:22.9d}-type configurations, as in \emph{Everyone\tss{i}’s
        mother likes him}\tss{i/j}, where (importantly) the
        non-coreferential/non-covariant interpretation of \emph{him} is also an
        option. The same goes for non-local variable \isi{binding}, as in \emph{Every
        criminal\tss{i} thinks the police are after him}\tss{i/j},
        instantiating \REF{ex:22.9c}, where again the bound reading is only
        optional.  As discussed in~\Cref{sec:22.1}, there is no need for
    the grammar to legislate for optionality, as this is the default state of
affairs from the minimalist perspective; only non-optional, forced readings are
unexpected and demand an explanation.}

\section{Conclusion}

In the same spirit as \textcite{Hornstein2009,Hornstein2013}, we have tried to
shed light on the question of why restrictions such as the binding
principles\is{binding} should exist at all (i.e.\ why they should be a
characteristic property of human language). The answer we have begun to develop
here offers a potential first step in \enquote{rethinking} the binding theory\is{binding}
from the ground up. It is based on the idea that whilst \isi{Merge} itself
might be free, its interpretation is not (up to the phase\is{phases} level),
due to the \gls{EOO} rationale in (\ref{ex:22.1}/\ref{ex:22.2}).  The \gls{MTB} in
conjunction with phases then delivers the \emph{effect} of interpretive
constraints (principles\is{binding} A and B).\footnote{Similarly, the phase
    delivers the \emph{effect} of the \gls{GB} binding domain, since it is at
    the phase\is{phases} level that these choices apply and are made to count.
    Clearly, this is not the same as claiming the phase\is{phases} to actually
    \emph{be} the \isi{binding} domain (redux) in any primitive sense, in which
    pronouns must be free and anaphors bound; see e.g.\
    \citealt{UriGal2006,Hicks2009,Sabel2012} for other ways to conceive the
relation between \isi{binding} domains and phases.} Binding conditions reduce
to the differential interpretation of free \isi{Merge} choices within a
phase\is{phases} (i.e.\ the maximum domain in which the system can
\enquote{care}): the choice between \gls{IM}\is{Merge} and \gls{EM}\is{Merge}
is cashed out at the interface in a complementary manner, yielding obligatorily
coreferent copies (local binding) versus obligatorily disjoint repetitions
(local obviation), respectively. By contrast, interpretive freedom (including
optional coreference) arises with cross-phasal\is{phases} dependencies, as default
optionality re-emerges beyond the phase level.

Finally, it should be noted that the sketch presented above leaves many
questions open and avenues unexplored. I am grateful to two anonymous reviewers
and an editor for highlighting some of these.  Amongst the most immediate
empirical challenges facing this approach are long-distance reflexives and
other cross-clausal referential dependencies, such as those holding between a
null embedded subject and a matrix overt subject in null-subject languages in
structures like \REF{ex:22.3}; non-local SE anaphors (contrasting with
local SELF anaphors) are another relevant point of variation here (cf.\
\citealt{ReiReu1993,Lidz2001}). Such cases present a problem for the model
proposed here, as they all involve obligatoriness effects that appear to hold
beyond the phase\is{phases} level, i.e.\ where optionality would be predicted (cf.
\ref{ex:22.9c}). An approach in terms of cancellation or extension of the
intermediate phases suggests itself for such cases of non-local \isi{binding} (see
\citealt{Livitz2016} for such an analysis of \ili{Russian} embedded \isi{null subjects}),
or else the relevant variation might be attributed to the nature of Transfer
itself (cf. the distinction between weak and strong Transfer implied in
\citealt{Chomsky2008}). A reviewer also asks about non-complementary
distribution, i.e.\ configurations in which both the pronoun and the anaphor
freely alternate and are equally acceptable (or indeed, equally unacceptable,
as in the cases of overlapping reference discussed in \citealt{ReiReu1993}). It
is important to note in this connection that the present approach takes only
obligatoriness, not optionality, to demand an explanation under the
\gls{SMT}\is{strong Minimalist thesis}
and a minimally specified \gls{UG}\is{Universal Grammar}
(cf.~\Cref{sec:22.1}; indeed, its main conceptual advantage is that it only
seeks to explain what needs to be explained, reducing the core \isi{binding} facts to
principled variation and leaving the rest open to free variation). More
specifically, interpretively constrained pronominal/anaphoric forms are
predicted to arise only where two \isi{Merge} options (internal and external)
compete at the phase\is{phases} level. Where either Internal or External Merge is
unavailable (cf.~\cref{fn:22.2}), interpretive optionality and thus
non-complementarity should re-emerge, at either or both interfaces. For \textsc{sem}, an
example of such non-complementarity has already been discussed (the freely
interpreted embedded pronoun in \REF{ex:22.9c}); the \textsc{phon} equivalent (i.e.\
multiple realizational options) is no less expected, and may be manifested in
the form of pronoun/anaphor interchangeability, as found in certain DP and PP
configurations. These tentative suggestions indicate at least some of the
empirical and theoretical directions in which the current approach might be
immediately extended.

\printchapterglossary{}

{\sloppy
\printbibliography[heading=subbibliography,notkeyword=this]
}

\end{document}
