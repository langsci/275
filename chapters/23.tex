\documentclass[output=paper]{langsci/langscibook}
\ChapterDOI{10.5281/zenodo.3972876}

\author{Cherry Chit-Yu Lam\affiliation{The Open University of Hong Kong}}
\title{Beyond one, two, three: Number matters in classifier languages}

% \chapterDOI{} %will be filled in at production

\abstract{Chinese has been widely recognised as a classic example of a
    numeral-licensing classifier language, where the presence of a classifier
    is obligatory for overt quantification of nouns. This paper presents new
    data from Mandarin\il{Mandarin} and \ili{Hong Kong Cantonese} (HKC) to show that the need of
    \isi{classifiers} for quantification is not always that absolute. Systematic
    variation has been found with an extended range of numerals examined
    (numerals larger than three), and a wider coverage of nouns in terms of
    animacy. The findings present a consistent pattern that HKC
    has a stricter requirement for classifiers in enumeration as bare common
    nouns are not definite in HKC, and it lacks the alternative
    strategies found in Mandarin.}

\maketitle

\rohead{\thechapter\hspace{0.5em}Beyond one, two, three}

\begin{document}\glsresetall

\section{Introduction}

Chinese, particularly Mandarin\il{Mandarin}, has been an exemplar
language with numeral-licensing \isi{classifiers}. This paper presents new data from
mainland Mandarin and \gls{HKC}\il{Hong Kong Cantonese} which contradicts such
a neat understanding.

It is generally understood that, in Mandarin\il{Mandarin} and \gls{HKC}\il{Hong
Kong Cantonese}, whenever overt quantification is expressed in a noun phrase,
whether by quantifiers like \emph{jǐ} (\gls{HKC} \emph{gei2}) ‘some’, or
\isi{numerals} like \emph{sān} (\gls{HKC} \emph{saam1}) ‘three’, a
classifier\is{classifiers} must be present, regardless of mass-count
distinction (\ref{ex:23.1}--\ref{ex:23.2}).\largerpage[2]

\ea\label{ex:23.1}Mandarin \parencites[92]{Chierchia1998}[519]{ChengSybesma1999} %1
	\relax{\multicolsep=.125\baselineskip%
	\begin{multicols}{2}
	\ea
    \gll    liǎng *(zhāng) zhuōzi\\
            two \hphantom{*(}\Clf{} table\\
	\glt    \enquote*{two tables}
	\ex
    \gll    sān *(píng) jiǔ\\
            three \hphantom{*(}bottle  wine\\
    \glt    ‘three bottles of wine’
	\z\end{multicols}
\ex \label{ex:23.2}\gls{HKC} \parencites[14]{Sio2006}[272]{ChengSybesma2005}%2
	\begin{multicols}{2}\ea
	\gll saam1 *(bun2) syu1\\
    three \hphantom{*(}\Clf{}  book\\
	\glt ‘three books’
	\ex
	\gll jat1 *(bui1) seoi2\\
    one \hphantom{*(}cup water\\
    \glt ‘a cup of water’
	\z\end{multicols}}
\z

This paper focuses only on cases of enumerated common count nouns such as
(\ref{ex:23.1}a) and (\ref{ex:23.2}a), since measure words are
necessary to license the counting of mass nouns even in non-classifier
languages like English. Indeed, measure words such as those in
(\ref{ex:23.1}b) and (\ref{ex:23.2}b) are termed as
\enquote{massifiers} in \citet{ChengSybesma1998}, which are different from
(count-)classifiers as in the (a) sentences.  Massifiers are there to
\emph{create} a unit of measure, while the count-classifiers, or
\isi{classifiers} in short, are only there to \emph{name} the unit of counting
which are inherent to the entity itself.\footnote{According to
    \textcite{ChengSybesma1998}, massifiers can be used with mass and count
    nouns, such as, \emph{liǎng bēi shuǐ} \enquote*{two glasses of water} and
    \emph{yī qún niǎo} \enquote*{a flock of birds} -- massifiers with count
nouns have also been known as \enquote{group classifiers} as pointed out by a
reviewer.} New data present a systematic pattern that the
classifier\is{classifiers} can be optional, sometimes even disfavoured, in a
[\Num{}\,+\,\Clf{}\,+\,\textsc{n}] structure when the numeral\is{numerals} size
reaches a certain point. Furthermore, \gls{HKC}\il{Hong Kong Cantonese} has
been found much less permissive with this exception than Mandarin\il{Mandarin}.
This new pattern challenges the traditional view (i.a.~\citealt{Krifka1995};
\citealt{Chierchia1998};
\citealt{ChengSybesma1999,ChengSybesma2005,ChengSybesma2012};
\citealt{Doetjes1996}) that \isi{numerals} in a classifier\is{classifiers}
language like Chinese obligatorily require licensing by the
classifier\is{classifiers}; and forms a consistent picture with the general
observation that Cantonese more strictly requires classifiers for individuation
than Mandarin\il{Mandarin}.

\section{Beyond one, two, three: A new perspective}

\subsection{Theoretical background: \citet{Krifka1995} and \citet{Chierchia1998}}

\citet{Krifka1995} and \citet{Chierchia1998} offer two classical analyses for
Chinese-style classifier\is{classifiers} languages, where \isi{classifiers}
license enumeration.\footnote{This numeral-licensing function of Chinese-style
    \isi{classifiers} contrasts with the classifier\is{classifiers} system in
    languages like \ili{Japanese} \citep{Watanabe2006}, \ili{Purepecha}
    \parencite{VazquezRojasMaldonado2012}, and \ili{Niuean} \citep{Massam2009},
    where \isi{numerals} are classifier-licensing, i.e.  \isi{classifiers} can
    only occur when a numeral\is{numerals} is present. This can be seen in the
    cases of [\Clf{}\,+\,\textsc{n}] in argument positions in both Cantonese and
    Mandarin, though the two varieties differ in terms of whether such noun
    phrases can appear as subjects or not (cf.\ \citealt{ChengSybesma1999},
\citealt{Sio2006}).} Krifka suggests that the presence and absence of (the need
for) \isi{classifiers} is determined by whether the \isi{numerals} in the
language have a built-in measure function. In Mandarin\il{Mandarin}, he argues,
\isi{numerals} do not come with such a measure function, hence whether the
measuring unit is an \enquote{\glsdesc{OU}} (\glsunset{OU}\gls{OU}) -- a unit
that measures the number of specimens of a kind, or a \enquote{\glsdesc{KU}}
(\glsunset{KU}\gls{KU}) -- a unit that measures subspecies, is left
underspecified (Krifka coined that as \enquote{\glsdesc{OKU}}
(\glsunset{OKU}\gls{OKU})). Assuming that \gls{OU} or \gls{KU} can only apply
to objects but not kinds, the presence of a classifier\is{classifiers} not only
specifies which measuring unit is in use, but also generates an
object-referring interpretation for the entity denoted by the noun. The
contrary is true in English. English \isi{numerals} have this measure function
inherently, and hence can express what [\Num{}\,+\,\Clf] does in
Mandarin\il{Mandarin}.  This distinction in measure function of \isi{numerals}
has been used to account for typological differences between
classifier\is{classifiers} and non-classifier languages in \citet{Krifka1995};
but \citet{BaleCoon2014} has found in \ili{Mi'gmaq} (Algonquian) and \ili{Chol}
(Mayan) that such a distinction can appear within a language. In other words,
while \isi{numerals} in different languages can vary in terms of
present/absence of measure function hence producing non-classifier and
classifier languages respectively, different \isi{numerals} within a language
can also vary in the same way. In the latter case, some \isi{numerals} can go
directly with count nouns, but some cannot. In \ili{Mi'gmaq}, for instance,
\citeauthor{BaleCoon2014} reported that “numerals 1--5 (along with numerals
morphologically built from 1--5) do not appear with \isi{classifiers}, while
numerals 6 and higher must” \parencite[700]{BaleCoon2014}, as illustrated in
(\ref{ex:23.3}--\ref{ex:23.4}).

\ea\label{ex:23.3} \ili{Mi'gmaq} %3
    \ea[]{
	\gll na’n-ijig ji’nm-ug\\
		five-\Agr{} man-\Pl{}\\
    \glt ‘five men’}
    \ex[*]{
    \gll na’n te’s-ijig ji’nm-ug\\
		five \Clf-\Agr{} man-\Pl{}\\
    \glt}
	\z
\ex \label{ex:23.4} \ili{Mi'gmaq} %4
    \ea[*]{%
	\gll asugom-ijig ji’nm-ug\\
		six-\Agr{} man-\Pl{}\\
    \glt}
    \ex[]{%
	\gll asugom te’s-ijig ji’nm-ug\\
		six \Clf-\Agr{} man-\Pl{}\\
    \glt ‘six men’}
	\z
\z

\largerpage[1]
On the other hand, \citet{Chierchia1998} explains such difference between
Mandarin and English, or rather classifier\is{classifiers} and non-classifier
languages in general, by the inherent properties of their nominals. He suggests
that all common nouns in (Mandarin) Chinese are mass nouns; and all mass nouns
are inherently plural (a.k.a. \enquote{inherent plurality hypothesis}).
Chierchia explains that count nouns are inherently singular, and become
pluralised when used to refer to a set of singularities. Singular count nouns
form singleton sets and are rudimentary building blocks of all other plural
sets (Chierchia termed them \emph{atoms}). Thus, plural count nouns denoting a
group of singularities are conceptualised as union relations
(\textbf{${\cup}$}). To Chierchia, mass nouns are plural-like; only that plural
count nouns are sets formed by union of \emph{atoms} while mass nouns are “the
closure under \textbf{${\cup}$} of \emph{a set of atoms}”
\citep[70]{Chierchia1998}. In other words, mass nouns denote an enclosed union
of all sets, and in that way, neutralize the difference between plural (i.e.\
sets) and singular (i.e.\ atoms). Therefore, Chierchia suggests that
Mandarin\il{Mandarin} common nouns provide a neat exemplar for the four mass
nouns criteria in \REF{ex:23.5}.

\ea\label{ex:23.5} Mass properties of Chinese nouns \citep[94]{Chierchia1998} %5
	\ea There is no plural marking.
	\ex A numeral\is{numerals} can combine with a noun only through a classifier\is{classifiers}.
	\ex There is no definite or indefinite article.
	\ex Nouns can occur bare in argument position.
	\z
\z

Focussing mainly on the second property concerning the distribution of numerals
and \isi{classifiers}, empirical data in \Cref{sub:23.2.2} shows that the claim made in
(\ref{ex:23.5}b) is too strong to hold. Turning back to Krifka’s alternative,
the proposal that the need for classifier\is{classifiers} stems from the absence of a measure
function in \isi{numerals} seem more plausible, especially with the re-interpretation
in \citet{BaleCoon2014}. However, the patterns in Mandarin\il{Mandarin} and \gls{HKC}\il{Hong Kong Cantonese} are not as
clear-cut as that in \ili{Mi'gmaq} and \ili{Chol}, which may pose a challenge
to an analysis that is purely along the lines of Krifka.

\subsection{Number size and classifiers}\label{sub:23.2.2}

One key observation made from the examples used in existing literature on
Chi\-nese \isi{classifiers} is that most (if not all) examples are confined to
the numerals one, two, and three. This study has examined \isi{numerals} beyond
three.  \Cref{tab:23.1} has the list of \isi{numerals} tested; these are
all cardinal numbers.

\begin{table}
\caption{Chinese numerals\label{tab:23.1}}
\resizebox{\textwidth}{!}{\begin{tabular}{llll}
\lsptoprule
Mandarin              & \gls{HKC}\il{Hong Kong Cantonese}                   &                               & \\
\midrule
\emph{yī}             & \emph{jat1}                 & one                           & 1\\
\emph{liǎng}          & \emph{loeng5}               & two                           & 2\\
\emph{sān}            & \emph{saam1}                & three                         & 3\\
\emph{sì}             & \emph{sei3}                 & four                          & 4\\
\emph{wǔ}             & \emph{m5}                   & five                          & 5\\
\emph{shí}            & \emph{sap6}                 & ten                           & 10\\
\emph{shí-yī}         & \emph{sap6-jat1}            & ten-one                       & 11\\
\emph{shí-wǔ}         & \emph{sap6-m5}              & ten-five                      & 15\\
\emph{èr-shí}         & \emph{ji6-sap6}             & two-ten                       & 20\\
\emph{èr-shí-yī}      & \emph{ji6-sap6-jat1}        & two-ten-one                   & 21\\
\emph{sān-shí}        & \emph{saam1-sap6}           & three-ten                     & 30\\
\emph{sān-shí-yī}     & \emph{saam1-sap6-jat1}      & three-ten-one                 & 31\\
\emph{sì-shí}         & \emph{sei3-sap6}            & four-ten                      & 40\\
\emph{wǔ-shí}         & \emph{m5-sap6}              & five-ten                      & 50\\
\emph{yī-bǎi}         & \emph{jat1-baak3}           & one-hundred                   & 100\\
\emph{yī-bǎi-líng-wǔ} & \emph{jat1-baak3-ling4-m5}  & one-hundred-zero-five         & 105\\
\emph{yī-qiān}        & \emph{jat1-cin1}            & one-thousand                  & 1000\\
\emph{yī-wàn}         & \emph{jat1-maan6}           & one-ten.thousand              & 10000\\
\emph{yī-wàn-yī-qiān} & \emph{jat1-maan6-jat1-cin1} & one-ten.thousand-one-thousand & 11000\\
\lspbottomrule
\end{tabular}}
\end{table}

These nineteen \isi{numerals}, ranging from 1 to 11000, are used with eight common
count nouns in Mandarin\il{Mandarin} and \gls{HKC}\il{Hong Kong Cantonese} to form noun phrases which appear as
either subject or object in simple declarative sentences. The eight nouns
considered are presented in \tabref{ex:23.2}. They vary in terms of degree
of \isi{animacy} (from human to inanimate) and number of syllables (mono- or
disyllabic). (\ref{ex:23.6}) and (\ref{ex:23.7}) are some sample sentences.

\begin{table}
\caption{Chinese nouns: Animacy and phonological size\label{tab:23.2}}
\begin{tabularx}{\textwidth}{llQQ}
\lsptoprule
                          & \Clf{}                 & Mandarin\il{Mandarin}               & \gls{HKC}\il{Hong Kong Cantonese}        \\
\midrule
{}[$+$\textsc{human}]       & \emph{gè}/\emph{go3}   & \emph{rén} \enquote*{person}        & \emph{jan4} \enquote*{person}            \\
		                  &                        & \emph{xuéshēng} \enquote*{student}  & \emph{hok6saang1} \enquote*{student}     \\
{}[$+$\textsc{animate}]     & \emph{zhī}/\emph{zek3} & \emph{gǒu} \enquote*{dog}           & \emph{gau2} \enquote*{dog}               \\
		                  &                        & \emph{lánggǒu} \enquote*{wolfhound} & \emph{long4gau2} \enquote*{wolfhound}    \\
{}[$-$\textsc{animate}]     & \emph{kē}/\emph{po1}   & \emph{shù} \enquote*{tree}          & \emph{syu6} \enquote*{tree}              \\
		                  &                        & \emph{sōngshù} \enquote*{pine tree} & \emph{cung4syu6} \enquote*{pine tree}    \\
{}[$-$\textsc{animate}]     & \emph{běn}/\emph{bun2} & \emph{shū} \enquote*{book}          & \emph{syu1} \enquote*{book}              \\
		                  &                        & \emph{zìdiǎn} \enquote*{dictoinary} & \emph{zi6din2} \enquote*{dictoinary}     \\
\lspbottomrule
\end{tabularx}
\end{table}

\ea\label{ex:23.6} Mandarin\il{Mandarin} %6
	\ea
        \gll    \textit{sān-shí-yī} \textit{(gè)} \textit{rén} cānjiā-le bǐsài\\
                \textit{three-ten-one} \hphantom{\textit{(}}\textit{\Clf}{} \textit{person} join-\Pfv{} competition\\
	    \glt    ‘Thirty-one people joined the competition.’
	\ex
        \gll    wǒ yāoqǐng-le \textit{yī-bǎi} \textsuperscript{?/*}\textit{(ge)} \textit{xuesheng}\\
                I  invite-\Pfv{} \textit{one-hundred} \hphantom{\textsuperscript{?/*}(}\textit{\Clf} \textit{student}\\
	    \glt    ‘I invited one hundred students.’
	\z
\ex \label{ex:23.7} \gls{HKC}\il{Hong Kong Cantonese} %7
    \ea[]{
        \gll    \textit{saap6} \textit{*(po1)} \textit{syu6} sei2-zo2\\
                \textit{ten} \hphantom{\textit{*(}}\textit{\Clf} \textit{tree} die-\Pfv{}\\
        \glt    ‘Ten trees died.’}
    \ex[]{
        \gll    ngo5 maai5-zo2 \textit{ji6-saap6-jat1} \textit{*(bun2)} \textit{zi6din2}\\
                I buy-\Pfv{} \textit{two-ten-one} \hphantom{\textit{*(}}\textit{\Clf} \textit{dictionary}\\
        \glt    ‘I bought twenty-one dictionaries.’}
	\z
\z

Regarding the classifier--noun pairings in the study, all the common nouns
under investigation are paired with the only appropriate
classifier\is{classifiers} in the language. In Mandarin\il{Mandarin} \emph{gǒu}
‘dog’ appears with the classifier \emph{zhí} (e.g.\ \emph{ten
*gè}/\emph{zhī gǒu}), and in \gls{HKC}\il{Hong Kong Cantonese} \emph{syu6}
‘tree’ with \emph{po1} (e.g. \emph{saap6 *go3}/\emph{po1 syu6}). The only
\enquote{exception} is with the [+\textsc{human}] nouns, as there are two
possible \isi{classifiers} for the noun \emph{student} -- a general classifier
\emph{gè}/\emph{go3} and a specific one \emph{wèi}/\emph{wai2}. But for better
comparison with the monosyllabic [+\textsc{human}] noun \emph{person}, which
cannot go with the specific classifier\is{classifiers} \emph{wèi}/\emph{wai2},
the classifier\is{classifiers} used for both \emph{student} and \emph{person}
in this paper is the general classifier \emph{gè}/\emph{go3}.

In the acceptability judgment task, Mandarin\il{Mandarin} and \gls{HKC}\il{Hong Kong Cantonese} native
speakers\footnote{The results reported in this paper are taken from the
    acceptability judgment questionnaire from 2014. Four native
    Mandarin\il{Mandarin} speakers and four native \ili{Hong Kong Cantonese}
    speakers, aged 25--30,
    were consulted. Two of the Mandarin\il{Mandarin} speakers were from Guangdong
    province, and the other two from northern China near Tianjin; samples of
    both varieties were gender-balanced. Participants were asked to rate
    sentences on a four-point scale (0--3). By comparing with \isi{control} sentences,
    the scale of acceptability was established (in terms of average score):
    2.8--3.0 = completely acceptable (\ding{51}), 1.8--2.7 = marginally acceptable (?),
    1.3--1.7 = unacceptable (?/*), 0.0--1.2 = absolutely unacceptable (*). These
    terminologies will be consistently adopted in this paper. Since little
    regional variation has been found between southern and northern Mandarin
speakers, and for the convenience of exposition, the average judgment scores
will be presented in the text.} were asked to judge the acceptability of these
sentences with and without \isi{classifiers}. The judgement results have revealed
several interesting patterns.  First, both Mandarin\il{Mandarin} and \gls{HKC}\il{Hong Kong Cantonese} speakers
allow the [+\textsc{human}] count noun, \emph{person}, to take the [\Num{}\,+\,$\varnothing$\,+\,\textsc{n}]
structure, regardless of the value of the numeral\is{numerals}. However,
more precisely, in Mandarin\il{Mandarin}, tested \isi{numerals} higher than 10 are all rated
acceptable whether in subject or object positions.  In \gls{HKC}\il{Hong Kong Cantonese}, when the noun
phrase appears as object, the \isi{numerals} have to be greater than 30, but when the
noun phrase appears as subject, only \isi{numerals} higher than 100 are rated
acceptable. All other sentences with [\Num{}\,+\,${\varnothing}$\,+\,\emph{person}]
(as subject or object) are considered marginally acceptable (none completely
ill-formed).

Down the scale of \isi{animacy}, while \gls{HKC}\il{Hong Kong Cantonese} has a
pattern consistent with the traditional understanding, i.e.\ \isi{numerals}
must be licensed by \isi{classifiers}; Mandarin speakers allow null-classifier
\isi{enumeration} more liberally, especially with two sets of \isi{numerals}.
The first set involves high \isi{numerals} 1000, 10000, and 11000. In
Mandarin\il{Mandarin}, subject noun phrases allow these three \isi{numerals} to
occur without the mediation of a classifier\is{classifiers} whenever the noun
is animate (object noun phrases require a human noun).\footnote{In any case,
the noun concerned has to be disyllabic.} Even with nouns of lower
\isi{animacy}, these three numerals consistently show a higher score in
Mandarin\il{Mandarin} null-classifier noun phrases. More importantly, in
Mandarin\il{Mandarin}, the presence of a classifier\is{classifiers} is not
preferred when the noun \emph{rén} ‘person’ occurs with these three high
numerals: those Mandarin\il{Mandarin} sentences are considered marginally
acceptable (2.5 for subject, and 2.0 for object) when the
classifier\is{classifiers} is present, and completely acceptable (3.0) when it
is not. \gls{HKC}\il{Hong Kong Cantonese} noun phrases are much more restricted
for such exceptions: apart from the noun \emph{jan4} ‘person’, no other nouns
can be enumerated without the presence of a classifier, however large the
numeral\is{numerals} is.

One possible explanation for such unmediated quantification could be that the
classifier\is{classifiers} is still present in the structure but phonologically
(partially) covert. An anonymous reviewer has pointed out that there is often a
glottal stop between the numeral\is{numerals} and the noun whenever the
classifier is absent, presumably, where the noun is [+\textsc{human}] and hence
the potential classifier\is{classifiers} would be \emph{gè} in
Mandarin\il{Mandarin} or \emph{go3} in \gls{HKC}. In the Jin varieties of
northern China, for instance, their equivalent of \emph{gè} has been reported
to have a final glottal stop in addition to the one in the onset.\footnote{I
    thank a reviewer for introducing me to the observations in the Jin
varieties.} If the same unmediated quantification is found in the Jin
varieties, then what happens there could be that since there are two glottal
stops in the classifier \emph{gè}, one of them remains as the
\enquote{residue} of the classifier and licenses the numeral in the place of
the classifier itself.

However, empirically, the Mandarin\il{Mandarin} and Cantonese\il{Cantonese}
speakers consulted in this study have not displayed such an articulatory
feature, and even if it is indeed the case, the phonological reduction process
could only be acting as an additional trigger for the omission of the
classifier when the noun is [+\textsc{human}], but not as a sufficient
condition to account for the selective permissiveness of [\Num{}\,+\,$\varnothing$\,+\,N]
which is shown to be sensitive to \isi{animacy} and number
size. Otherwise, it would predict that (i) all [+\textsc{human}] nouns allow
[\Num{}\,+\,$\varnothing$\,+\,\textsc{n}] regardless of number size, and (ii) all
nouns that can appear with \emph{gè}/\emph{go3} (such as, \emph{apple},
\emph{ball}, and other [$-$\textsc{animate}] nouns) allow [\Num{}\,+\,$\varnothing$\,+\,\textsc{n}],
but neither is empirically true. In fact,
going back to the Mandarin and \gls{HKC} data, despite the absence of a glottal
stop in the coda position of the classifier \emph{gè}/\emph{go3}, there is one
in the onset. So, if, as the phonological reduction hypothesis goes, the
glottal stop between the numeral and the noun can act as a reduced form of the
classifier, then the glottal stop in the onset may work as well as the one in
the code position, but as aforementioned, such an articulatory feature has not
been observed and the phonological reduction hypothesis alone would have
overgeneralised the pattern of classifier-less enumeration in Mandarin and
\gls{HKC}.

Therefore, the classifier system in the Jin varieties certainly deserves
further investigation, but based on the Mandarin and \gls{HKC} data so far, a
more plausible explanation for the observed exception is that big numbers like
\emph{yì qían} ‘one thousand’ and \emph{yí wàn} ‘ten thousand’, like the
English \emph{thousands} and \emph{millions,} are not \isi{numerals}, but
measure words (Lisa Cheng, p.c.). It is indeed the case that a measure word
cannot co-occur with a classifier\is{classifiers}, as in \eqref{ex:23.8}.

\ea\label{ex:23.8} %8
	\ea Mandarin\\
    \gll * wǔ jīn kē cài\\
		{} five catty \Clf{} vegetable\\
    \ex \gls{HKC}\\%
	\gll * saap6-jat1 doi6 go3 ping4guo2\\
		{} ten-one bag \Clf{} apple\\
	\z
\z

Nevertheless, it is important to note that even though the presence of a
classifier may be disfavoured at times, [\Num{}\,+\,\Clf{}\,+\,\textsc{n}] is never
an unacceptable structure. In other words, the null-classifier structure is an
additional option, but never the only available option. Therefore, I suggest
that these high \isi{numerals} have an inherent measure function emerging in
Mandarin (à la \citealt{Krifka1995}), but has not yet been
grammaticalized\is{grammaticalization} into a proper measure word. Therefore,
when these high \isi{numerals} occur, the noun can either be individuated by
the measure function of the \isi{numerals} and does not require a
classifier\is{classifiers}, or be individuated by the
classifier\is{classifiers}. The preference for either of the two individuation
strategies varies from one speaker to another.

Another exception happens with the numeral\is{numerals} \emph{one}. Mandarin\il{Mandarin} speakers
consider direct \isi{enumeration} marginally acceptable when the count noun is
disyllabic and non-human. More specifically, when the noun phrase is a subject,
\emph{one} can go directly with any non-human count nouns (the scores range
from 2.0 to 2.3); and when it is an object, the count noun must denote an
animal or a plant (both scored 1.8) but not a completely inanimate object like
\emph{dictionary} (scored 1.3). A possible explanation for this pattern is
that Mandarin\il{Mandarin} is developing an indefinite article: the slight subject-object
asymmetry in the acceptability of [\emph{one}\,+\,\textsc{n}] may be a sign of this being
a still ongoing development. \citet{Chierchia1998} suggests that the indefinite
article is simply a variant for the first numeral\is{numerals}, and this is a
well-established grammaticalisation\is{grammaticalization} pathway (\citealt{HeineKuteva2002}).
Therefore, what \citeauthor{Chierchia1998} predicts for Mandarin\il{Mandarin} -- there is
“no morpheme that combines directly with a noun and means what \emph{a} means
in English” \parencite[91]{Chierchia1998} -- may not be correct, since the
presence of \emph{one} without the mediation of a classifier\is{classifiers}
can be interpreted as an indefinite article \eqref{ex:23.9}.

\ea\label{ex:23.9} Mandarin\il{Mandarin}\\
    \gll yì sōngshu sǐ-le\\
    	one pine.tree die-\Pfv{}\\
    \glt `One/a pine tree died.'
\z

\subsection{More than numbers}\label{sub:23.2.3}

The data presented in \Cref{sub:23.2.2} boils down to one general conclusion:
classifiers can be optional in licensing a numeral\is{numerals}, especially in
Mandarin\il{Mandarin}, depending on the size of the numeral\is{numerals}. This
observation points to two issues: (i) numeral\is{numerals} size can determine
the necessity of \isi{classifiers} for individuation -- \emph{one} and high
\isi{numerals} behave differently, and (ii) \gls{HKC} classifiers are much more
obligatory for individuation than Mandarin classifiers. The first issue has
been discussed in the previous section, thus this section is devoted to
discussing the cross-linguistic variations in the use of \isi{classifiers}.

The difference between Mandarin\il{Mandarin} and \gls{HKC}\il{Hong Kong Cantonese} in permitting [\Num{}\,+\,${\varnothing}$\,+\,\textsc{n}]
structures is consistent with a more general pattern that
\gls{HKC} more strictly requires the presence of \isi{classifiers} for individuation.
\Cref{tb:3,tb:4} summarise the Mandarin\il{Mandarin} and \gls{HKC}\il{Hong Kong Cantonese} classifier
paradigms.

\begin{figure}
    \renewcommand*{\arraystretch}{1.25}
    \fbox{\begin{small}
    \begin{tabularx}{\textwidth - 4\fboxsep}{@{}XXXXX@{}}
                             &                                 &                                                     &                                  & \\
\gls{PN}                     & \tn{xiao}{\emph{Xiǎomíng}}      & \cellcolor{gray!33!white}                           & \emph{Lǐsī} & \gls{PN}\\
                             & \enquote*{Xiaoming}             & \cellcolor{gray!33!white}                           & \enquote*{Lisi}                  & \\
common noun used as \gls{PN} & \emph{láobǎn}\newline\enquote*{boss}   & \cellcolor{gray!33!white}                           & \emph{làoshī} \enquote*{teacher} & common noun used as \gls{PN}\\
                             &                                 & \cellcolor{gray!33!white}\emph{{kànjiàn}}    &                                  & \\
\gls{CN}                     & \emph{mìfēng}                   & \cellcolor{gray!33!white}{see}        & \emph{dàngāo} & \gls{CN} \\
                             & \enquote*{(the) bee}            & \cellcolor{gray!33!white}{\enquote*{sees}} & \enquote*{(the) cake}            & \\
\tn{d}{*\Clf+\textsc{n}}     & \tn{mifeng}{*\emph{zhī mìfēng}} & \cellcolor{gray!33!white}                           & \tn{gedan}{\emph{gè dàngāo}}     & \tn{dprime}{\Clf{}+\textsc{n}} \\
                             & \hphantom{*}\Clf{} bee          & \cellcolor{gray!33!white}                           & \Clf{} cake                      & \\
                             & \tn{bee}{\enquote*{the bee}}    & \cellcolor{gray!33!white}                           & \tn{cake}{\enquote*{the/a cake}} & \\
\emph{one}+\Clf+\textsc{n}   & \emph{yì zhī mìfēng}            & \cellcolor{gray!33!white}                           & \tn{yige}{\emph{yī gè dàngāo}}   & \emph{one}+\Clf+\textsc{n} \\
                             & one \Clf{} bee                  & \cellcolor{gray!33!white}                           & one \Clf{} cake                  & \\
                             & \enquote*{a bee}                & \cellcolor{gray!33!white}                           & \tn{acake}{\enquote*{a cake}}    & \\
                             &                                 &                                                     &                                  & \\
    \end{tabularx}
    \end{small}}
    \tikz[remember picture, overlay] {
%        \node [fit=(d) (mifeng) (bee), draw, very thick, inner sep=1mm] {};
%        \node [fit=(gedan) (dprime) (cake), draw, very thick, inner sep=1mm] {};
        \node [fit=(xiao) (yige) (acake), draw, thick, inner sep=1.5mm] {};
    }
    \caption{Mandarin classifier\is{classifiers} paradigm}\label{tb:3}
\end{figure}

\begin{figure}
    \renewcommand*{\arraystretch}{1.25}
    \fbox{\begin{small}
    \begin{tabularx}{\textwidth - 4\fboxsep}{@{}XlXlX@{}}
                          &                                   &                                                     &                                             & \\
\gls{PN}                  & \tn{xiao}{\emph{Siu2ming4}}       & \cellcolor{gray!33!white}                           & \emph{Daai6man4} & \gls{PN}\\
                          & \enquote*{Siuming}                & \cellcolor{gray!33!white}                           & \enquote*{Daaiman}                          & \\
\gls{CN} used             & \emph{lou5ban2}                   & \cellcolor{gray!33!white}                           & \emph{lou5si1} & \gls{CN} used\\
as \gls{PN}               & \enquote*{boss}                   & \cellcolor{gray!33!white}                           & \enquote*{teacher}                          &  as \gls{PN}\\
                          &                                   & \cellcolor{gray!33!white}{\emph{gin3-dou2}}  & & \\
\tn{d}{*\gls{CN}}         & \tn{mifeng}{*\emph{mat6fong1}}    &
\cellcolor{gray!33!white}{see-\Compl{}}        &
\tn{gedan}{\textsuperscript{?}\emph{dan6go1}} & \tn{N}{\textsuperscript{?}\gls{CN}} \\
                          & \tn{bee}{\hphantom{*}\enquote*{bee}}          & \cellcolor{gray!33!white}{\enquote*{saw}} & \tn{cake}{\hphantom{\textsuperscript{?}}\enquote*{cake}}                  & \\
\Clf{}+\gls{CN}           & \emph{zak3 mat6fong1}            & \cellcolor{gray!33!white} & \emph{go3 dan6go1}      & \Clf+\gls{CN} \\
                          & \Clf{} bee            & \cellcolor{gray!33!white}                           & \Clf{} cake                                 & \\
                          & \enquote*{the bee}                & \cellcolor{gray!33!white}                           & \enquote*{the/a cake}                       & \\
\emph{one}+\Clf+\textsc{n}& \emph{jat1 zak3 mat6fong1}        & \cellcolor{gray!33!white} & \tn{yige}{\emph{jat1 go3 dan6go1}}                  & \emph{one}+\Clf{}+\gls{CN} \\
                          & one \Clf{} bee                    & \cellcolor{gray!33!white}                           & one \Clf{} cake                             & \\
                          & \enquote*{a bee}                  & \cellcolor{gray!33!white}                           & \tn{acake}{\enquote*{a cake}}               & \\
                          &                                   &                                                     &                                             & \\
    \end{tabularx}
    \end{small}}
    \tikz[remember picture, overlay] {
%        \node [fit=(d) (mifeng) (bee), draw, very thick, inner sep=1mm] {};
%        \node [fit=(gedan) (N) (cake), draw, very thick, inner sep=1mm] {};
        \node [fit=(xiao) (yige) (acake), draw, thick, inner sep=1.5mm] {};
    }
    \caption{\gls{HKC} classifier\is{classifiers} paradigm}\label{tb:4}
\end{figure}

On the one hand, \Cref{sub:23.2.2} has shown that \gls{HKC}\il{Hong Kong
Cantonese} only allows null-classifier \isi{enumeration} with the noun
\emph{jan4} ‘person’ and when the numeral is greater than 100 (for subject) or
30 (for object); on the other hand, \citet{ChengSybesma1999} have famously
identified that \gls{HKC}\il{Hong Kong Cantonese} allows [\Clf{}\,+\,\textsc{n}]
as both subject and object, whereas Mandarin\il{Mandarin} only allows it as
object.  What appears to be two separate issues, can be rethought as one if we
take another perspective on the second issue. \gls{HKC}\il{Hong Kong
Cantonese}, in fact, does not allow bare common nouns in subject position
(\emph{mat6fong1} \enquote*{bee} in \Cref{tb:4}), except when they act as
proper names (\emph{lou5ban2} \enquote*{boss} in \Cref{tb:4}).  Therefore,
instead of viewing the second issue as Mandarin\il{Mandarin} disallowing
[\Clf{}\,+\,\textsc{n}] as subjects, it is more appropriate to see it as
\gls{HKC}\il{Hong Kong Cantonese} requires a classifier\is{classifiers} for
subject noun phrases with a common count noun. In that case, the two issues are
unified to a general cross-linguistic variation that \gls{HKC}\il{Hong Kong
Cantonese} more obligatorily requires the presence of a
classifier\is{classifiers} for individuation, regardless of the need for
enumeration. To account for this requirement in \gls{HKC}\il{Hong Kong
Cantonese}, \citet{ChengSybesma1999} have suggested that \isi{classifiers}
express definiteness like the English determiner \emph{the}, hence a
classifier\is{classifiers} phrase (\textsc{clfp}) is projected whenever a definite
reading arises. Since they report that both \gls{HKC} [\Clf{}\,+\,\textsc{n}]s
and Mandarin\il{Mandarin} bare common nouns have a definite reading, the
difference between the \gls{HKC}\il{Hong Kong Cantonese} strategy and the
Mandarin\il{Mandarin} one is that the former has an overtly articulated
\Clf{}\textsuperscript{0} while the latter has an empty
\Clf{}\textsuperscript{0}. In contrast, since bare common nouns in \gls{HKC}
are not definite, the classifier\is{classifiers} phrase which encodes
definiteness is not projected in \gls{HKC}\il{Hong Kong Cantonese} bare common
nouns. Therefore, assuming that Chinese requires a definite subject, bare
common nouns cannot be subjects in \gls{HKC}\il{Hong Kong Cantonese}.

The issue of referentiality or definiteness can be a plausible explanation for
the [\Clf{}\,+\,\textsc{n}] and bare noun distinction in \gls{HKC}\il{Hong Kong Cantonese} and Mandarin\il{Mandarin}, but it
does not provide an answer for the difference in numeral-licensing function of
classifiers in the two Chinese varieties, since both [\Num{}\,+\,\Clf{}\,+\,\textsc{n}] and
[\Num{}\,+\,${\varnothing}$\,+\,\textsc{n}] are indefinite.\footnote{\citet{Huang2015} views
    this [\Clf{}\,+\,\textsc{n}] pattern from another perspective: numeral\is{numerals} requirement
    (more specifically, \emph{one} requirement). He interprets that Cantonese
    allows bare classifier\is{classifiers} phrases in both subject and object positions,
    Mandarin\il{Mandarin} restricts their occurrences to environments with a governing verb
    or preposition, and generally prohibits them in subject position. This
    observation is captured in the null numeral\is{numerals} ‘one’ micro-parameter (i).

\begin{exe}
    \exi{(i)}
    \begin{xlist}
	\ex In Mandarin\il{Mandarin}, [\textsubscript{one} e] is [$-$strong], triggering \isi{Agree} with \Clf{}.
	\ex In Cantonese, [\textsubscript{one} e] is [+strong], triggering Move of \Clf{}.
    \end{xlist}
\end{exe}

In short, \citeauthor{Huang2015} claims that Cantonese has a [+strong] number
head, and Mandarin\il{Mandarin} a [$-$strong] one. This interpretation of the classifier
paradigms is insightful, but still fails to capture the new data on
null-classifier \isi{enumeration} presented in this paper.} The answer to this
cross-linguistic variation in classifier\is{classifiers} use can be found in three related
phenomena in Mandarin\il{Mandarin} (none attested in \gls{HKC}): (i) the development of
\emph{one} as an indefinite article (see \Cref{sub:23.2.2}); (ii) the presence
of special forms for \emph{two} and \emph{three} -- \emph{liǎ} ‘two/two of’
and \emph{sā} ‘three/three of’ (\ref{ex:23.12}); (iii) the use of plural
marker \emph{-men} for animate nouns/noun phrases (\ref{ex:23.13}).\largerpage[-2]

\ea\label{ex:23.12} Mandarin\il{Mandarin} %12
	\ea
        \gll    wǒ-men \textit{liǎ} / \textit{sā} \textit{(*gè)} shì hǎo píngyou\\
                \Fpl{} \textit{two} {} \textit{three} \hphantom{\textit{(*}}\textit{\Clf{}} be good friend\\
	    \glt    ‘We two/three are good friends.’
	\ex
	    \gll    \textit{liǎ} \textit{(*kē)} shù sǐ-le\\
                \textit{two} \hphantom{\textit{*(}}\textit{\Clf} tree die-\Pfv{}\\
	    \glt    ‘Two trees died.’
	\z
\ex \label{ex:23.13} Mandarin\il{Mandarin} %13
    \ea[]{%
	    \gll    xuésheng-\textit{men} xǐhuan chī miàn\\
	    	    student-\textit{\Pl} like  eat  noodles\\
        \glt    ‘The students like to eat noodles.’}
    \ex[]{%
	    \gll    shí-èr \textit{gè} xuésheng xǐhuan chī miàn\\
                ten-two \textit{\Clf} student  like  eat  noodles\\
        \glt    ‘Twelve students like to eat noodles.’}
    \ex[*]{
        \gll    shí-èr \textit{gè} xuésheng\textit{-men} xǐhuan chī miàn\\
	    	    ten-two  \textit{\Clf}  student-\textit{\Pl}  like  eat  noodles\\
        \glt    intended: ‘Twelve students like to eat noodles.’}
	\z
\z

All three developments have one common property: the presence of classifiers
become either optional or disallowed. The development of \emph{one} as
indefinite article in Mandarin\il{Mandarin} allows the classifier\is{classifiers} to be optional when
\emph{one} appears with non-human (disyllabic) count nouns. The two special
forms for \emph{two} and \emph{three} in Mandarin\il{Mandarin} cannot occur with
classifiers, because they themselves mean ‘two of’ and ‘three of’ respectively,
meaning that they have inherent measure functions, just as the three high
numerals 1000, 10000, and 11000. Finally, the fact that the Mandarin\il{Mandarin} plural
\emph{-men} is much more developed than its \gls{HKC}\il{Hong Kong Cantonese} counterpart
(\emph{-dei6}) which can only suffix on pronouns, is another piece of evidence
showing that Mandarin\il{Mandarin} \isi{enumeration} is less dependent on the use of \isi{classifiers}.
However, this only suggests that plural-marking and \isi{classifiers} are competing
strategies for the \isi{enumeration} function, but not that they are
morpho-phonological competitors, as they take up different structural
positions. In Mandarin\il{Mandarin} and Cantonese, for instance, \isi{classifiers} are in
pre-nominal position, while plural markers are post-nominal. Borer formalises
the difference as: “the plural marker is a spell-out of an abstract head
feature \tuple{div} [divided] on a moved N-stem, while the
classifier\is{classifiers} is an independent f[unction]-morph occurring in the
left-periphery of the N” (\citeyear[95]{Borer2005}), as represented in\largerpage
(\ref{ex:23.14}a,b)
below:\footnote{Adapted from \citet[95]{Borer2005}, the open value
    \tuple{e}\textsubscript{DIV} is the classifier\is{classifiers} head, and
\tuple{div} is the plural head feature. The co-superscripts (e.g.\ \emph{max})
indicate range assignment relations.}\pagebreak

\begin{exe}
\ex\label{ex:23.14}
{\premulticols=0pt\postmulticols=0pt
\begin{multicols}{2}\raggedcolumns
\begin{xlist}
\ex\relax [$+$\Clf{} $-$\Pl]\\
        \begin{tikzpicture}[baseline]

            \Tree 	[.\Clf\textsuperscript{max}
                        \Clf\textsuperscript{2}
                        [.{}
                            \tuple{e\textsuperscript{2}}\tss{DIV}
                            N\textsuperscript{max}
                        ]
                    ]

        \end{tikzpicture}
\columnbreak
    \ex\relax [$-$\Clf{} $+$\Pl]\\
        \begin{tikzpicture}[baseline]

            \Tree 	[.\Clf\textsuperscript{max}
                        \node (n) {N.\tuple{div\textsuperscript{2}}};
                        [.{}
                            \tuple{e\textsuperscript{2}}\tss{DIV}
                            \node (t) {N\textsuperscript{max}\\t};
                        ]
                    ]

                    \draw [arrow, bend left = 60] (t.south) to (n.south);

        \end{tikzpicture}
\end{xlist}
\end{multicols}}
\end{exe}

Borer’s explanation suffices for the complementary distribution of classifiers
and plural markers but does not account for differences in the distribution of
Cantonese and Mandarin\il{Mandarin} \isi{classifiers}.

\section{Implications}

This paper has presented new empirical data from mainland Mandarin\il{Mandarin} and \gls{HKC}, and a new perspective in viewing the
classifier\is{classifiers} paradigms of the two Chinese varieties, particularly
regarding the variation in distribution of bare classifier phrases in subject
position. While previous studies have examined the issue from the angle of
definiteness-encoding (\citealt{ChengSybesma1999}) -- bare nouns vs.\ bare
classifier\is{classifiers} phrases, and strength of numeral\is{numerals} head
\citep{Huang2015} -- numeral\is{numerals} phrases with \emph{one} vs.\ bare classifier
phrases, neither can account for empirical cases where \isi{classifiers} are
optional in licensing \isi{numerals} in Chinese (especially Mandarin). Therefore,
this paper opens a new way to rethink this puzzle by showing (i) how numeral
size, \isi{animacy}, and phonological size can determine classifier\is{classifiers}
obligatoriness, and (ii) three related phenomenon that happened exclusively in
Mandarin\il{Mandarin} which weaken the need for \isi{classifiers} in
its individuation function -- \emph{one} as indefinite article, special forms
for \emph{two} and \emph{three}, and plural marker with animate count nouns.
These together should offer a more unified picture for the use of
\isi{classifiers} in Mandarin\il{Mandarin} and \gls{HKC}\il{Hong Kong
Cantonese}.

\printchapterglossary{}

{\sloppy\printbibliography[heading=subbibliography,notkeyword=this]}

\end{document}
